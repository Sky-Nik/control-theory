% cd ..\..\Users\NikitaSkybytskyi\Desktop\control-theory
% cls && pdflatex ControlTheoryAlgorithms.tex && cls && pdflatex ControlTheoryAlgorithms.tex && start ControlTheoryAlgorithms.pdf

\documentclass[a5paper, 10pt]{article}
\usepackage[T2A,T1]{fontenc}
\usepackage[utf8]{inputenc}
\usepackage[english, ukrainian]{babel}
\usepackage{amsmath, amssymb}
\usepackage[top = 2 cm, left = 1 cm, right = 1 cm, bottom = 2 cm]{geometry} 

\usepackage{fancyhdr}
\pagestyle{fancy}
\lhead{Алгоритми теорії керування}
\rhead{Нікіта Скибицький}
\cfoot{\thepage}

\usepackage{amsthm}
\newtheorem{definition}{Визначення}
\theoremstyle{definition}
\newtheorem*{problem*}{\normalfont{\textit{Задача}}}
\newtheorem{problem}{\normalfont{\textit{Задача}}}[section]
\newtheorem{algorithm}{\tt Алгоритм}[section]
\newtheorem*{solution}{Розв'язок}

\allowdisplaybreaks
\setlength\parindent{0pt}
\numberwithin{equation}{section}

\usepackage{graphicx}

\newcommand{\JJ}{\mathcal{J}}
\newcommand{\KK}{\mathcal{K}}
\newcommand{\MM}{\mathcal{M}}
\newcommand{\UU}{\mathcal{U}}
\newcommand{\XX}{\mathcal{X}}
\newcommand{\BB}{\mathcal{B}}
\newcommand{\NN}{\mathcal{N}}
\newcommand{\HH}{\mathcal{H}}
\newcommand{\EE}{\mathcal{E}}
\newcommand{\RR}{\mathbb{R}}
\newcommand{\Max}{\displaystyle\max\limits}

\newcommand*\diff{\mathop{}\!\mathrm{d}}

\renewcommand{\phi}{\varphi}
\renewcommand{\SS}{\mathcal{S}}
\renewcommand{\epsilon}{\varepsilon}

\DeclareMathOperator{\erf}{erf}
\DeclareMathOperator{\erfi}{erfi}
\DeclareMathOperator{\signum}{sgn}
\DeclareMathOperator*{\argmin}{arg\,min}

\begin{document}

\section{Системи керування. Постановка задачі оптимального керування}

\subsection{Алгоритми}

\begin{problem*}
	Задана лінійна система керування 
	\begin{equation} 
	    \label{eq:algo-1-1}
	    \frac{\diff x(t)}{\diff t} = A (t) \cdot x(t) + B(t) \cdot u(t),
	\end{equation} 
	де $x(t) \in \RR^n$ -- вектор фазових координат, $A(t) \in \RR^{n\times n}$, $u \in \RR^m$ -- відоме керування, $B(t) \in \RR^{n \times m}$, $t \in [t_0, T]$, з початковими умовами $x(t_0) = x_0$, де $x_0 \in \RR^n$. Необхідно:
	\begin{enumerate}
		\item Визначити клас керування (програмне чи з оберненим зв'язком).
		\item Знайти траєкторію системи, що відповідає заданому керуванню.
		\item Звести задане керування до програмного.
		\item Перевірити траєкторію на неперервну диференційованість.
		\item Порівняти задане керування з іншим керуванням відносно заданого критерію якості 
		\begin{equation}
		    \label{eq:algo-1-2}
		    \JJ(u) = \int_{t_0}^T f(x(t), u(t), t) \diff t + \Phi(x(T)) \to \min    
		\end{equation}
		\item Знайти фундаментальну матрицю системи, нормовану за моментом $s$. %, де $s \in \RR^1$.
		\item Побудувати спряжену систему диференціальних рівнянь.
	\end{enumerate}
\end{problem*}

\begin{algorithm}
    \label{algo-1-1}
	Розглянемо всі пункти задачі описані вище.
	\begin{enumerate}
		\item Якщо $u = u(t)$ не залежить від $x$, то керування програмне, інакше ($u = u(x, t)$) керування з оберненим зв'язком.
		\item Для знаходження траєкторії %просто
		розв'язується система (\ref{eq:algo-1-1}) з підставленим $u$.
		\item У визначенні $u = u(x, t)$ $x$ замінюється на знайдену у попередньому пункті траєкторію $x = x(t)$.
		\item Це питання має сенс якщо керування кусково-неперервне, тоді у всіх точках розриву % негладкості?
		необхідно розглянути односторонні похідні і перевірити їх на рівність. Якщо похідні рівні то траєкторія неперервно диференційовна, інакше ні.
		\item Значення критерію якості (\ref{eq:algo-1-2}) обчислюється на обох керуваннях і порівнюється.
		\item Розв'язується система диференціальних рівнянь 
		\begin{equation}
		    \label{eq:algo-1-3}
		    \frac{\diff \Theta(t, s)}{\diff t} = A(t) \cdot \Theta(t, s), \quad \Theta(s, s) = I.
		\end{equation}
		\item Спряженою системою до системи (\ref{eq:algo-1-1}) називається система вигляду
		\begin{equation}
		    \label{eq:algo-1-4}
		    \frac{\diff y(t)}{\diff t} = - A^*(t) \cdot y(t),
		\end{equation}
		де $y = (y_1, \ldots, y_n)^*$.
	\end{enumerate}
\end{algorithm}

\vspace*{\baselineskip}

\begin{problem*}
	Звести задачу Лагранжа (або Больца) з функціоналом вигляду (\ref{eq:algo-1-2}) за умов (\ref{eq:algo-1-1}) до задачі Майєра.
\end{problem*}

\begin{algorithm}
    \label{algo-1-2}
    Зведення відбувається у кілька кроків:
	\begin{enumerate}
		\item Вводиться змінна \[x_{n+1} (t) \overset{\text{def}}{=} \int_{t_0}^t f(x(t), u(t), t) \diff t.\]
		\item Тоді \[ \JJ(u) = x_{n+1} (T) + \Phi(T) \to \inf. \]
		\item До системи додається умова \[ \frac{\diff x_{n+1}(t)}{\diff t} = f(x(t), u(t), t). \]
	\end{enumerate}
\end{algorithm}
 \newpage

\section{Елементи багатозначного аналізу. Множина досяжності}

\subsection{Алгоритми}

\begin{problem*}
	Знайти
	\begin{enumerate}
		\item $A+B$;
		\item $\lambda A$;
		\item $\alpha(A,B)$;
		\item $MA$,
	\end{enumerate}
	де множини $A \subset\RR^m$, $B\subset\RR^m$, скаляр $\lambda\in\RR^1$, матриця $M\in\RR^{n\times m}$.
\end{problem*}

\begin{algorithm}
	\label{algo-2-1}
	Розглянемо всі пункти задачі вище.
	\begin{enumerate}
		\item Знаходимо за визначенням, $A+B=\{a+b|a\in A,b\in B\}$.
		\item Знаходимо за визначенням, $\lambda A =\{\lambda a|a\in A\}$. 
		\item \begin{enumerate}
			\item Знаходимо відхилення $\beta(A,B)$ і $\beta(B,A)$ за визначенням, 
			\begin{equation}
				\label{eq:2.1}
				\beta(A,B) = \max_{a\in A}\rho(a,B),
			\end{equation}
			де 
			\begin{equation}
				\label{eq:2.2}
				\rho(a,B) = \min_{b\in B} \rho(a,b).
			\end{equation}
			\item Знаходимо $\alpha(A,B)$ за визначенням, 
			\begin{equation}
				\label{eq:2.3}
				\alpha(A,B)=\max\{\beta(A,B),\beta(B,A).
			\end{equation}
		\end{enumerate} 
		\item Знаходимо за визначенням, $MA=\{Ma|a\in A\}$.
	\end{enumerate}
\end{algorithm}

\vspace*{\baselineskip}

\begin{problem*}
	Знайти опорну функцію множини $A \subset \RR^n$.
\end{problem*}

\begin{algorithm}
	\label{algo-2-2}
	\begin{enumerate}
		\item Намагаємося знайти за визначенням,
		\begin{equation}
		 	\label{eq:2.4}
		 	c(A,\psi) = \Max_{a\in A} \langle a, \psi \rangle.
		\end{equation}
		\item Якщо не вийшло, то намагаємося знайти за геометричною властивістю: $c(A,\psi)$ -- (орієнтована) відстань від початку координат до опорної площини множини $A$, для якої напрямок-вектор $\psi$ є вектором нормалі.
	\end{enumerate}
\end{algorithm}

\vspace*{\baselineskip}

\begin{problem*}
	Знайти інтеграл Аумана $\JJ = \int F(x) \diff x$, де $F(x)\subset\RR^n$.
\end{problem*}

\begin{algorithm}
	\label{algo-2-3}
	\begin{enumerate}
		\item Знаходимо опорну функцію від інтегралу:
		\begin{equation}
		 	\label{eq:2.5}
		 	c(\JJ, \psi) = \int c (F(x), \psi) \diff x.
		\end{equation}
		\item Знаходимо $\JJ$ як опуклий компакт з відомою опорною функцією $c(\JJ, \psi)$.
	\end{enumerate}
\end{algorithm}

\vspace*{\baselineskip}

\begin{problem*}
	Знайти множину досяжності системи $\dot x = A x + B u$, де $x(t_0) \in \mathcal{M}_0$, $u \in \mathcal{U}$.
\end{problem*}

\begin{algorithm}
	\label{algo-2-4}
	\begin{enumerate}
		\item Знаходимо фундаментальну матрицю $\Theta(t,s)$ системи нормовану за моментом $s$.
		\item Знаходимо інтеграл Аумана
		\begin{equation}
			\label{eq:2.6}
			\int_{t_0}^t \Theta(t, s) B(s) \UU(s) \diff s
		\end{equation}
		за алгоритмом \ref{algo-2-3}.
		\item Використовуємо теорему про вигляд множини досяжності лінійної системи керування:
		\begin{equation}
			\label{eq:2.7}
		 	\XX(t, \MM_0) = \Theta(t, t_0) \MM_0 + \int_{t_0}^t \Theta(t, s)B(s)\UU(s) \diff s.
		\end{equation}
	\end{enumerate}
\end{algorithm}

\vspace*{\baselineskip}

\begin{problem*}
	Знайти опорну функцію множини досяжності системи $\dot x = A x + B u$, де $x(t_0) \in \mathcal{M}_0$, $u \in \mathcal{U}$.
\end{problem*}

\begin{algorithm}
	\label{algo-2-5}
	\begin{enumerate}
		\item Знаходимо фундаментальну матрицю $\Theta(t,s)$ системи нормовану за моментом $s$.
		\item Знаходимо опорну функцію $c(\MM_0, \Theta^*(t, t_0) \psi)$ за алгоритмом \ref{algo-2-2}.
		\item Знаходимо опорну функцію $c(\UU(s), B^*(s) \Theta^*(t, s) \psi)$ за алгоритмом \ref{algo-2-2}.
		\item Використовуємо теорему про вигляд опорної функції множини досяжності лінійної системи керування: 
		\begin{equation}
			\label{eq:2.8}
			c(\XX(t, \MM_0), \psi) = c(\MM_0, \Theta^*(t, t_0) \psi) + \int_{t_0}^t c(\UU(s), B^*(s) \Theta^*(t, s) \psi) \diff s.
		\end{equation}
	\end{enumerate}
\end{algorithm}

 \newpage

\section{Задача про переведення системи з точки в точку. Критерії керованості лінійної системи керування}

\subsection{Алгоритми}

\begin{problem*}
	Перевести систему $\dot x = A x + B u$ з точки $x_0$ в точку $x_T \in \RR^1$ за допомогою керування з класу $K$ (керування, залежні від вектору параметрів $c$).
\end{problem*}

\begin{algorithm}
	\begin{enumerate}
		\item Знаходимо траєкторію системи при заданому керуванні (залежну від параметра $c$). 
		\item Знаходимо з отриманого алгебраїчного рівняння параметр $c$.
	\end{enumerate}
\end{algorithm}

\begin{problem*}
	\begin{enumerate}
		\item Знайти грамміан керованості системи $\dot x = A x + B u$ за визначенням.
		\item Записати систему диференційних рівнянь для знаходження грамміана керованості.
		\item Використовуючи грамміан керованості, знайти інтервал повної керованості системи.
		\item Для цього інтервалу записати керування яке певеродить систему з точки $x_0$ в точку $x_T$ (або розв'язати задачу оптимального керування).
	\end{enumerate}
\end{problem*}

\begin{algorithm}
	Розглянемо всі пункти задачі вище.
	\begin{enumerate}
		\item \begin{enumerate}
			\item Знаходимо $\Theta(T,s)$.
			\item Використовуємо формулу \[\Phi(T, t_0) = \int_{t_0}^T \Theta(T, s) B(s) B^*(s) \Theta^*(T, s) \diff s.\]
		\end{enumerate}
		\item Записуємо систему \[ \dot \Phi(t, t_0) = A(t) \cdot \Phi(t,t_0)+\Phi(t,t_0)\cdot A^*(t)+B(t)\cdot B^*(t), \Phi(t_0,t_0) = 0. \]
		\item Це інтервал на якому $\Phi(t,t_0) \ne 0$.
		\item Використовуємо формулу \[ u (t) = B^*(t) \cdot \Theta(T,t)\cdot\Phi^{-1}(T,t_0)(x_T-\Theta(T,t_0)\cdot x_0). \]
	\end{enumerate}
\end{algorithm}

\begin{problem*}
	Дослідити стаціонарну систему $\dot x = A x + B u$ на керованість використовуючи другий критерій керованості.
\end{problem*}

\begin{algorithm}
	\begin{enumerate}
		\item Знаходимо $D = \left(B \vdots AB \vdots A^nB \vdots\ldots\vdots A^{n-1}B\right)$.
		\item Якщо $rang D = n$ то стаціонарна системи цілком керована, інакше ні.
	\end{enumerate}
\end{algorithm}
 \newpage

\section{Критерії спостережуваності. Критерій двоїстості}

\subsection{Алгоритми}

\begin{problem*}
    Побудувати систему для знаходження грамміана спостережуваності для системи $\dot x = A x$, $y = H x$.
\end{problem*}

\begin{algorithm} \tt
    Записуємо систему \[ \dot \NN(t, t_0) = -A(t) \cdot \NN(t, t_0) - \NN(t, t_0) \cdot A^*(t) + H^*(t) \cdot H(t). \]
\end{algorithm}

\begin{problem*}
    Чи буде стаціонарна система $\dot x = A x + B u$ цілком спостережуваною?
\end{problem*}

\begin{algorithm} \tt 
    \begin{enumerate}
        \item Знаходимо \[ \mathcal{R} = \left(H^* \vdots A^* H^* \vdots (A^*)^2 H^* \vdots \ldots \vdots (A^*)^{n-1} H^*\right).\] 
        \item Якщо $rang \mathcal{R} = n$ то система цілком спостережувана інакше ні.
    \end{enumerate}
\end{algorithm}

\begin{problem*}
    Дослідити на спостережуваність систему $\dot x = A x$, $y = H x$, використовуючи критерій двоїстості і відповідний критерій керованості.
\end{problem*}

\begin{algorithm} \tt
    \begin{enumerate}
        \item Будуємо спряжену систему \[ \frac{\diff z(t)}{\diff t} = - A^*(t) \cdot z(t) + H^*(t) \cdot u(t). \]
        \item Досліджуємо її на керованість, якщо вона керована, то по\-чат\-ко\-ва \allowbreak сис\-те\-ма спостережувана, інакше ні.
    \end{enumerate}
\end{algorithm}

\begin{problem*}
    Побудувати спостерігач для системи $\dot x = A x$, $y = H x$.
\end{problem*}

\begin{algorithm} \tt
    \begin{enumerate}
        \item Записуємо систему (спостерігач) \[ \dot{\hat{x}} (t) = (A (t) - K (t) H (t)) \cdot \hat x (t) + K(t) \cdot y(t).\]
        \item Або систему (спостерігач) \[ \dot{\hat{x}} (t) = A (t) \cdot \hat x (t) + K(t) \cdot (y(t) - H(t) \cdot \hat x(t)).\]
    \end{enumerate}
    Вибір вільний, це різні форми запису одного і того ж.
\end{algorithm}

\begin{problem*}
    Задана динамічна система $\dot x = A x$, $y = H x$. Знайти розв'язок задачі спостереження з використання грамміана спостержуваності.
\end{problem*}

\begin{algorithm} \tt
    \begin{enumerate}
        \item Знаходимо грамміан спостережуваності $\NN(t,t_0)$ з сис\-те\-ми \[ \dot \NN(t, t_0) = -A(t) \cdot \NN(t, t_0) - \NN(t, t_0) \cdot A^*(t) + H^*(t) \cdot H(t). \]

        \item Знаходимо $R(t) = \NN^{-1}(t, t_0)$.

        \item Розв'язуємо рівняння \[ \dot x(t) = A(t) \cdot x(t) + R(t) \cdot H^*(t) \cdot (y(t) - H(t) \cdot x(t)). \]
    \end{enumerate}
\end{algorithm}

 \newpage

\section{Задача фільтрації. Множинний підхід}

\subsection{Алгоритми}

\begin{problem*}
	Задана динамічна система $\dot x = A x + v$, $y = G x + w$, де $v(t) \in \RR^1$, $w(t) \in \RR^1$ -- невідомі шуми, $x_0 \in \RR^1$ -- невідома початкова умова, $y(t) \in \RR^1$ -- відомі спостереження. 

	\begin{enumerate}
		\item Побудувати інформаційну множину такої системи в момент $\tau \in [0, T]$ за умови, що \[ \int_0^\tau (Mv^2(s) + Nw^2(s)) \diff s + p_0x^2(0) \le \mu^2. \]
		\item Знайти похибку оцінювання.
	\end{enumerate}
\end{problem*}

\begin{algorithm} \tt
	\begin{enumerate}
		\item \begin{enumerate}
			\item Знайдемо $R(t)$ з рівняння Бернуллі \[\dot R (t)= A (t) \cdot R (t)+ R (t) \cdot A^* (t)- R (t) \cdot G^* (t) \cdot N (t) \cdot G (t) \cdot R(t), \quad R(t_0) = p_0^{-1}. \]
			\item Знайдемо $K(t)$ за формулою \[K (t)= R (t) \cdot G^* (t) \cdot N(t).\]
			\item Знайдемо фільтр (спостерігач) за формулою \[ \dot{\hat{x}} (t) = A (t) \cdot \hat x (t) + K (t) \cdot (y (t) - G (t) \cdot \hat x (t)). \]
			\item Знайдемо $k(s)$ з системи \[ \dot k (s) = \langle N(s) (y(s) - G(s) \cdot \hat x(s)), y(s) - G(s) \cdot \hat x(s)\rangle, \quad k(t_0) = 0. \]
			\item Знайдемо $\XX(\tau)$ за формулою \[ \XX(\tau) = \EE (\hat x(\tau), (\mu^2 - k(\tau)) \cdot R(\tau)). \]
		\end{enumerate}
		\item Похибка $e(\tau)$ оцінювання задовольняє оцінці \[ |e(\tau)| \le \sqrt{\mu^2-k(\tau)} \cdot \sqrt{\lambda_* (R(\tau))}.\]
	\end{enumerate}
\end{algorithm}
 \newpage

\section{Варіаційний метод в задачі оптимального керування}

\subsection{Алгоритми}

\begin{problem*}
	Знайти першу варіацію за Лагранжем і похідну Фреше в просторі інтегрованих з квадратом функцій для функціоналу $\JJ = \int f(u) \diff s$.
\end{problem*}

\begin{algorithm} \tt
	\begin{enumerate}
		\item Записуємо $\JJ(u + \alpha h)$.
		\item Знаходимо $\frac{\diff}{\diff \alpha} \JJ(u + \alpha h)$.
		\item Знаходимо першу варіацію $\delta \JJ (u, h)$ за Лагранжем за формулою \[ \delta \JJ(u, h) = \frac{\diff}{\diff \alpha} \left.\JJ(u + \alpha h)\right|_{\alpha = 0}. \]
		\item Якщо \[\delta \JJ (u, h) = \int h(s) \cdot g(s) \diff s,\] то $g(s)$ -- похідна за Фреше.
	\end{enumerate}
\end{algorithm}

\begin{problem*}
	Побудувати рівняння у варіаціях для системи керування $\dot x = A x + B u$.
\end{problem*}

\begin{algorithm} \tt
	Рівняння у варіаціях має загальний вигляд \[ \frac{\diff z(t)}{\diff t} = \frac{\partial f(x(t),u(t),t)}{\partial x} \cdot z(t) + \frac{\partial f(x(t),u(t),t)}{\partial u} \cdot h(t), \quad z(0) = 0. \]
\end{algorithm}

\begin{problem*}
	Знайти першу варіацію за Лагранжем і похідну Фреше в просторі інтегрованих з квадратом функцій для задачи оптимального керування варіаційним методом \[ \JJ = \int f \diff s + \Phi(T) \to \inf \] за умови, що \[ \dot x = f_0 (x, u), \] і розв'язати цю задачу.
\end{problem*}

\begin{algorithm} \tt
	\begin{enumerate}
	\item Позначимо $\phi (\alpha) = \JJ (u + \alpha h)$. 

	\item Знайдемо $\phi' (\alpha)$.

	\item Підставляючи $\alpha = 0$, знаходимо \[\phi' (0) = \int ... \diff s + \underset{=-\psi(T)}{\underbrace{\Phi_1(T)}} \cdot z (T). \]

	\item Запишемо рівняння у варіаціях на функцію $z(t)$.

	\item Введемо додаткові, спряжені змінні $\psi$ такі, що \[ \psi (T) = - \frac{\partial \Phi (x (T))}{\partial x}. \] 

	\item Тоді \[\left\langle \frac{\partial \Phi (x (T))}{\partial x}, z (T) \right\rangle = - \langle \psi (T), z (T) \rangle.\] 

	\item Враховуючи рівняння у варіаціях, маємо
	\begin{align*}
		\psi (T) \cdot z (T) &= \psi (T) \cdot z (T) - \psi (t_0) \cdot z (t_0) = \\
		&= \int_{t_0}^T \left( \psi (s) \cdot z' (s) + \psi' (s) \cdot z (s) \right) \diff s = ...
	\end{align*}

	\item Підставимо це у вигляд $\phi' (0)$:
	\[ \phi' (0) = - \int (\psi' + ...) \cdot z \diff s + \int (...) \cdot h(s)) \diff s.\]

	\item Накладаємо на функцію $\psi(t)$ умову (спряжену систему) \[ \frac{\diff \psi (t)}{\diff t} = - \frac{\partial f (x (t), u (t), t)}{\partial x} \cdot \psi (t) + \frac{\partial f_0 (x (t), u (t), t)}{\partial x} = 0, \]

	\item Завдяки цьому у $\delta \JJ (u, h) = \phi' (0)$ перший інтеграл зануляється. \\

	\item Знаходимо $\JJ ' (u)$

	\item З необхідної умову екстремуму функціоналу, $\JJ' (u_*) = 0$, зна\-хо\-ди\-мо $u_*$.

	\item Далі \[x_*(t) = x_0 + \int_0^t f(x(s), u_*(s), s) \diff s.\]

	\item Покладаючи $t = T$ знаходимо $x (T)$.

	\item Остаточно знаходимо $u_*$, $x_*$.
	\end{enumerate}
\end{algorithm}

 \newpage

\section{Принцип максимуму Понтрягіна для задачі з вільним правим кінцем}

\subsection{Алгоритми}

\begin{problem*}
    Записати крайову задачу принципу максимуму для задачі оптимального керування: \[ \JJ = \int f \diff s + \Phi(T) \to \inf \] за умови, що \[ \dot x = f_0. \] Розв'язати задачу оптимального керування.
\end{problem*}

\begin{algorithm} \tt
    \begin{enumerate}
        \item Записуємо функцію Гамільтона-Понтрягіна: \[ \mathcal{H} (x, u, \psi, t) = - f_0(x, u, t) + \langle \psi, f(x, u, t) \rangle. \]
    
        \item Записуємо спряжену систему: \[ \dot \psi = - \nabla_x \mathcal{H}, \quad \psi(T) = - \nabla \Phi(x(T)). \]
    
    
        \item Знаходимо $u(\psi)$ з умови оптимальності: \[ \dfrac{\partial \mathcal{H}(x, u, \psi, t)}{\partial u} = 0. \]

        \item Підставляємо знайдене керування у початкову систему, от\-ри\-ма\-ли \allowbreak край\-о\-ву задачу, систему диференціальних рівнянь на $x$ і $\psi$ з гра\-нич\-ни\-ми  \allowbreak у\-мо\-ва\-ми.

        \item Розв'язуємо крайову задачу і знаходимо $x$.

        \item Відновлюємо $u = u (\psi)$ за знайденим $\psi$.
    \end{enumerate}
\end{algorithm}
 \newpage

\section{Принцип максимуму Понтрягіна: загальний випадок}

\subsection{Алгоритми}

\begin{problem*}
	Розв'язати задачу оптимального керування за допомогою принципу максимуму Понтрягіна: \[ \JJ  = \int f_0 \diff s + \Phi_0 \to \inf \] за умов, що \[ \dot x = f, \] а також \[ \int f_i \diff s + \Phi_i = 0 , \quad i = \overline{1..k}. \]
\end{problem*}

\begin{algorithm} \tt
	\begin{enumerate}
		\item Запишемо функцію Гамільтона-Понтрягіна: \[ \HH = -F + \langle \psi, f \rangle. \]
		\item Запишемо термінант: \[ F = \sum_i \lambda_i f_i, \quad \ell = \sum_i \lambda_i \Phi_i. \]
		\item Випишемо тепер всі (необхідні) умови принципу максимуму:
		\begin{enumerate}
			\item оптимальність: \[\frac{\partial \HH}{\partial u} = 0;\]
			\item стаціонарність (спряжена система): \[\dot \psi = - \nabla_x \HH;\]
			\item трансверсальність: \[\psi(t_0) = \frac{\partial \ell}{\partial x_0}, \quad \psi(T) = - \frac{\partial \ell}{\partial x_T};\]
			\item стаціонарність за кінцями: відсутня, бо час фіксований;
			\item доповнююча нежорсткість: відсутня, бо немає інтегральних \allowbreak об\-ме\-жень виду нерівність на задачу;
			\item невід'ємність: $\lambda_i \ge 0$.
		\end{enumerate}
		\item Методом від супротивного показуємо, що $\lambda_i \ne 0$.
		\item З умов принципу максимуму визначаємо $u = u(\psi)$.
		\item Записуємо крайову задачу -- систему диференціальних рівнянь на $x$ і $\psi$ з граничними умовами.
		\item Знаходимо її розв'язок $x_*$.
		\item Відновлюємо $u_* = u_*(\psi)$.
	\end{enumerate}
\end{algorithm}

\begin{problem*}
	Розв'язати задачу оптимальної швидкодії за допомогою принципу максимуму Понтрягіна: \[ \JJ  = \int f_0 \diff s + \Phi_0 \to \inf \] за умов, що \[ \dot x = f, \] а також \[ \int f_i \diff s + \Phi_i = 0 , \quad i = \overline{1..k}. \]
\end{problem*}


\begin{algorithm} \tt
	\begin{enumerate}
		\item Запишемо функцію Гамільтона-Понтрягіна: \[ \HH = -F + \langle \psi, f \rangle. \]
		\item Запишемо термінант: \[ F = \sum_i \lambda_i f_i, \quad \ell = \sum_i \lambda_i \Phi_i. \]
		\item Випишемо тепер всі (необхідні) умови принципу максимуму:
		\begin{enumerate}
			\item оптимальність: \[\frac{\partial \HH}{\partial u} = 0;\]
			\item стаціонарність (спряжена система): \[\dot \psi = - \nabla_x \HH;\]
			\item трансверсальність: \[\psi(t_0) = \frac{\partial \ell}{\partial x_0}, \quad \psi(T) = - \frac{\partial \ell}{\partial x_T};\]
			\item стаціонарність за кінцями: \[ \HH(T) = \frac{\partial \ell}{\partial T}; \]
			\item доповнююча нежорсткість: відсутня, бо немає інтегральних \allowbreak об\-ме\-жень виду нерівність на задачу;
			\item невід'ємність: $\lambda_i \ge 0$.
		\end{enumerate}
		\item Методом від супротивного показуємо, що $\lambda_i \ne 0$.
		\item З умов принципу максимуму визначаємо $u = u(\psi)$.
		\item Записуємо крайову задачу -- систему диференціальних рівнянь на $x$ і $\psi$ з граничними умовами.
		\item З умов принципу максимуму і крайової задачі визначаємо $T$.
		\item Знаходимо розв'язок крайової задачі $x_*$.
		\item Відновлюємо $u_* = u_*(\psi)$.
	\end{enumerate}
\end{algorithm}



 \newpage

\section{Дискретний варіант методу динамічного програмування}

\subsection{Алгоритми}

\begin{problem*}
	Розглядається задача оптимального керування \[ \JJ (u, x) = \sum_{k=0}^{N-1} g_k(x(k),u(k))+\Phi(x(N))\to\min \] при умовах \[ x(k + 1) = f_k(x(k),u(k)), \quad k=0,1,\ldots,N-1, \] \[ x(k)\in\XX_k, \quad k=0,1,\ldots,N, \] \[ u(k)\in\UU_k, \quad k=0,1,\ldots,N-1. \] Знайти оптимальне керування, оптимальну траєкторію, функцію Белмана і оптимальне значення критерію якості.
\end{problem*}

\begin{algorithm} \tt
	\begin{enumerate}
		\item $\BB_N(z) = \Phi(z)$.
		\item Для $s=\overline{N-1..0}$ записуємо і розв'язуємо дискретне рівняння Белмана: \[\BB_s(z) = \min_{u\in \UU_s} (g_s(z,u)+\BB_{s+1}(f_s(z,u))) \] для всіх $z \in \XX_s$, запам'ятовуючи $\{u_*(s)\}$.
		\item Знаходимо $x_*(0)$ як \[ x_*(0) = \argmin_{z\in \XX_0} \BB_0(z).\]
		\item Знаходимо $\JJ_*$ як $\JJ_* = \BB_0(x_*(0))$.
		\item Для $s=\overline{0..N-1}$ відновлюємо $x_*(s+1)$ за відомим керуванням: \[x_*(s+1) = f_s(x_*(s),u_*(s)).\] 
	\end{enumerate}
\end{algorithm} \newpage

\section{Метод динамічного програмування}

\subsection{Алгоритми}

\begin{problem*}
	Розв'язати задачу оптимального керування і знайти функцію Белмана: \[ \JJ = \int f_0 \diff s + \Phi(T) \to \inf \] за умови, що \[ \dot x = f. \]
\end{problem*}

\begin{algorithm} \tt
	\begin{enumerate}
		\item Відрізок $[t_0, T]$ розбивається сіткою $t_0 < t_1 < t_2 < \ldots < t_N = T$ з деяким кроком $h$.
		\item Позначаємо $\XX_i = \XX (t_i)$, $\UU_i = \UU (t_i)$, $i = 0,1,2,\ldots,N$.
		\item $\BB_N(z) = \Phi(z)$.
		\item Для $s=\overline{N-1..0}$ записуємо і розв'язуємо рівняння Белмана: \begin{multline*}\BB_s(z) = \inf_{u\in \UU_s} \left(\int_{t_s}^{t_{s+1}} f_0(x(\tau,u(\tau),\tau)\diff\tau\right.+\\+\left.\BB_{s+1}\left(z+\int_{t_s}^{t_{s+1}} f(x(\tau),u(\tau),\tau) \diff \tau\right)\right) \end{multline*} для всіх $z \in \XX_s$, запам'ятовуючи $\{u_*(s)\}$. \\

		Розв'язуємо ми його через рівняння Гамільтона-Якобі-Белмана, \[\dot \BB(z) + \inf_u (\langle \nabla_z \BB(z), f \rangle + f_0(z) ) = 0. \]

		\item Знаходимо $x_*(t_0)$ як \[ x_*(t_0) = \argmin_{z\in \XX_0} \BB_0(z).\]
		\item Знаходимо $\JJ_*$ як $\JJ_* = \BB_0(x_*(t_0))$.
		\item Для $s=\overline{0..N-1}$ відновлюємо $x_*(t_{s+1})$ за відомим керуванням: \[x_*(t_{s+1}) = x_*(t_s) + \int_{t_s}^{t_{s+1}} f(x_*(\tau),u_*(\tau),\tau) \diff \tau.\] 
	\end{enumerate}
\end{algorithm} \newpage

\end{document}