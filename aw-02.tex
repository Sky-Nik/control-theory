% OK, complete

\section{Елементи багатозначного аналізу. Множина досяжності}

\subsection{Алгоритми}

\begin{problem*}
	Знайти
	\begin{enumerate}
		\item $A+B$;
		\item $\lambda A$;
		\item $\alpha(A,B)$;
		\item $MA$,
	\end{enumerate}
	де множини $A \subset\RR^m$, $B\subset\RR^m$, скаляр $\lambda\in\RR^1$, матриця $M\in\RR^{n\times m}$.
\end{problem*}

\begin{algorithm} \tt
	Розглянемо всі пункти задачі вище.
	\begin{enumerate}
		\item Знаходимо за визначенням, \[A+B=\{a+b|a\in A,b\in B\}.\]
		\item Знаходимо за визначенням, \[\lambda A =\{\lambda a|a\in A\}.\] 
		\item \begin{enumerate}
			\item Знаходимо $\beta(A,B)$ і $\beta(B,A)$ за визначенням, \[ \beta(A,B) = \max_{a\in A}\rho(a,B), \]
			де \[\rho(a,B) = \min_{b\in B} \rho(a,b).\]
			\item Знаходимо $\alpha(A,B)$ за визначенням, \[\alpha(A,B)=\max\{\beta(A,B),\beta(B,A).\]
		\end{enumerate} 
		\item Знаходимо за визначенням, \[MA=\{Ma|a\in A\}.\]
	\end{enumerate}
\end{algorithm}

\begin{problem*}
	Знайти опорну функцію множини $A \subset \RR^n$.
\end{problem*}

\begin{algorithm} \tt
	\begin{enumerate}
		\item \textbf{Намагаємося} знайти за визначенням, \[ c(A,\psi) = \max_{a\in A} \langle a, \psi \rangle. \]
		\item Якщо не вийшло, то намагаємося знайти за геометричною властивістю: $c(A,\psi)$ -- (орієнтована) відстань від початку координат до опорної \allowbreak пло\-щи\-ни множини $A$, для якої напрямок-вектор $\psi$ є вектором нормалі.
	\end{enumerate}
\end{algorithm}

\begin{problem*}
	Знайти інтеграл Аумана $\JJ = \int F \diff x$, де $F = F(x)\subset\RR^n$.
\end{problem*}

\begin{algorithm} \tt
	\begin{enumerate}
		\item Знаходимо опорну функцію від інтегралу: \[ c(\JJ, \psi) = \int c (F, \psi) \diff x.\]
		\item Знаходимо $\JJ$ як опуклий компакт з відомою опорною функцією $c(\JJ, \psi)$.
	\end{enumerate}
\end{algorithm}

\begin{problem*}
	Знайти множину досяжності системи $\dot x = A x + B u$, де $x(t_0) \in \mathcal{M}_0$, $u \in \mathcal{U}$.
\end{problem*}

\begin{algorithm} \tt
	\begin{enumerate}
		\item Знаходимо фундаментальну матрицю $\Theta(t,s)$ системи нор\-мо\-ва\-ну за моментом $s$.
		\item Знаходимо інтеграл Аумана \[\int_{t_0}^t \Theta(t, s) B(s) \UU(s) \diff s.\]
		\item Використовуємо теорему про вигляд множини досяжності лінійної сис\-те\-ми керування: \[ \XX(t, \MM_0) = \Theta(t, t_0) \MM_0 + \int_{t_0}^t \Theta(t, s) B(s) \UU(s) \diff s. \]
	\end{enumerate}
\end{algorithm}

\begin{problem*}
	Знайти опорну функцію множини досяжності системи $\dot x = A x + B u$, де $x(t_0) \in \mathcal{M}_0$, $u \in \mathcal{U}$.
\end{problem*}

\begin{algorithm} \tt
	\begin{enumerate}
		\item Знаходимо фундаментальну матрицю $\Theta(t,s)$ системи нор\-мо\-ва\-ну за моментом $s$.
		\item Знаходимо опорну функцію $c(\MM_0, \Theta^*(t, t_0) \psi)$.
		\item Знаходимо опорну функцію $c(\UU(s), B^*(s) \Theta^*(t, s) \psi)$.
		\item Використовуємо теорему про вигляд опорної функції множини до\-сяж\-но\-с\-ті лінійної системи керування: \[ c(\XX(t, \MM_0), \psi) = c(\MM_0, \Theta^*(t, t_0) \psi) + \int_{t_0}^t c(\UU(s), B^*(s) \Theta^*(t, s) \psi) \diff s. \]
	\end{enumerate}
\end{algorithm}

\newpage

\subsection{Аудиторне заняття}

\begin{problem}
	Знайти $A + B$ і $\lambda A$, а також метрику Хаусдорфа $\alpha (A, B)$, якщо:
	
	\begin{enumerate}
	    \item $A = \{-3, 2, -1\},  B = \{-2, 5, 1\},  \lambda = 3$;
	    
	    \item $A = \{4, 2, -4\},  B = [-2, 3],  \lambda = -1$;
	    
	    \item $A = [-1, 2],  B = [3, 7],  \lambda = -2$;
	    
	\end{enumerate}
	
\end{problem}

\begin{solution}

	\begin{enumerate}
	    \item За визначенням операції $A + B = \{-5, 2, -2, 0, 7, 3, -3, 4, 0\}$, $\lambda A = \{-9, 6, -3 \}$. \\
	    
	    Метрика Хаусдорфа визначатиметься як \[\alpha (A, B) = \max\{\beta (A, B), \beta (B, A)\},\] в свою чергу $\beta (A, B)$ визначається як максимум з мінімумів відхилень множини, тобто, у нашому випадку, $\beta (A, B) = \max\{1, 1, 1\}; \beta (B, A) = \max\{1, 3, 1\}$, тоді $\alpha (A, B) = 3$.
	    
	    \item За визначенням операції $A + B = [-6,7]$, $\lambda A = \{-4, -2, 4 \}$.\\
	    
	    Метрика Хаусдорфа визначатиметься як \[\alpha (A, B) = \max\{\beta (A, B), \beta (B, A)\},\] в свою чергу $\beta (A, B)$ визначається як максимум з мінімумів відхилень множини, тобто, у нашому випадку, $\beta (A, B) = \max\{1, 0, 2\}; \beta (B, A) = \max[0, 3]$, оскільки $-1$ відхиляється від найближчих елементів на $3$ і це є максимумом, тоді $\alpha (A, B) = 3$.
	    
	    \item За визначенням операції $A + B = [2,9]$, $\lambda A = [-4,2]$. \\
	    
	    Метрика Хаусдорфа визначатиметься як \[\alpha (A, B) = \max\{\beta (A, B), \beta (B, A)\},\] в свою чергу $\beta (A, B)$ визначається як максимум з мінімумів відхилень множини, тобто, у нашому випадку, $\beta (A, B) = \max[1,4]; \beta (B, A) = \max[1, 5]$, оскільки відповідні краї відхиляються на $4$ та $5$ відповідно ($-1$ від $3$ та $7$ від $2$), тоді $\alpha (A, B) = 5$.
	    
	\end{enumerate}
	
\end{solution}

\begin{problem}
	Знайти $MA$, якщо
	\[
	M=
  \begin{pmatrix}
    -2 & 4 \\
    3 & 5
  \end{pmatrix}
  , A= 
  \left\{
  \begin{pmatrix}
    -1 \\
    2 
  \end{pmatrix},
    \begin{pmatrix}
    3 \\
    -4 
  \end{pmatrix},
    \begin{pmatrix}
    0 \\
    -2 
  \end{pmatrix}
  \right\}
  .\]
\end{problem}

\begin{solution}
	За означенням $MA = \{Ma\in \RR^m, a\in A\}$, тому
	\newline
   \[ \begin{pmatrix}
   -2 & 4 \\
   3 & 5
  \end{pmatrix}
  \begin{pmatrix}
    -1 \\
    2 
  \end{pmatrix}=
  \begin{pmatrix}
    10 \\
    7 
  \end{pmatrix};
   \begin{pmatrix}
   -2 & 4 \\
   3 & 5
  \end{pmatrix}
  \begin{pmatrix}
    3 \\
    -4 
  \end{pmatrix}=
  \begin{pmatrix}
    -22 \\
    -11 
  \end{pmatrix};
   \begin{pmatrix}
   -2 & 4 \\
   3 & 5
  \end{pmatrix}
  \begin{pmatrix}
    0 \\
    -2 
  \end{pmatrix}=
  \begin{pmatrix}
    -8 \\
    -10 
  \end{pmatrix};
  \]
  
  Отже, отримаємо \[MA = 
  \left\{
  \begin{pmatrix}
    10 \\
    7 
  \end{pmatrix},
  \begin{pmatrix}
    -22 \\
    -11 
  \end{pmatrix},
  \begin{pmatrix}
    -8 \\
    -10 
  \end{pmatrix}
  \right\}.\]
\end{solution}

\begin{problem}
	Знайти опорні функції таких множин:

	\begin{enumerate}
		\item $A = [0, r]$;

		\item $A = [-r, r]$;

		\item $A = \{ (x_1, x_2): |x_1| \le 1, |x_2| \le 2 \}$;

		\item $A = \KK_r (0) = \{ x \in \RR^n: \|x\| \le r \}$;

		\item $A = \SS^n = \{ x \in \RR^n: \|x\| = 1 \}$.
	\end{enumerate}
\end{problem}

\begin{solution}
	\begin{enumerate}
		\item За означення опорної функції, \[ c(A, \psi) = \max_{a \in A} \langle a, \psi \rangle = \begin{cases} 0, & \psi < 0 \\ r \psi, & 0 \le \psi \end{cases} = \max \{ 0, r \psi \}. \]

		\item За означення опорної функції, \[ c(A, \psi) = \max_{a \in A} \langle a, \psi \rangle = \begin{cases} - r \psi, & \psi < 0 \\ r \psi, & 0 \le \psi \end{cases} = r |\psi |. \]

		\item За означення опорної функції, \[ c(A, \psi) = \max_{a \in A} \langle a, \psi \rangle = \max_{a \in A} (\psi_1 x_1 + \psi_2 x_2) = |\psi_1| + 2 |\psi_2|. \]

		\item За властивістю опорної функції (вона дорівнює орієнтованій відстані від початку координат до опорної площини множини $A$ яка відповідає напрямку $\psi$), маємо $c(\KK_r (0), \psi) = r \| \psi \|$.

		\item За тією ж властивістю опорної функції маємо $c(\SS^n, \psi) = \| \psi \|$.
	\end{enumerate}
\end{solution}

\begin{problem}
	Знайти інтеграл Аумана $\JJ = \int_0^1 F(x) \diff x$ таких багатозначних відображень:

	\begin{enumerate}
		\item $F(x) = [0, x]$, $x \in [0, 1]$;

		\item $F(x) = \KK_x (0) = \{ y \in \RR^n: \|y\| \le x \}$, $x \in [0, 1]$.
	\end{enumerate}
\end{problem}

\begin{solution}
	Скористаємося рівністю \[ c \left(\int_0^1 F(x), \psi\right) \diff x = \int_0^1 c(F(x), \psi) \diff x, \]

	яка виконується в умовах теореми Ляпунова про опуклість інтегралу Аумана.

	\begin{enumerate}
		\item \[c(\JJ) = \int_0^1 c([0, x], \psi) \diff x = \begin{cases} \psi / 2, & 0 \le \psi \\ 0, & \psi < 0 \end{cases}. \]

		А далі наші знання опорних функцій підказують, що $\JJ = [0, 1 / 2]$.

		\item \[c(\JJ) = \int_0^1 c(\KK_x(0), \psi) \diff x = \int_0^1 x \| \psi \| \diff x = \| \psi \| / 2. \]

		А далі наші знання опорних функцій підказують, що $\JJ = \KK_{1 / 2} (0)$.
	\end{enumerate}
\end{solution}

\begin{problem}
	Знайти множину досяжності такої системи керування: \[ \frac{\diff x}{\diff t} = x + u, \]

	де $x (0) = x_0 \in \MM_0$, $u (t) \in \UU$, $t \ge 0$, \[ \MM_0 = \{ x: |x| \le 1 \}, \] \[ \UU = \{ u: |u| \le 1 \}. \]
\end{problem}

\begin{solution}
	Скористаємося теоремою про вигляд множини досяжності лінійної системи керування: \[ \XX(t, \MM_0) = \Theta(t, t_0) \MM_0 + \int_{t_0}^t \Theta(t, s) B(s) \UU(s) \diff s. \]

	Підставимо вже відомі значення: \[ \XX(t, [-1, 1]) = \Theta(t, 0) \cdot [-1, 1] + \int_0^t \left( \Theta(t, s) \cdot 1 \cdot [-1, 1] \right) \diff s, \] 

	тобто залишилося знайти $\Theta$. Знайдемо її з системи \[ \frac{\diff \Theta(t, s)}{\diff t} = A(t) \cdot \Theta(t, s) = \Theta(t, s). \]

	Нескладно бачити, що $\Theta(t, s) = e^{t - s}$, тому \begin{multline*} \XX(t, [-1, 1]) = [-e^t, e^t] + \int_0^t [-e^{t - s}, e^{t - s}] \diff s = \\ = [-e^t, e^t] + [1 - e^t, e^t - 1] = [1 - 2 e^t, 2 e^t - 1]. \end{multline*} 
\end{solution}

\begin{problem}
	Знайти опорну функцію множини досяжності для системи керування: \[
	\left\{
		\begin{aligned}
			\frac{\diff x_1}{\diff t} &= 2x_1 + x_2 + u_1, \\
			\frac{\diff x_2}{\diff t} &= 3x_1 + 4x_2 + u_2,
		\end{aligned}
	\right.
	\]

	де $x (0) = (x_{01}, x_{02}) \in \MM_0$, $u(t) = (u_1(t), u_2(t)) \in \UU$, $t \ge 0$, \[ \MM_0 = \{ (x_{01}, x_{02}): |x_{01}| \le 1, |x_{02}| \le 1 \}, \] \[ \UU = \{(u_1, u_2): |u_1| \le 1, |u_2| \le 1\}. \]
\end{problem}

\begin{solution}
	Скористаємося теоремою про вигляд опорної функції множини досяжності лінійної системи керування: \[ c(\XX(t, \MM_0), \psi) = c(\MM_0, \Theta^*(t, t_0) \psi) + \int_{t_0}^t c(\UU(s), B^*(s) \Theta^*(t, s) \psi) \diff s. \]

	Підставимо вже відомі значення: \[ c(\XX(t, [-1,1]^2), \psi) = c([-1,1]^2, \Theta^*(t, 0) \psi) + \int_0^t c([-1,1]^2, \begin{pmatrix} 1 & 1 \end{pmatrix} \Theta^*(t, s) \psi) \diff s, \] 

	тобто залишилося знайти $\Theta$. Знайдемо її з системи \[ \frac{\diff \Theta(t, s)}{\diff t} = A(t) \cdot \Theta(t, s) = \begin{pmatrix} 2 & 1 \\ 3 & 4 \end{pmatrix} \Theta(t, s). \]

	Нескладно бачити, що \[ \Theta(t, s) = \frac{1}{4}
	\begin{pmatrix}
		3 e^{t - s} + e^{5 (t - s)} & - e^{t - s} + e^{5 (t - s)} \\ -3 e^{t - s} + 3 e^{5 (t - s)} & e^{t - s} + 3 e^{5 (t - s)}
	\end{pmatrix} 
	\]

	Тому
	\begin{multline*} 
		c\left(\XX(t, [-1,1]^2), \begin{pmatrix} \psi_1 \\ \psi_2 \end{pmatrix}\right) = c\left([-1,1]^2, \frac{1}{4} \begin{pmatrix} 3 e^t + e^{5 t} & -3 e^t + 3 e^{5 t} \\ - e^t + e^{5 t} & e^t + 3 e^{5 t}	\end{pmatrix} \begin{pmatrix} \psi_1 \\ \psi_2 \end{pmatrix} \right) + \\
		+ \int_0^t c\left([-1,1]^2, \begin{pmatrix} 1 & 0 \\ 0 & 1 \end{pmatrix} \frac{1}{4} \begin{pmatrix} 3 e^{t - s} + e^{5 (t - s)} & -3 e^{t - s} + 3 e^{5 (t - s)} \\ - e^{t - s} + e^{5 (t - s)} & e^{t - s} + 3 e^{5 (t - s)} \end{pmatrix} \begin{pmatrix} \psi_1 \\ \psi_2 \end{pmatrix} \right) \diff s = \\
		= c\left([-1,1]^2, \frac{1}{4} \begin{pmatrix} (3 e^t + e^{5 t}) \psi_1 + (-3 e^t + 3 e^{5 t}) \psi_2 \\ (- e^t + e^{5 t}) \psi_1 + (e^t + 3 e^{5 t}) \psi_2 \end{pmatrix}  \right) + \\
		+ \int_0^t c\left([-1,1]^2, \frac{1}{4} \begin{pmatrix} (3 e^{t - s} + e^{5 (t - s)}) \psi_1 + (-3 e^{t - s} + 3 e^{5 (t - s)}) \psi_2 \\ (- e^{t - s} + e^{5 (t - s)}) \psi_1 + (e^{t - s} + 3 e^{5 (t - s)}) \psi_2 \end{pmatrix} \right) \diff s = \\
		= \frac{1}{4} \left( \left|(3 e^t + e^{5 t}) \psi_1 + (-3 e^t + 3 e^{5 t}) \psi_2\right| + \left|(- e^t + e^{5 t}) \psi_1 + (e^t + 3 e^{5 t}) \psi_2\right| \right) + \\
		+ \frac{1}{4} \int_0^t \left|(3 e^{t - s} + e^{5 (t - s)}) \psi_1 + (-3 e^{t - s} + 3 e^{5 (t - s)}) \psi_2\right| \diff s + \\
		+ \frac{1}{4} \int_0^t \left|(- e^{t - s} + e^{5 (t - s)}) \psi_1 + (e^{t - s} + 3 e^{5 (t - s)}) \psi_2\right| \diff s.
	\end{multline*}
\end{solution}