\subsection{Задачі з розв'язками}

\begin{problem}
	Перевести систему \[ \frac{\diff x}{\diff t} = u, \quad t \in [0, T], \] з точки $x (0) = x_0$ в точку $x (T) = y_0$ за допомогою керування з класу:
	\begin{enumerate}
		\item постійних функцій $u (t) = c$, $c$ -- константа;

		\item кусково-постійних функцій \[ u (t) = \begin{cases} c_1, & t \in [0, t_1], \\ c_2, & t \in [t_1, T]. \end{cases} \]

		Тут $c_1$, $c_2$ -- константи, $c_1 \ne c_2$, $0 < t_1 < T$;

		\item програмних керувань $u(t) = c t$, $c$ -- константа;

		\item керувань з оберненим зв'язком $u(x) = c x$, $c$ -- константа.
	\end{enumerate}
\end{problem}

\begin{solution}
	Скористаємося формулою $x (T) = x (0) + \int_0^T \frac{\diff x}{\diff t} \diff t$:

	\begin{enumerate}
		\item \[x (T) = x (0) + \int_0^T c \diff t = x (0) + c T,\] звідки \[c = \frac{x (T) - x (0)}{T} = \frac{y_0 - x_0}{T};\]

		\item \[ x (T) = x (0) + \int_0^{t_1} c_1 \diff t + \int_{t_1}^T c_2 \diff t = x (0) + c_1 t_1 + c_2 (T - t_1). \] Розв'язок не єдиний, \[ c_2 = \frac{x (T) - x (0) - c_1 t_1}{T - t_1}, \] де $c_1$ -- довільна стала, наприклад $c_1 = 0$, тоді \[ c_2 = \frac{x (T) - x (0)}{T - t_1} = \frac{y_0 - x_0}{T - t_1}. \]

		\item \[ x (T) = x (0) + \int_0^T c t \diff t = x (0) + \frac{cT^2}{2}, \] звідки \[ c = \frac{2 (x (T) - x (0))}{T^2} = \frac{2 (y_0 - x_0)}{T^2}. \]

		\item У цьому випадку проінтегрувати не можна, бо $u$ залежить від $x$, тому просто запишемо за формулою Коші \[ x (T) = x (0) \cdot e^{c T}, \] звідки \[c = \frac{\ln(x (T) / x (0))}{T} = \frac{\ln(y_0) - \ln(x_0)}{T}. \]

		Варто зауважити, не для всіх пар $x_0$ і $y_0$ коректно визначається значення $c$. А саме, необхідно щоб $y_0$ було того ж знаку, що і $x_0$.
	\end{enumerate}
\end{solution}

\begin{problem}
	\begin{enumerate}
		\item Використовуючи означення, знайти грамміан керованості для системи керування \[ \frac{\diff x(t)}{\diff t} = t x (t) + \cos (t) \cdot u(t), \quad t \ge 0. \]

		\item Записати диференціальне рівняння для грамміана керованості і за його допомогою знайти грамміан керованості.

		\item Використовуючи критерій керованості, вказати інтервал повної керованості вказаної системи керування. Для цього інтервала записати керування, яке розв'язує задачу про переведення системи з точки $x_0$ у стан $x_T$.
	\end{enumerate}
\end{problem}

\begin{solution}
	\begin{enumerate}
		\item Скористаємося формулою \[\Phi(T, t_0) = \int_{t_0}^T \Theta(T, s) B(s) B^*(s) \Theta^*(T, s) \diff s.\] $\Theta(T, s)$ знаходимо з системи \[ \frac{\diff \Theta(t,s)}{\diff t} = A(t) \cdot \Theta(t, s) = t \cdot \Theta(t, s),\] а саме $\Theta(t, s) = \exp\left\{\frac{t^2 - s^2}{2}\right\}$. Підставляючи всі знайдені значення, отримаємо \[\Phi(T, t_0) = \cos^2(T) \cdot e^{T^2} \int_{t_0}^T e^{-s^2} \diff s = \frac{1}{2} \sqrt{\pi} \cdot \cos^2(T) \cdot e^{T^2} \cdot \erf(T).\]

		\item Запишемо систему \[ \frac{\diff \Phi (t, t_0)}{\diff t} = A(t) \cdot \Phi(t, t_0) + \Phi(t, t_0) \cdot A^*(t) + B(t) \cdot B^*(t), \quad \Phi(t_0, t_0) = 0. \]

		І підставимо відомі значення: \[ \frac{\diff \Phi (t, 0)}{\diff t} = 2 t \Phi(t, 0)  + \cos^2(t), \quad \Phi(0, 0) = 0. \]

		Звідси \[ \Phi(t, 0) = \frac{1}{8} \sqrt{\pi} e^{t^2 - 1} ( - 2 e \erf(t) + i (\erfi(1 + i t) - i \erfi(1 - i t)), \]

		а \[ \Phi(T, 0) = \frac{1}{8} \sqrt{\pi} e^{T^2 - 1} ( - 2 e \erf(T) + i (\erfi(1 + i T) - i \erfi(1 - i T)), \]

		\item З вигляду грамміану керованості отриманого у першому пункті очевидно, що система цілком керована на півінтервалі $[0, \pi / 2)$, зокрема на інтервалі $[0, 1]$. \\

		Підставимо тепер граміан у формулу для керування що розв'язує задачу про переведення системи із стану $x_0$ у стан $x_T$:
		\begin{multline*} u(t) = B^*(t) \Theta^*(T, t) \Phi^{-1}(T, t_0) (x_T - \Theta(T, t_0) x_0) = \\ = \cos(t) \cdot \exp\left\{\frac{T^2-t^2}{2}\right\} \Phi^{-1} (T, 0) \left(x_T - \exp\left\{\frac{T^2}{2}\right\} x_0\right) \end{multline*}
	\end{enumerate}
\end{solution}

\begin{problem}
	За допомогою грамміана керованості розв'язати таку задачу оптимального керування: мінімізувати критерій якості \[ \JJ (u) = \int_0^T u^2(s) \diff s\] за умов, що \[ \frac{\diff x(t)}{\diff t} = \sin(t) \cdot x(t) + u(t), \quad x (0) = x_0, x (T) = x_T. \]

	Тут $x$ -- стан системи, $u(t)$ -- скалярне керування, $x_0$, $X_T$ -- задані точки, $t \in [t_0, T]$.
\end{problem}

\begin{solution}
	Знайдемо шукане керування за формулою \[ u(t) = B^*(t) \Theta^*(T, t) \Phi^{-1}(T, t_0) (x_T - \Theta(T, t_0) x_0). \]

	У цій задачі $\Theta(t, s) = e^{\cos (s) - \cos (t)}$, знайдене з системи $\dot \Theta = A \Theta$, $\Phi(T, t_0) = e^{-2 \cos (T)} \int_0^T e^{2 \cos (s)} \diff s$, підставляючи знаходимо \[ u(t) = \frac{e^{\cos (t) + \cos (T)} \cdot (x_T - e^{1 - \cos (T)} x_0)}{\int_0^T e^{2 \cos (s)} \diff s}. \]
\end{solution}

\begin{problem}
    За допомогою грамміана керованості розв'язати таку задачу оптимального керування: мінімізувати критерій якості
    \[ \mathcal{J}(u) = \int_0^T u^2(s) dx \]
    за умов, що
    \[ \dfrac{d^2x(t)}{dt^2} - 5\dfrac{dx(t)}{dt} + 6x(t) = u(t), \]
    \[ x(0) = x_0, x'(0) = y_0, x(T) = x'(T) = 0.\]
    Тут $x$ -- стан системми, $u(t)$ -- скалярне керування, $t \in [0, T]$.
\end{problem}

\begin{solution}
    Почнемо з того що зведемо рівняння другого порядку до системи рівнянь заміною $x_1 = x$, $x_2 = \dot x_1$, тоді маємо систему
    \[ \begin{pmatrix} \dot x_1 \\ \dot x_2 \end{pmatrix} (t) = \begin{pmatrix} 0 & 1 \\ -6 & 5 \end{pmatrix} \begin{pmatrix} x_1 \\ x_2 \end{pmatrix} (t) + \begin{pmatrix} 0 \\ 1 \end{pmatrix} u(t). \]
    
    Знайдемо власні числа матриці $A - \lambda E$: $\det(A - \lambda E) = \begin{vmatrix} -\lambda & 1 \\ -6 & 5-\lambda \end{vmatrix} = \lambda^2 - 5\lambda + 6 = (\lambda - 2) (\lambda - 3) = 0$, звідки $\lambda_1 = 2$, $\lambda_2 = 3$. Знайдемо власні вектори, вони будуть $\begin{pmatrix} 1 \\ 2 \end{pmatrix}$ і $\begin{pmatrix} 1 \\ 3 \end{pmatrix}$ відповідно. Звідси знаходимо загальний розв'язок
    \[ \begin{pmatrix} x_1 \\ x_2 \end{pmatrix} (t) = c_1 \begin{pmatrix} e^{2t} \\ 2e^{2t} \end{pmatrix} + c_2 \begin{pmatrix} e^{3t} \\ 3e^{3t} \end{pmatrix}. \]
    
    З рівняння
    \[c_1 \begin{pmatrix} e^{2s} \\ 2e^{2s} \end{pmatrix} + c_2 \begin{pmatrix} e^{3s} \\ 3e^{3s} \end{pmatrix} = \begin{pmatrix} 1 \\ 0 \end{pmatrix} \]
    знаходимо $c_1 = 3e^{-2s}$, $c_2 = -2e^{-3s}$, а з рівняння
    \[c_1 \begin{pmatrix} e^{2s} \\ 2e^{2s} \end{pmatrix} + c_2 \begin{pmatrix} e^{3s} \\ 3e^{3s} \end{pmatrix} = \begin{pmatrix} 0 \\ 1 \end{pmatrix} \]
    знаходимо $c_1 = -e^{-2s}$, $c_2 = e^{-3s}$, тобто
    \[ \Theta(t, s) = \begin{pmatrix} 3e^{2(t-s)} - 2e^{3(t-s)} & -e^{2(t-s)} + e^{3(t-s)} \\ 6e^{2(t-s)} - 6e^{3(t-s)} & -2e^{2(t-s)} + 3e^{3(t-s)} \end{pmatrix}. \]
    
    Знайдемо грамміан за формулою \[\Phi(T, 0) = \int_0^T \Theta(T, s) B(s) B^* (s) \Theta^*(T, s) ds. \]
    
    \[ \Theta(T, s) B(s) = \begin{pmatrix} -e^{2(T - s)} + e^{3(T - s)} \\ -2e^{2(T - s)} + 3e^{3(T - s)} \end{pmatrix}. \]
    \[ B^* (s) \Theta^*(T, s)  =  (\Theta(T, s) B(s))^\star = \begin{pmatrix} -e^{2(T - s)} + e^{3(T - s)} & -2e^{2(T - s)} + 3e^{3(T - s)} \end{pmatrix}. \]
    
    \begin{align*} 
        \Phi(T, 0) &= \int_0^T \begin{pmatrix} -e^{2(T - s)} + e^{3(T - s)} \\ -2e^{2(T - s)} + 3e^{3(T - s)} \end{pmatrix} \begin{pmatrix} -e^{2(T - s)} + e^{3(T - s)} & -2e^{2(T - s)} + 3e^{3(T - s)} \end{pmatrix} ds = \\
        &= \int_0^T \begin{pmatrix} e^{4(T - s)} - 2 e^{5(T - s)} + e^{6(T - s)} & 2 e^{4(T - s)} - 5 e^{5(T - s)} + 3 e^{6(T - s)} \\ 2e^{4(T - s)} - 5 e^{5(T - s)} + 3 e^{6(T - s)} & 4 e^{4(T - s)} - 12 e^{5(T - s)} + 9  e^{6(T - s)} \end{pmatrix} ds = \\
        &= \begin{pmatrix} \dfrac{e^{4T} - 1}{4} - \dfrac{2(e^{5T} - 1)}{5} + \dfrac{e^{6T} - 1}{6} & \dfrac{e^{4T} - 1}{2} - (e^{5T} - 1) + \dfrac{e^{6T} - 1}{2} \\ \\ \dfrac{e^{4T} - 1}{2} - (e^{5T} - 1) + \dfrac{e^{6T} - 1}{2} & (e^{4T} - 1) - \dfrac{12(e^{5T} - 1)}{5} + \dfrac{3(e^{6T} - 1)}{2} \end{pmatrix}
    \end{align*}
    
    Обчислення визначника грамміану є громіздкою обчислювальною задачею, ми залишаємо її як вправу для читача.
\end{solution}

\begin{problem}
	Записати систему диференціальних рівнянь для знаходження першої матриці керованості (грамміана керованості) і сформулювати критерій керованості на інтервалі $[0, T]$ у випадку, якщо система керування має вигляд:
	\begin{enumerate}
		\item \[ 
		\left\{
			\begin{aligned}
				\frac{\diff x_1(t)}{\diff t} = t x_1 (t) + x_2 (t) + u_1 (t), \\
				\frac{\diff x_2(t)}{\diff t} = - x_1 (t) + 2 x_2 (t) + t^2 u_2 (t).
			\end{aligned}
		\right.
		\]

		Тут $x = (x_1, x_2)^*$ -- вектор стану, $u = (u_1, u_2)^*$ -- вектор керування, $t \in [0, T]$.

		\item \[ \frac{\diff^2 x(t)}{\diff t^2} + \sin(t) \cdot x(t) = u(t). \]

		Тут $x$ -- стан системи, $u(t)$ -- скалярне керування, $t \in [0, T]$.
	\end{enumerate}
\end{problem}

\begin{solution}
	\begin{enumerate}
		\item $A = \begin{pmatrix} t & 1 \\ -1 & 2 \end{pmatrix}$, $B = \begin{pmatrix} 1 & 0 \\ 0 & t^2 \end{pmatrix}$, \[
		\left\{
			\begin{aligned}
				\dot \phi_{11} &= 2 t \phi_{11} + 2 \phi_{12} + 1, \\
				\dot \phi_{12} &= - \phi_{11} + (t + 2) \phi_{12} + \phi_{22}, \\
				\dot \phi_{21} &= \ldots 
			\end{aligned}
		\right.
		\]

		\item Введемо нову змінну $x_2 = \dot x$, тоді $A = \begin{pmatrix} 0 & 1 \\ -\sin(t) & 0 \end{pmatrix}$, $B = \begin{pmatrix} 0 \\ 1 \end{pmatrix}$, \[
		\left\{
			\begin{aligned}
				\dot \phi_{11} &= 2 \phi_{12}, \\
				\dot \phi_{12} &= (1 - \sin(t)) \phi_{11}, \\
				\ldots 
			\end{aligned}
		\right.
		\]
	\end{enumerate}
\end{solution}

\begin{problem}
    Дослідити системи на керованість використовуючи другий критерій керованості:
    \begin{enumerate}
        \item \[\ddot x + a \dot x + b x = u; \]
        \item \[ \left\{ \begin{aligned} \dot x_1 &= 2x_1 + x_2 + au \\ \dot x_2 &= x_1 + 4 x_2 + u \end{aligned} \right. \]
        \item \[ \left\{ \begin{aligned} \dot x_1 &= 2x_1 + x_2 + u_1 \\ \dot x_2 &= x_1 + 3 x_3 + u_2 \\ \dot x_3 &= x_2 + x_3 + u_2  \end{aligned} \right. \]
    \end{enumerate}
\end{problem}

\begin{solution}
    \begin{enumerate}
        \item Почнемо з того що зведемо рівняння другого порядку до системи рівнянь заміною $x_1 = x$, $x_2 = \dot x_1$, тоді маємо систему
        \[ \left\{ \begin{aligned} \dot x_1 &= x_2 \\ \dot x_2 &= - a x_2 - b x_1 + u  \end{aligned} \right. \]
        Тоді
        \[ A = \begin{pmatrix} 0 & 1 \\ -b & -a \end{pmatrix} \qquad B = \begin{pmatrix} 0 \\ 1 \end{pmatrix}. \]
        \[ D = \begin{pmatrix} B & AB \end{pmatrix} = \begin{pmatrix} 0 & 1 \\ 1 & - a \end{pmatrix}. \]
        Її ранг дорівнює 2 якщо за будь-яких $a$ і $b$, тобто система завжди цілком керована.
        \item 
        \[ A = \begin{pmatrix} 2 & 1 \\ 1 & 4 \end{pmatrix} \qquad B = \begin{pmatrix} a \\ 1 \end{pmatrix}. \]
        \[ D = \begin{pmatrix} B & AB \end{pmatrix} = \begin{pmatrix} a & 2 a + 1 \\ 1 & a + 4 \end{pmatrix}. \]
        Її визначник $a^2 + 4a - 2a - 1 = a^2 + 2a - 1 = 0$ якщо $a = -1 \pm \sqrt 2$, тоді система не є цілком керованою, а інакше є.
        \item 
        \[ A = \begin{pmatrix} 2 & 1 & 0 \\ 1 & 0 & 3 \\ 0 & 1 & 1 \end{pmatrix} \qquad B = \begin{pmatrix} 1 & 0 \\ 0 & 1 \\ 0 & 1  \end{pmatrix}. \]
        \[ D = \begin{pmatrix} B & AB & A^2B \end{pmatrix} = \begin{pmatrix} 1 & 0 & 2 & 1 & 5 & 5 \\ 0 & 1 & 1 & 3 & 2 & 7 \\ 0 & 1 & 0 & 2 & 1 & 5 \end{pmatrix} .\]
        Її ранг дорівнює 3, тобто система цілком керована.
    \end{enumerate} 
\end{solution}


\begin{problem}
    Перевести систему 
    \[ \dfrac{dx}{dt} = 2tx + u, t \in [0, T], \]
    з точки $x(0) = x_0$ в точку $x(T) = y_0$ за допомогою керування з класу:
    \begin{enumerate}
        \item постійних функцій $u(t) = c$, $c$ -- константа; 
        \item кусково-постійних функцій 
        \[
        u(t) = \begin{cases}
            c_1 & t \in [0, t_1), \\
            c_2 & t \in (t_1, T].
        \end{cases}
        \]
        Тут $c_1$, $c_2$ -- константи, $c_1 \ne c_2$, $0 < t_1 < T$;
        \item програмних керувань $u(t) = ct$, $c$ -- константа;
        \item керувань з оберненим зв'язок $u(x) = cx$, $c$ -- константа.
    \end{enumerate}
\end{problem}

\begin{solution}
Будемо просто підставляти керування у диференційне рівняння і розв'язувати його:
\begin{enumerate}
\item Зводимо до канонічного вигляду лінійного рівняння:
\[ \dfrac{dx}{dt} - 2t \cdot x(t) = c. \]
Домножаємо на множник що інтегрує:
\begin{align*}
    \exp\{-t^2\} \cdot \dfrac{dx}{dt} - 2 t \cdot \exp\{-t^2\} \cdot x(t) &= c \cdot \exp\{-t^2\} \\
    \\
    \exp\{-t^2\} \cdot \dfrac{dx}{dt} + x(t) \cdot \dfrac {d \exp\{-t^2\}} {dt} &= c \cdot \exp\{-t^2\}.
\end{align*}
Згортаємо похідну добутку:
\[ \dfrac {d (\exp\{-t^2\} \cdot x(t))} {dt} = c \cdot \exp\{-t^2\}. \]
Інтегруємо:
\begin{align*}
    \left.(\exp\{-t^2\} \cdot x(t))\right|_0^T &= \int_0^T c \cdot \exp\{-t^2\} dt \\
    \\
    \exp\{-T^2\} \cdot y_0 - x_0 &= c \cdot \dfrac {\sqrt \pi} 2 \cdot \erf (T),
\end{align*}
і виражаємо звідси $c$:
\[ c = 2 \cdot \dfrac{\exp\{-T^2\} \cdot y_0 - x_0} {\sqrt \pi \cdot \erf (T)}, \]
де $\erf$ позначає функцію помилок, тобто $\erf(x) = \dfrac 2 {\sqrt \pi} \cdot \int_0^x \exp\{-t^2\} dt$.\\

Зауважимо, що задача має розв'язок завжди.
\item Нескладно зрозуміти, що нас задовольнить довільне керування вигляду
\[ c_2 = 2 \cdot \dfrac{\exp\{-T^2\} \cdot y_0 - \exp\{-t_1^2\} \cdot x_1} {\sqrt \pi \cdot (\erf (T) - \erf (t_1))},\] де \[ x_1 = \dfrac{2x_0 + c_1\sqrt \pi \cdot \erf (T)}{2\cdot \exp\{-t_1^2\}}, \]
тобто  ми просто дозволили $c_1$ бути довільною сталою, обчислили $x(t_1)$, а потім розв'язали задачу переведення системи з точки $(t_1, x_1)$ у точку $(T, y_0)$ як у першому пункті, з мінімальними поправками на межі інтегрування. \\

Зокрема, якщо $c_1 = 0$, то $x_1 = \dfrac {x_0} {\exp\{-t_1^2\}}$, тому $c_2 = 2 \cdot \dfrac{\exp\{-T^2\} \cdot y_0 - x_0} {\sqrt \pi \cdot (\erf (T) - \erf (t_1))}$.\\

Зауважимо, що задача має розв'язок завжди.
\item 
\begin{align*}
    \dfrac{dx}{2x(t) + c} &= t dt \\
    \\
    \int_0^T \dfrac{dx}{2x(t) + c} &= \int_0^T t dt \\
    \\
    \left.\left(\dfrac 12 \ln(2x(t) + c)\right)\right|_0^T &= \dfrac {T^2} 2 \\
    \\
    \ln(2y_0 + c) - \ln(2x_0 + c) &= T^2 \\
    \\
    \ln\left(\dfrac{2y_0 + c}{2x_0 + c}\right) &= T^2 \\
    \\
    \dfrac{2y_0 + c}{2x_0 + c} &= \exp\{T^2\} \\
    \\
    2y_0 + c &= (2x_0 + c)\cdot \exp\{T^2\} \\
    \\
    2(y_0 - x_0 \cdot \exp\{T^2\}) &= c \cdot (\exp\{T^2\} - 1) 
\end{align*}
звідки
\[c = 2\cdot \dfrac{y_0 - x_0 \cdot \exp\{T^2\}}{\exp\{T^2\} - 1}. \]
Зауважимо, що задача має розв'язок завжди.
\item 
\begin{align*}
    \dfrac{dx}{dt} &= 2t\cdot x(t) + c\cdot x(t) \\
    \\
    \dfrac{dx}{x(t)} &= (2t + c) dt \\
    \\
    \int_0^T \dfrac{dx}{x(t)} &= \int_0^T (2t + c) dt \\
    \\
    (\ln(x(t))|_0^T &= T^2 + cT \\
    \\
    \ln(y_0) - \ln(x_0) &= T^2 + cT \\ 
    \\
    \ln\left(y_0/x_0\right) &= T^2 + cT 
\end{align*}
звідки
\[ c = \dfrac{\ln\left(y_0/x_0\right) - T^2}{T}. \]
Зауважимо, що задача має розв'язок тільки якщо $\signum(x_0) = \signum(y_0)$.
\end{enumerate}
\end{solution} 

\begin{problem}
\begin{enumerate}
    \item Знайти грамміан керованості для системи керування \[\dfrac{dx(t)}{dt} = tx(t) + u(t)\] і дослідити її на керованість, використовуючи перший критерій керованості.
    \item За допомогою грамміана керованості розв'язати таку задачу оптимального керування: \[\mathcal{J}(u) = \int_0^T u^2(s) ds \to \min\]
    за умов, що \[\dfrac{dx(t)}{dt} = tx(t) + u(t), x(0) = x_0, x(T) = x_T. \]
    Тут $x$ -- стан системи. $u(t)$ -- скалярне керування, $x_0$, $x_T$ -- задані точки, $t\in[0,T]$.
\end{enumerate}
\end{problem}

\begin{solution}
\begin{enumerate}
    \item Одразу помітимо, що $A(t) = (t)$, $B(t) = (1)$. Далі, з рівняння $\dfrac{d\Theta(t,s)}{dt} = A(t) \cdot \Theta(t,s)$ знаходимо $\Theta(t,s) = \exp\{t^2 / 2 - s^2 / 2\}$. Залишилося всього нічого, знайти власне грамміан:
    \begin{multline*} \Phi(T,0) = \int_0^T \Theta(T,s)B(s)B^*(s),\Theta^*(T,s) ds = \\ = \int_0^T (\exp\{T^2 - s^2\}) ds = \begin{pmatrix}\dfrac{\sqrt \pi}{2} \cdot \exp\{T^2\} \cdot \erf(T) \end{pmatrix}, \end{multline*} і $\det\Phi(T,0)\ne0$, тобто система цілком керована на $[0, T]$.
    \item Пригадаємо наступний результат: розв'язком вищезгаданої задачі про оптимальне керування є функція
    \begin{align*}
        u(t) &= B^*(t) \Theta^*(T,t)\Phi^{-1}(T,0)(x_T-\Theta(T,0) x_0) = \\
        \\
        &= \exp\{T^2 / 2 - t^2 / 2\} \begin{pmatrix}\dfrac2{\sqrt \pi} \cdot \exp\{-T^2\} \cdot \dfrac1{\erf(T)} \end{pmatrix} (x_T - \exp\{T^2 / 2\} x_0) = \\
        \\
        &= \dfrac{2}{\sqrt{\pi}\cdot \erf(T)} \cdot \left(x_T \cdot \exp\left\{-\dfrac{T^2+t^2}2\right\} - x_0 \cdot \exp\left\{- \dfrac{t^2}2\right\}\right).
    \end{align*} 
\end{enumerate}
\end{solution} 

\begin{problem}
    Мінімізувати критерій якості 
    \[ \mathcal{J}(u) = \int_0^T (u_1^1(s) + u_2^2(s)) ds \]
    за умов \[ \left\{ \begin{aligned} \dfrac{dx_1(t)}{dt} = 6x_1(t) - 2x_2(t) + u_1(t), \\ \dfrac{dx_2(t)}{dt} = 5x_1(t) - x_2(t) + u_2(t), \end{aligned} \right. \]
    \[ x_1(0) = x_{10}, x_2(0) = x_{20}, \] 
    \[ x_1(T) = x_2(T) = 0. \]
    Тут $x = (x_1, x_2)^\star$ -- вектор фазових координат з $\RR^2$, $u=(u_1,u_2)^\star$ -- вектор керування, $x = (x_{10}, x_{20})^\star$ -- відома точка, $t \in [0, T]$.
\end{problem}

\begin{solution}
    Запишемо систему у людському вигляді:
    \[ \begin{pmatrix} \dot x_1 \\ \dot x_2 \end{pmatrix} (t) = \begin{pmatrix} 6 & -2 \\ 5 & -1 \end{pmatrix} \begin{pmatrix} x_1 \\ x_2 \end{pmatrix} (t) + \begin{pmatrix} 1 & 0 \\ 0 & 1 \end{pmatrix} \begin{pmatrix} u_1 \\ u_2 \end{pmatrix} (t) \]
    
    Знайдемо власні числа матриці $A - \lambda E$: $\det(A - \lambda E) = \begin{vmatrix} 6 - \lambda & -2 \\ 5 & -1 - \lambda \end{vmatrix} = \lambda^2 - 5\lambda + 4 = (\lambda - 1) (\lambda - 4) = 0$, звідки $\lambda_1 = 1$, $\lambda_2 = 4$. Знайдемо власні вектори, вони будуть $\begin{pmatrix} 2 \\ 5 \end{pmatrix}$ і $\begin{pmatrix} 1 \\ 1 \end{pmatrix}$ відповідно. Звідси знаходимо загальний розв'язок
    \[ \begin{pmatrix} x_1 \\ x_2 \end{pmatrix} (t) = c_1 \begin{pmatrix} 2e^t \\ 5e^t \end{pmatrix} + c_2 \begin{pmatrix} e^{4t} \\ e^{4t} \end{pmatrix}. \]
    
    З рівняння
    \[ c_1 \begin{pmatrix} 2e^s \\ 5e^s \end{pmatrix} + c_2 \begin{pmatrix} e^{4s} \\ e^{4s} \end{pmatrix} = \begin{pmatrix} 1 \\ 0 \end{pmatrix} \]
    знаходимо $c_1 = -\dfrac13 e^{-s}$, $c_2 = \dfrac53 e^{-4s}$, а з рівняння
    \[ c_1 \begin{pmatrix} 2e^s \\ 5e^s \end{pmatrix} + c_2 \begin{pmatrix} e^{4s} \\ e^{4s} \end{pmatrix} = \begin{pmatrix} 0 \\ 1 \end{pmatrix} \]
    знаходимо $c_1 = \dfrac13 e^{-s}$, $c_2 = -\dfrac23 e^{-4s}$, тобто
    \[ \Theta(t, s) = \dfrac13\begin{pmatrix} -2 e^{t-s} + 5 e^{4(t-s)} & 2 e^{t-s} - 2 e^{4(t-s)} \\ -5 e^{t-s} + 5 e^{4(t-s)} & 5 e^{t-s} - 2 e^{4(t-s)} \end{pmatrix}. \]
    
    Знайдемо грамміан за формулою \[\Phi(T, 0) = \int_0^T \Theta(T, s) B(s) B^* (s) \Theta^*(T, s) ds. \]
    
    \[ \Theta(T, s) B(s) =  \dfrac13\begin{pmatrix} -2 e^{t-s} + 5 e^{4(t-s)} & 2 e^{t-s} - 2 e^{4(t-s)} \\ -5 e^{t-s} + 5 e^{4(t-s)} & 5 e^{t-s} - 2 e^{4(t-s)} \end{pmatrix}. \]
    \[ B^* (s) \Theta^*(T, s)  =  (\Theta(T, s) B(s))^\star =  \dfrac13\begin{pmatrix} -2 e^{t-s} + 5 e^{4(t-s)} & 5 e^{t-s} - 5 e^{4(t-s)} \\ -2 e^{t-s} + 2 e^{4(t-s)} & 5 e^{t-s} - 2 e^{4(t-s)} \end{pmatrix}. \]
    
    Обчислення грамміану є громіздкою обчислювальною задачею, ми залишаємо її як вправу для читача.
\end{solution}

\begin{problem}
    Записати систему диференціальних рівнянь для знаходження першої матриці керованості (грамміана керованості) і сформулювати критерій керованості на інтервалі $[0, T]$ у випадку, якщо система керування має вигляд:
    \[ \dfrac{d^2x(t)}{dt^2} + tx(t) = u(t). \]
    Тут $x$ -- стан системи, $u(t)$ -- скалярне керування, $t \in [0, T]$.
\end{problem}

\begin{solution}
    Зробимо заміну $x_1 = x$, $x_2 = \dot x$, тоді маємо систему \[ \begin{pmatrix} \dot x_1 \\ \dot x_2 \end{pmatrix} (t) = \begin{pmatrix} 0 & 1 \\ -t & 0 \end{pmatrix} \begin{pmatrix} x_1 \\ x_2 \end{pmatrix} (t) + \begin{pmatrix} 0 \\ 1 \end{pmatrix} \begin{pmatrix} u \end{pmatrix} (t). \]
    Звідси можемо записати систему диференціальних рівнянь для знаходження грамміана керованості:
    \[ \dfrac{\Phi(t, t_0)}{dt} = A(t) \Phi(t, t_0) + \Phi(t, t_0) A^\star(t) + B(t) B^*(t). \]
    \[ \dfrac{\Phi(t, 0)}{dt} = \begin{pmatrix} 0 & 1 \\ -t & 0 \end{pmatrix} \Phi(t, 0) + \Phi(t, 0) \begin{pmatrix} 0 & t \\ -1 & 0 \end{pmatrix} + \begin{pmatrix} 0 & 0 \\ 0 & 1 \end{pmatrix}. \]
    Окрім цього, не забуваємо про умову $\Phi(0, 0) = 0$.\\
    
    Щодо критерію керованості, то тут все просто (чи радше стандартно), для того щоб система була цілком керованою на $[0, T]$ необхідно і достатньо, щоб грамміан керованості $\Phi(T, 0)$ був невиродженим, тобто щоб $\det \Phi(T, 0) \ne 0$ або (що те саме у випадку невід'ємно-визначеної матриці) щоб $\det \Phi(T, 0) > 0$.
\end{solution}

\begin{problem}
    Дослідити на керованість, використовуючи другий критерій керованості:
    \begin{enumerate}
        \item \[ \left\{ \begin{aligned} \dfrac{dx_1}{dt} &= -x_1 + x_2 + au, \\ \dfrac{dx_2}{dt} &= x_1 + \dfrac ua; \end{aligned} \right. \]
        \item \[ \left\{ \begin{aligned} \dfrac{dx_1}{dt} &= x_1 - x_2 + au, \\ \dfrac{dx_2}{dt} &= x_1 + \dfrac ua; \end{aligned} \right. \] 
        \item \[ \left\{ \begin{aligned} \dfrac{dx_1}{dt} &= x_1 + x_2 + au, \\ \dfrac{dx_2}{dt} &= - x_1 + x_2 + a^2u; \end{aligned} \right. \]
        \item \[ \left\{ \begin{aligned} \dfrac{dx_1}{dt} &= 2x_1 + x_2 - au, \\ \dfrac{dx_2}{dt} &= - x_1 + au; \end{aligned} \right. \]
        \item \[ x^{(n)}(t) + a_1 x^{(n-1)}(t) + \ldots + a_{n-1}x'(t) + a_n x(t) = u(t). \]
        \item \[ \left\{ \begin{aligned} \dfrac{dx_1}{dt} &= x_1 + 2x_2 - x_3 + u_1 - u_2 \\ \dfrac{dx_2}{dt} &= -x_1 + x_2 + 3x_3 + u_1 \\ \dfrac{dx_3}{dt} &= x_2 + x_3 + 2u_2 \end{aligned} \right. \]
    \end{enumerate}
\end{problem}

\begin{solution}
    \begin{enumerate}
        \item \[ A = \begin{pmatrix} -1 & 1 \\ 1 & 0 \end{pmatrix} \qquad B = \begin{pmatrix} a \\ 1 / a \end{pmatrix} \]
        \[ D = \begin{pmatrix} B & AB \end{pmatrix} = \begin{pmatrix} a & 1 / a - a \\ 1 / a & a \end{pmatrix} \]
        \[ \det D = a^2 + 1 - 1/a^2 \ne 0, \]
        тобто система цілком керована якщо тільки $a \ne \pm \sqrt{\dfrac{\sqrt{5} - 1}{2}}$.
        \item \[ A = \begin{pmatrix} 1 & -1 \\ 1 & 0 \end{pmatrix} \qquad B = \begin{pmatrix} a \\ 1 / a \end{pmatrix} \]
        \[ D = \begin{pmatrix} B & AB \end{pmatrix} = \begin{pmatrix} a & a - 1 / a \\ 1 / a & a \end{pmatrix} \]
        \[ \det D = a^2 - 1 + 1/a^2 \ne 0, \]
        тобто система цілком керована для будь-яких $a$ (навіть $\det D \ge 1$ за нерівністю Коші).
        \item \[ A = \begin{pmatrix} 1 & 1 \\ -1 & 1 \end{pmatrix} \qquad B = \begin{pmatrix} a \\ a^2 \end{pmatrix} \]
        \[ D = \begin{pmatrix} B & AB \end{pmatrix} = \begin{pmatrix} a & a + a^2 \\ a^2 & a^2 - a \end{pmatrix} \]
        \[ \det D = a^3 - a^2 - a^4 - a^3 = -a^4 - a^2 \ne 0, \]
        тобто система цілком керована якщо тільки $a \ne 0$.
        \item \[ A = \begin{pmatrix} 2 & 1 \\ -1 & 0 \end{pmatrix} \qquad B = \begin{pmatrix} -a \\ a \end{pmatrix} \]
        \[ D = \begin{pmatrix} B & AB \end{pmatrix} = \begin{pmatrix} -a & -a \\ a & a \end{pmatrix} \]
        \[ \det D = 0, \]
        тобто система не є цілком керованою для будь-яких $a$.
        \item Зробимо заміну $x_0 = x$, $x_1 = x'$, $\ldots$, $x_n = x^{(n)}$, тоді отримаємо систему
        \[ \left\{ \begin{aligned} \dot x_0 &= x_1 \\ \dot x_1 &= x_2 \\ \cdots \\ \dot x_{n-1} &= x_n \\ \dot x_n &= u - a_n x_0 - a_{n-1} x_1 - \ldots - a_1 x_{n-1} \end{aligned} \right. \]
        тобто
        \[ A = \begin{pmatrix} 0 & 1 & \ddots & 0 & 0 \\ 0 & 0 & \ddots & \ddots & 0 \\ \vdots & \vdots & \ddots & \ddots & \ddots \\ 0 & 0 & \cdots & 0 & 1 \\ -a_n & -a_{n-1} & \cdots & -a_1 & 0 \end{pmatrix} \qquad B = \begin{pmatrix} 0 \\ 0 \\ \vdots \\ 0 \\ 1 \end{pmatrix} \]
        \[ D = \begin{pmatrix} B & AB & A^2B & \cdots & A^nB \end{pmatrix} = \begin{pmatrix} 0 & \cdots & 0 & 0 & 1 \\ \vdots & \reflectbox{$\ddots$} & \reflectbox{$\ddots$} & \reflectbox{$\ddots$} & \cdots \\ 0 & 0 & 1 & 0 & \cdots \\ 0 & 1 & 0 & -a_1 & \cdots \\ 1 & 0 & -a_1 & -a_2 & \cdots \end{pmatrix} \]
        \[ \det D = -1 \ne 0, \] тобто система цілком керована для довільних $a_1$, $a_2$, $\ldots$, $a_n$.
        \item \[ A = \begin{pmatrix} 1 & 2 & -1 \\ -1 & 1 & 3 \\ 0 & 1 & 1 \end{pmatrix} \qquad B = \begin{pmatrix} 1 & -1 \\ 1 & 0 \\ 0 & 2 \end{pmatrix} \]
        \[ D = \begin{pmatrix} B & AB & A^2B  \end{pmatrix} = \begin{pmatrix} 1 & -1 & 3 & -3 & 2 & 8 \\ 1 & 0 & 0 & 7 & 0 & 16 \\ 0 & 2 & 1 & 2 & 1 & 9 \end{pmatrix} \]
        Її ранг дорівнює 3, тобто система цілком керована.
    \end{enumerate}
\end{solution}

