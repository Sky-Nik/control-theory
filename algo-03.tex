\section{Задача про переведення системи з точки в точку. Критерії керованості лінійної системи керування}

\subsection{Алгоритми}

\begin{problem*}
	Перевести систему $\dot x = A x + B u$ з точки $x_0$ в точку $x_T \in \RR^1$ за допомогою керування з класу $K$ (керування, залежні від вектору параметрів $c$).
\end{problem*}

\begin{algorithm} \tt
	\begin{enumerate}
		\item Знаходимо траєкторію системи при заданому керуванні (залежну від параметра $c$). 
		\item Знаходимо з отриманого алгебраїчного рівняння параметр $c$.
	\end{enumerate}
\end{algorithm}

\begin{problem*}
	\begin{enumerate}
		\item Знайти грамміан керованості системи $\dot x = A x + B u$ за визначенням.
		\item Записати систему диференційних рівнянь для знаходження грамміана керованості.
		\item Використовуючи грамміан керованості, знайти інтервал повної керованості системи.
		\item Для цього інтервалу записати керування яке певеродить систему з точки $x_0$ в точку $x_T$/розв'язати задачу оптимального керування.
	\end{enumerate}
\end{problem*}

\begin{algorithm} \tt
	Розглянемо всі пункти задачі вище.
	\begin{enumerate}
		\item \begin{enumerate}
			\item Знаходимо $\Theta(T,s)$.
			\item Використовуємо формулу \[\Phi(T, t_0) = \int_{t_0}^T \Theta(T, s) B(s) B^*(s) \Theta^*(T, s) \diff s.\]
		\end{enumerate}
		\item Записуємо систему \[ \dot \Phi(t, t_0) = A(t) \cdot \Phi(t,t_0)+\Phi(t,t_0)\cdot A^*(t)+B(t)\cdot B^*(t), \Phi(t_0,t_0) = 0. \]
		\item Це інтервал на якому $\Phi(t,t_0) \ne 0$.
		\item Використовуємо формулу \[ u (t) = B^*(t) \cdot \Theta(T,t)\cdot\Phi^{-1}(T,t_0)(x_T-\Theta(T,t_0)\cdot x_0). \]
	\end{enumerate}
\end{algorithm}

\begin{problem*}
	Дослідити стаціонарну систему $\dot x = A x + B u$ на керованість використовуючи другий критерій керованості.
\end{problem*}

\begin{algorithm} \tt
	\begin{enumerate}
		\item Знаходимо $D = \left(B \vdots AB \vdots A^nB \vdots\ldots\vdots A^{n-1}B\right)$.
		\item Якщо $rang D = n$ то стаціонарна системи цілком керована, інакше ні.
	\end{enumerate}
\end{algorithm}
