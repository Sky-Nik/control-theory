% cd ..\..\Users\NikitaSkybytskyi\Desktop\control-theory

% cls && pdflatex ahw.tex && cls && pdflatex ahw.tex && start ahw.pdf

\documentclass[a5paper, 10pt]{article}
\usepackage[T2A,T1]{fontenc}
\usepackage[utf8]{inputenc}
\usepackage[english, ukrainian]{babel}
\usepackage{amsmath, amssymb}
\usepackage[top = 2 cm, left = 1 cm, right = 1 cm, bottom = 2 cm]{geometry} 

\title{{\Huge ТЕОРІЯ КЕРУВАННЯ}}
\date{}

\usepackage{fancyhdr}
\pagestyle{fancy}
\lhead{Семінари з теорії керування, 2018}
\rhead{Нікіта Скибицький, ОМ-3}
\cfoot{\thepage}

\usepackage{amsthm}
\newtheorem{definition}{Визначення}
\theoremstyle{definition}
\newtheorem*{problem*}{\normalfont{\textit{Задача}}}
\newtheorem{problem}{\normalfont{\textit{Задача}}}[section]
\newtheorem{algorithm}{\tt Алгоритм}[section]
\newtheorem*{solution}{Розв'язок}

\allowdisplaybreaks
\setlength\parindent{0pt}
\numberwithin{equation}{section}

\usepackage{xcolor}
\usepackage{hyperref}
\hypersetup{unicode=true,colorlinks=true,linktoc=all,linkcolor=red}

\usepackage{graphicx}

\newcommand{\JJ}{\mathcal{J}}
\newcommand{\KK}{\mathcal{K}}
\newcommand{\MM}{\mathcal{M}}
\newcommand{\UU}{\mathcal{U}}
\newcommand{\XX}{\mathcal{X}}
\newcommand{\NN}{\mathcal{N}}
\newcommand{\HH}{\mathcal{H}}
\newcommand{\EE}{\mathcal{E}}
\newcommand{\RR}{\mathbb{R}}
\newcommand{\Max}{\displaystyle\max\limits}

\newcommand*\diff{\mathop{}\!\mathrm{d}}

\renewcommand{\phi}{\varphi}
\renewcommand{\SS}{\mathcal{S}}
\renewcommand{\epsilon}{\varepsilon}

\DeclareMathOperator{\erf}{erf}
\DeclareMathOperator{\erfi}{erfi}
\DeclareMathOperator{\signum}{sgn}

\begin{document}

\maketitle \thispagestyle{empty} \newpage 

У ваших руках конспект семінарських занять з нормативного курсу ``Теорія керування'', прочитаного доц., д.ф.-м.н. Пічкуром Володимиром Володимировичем на третьому курсі спеціальності прикладна математика факультету комп\-'ю\-тер\-них наук та кібернетики Київського національного університету імені Тараса Шевченка восени 2018-го року. \\

Конспект у компактній формі відображає матеріал курсу, допомагає сформувати загальне уявлення про предмет вивчення, правильно зорієнтуватися в даній галузі знань. Конспект лекцій з названої дисципліни сприятиме більш успішному вивченню дисципліни, причому більшою мірою для студентів заочної форми, екстернату, дистанційного та індивідуального навчання. \\

Упорядник безмежно вдячний Живолович Олександрі та Мельник Катерині а також решті групи ОМ-3, чиї безцінні конспекти лягли в основу цього збірника, та Антиповій Алісі за верстку частини задач. \\

Структура конспекту наступна: задачі розділені за темами (\tt{section}\normalfont), кожна тема містить три частини (\tt{subsection}\normalfont): 
\begin{enumerate}
	\item Алгоритми -- типові задачі теми із загальними алгоритмами розв'язування.
	\item Аудиторне заняття -- задачі, що пропонувалися для роботи на семінарі, абсолютна більшість із розв'язаннями.
	\item Домашнє завдання -- задачі, які пропонувалися (не всі) на домашню роботу, майже всі із розв'язаннями.
\end{enumerate}

Комп\-'ю\-тер\-ний набір та верстка -- Скибицький Нікіта Максимович.

\newpage

\tableofcontents \newpage

\subsection{Аудиторне заняття}

\begin{problem}
	Задана скалярна система керування 
	
	\begin{equation}
		\label{eq:1.1}
		\frac{\diff x(t)}{\diff t} = u(t), \quad x(0) = 1.
	\end{equation}

	Тут $x$ -- стан системи, $t \in [0, 1]$. Керування задане у вигляді

	\begin{equation}
		\label{eq:1.2}
		u(x) = a x.
	\end{equation}

	Тут $a$ -- скалярний параметр.

	\begin{enumerate}
		\item Знайти траєкторію системи (\ref{eq:1.1}) при керуванні (\ref{eq:1.2}).

		\item Знайти програмне керування $u(t) = a x (t)$, яке відповідає знайденій траєкторії. 

		\item Оцінити, при якому значення параметра $a \in \{2, 4, -3\}$ критерій якості $\JJ(u) = x^2 (1)$ буде мати менше значення.
	\end{enumerate}
\end{problem}

\begin{solution}
    Скористаємося пунктами 2, 3, 5 алгоритму \ref{algo-1-1}:
    
	\begin{enumerate}
		\item Підставляючи керування (\ref{eq:1.2}) у систему (\ref{eq:1.1}), отримаємо систему \[ \frac{\diff x(t)}{\diff t} = a x(t), \quad x(0) = 1. \]
		Її розв'язок має вигляд \[ x(t) = x(0) \cdot e^{a t} = e^{a t}. \]
		\item Підставляємо знайдену у попередньому пункті траєкторію у вигляд (\ref{eq:1.2}) керування: \[ u(t) = a x(t) = a \cdot e^{a t}. \]
		\item Множина $\{2, 4, -3\}$ скінченна, тому можна просто перебрати всі її елементи та обчислити значення критерію якості на кожному з них: \[ \begin{aligned} \left. \JJ (u) \right|_{a = 2} &= x^2 (1) = \left. e^{2 a t} \right|_{t = 1} = e^4, \\ \left. \JJ (u) \right|_{a = 4} &= x^2 (1) = \left. e^{2 a t} \right|_{t = 1} = e^8, \\ \left. \JJ (u) \right|_{a = -3} &= x^2 (1) = \left. e^{2 a t} \right|_{t = 1} = e^{-6}. \end{aligned} \]
		Найменшим з цих значень є $e^{-6}$ яке досягається при $a = -3$.
	\end{enumerate}
\end{solution}

\begin{problem}
	Задана система керування

	\begin{equation}
		\label{eq:1.3}
		\left\{
			\begin{aligned}
				\frac{\diff x_1 (t)}{\diff t} &= x_1 (t) + x_2 (t) + u (t), \\
				\frac{\diff x_2 (t)}{\diff t} &= - x_1 (t) + x_2 (t) + u(t),
			\end{aligned}
		\right.
		\quad
		x_1 (0) = 2, x_2 (0) = 1.
	\end{equation}

	Тут $x = (x_1, x_2)^*$ -- вектор фазових координат з $\RR^2$, $t \in [0, T]$. Керування задане у вигляді

	\begin{equation}
		\label{eq:1.4}
		u(x_1, x_2) = 2 x_1 + x_2.
	\end{equation}

	\begin{enumerate}
		\item Знайти траєкторію системи (\ref{eq:1.3}) при керуванні (\ref{eq:1.4}).

		\item Знайти програмне керування $u(t) = 2 x_1 (t) + x_2 (t)$, яке відповідає знайденій траєкторії.

		\item Якою буде фундаментальна матриця, нормована за моментом $s$, системи. що одержана при підстановці керування (\ref{eq:1.4}) у систему (\ref{eq:1.3})?

		\item Побудувати спряжену систему до системи, одержаної при підстановці керування (\ref{eq:1.4}) у систему (\ref{eq:1.3}), та її фундаментальну матрицю.
	\end{enumerate}
\end{problem}

\begin{solution}
    Скористаємося пунктами 2, 3, 5-7 алгоритму \ref{algo-1-1}:
    
	\begin{enumerate}
		\item Підставляючи керування (\ref{eq:1.4}) у систему (\ref{eq:1.3}), отримаємо систему \[
		\left\{
			\begin{aligned}
				\frac{\diff x_1 (t)}{\diff t} &= 3 x_1 (t) + 2 x_2 (t), \\
				\frac{\diff x_2 (t)}{\diff t} &= x_1 (t) + 2 x_2 (t),
			\end{aligned}
		\right.
		\quad
		x_1 (0) = 2, x_2 (0) = 1.
		\]

		Її розв'язок \[
		\left\{
			\begin{aligned}
				x_1 (t) &= 2 e^{4 t}, \\
				x_2 (t) &= e^{4 t}.
			\end{aligned}
		\right.
		\]

		\item Підставляємо знайдену у попередньому пункті траєкторію у вигляд (\ref{eq:1.4}) керування: \[ u(t) = 2 x_1 (t) + x_2 = 2 \cdot \left( 2 e^{4 t} \right) + e^{4 t} = 5 e^{4 t}. \]

		\item Загальним розв'язком системи (\ref{eq:1.3}) з підставленим керуванням (\ref{eq:1.4}) є \[
		\left\{
			\begin{aligned}
				x_1 (t) &= c_1 e^t + 2 c_2 e^{4 t}, \\
				x_2 (t) &= - c_1 e^t + c_2 e^{4 t}.
			\end{aligned}
		\right.
		\]

		Це означає, що фундаментальна матриця цієї системи матиме вигляд \[
			\Theta(t) = 
			\begin{pmatrix}
				c_{11} e^t + 2 c_{12} e^{4 t} & c_{21} e^t + 2 c_{22} e^{4 t} \\
				- c_{11} e^t + c_{12} e^{4 t} & - c_{21} e^t + c_{22} e^{4 t}
			\end{pmatrix}
		\]

		Залишається нормувати її за моментом $s$, тобто знайти такі $c_{11} (s)$, $c_{12} (s)$, $c_{21} (s)$, $c_{22} (s)$, що $\Theta(s, s) = I$. Отримаємо \[
			\Theta(t, s) = \frac{1}{3}
			\begin{pmatrix}
				e^{t - s} + 2 e^{4 (t - s)} & -2 e^{t - s} + 2 e^{4 (t - s)} \\
				e^{t - s} - e^{4 (t - s)} & 2 e^{t - s} + e^{4 (t - s)}
			\end{pmatrix}
		\]

		\item Спряжена система
		\[
		\left\{
			\begin{aligned}
				\frac{\diff z_1 (t)}{\diff t} &= - 3 z_1 (t) - z_2 (t), \\
				\frac{\diff z_2 (t)}{\diff t} &= - 2 z_1 (t) - 2 z_2 (t),
			\end{aligned}
		\right.
		\]

		а відповідна фундаментальна матриця \[ \Psi(t, s) = \Theta^*(s, t) = \frac{1}{3}
			\begin{pmatrix}
				e^{s - t} + 2 e^{4 (s - t)} & e^{s - t} - e^{4 (s - t)} \\
				-2 e^{s - t} + 2 e^{4 (s - t)} & 2 e^{s - t} + e^{4 (s - t)}
			\end{pmatrix}
		\]
	\end{enumerate}
\end{solution}

\begin{problem}
	Розглядається задача Больца \[ \JJ (u) = \int_0^1 u^2 (s) \diff s + ( x (1) - 2 )^2 \to \inf \]

	за умови, що \[ \frac{\diff x (t)}{\diff t} = x^2 (t) + u (t), \quad x (0) = x_0. \]

	Тут $x (t) \in \RR^1$, $u (t) \in \RR^1$, $t \in [0, 1]$. Точка $x_0 \in \RR^1$ задана. Звести цю задачу до задачі Майєра.
\end{problem}

\begin{solution}
    Скористаємося алгоритмом \ref{algo-1-2}:

	Введемо нову змінну \[ x_2 = \int_0^t u^2 (s) \diff s, \] 

	тоді \[ \JJ (u) = x_2 (1) + ( x_1 (1) - 2)^2 \to \inf, \]

	за умов, що \[ 
	\left\{
		\begin{aligned}
			\frac{\diff x_1 (t)}{\diff t} &= x_1^2 (t) + u (t), \\
			\frac{\diff x_2 (t)}{\diff t} &= u^2 (t).
		\end{aligned}
	\right.
	\quad
	x_1 (0) = x_0, x_2 (0) = 0.
	\]
\end{solution}

\begin{problem}
	Задана система керування 
	\[
		\left\{
			\begin{aligned}
				\dfrac{\diff x_1(t)}{\diff t} &= 2x_1(t) + x_2(t) + u(t), \\
				\dfrac{\diff x_2(t)}{\diff t} &= 3x_1(t) + 4x_2(t),
			\end{aligned}
		\right.
		\quad
		x_1(0) = 1, x_2(0) = -1.
	\]

	Тут $x = (x_1, x_2)^*$ -- вектор фазових координат з $\RR^2$, $t \in [0, 2]$. Керування задане у вигляді
	\[
		u(t) 
		=
		\begin{cases}
			0, & \text{якщо } t \in [0, 1], \\
			1, & \text{якщо } t \in (1, 2].
		\end{cases}
	\]
	
	\begin{enumerate}
		\item Знайти траєкторію системи, яка відповідає цьому керуванню.
		\item Чи буде ця траєкторія неперервно диференційовною?
		\item Чи буде таке керування кращим в порівнянні з керуванням $u(t) = 0$, $t \in [0, 2]$ 
		за умови, що критерій якості має вигляд 
		\[ \JJ (u) = x_1^2(2) + x_2^2(2) \to \min. \]
	\end{enumerate}
\end{problem}

\begin{solution}
    Скористаємося пунктами 2-5 алгоритму \ref{algo-1-1}:
    
	\begin{enumerate}
		\item При $t \in [0, 1]$ маємо 
		\[ \begin{pmatrix} \dot x_1 \\ \dot x_2 \end{pmatrix} = \begin{pmatrix}	2 & 1 \\ 3 & 4 \end{pmatrix} \cdot \begin{pmatrix} x_1 \\ x_2 \end{pmatrix}.\]
		
		Характеристичне рівняння 
		\[ \begin{vmatrix} 2 - \lambda & 1 \\ 3 & 4 - \lambda \end{vmatrix} = \lambda^2 - 6\lambda + 5 = 0,\]
		звідки $\lambda_1 = 1$, $\lambda_2 = 5$.\\
		
		З курсу диференціальних рівнянь відомо, що загальний розв'язок має вигляд
		\[ x = c_1 v_1 e^t + c_2 v_2 e^{5t},\]
		де $v_1$, $v_2$ -- власні вектори, що відповідають $\lambda_1$ та $\lambda_2$ відповідно. Підставляючи $\lambda_i$, $i=1,2$, знаходимо
		\[ \begin{pmatrix} x_1 \\ x_2 \end{pmatrix} = c_1 \begin{pmatrix} 1 \\ -1 \end{pmatrix} e^t + c_2 \begin{pmatrix} 1 \\ 3 \end{pmatrix} e^{5t}.\]
		
		Підставляючи $t = 0$ отримуємо $c_1 = 1$, $c_2 = 0$. При $t \in (1, 2]$ маємо 
		\[ \begin{pmatrix} x_1 \\ x_2 \end{pmatrix} = c_1 v_1 e^t + c_2 v_2 e^{5t} + \begin{pmatrix} c_3 \\ c_4 \end{pmatrix}, \]
		де $c_3$, $c_4$ задовольняють систему 
		\[
			\left\{
				\begin{aligned}
					2c_3 &+ c_4 + 1 &= 0, \\
					3c_3 &+ 4c_4 &= 0,
				\end{aligned}
			\right.
		\]
		звідки $c_3 = -4/5$, $c_4 = 3/5$ і 
		\[ \begin{pmatrix} x_1 \\ x_2 \end{pmatrix} = c_1 v_1 e^t + c_2 v_2 e^{5t} + \begin{pmatrix} -4/5 \\ 3/5 \end{pmatrix},\]
		
		Підставляючи $t = 1$ отримуємо $c_1 = \left(1 + \dfrac{3}{4e}\right)$, $c_2 = \dfrac{1}{20e^5}$. Остаточно маємо 
		\[ \begin{pmatrix} x_1 \\ x_2 \end{pmatrix} = \begin{cases} \begin{pmatrix} 1 \\ -1 \end{pmatrix} e^t, & t \in [0, 1], \\ \left(1 + \dfrac{3}{4e}\right) \begin{pmatrix} 1 \\ -1 \end{pmatrix} e^t + \dfrac{1}{20e^5} \begin{pmatrix} 1 \\ 3 \end{pmatrix} e^{5t} + \begin{pmatrix} -4/5 \\ 3/5 \end{pmatrix}, & t \in (1, 2]. \end{cases}\]
		
		\item 
		\[ \dot x (1-) = \begin{pmatrix} 2 & 1 \\ 3 & 4	\end{pmatrix} \cdot x(1-). \]
		
		З неперервності $x_1$, $x_2$ маємо:
		\[ \begin{pmatrix} 2 & 1 \\ 3 & 4 \end{pmatrix} \cdot x(1-) = \begin{pmatrix} 2 & 1 \\ 3 & 4 \end{pmatrix} \cdot x(1). \]
		
		З іншого боку,
		\[ \dot x (1+) = \begin{pmatrix} 2 & 1 \\ 3 & 4 \end{pmatrix} \cdot x(1+) + \begin{pmatrix} 1 \\ 0 \end{pmatrix} = \begin{pmatrix} 2 & 1 \\ 3 & 4 \end{pmatrix} \cdot x(1) + \begin{pmatrix} 1 \\ 0 \end{pmatrix} \]
		
		Нескладно бачити, що 
		\[ \dot x (1-) \ne \dot x (1+), \]
		тобто траєкторія не є неперервно диференційовною в точці $1$.
		
		\item 
		Підставимо $t=2$ в розв'язки для обох керувань (%попутно зауваживши
		зауважимо, що для нового керування розв'язок ми вже знаємо, це просто продовження вже знайденого розв'язку для $t \in [0, 1]$):
		\[
			\left(e^2 + \dfrac34e + \dfrac{e^5}{20} - \dfrac45\right)^2 + \left(-e^2 - \dfrac34e + \dfrac{3e^5}{20} + \dfrac35\right)^2
			\lor
			(e^2)^2 + (-e^2)^2
		\]
	    Виконавши обчислення знаходимо, що права частина менше, тобто нове керування є кращим за початкове.
	\end{enumerate}
\end{solution}
 \newpage

% OK, incomplete, missing 1.7

\subsection*{Домашнє завдання}

\begin{problem}
	Задана система керування 
	\begin{equation}
		\label{eq:1.5}
		\left\{
			\begin{aligned}
				\dfrac{dx_1(t)}{dt} &= -8x_1(t)  -x_2(t) + u(t),\\
				\dfrac{dx_2(t)}{dt} &= 6x_1(t) + 3x_2(t),
			\end{aligned}
		\right.
		\quad
		x_1(0) = -2, x_2(0) = 1.
	\end{equation}

	Тут $ x =(x_1, x_2)^*$ -- вектор фазових координат з $\RR^2$, $t \in [0, 1]$. Керування задане у вигляді
	\begin{equation}
		\label{eq:1.6}
		u(x_1, x_2) = 4x_1 - x_2.
	\end{equation}

	\begin{enumerate}
		\item До якого класу керувань належить керування (\ref{eq:1.6}): програмних керувань, чи керувань з оберненим зв'язком?
		\item Знайти траєкторію системи при керуванні (\ref{eq:1.6}).
		\item Знайти програмне керування $u(t) = 4x_1(t) - x_2(t)$, яке відповідає знайденій траєкторії.
		\item Якою буде фундаментальна матриця, нормована за моментом $s$, системи, що одержана при підстановці керування (\ref{eq:1.6}) в систему (\ref{eq:1.5})?
		\item Побудувати спряжену систему до системи, одержаної при підстановці керування (\ref{eq:1.6}) в систему (\ref{eq:1.5}), та її фундаментальну матрицю.
	\end{enumerate}
\end{problem}

\begin{solution}
	\begin{enumerate}
		\item З оберненим зв'язком.
		\item 
		\[
			\begin{pmatrix}
				\dot x_1 \\ 
				\dot x_2
			\end{pmatrix}
			=
			\begin{pmatrix}
				-4 & -2 \\
				6 & 3
			\end{pmatrix}
			\begin{pmatrix}
				x_1 \\
				x_2
			\end{pmatrix}.
		\]
		Характеристичне рівняння 
		\[
			\begin{vmatrix}
				-4 - \lambda & -2 \\
				6 & 3 - \lambda 
			\end{vmatrix}
			=
			\lambda^2 + \lambda 
			=
			(\lambda + 1) \lambda
			=
			0,
		\]
		звідки $\lambda_1 = -1$, $\lambda_2 = 0$.\\
		
		З курсу диференційних рівнянь відомо, що тоді загальний розв'язок має вигляд
		\[
			\begin{pmatrix}
				x_1 \\
				x_2
			\end{pmatrix}
			=
			c_1 v_1 e^{-t} + c_2 v_2,
		\]
		де $v_1$, $v_2$ -- власні вектори, що відповідають $\lambda_1$ та $\lambda_2$ відповідно.\\
		
		Нескладно бачити, що 
		\[
			\begin{pmatrix}
				x_1 \\
				x_2
			\end{pmatrix}
			=
			c_1 
			\begin{pmatrix}
				1 \\
				-2
			\end{pmatrix} 
			+ 
			c_2 
			\begin{pmatrix}
				2 \\
				-3
			\end{pmatrix}
			e^{-t} .
		\]
		
		Підставляючи $t=0$ отримуємо $c_1 = 4$, $c_2 = -3$.\\
		
		Остаточно маємо:
		\[
			\begin{pmatrix}
				x_1 \\
				x_2
			\end{pmatrix}
			=
			4
			\begin{pmatrix}
				1 \\
				-2
			\end{pmatrix} 
			-3
			\begin{pmatrix}
				2 \\
				-3
			\end{pmatrix}
			e^{-t}
			.
		\]
		\item Просто підставляємо знайдені $x_1(t)$, $x_2(t)$ в $u(x_1, x_2)$:
		\[
			u(t)
			=
			4\left(4\cdot (1) - 3\cdot (2)\cdot e^{-t}\right)
			-
			\left(4\cdot (-2) - 3\cdot (-3)\cdot e^{-t}\right)
			=
			24 - 33e^{-t}.
		\]
		\item З вигляду загального розв'язку бачимо, що вищезгадана фундаментальна матриця матиме вигляд
		\[
			\Theta(t,s)
			=
			\begin{pmatrix}
				c_1 + 2c_2 e^{s-t} & c_3 + 2c_4 e^{s-t} \\
				-2c_1 - 3c_2 e^{s-t} & -2c_3 - 3c_4 e^{s-t} 
			\end{pmatrix},
		\]
		причому 
		\[
			\left\{
				\begin{aligned}
					c_1   &+ 2c_2 &= 1 \\
					-2c_1 &- 3c_2 &= 0
				\end{aligned}
			\right.		
		\]
		(і аналогічна система для $c_3$, $c_4$).\\
		
		Знаходимо $c_1 = -3$, $c_2 = 2$, $c_3 = -2$, $c_4 = 1$ і підставляємо у матрицю:
		\[
			\Theta(t,s)
			=
			\begin{pmatrix}
				-3 + 4e^{s-t} & -2 + 2e^{s-t} \\
				6 - 6e^{s-t} & 4 - 3e^{s-t} 
			\end{pmatrix},
		\]
		
		\item Спряжена система буде
		\[
			\begin{pmatrix}
				\dot y_1 \\
				\dot y_2
			\end{pmatrix}
			=
			\begin{pmatrix}
				4 & -6 \\
				2 & -3
			\end{pmatrix}
			\begin{pmatrix}
				y_1 \\
				y_2
			\end{pmatrix},
		\]
		а відповідна фундаметальна матриця
		\[
			\Psi(t,s)
			=
			\Theta^*(s,t)
			=
			\begin{pmatrix}
				-3 + 4e^{t-s} & 6 - 6e^{t-s} \\
				-2 + 2e^{t-s} & 4 - 3 e^{t-s}
			\end{pmatrix},
		\]
	\end{enumerate}
\end{solution}

\begin{problem}
	Розглядається задача Лагранжа
		\[
		\JJ(u)
		=
		\int_0^T u^2(s) \diff s \to \inf
		\]
		за умови, що
		\[
			\left\{
				\begin{aligned}
					\dfrac{dx_1(t)}{dt} &= -x_1(t) + x_2(t) + u(t), \\
					\dfrac{dx_2(t)}{dt} &= x_1(t)x_2(t),
				\end{aligned}
			\right.
			\quad
			x_1(0)=0,x_2(0)=1.
		\]
		Тут $x=(x_1,x_2)^*$ -- вектор фазових координат з $\RR^2$, $t\in[0,T]$. Звести цю задачу до задачі Майєра.
\end{problem}

\begin{solution}
	Введемо нову фазову координату $x_3(t)=\int_0^t u^2(s) \diff s$, тоді до системи додається початкова умова $x_3(0)=0$, рівняння $\dot x_3 = u^2$, а функціонал якості переписується у вигляді $x_3(T) \to \inf$.
\end{solution}


\begin{problem}
	% 1.7
\end{problem}

\begin{solution}
	% 1.7
\end{solution}
 \newpage

% OK, incomplete, missing 2.1, 2.2

\section{Елементи багатозначного аналізу. Множина досяжності}

\subsection*{Аудиторне заняття}

\begin{problem}
	% 2.1
\end{problem}

\begin{solution}
	% 2.1
\end{solution}

\begin{problem}
	% 2.2
\end{problem}

\begin{solution}
	% 2.2
\end{solution}

\begin{problem}
	Знайти опорні функції таких множин:

	\begin{enumerate}
		\item $A = [0, r]$;

		\item $A = [-r, r]$;

		\item $A = \{ (x_1, x_2): |x_1| \le 1, |x_2| \le 2 \}$;

		\item $A = \KK_r (0) = \{ x \in \RR^n: \|x\| \le r \}$;

		\item $A = \SS^n = \{ x \in \RR^n: \|x\| = 1 \}$.
	\end{enumerate}
\end{problem}

\begin{solution}
	\begin{enumerate}
		\item За означення опорної функції, \[ c(A, \psi) = \max_{a \in A} \langle a, \psi \rangle = \begin{cases} 0, & \psi < 0 \\ r \psi, & 0 \le \psi \end{cases} = \max \{ 0, r \psi \}. \]

		\item За означення опорної функції, \[ c(A, \psi) = \max_{a \in A} \langle a, \psi \rangle = \begin{cases} - r \psi, & \psi < 0 \\ r \psi, & 0 \le \psi \end{cases} = r |\psi |. \]

		\item За означення опорної функції, \[ c(A, \psi) = \max_{a \in A} \langle a, \psi \rangle = \max_{a \in A} (\psi_1 x_1 + \psi_2 x_2) = |\psi_1| + 2 |\psi_2|. \]

		\item За властивістю опорної функції (вона дорівнює орієнтованій відстані від початку координат до опорної площини множини $A$ яка відповідає напрямку $\psi$), маємо $c(\KK_r (0), \psi) = r \| \psi \|$.

		\item За тією ж властивістю опорної функції маємо $c(\SS^n, \psi) = \| \psi \|$.
	\end{enumerate}
\end{solution}

\begin{problem}
	Знайти інтеграл Аумана $\JJ = \int_0^1 F(x) \diff x$ таких багатозначних відображень:

	\begin{enumerate}
		\item $F(x) = [0, x]$, $x \in [0, 1]$;

		\item $F(x) = \KK_x (0) = \{ y \in \RR^n: \|y\| \le x \}$, $x \in [0, 1]$.
	\end{enumerate}
\end{problem}

\begin{solution}
	Скористаємося рівністю \[ c \left(\int_0^1 F(x), \psi\right) \diff x = \int_0^1 c(F(x), \psi) \diff x, \]

	яка виконується в умовах теореми Ляпунова про опуклість інтегралу Аумана.

	\begin{enumerate}
		\item \[c(\JJ) = \int_0^1 c([0, x], \psi) \diff x = \begin{cases} \psi / 2, & 0 \le \psi \\ 0, & \psi < 0 \end{cases}. \]

		А далі наші знання опорних функцій підказують, що $\JJ = [0, 1 / 2]$.

		\item \[c(\JJ) = \int_0^1 c(\KK_x(0), \psi) \diff x = \int_0^1 x \| \psi \| \diff x = \| \psi \| / 2. \]

		А далі наші знання опорних функцій підказують, що $\JJ = \KK_{1 / 2} (0)$.
	\end{enumerate}
\end{solution}

\begin{problem}
	Знайти множину досяжності такої системи керування: \[ \frac{\diff x}{\diff t} = x + u, \]

	де $x (0) = x_0 \in \MM_0$, $u (t) \in \UU$, $t \ge 0$, \[ \MM_0 = \{ x: |x| \le 1 \}, \] \[ \UU = \{ u: |u| \le 1 \}. \]
\end{problem}

\begin{solution}
	Скористаємося теоремою про вигляд множини досяжності лінійної системи керування: \[ \XX(t, \MM_0) = \Theta(t, t_0) \MM_0 + \int_{t_0}^t \Theta(t, s) B(s) \UU(s) \diff s. \]

	Підставимо вже відомі значення: \[ \XX(t, [-1, 1]) = \Theta(t, 0) \cdot [-1, 1] + \int_0^t \left( \Theta(t, s) \cdot 1 \cdot [-1, 1] \right) \diff s, \] 

	тобто залишилося знайти $\Theta$. Знайдемо її з системи \[ \frac{\diff \Theta(t, s)}{\diff t} = A(t) \cdot \Theta(t, s) = \Theta(t, s). \]

	Нескладно бачити, що $\Theta(t, s) = e^{t - s}$, тому \begin{multline*} \XX(t, [-1, 1]) = [-e^t, e^t] + \int_0^t [-e^{t - s}, e^{t - s}] \diff s = \\ = [-e^t, e^t] + [1 - e^t, e^t - 1] = [1 - 2 e^t, 2 e^t - 1]. \end{multline*} 
\end{solution}

\begin{problem}
	Знайти опорну функцію множини досяжності для системи керування: \[
	\left\{
		\begin{aligned}
			\frac{\diff x_1}{\diff t} &= 2x_1 + x_2 + u_1, \\
			\frac{\diff x_2}{\diff t} &= 3x_1 + 4x_2 + u_2,
		\end{aligned}
	\right.
	\]

	де $x (0) = (x_{01}, x_{02}) \in \MM_0$, $u(t) = (u_1(t), u_2(t)) \in \UU$, $t \ge 0$, \[ \MM_0 = \{ (x_{01}, x_{02}): |x_{01}| \le 1, |x_{02}| \le 1 \}, \] \[ \UU = \{(u_1, u_2): |u_1| \le 1, |u_2| \le 1\}. \]
\end{problem}

\begin{solution}
	Скористаємося теоремою про вигляд опорної функції множини досяжності лінійної системи керування: \[ c(\XX(t, \MM_0), \psi) = c(\MM_0, \Theta^*(t, t_0) \psi) + \int_{t_0}^t c(\UU(s), B^*(s) \Theta^*(t, s) \psi) \diff s. \]

	Підставимо вже відомі значення: \[ c(\XX(t, [-1,1]^2), \psi) = c([-1,1]^2, \Theta^*(t, 0) \psi) + \int_0^t c([-1,1]^2, \begin{pmatrix} 1 & 1 \end{pmatrix} \Theta^*(t, s) \psi) \diff s, \] 

	тобто залишилося знайти $\Theta$. Знайдемо її з системи \[ \frac{\diff \Theta(t, s)}{\diff t} = A(t) \cdot \Theta(t, s) = \begin{pmatrix} 2 & 1 \\ 3 & 4 \end{pmatrix} \Theta(t, s). \]

	Нескладно бачити, що \[ \Theta(t, s) = \frac{1}{4}
	\begin{pmatrix}
		3 e^{t - s} + e^{5 (t - s)} & - e^{t - s} + e^{5 (t - s)} \\ -3 e^{t - s} + 3 e^{5 (t - s)} & e^{t - s} + 3 e^{5 (t - s)}
	\end{pmatrix} 
	\]

	Тому
	\begin{multline*} 
		c\left(\XX(t, [-1,1]^2), \begin{pmatrix} \psi_1 \\ \psi_2 \end{pmatrix}\right) = c\left([-1,1]^2, \frac{1}{4} \begin{pmatrix} 3 e^t + e^{5 t} & -3 e^t + 3 e^{5 t} \\ - e^t + e^{5 t} & e^t + 3 e^{5 t}	\end{pmatrix} \begin{pmatrix} \psi_1 \\ \psi_2 \end{pmatrix} \right) + \\
		+ \int_0^t c\left([-1,1]^2, \begin{pmatrix} 1 & 0 \\ 0 & 1 \end{pmatrix} \frac{1}{4} \begin{pmatrix} 3 e^{t - s} + e^{5 (t - s)} & -3 e^{t - s} + 3 e^{5 (t - s)} \\ - e^{t - s} + e^{5 (t - s)} & e^{t - s} + 3 e^{5 (t - s)} \end{pmatrix} \begin{pmatrix} \psi_1 \\ \psi_2 \end{pmatrix} \right) \diff s = \\
		= c\left([-1,1]^2, \frac{1}{4} \begin{pmatrix} (3 e^t + e^{5 t}) \psi_1 + (-3 e^t + 3 e^{5 t}) \psi_2 \\ (- e^t + e^{5 t}) \psi_1 + (e^t + 3 e^{5 t}) \psi_2 \end{pmatrix}  \right) + \\
		+ \int_0^t c\left([-1,1]^2, \frac{1}{4} \begin{pmatrix} (3 e^{t - s} + e^{5 (t - s)}) \psi_1 + (-3 e^{t - s} + 3 e^{5 (t - s)}) \psi_2 \\ (- e^{t - s} + e^{5 (t - s)}) \psi_1 + (e^{t - s} + 3 e^{5 (t - s)}) \psi_2 \end{pmatrix} \right) \diff s = \\
		= \frac{1}{4} \left( \left|(3 e^t + e^{5 t}) \psi_1 + (-3 e^t + 3 e^{5 t}) \psi_2\right| + \left|(- e^t + e^{5 t}) \psi_1 + (e^t + 3 e^{5 t}) \psi_2\right| \right) + \\
		+ \frac{1}{4} \int_0^t \left|(3 e^{t - s} + e^{5 (t - s)}) \psi_1 + (-3 e^{t - s} + 3 e^{5 (t - s)}) \psi_2\right| \diff s + \\
		+ \frac{1}{4} \int_0^t \left|(- e^{t - s} + e^{5 (t - s)}) \psi_1 + (e^{t - s} + 3 e^{5 (t - s)}) \psi_2\right| \diff s.
	\end{multline*}
\end{solution} \newpage

% OK, complete

\subsection{Домашнє завдання}

\begin{problem}
    Знайти $A + B$ і $\lambda A$, а також метрику Хаусдорфа $\alpha(A, B)$, якщо
    \begin{enumerate}
        \item $A = \{4,-2,3\}$, $B = \{7,-1,1\}$, $\lambda=2$;
        \item $A = \{5,-5,2\}$, $B = [1,3]$, $\lambda=-1$;
        \item $A = [-4,-2]$, $B = [-1,5]$, $\lambda=3$;
    \end{enumerate}
\end{problem}

\begin{solution}
    \begin{enumerate}
        \item $A = \{4,-2,3\}$, $B = \{7,-1,1\}$, $\lambda=2$;
        \begin{multline*} 
            A + B = \{4+7,4-1,4+1,-2+7,-2-1,-2+1,3+7,3-1,3+1\}= \\
            = \{11,3,5,5,-3,-1,10,2,4\} = \{-3,-1,2,3,4,5,10,11\}.
        \end{multline*} 
        \[ \lambda A = \{2 \cdot 4, 2 \cdot -2, 2 \cdot 3\} = \{8, -4, 6\}. \]
        \begin{multline*} 
            \alpha(A,B) = \max\{\beta(A,B),\beta(B,A)\} = \\
            = \max\{\max\{3,1,2\},\max\{3,1,2\}\}=\max\{3,3\}=3.
        \end{multline*} 
        \item $A = \{5,-5,2\}$, $B = [1,3]$, $\lambda=-1$;
        \begin{multline*} 
            A + B = (5 + [1,3]) \cup (-5 + [1,3]) \cup (2+[1,3])= \\
            = [6,8] \cup [-4,-1] \cup [3,5] = [-4,-1] \cup [3,5] \cup [6,8].
        \end{multline*} 
        \[ \lambda A = \{-1 \cdot 5, -1 \cdot -5, -1 \cdot 2\} = \{-5, 5, -2\}. \]
        \begin{multline*} 
            \alpha(A,B) = \max\{\beta(A,B),\beta(B,A)\} = \\
            = \max\{\max\{2,6,0\},\max_{b\in[1,3]}\{|b-2|\}\}=\max\{6,1\}=6.
        \end{multline*} 
        \item $A = [-4,-2]$, $B = [-1,5]$, $\lambda=3$;
        \[ A + B = [-4-1,-2+5] = [-5,3]. \]
        \[ \lambda A = [3 \cdot -4, 3 \cdot -2] = [-12, -6]. \]
        \begin{multline*} 
            \alpha(A,B) = \max\{\beta(A,B),\beta(B,A)\} = \\
            = \max\{\max\{|-4+1|,|-2+1|\},\max\{|-1+2|,|5+2|\}\}=\max\{3,7\}=7.
        \end{multline*}
    \end{enumerate}
\end{solution}

\begin{problem}
    Знайти $MA$, якщо \[ M = \begin{pmatrix} 2 & 1 \\ -5 & 3 \end{pmatrix}, A = \left\{ \begin{pmatrix} -1 \\ 1 \end{pmatrix}, \begin{pmatrix} 2 \\ -4 \end{pmatrix}, \begin{pmatrix} -3 \\ -2 \end{pmatrix} \right\}. \]
\end{problem}

\begin{solution}
    \begin{multline*} 
        M A = \left\{ M \begin{pmatrix} -1 \\ 1 \end{pmatrix}, M \begin{pmatrix} 2 \\ -4 \end{pmatrix}, M \begin{pmatrix} -3 \\ -2 \end{pmatrix} \right\} = \\
        = \left\{ \begin{pmatrix} -1 \\ 8 \end{pmatrix}, \begin{pmatrix} 0 \\ -22 \end{pmatrix}, \begin{pmatrix} -8 \\ 9 \end{pmatrix} \right\}.
    \end{multline*}        
\end{solution}


\begin{problem}
    Знайти опорні функції таких множин:
    \begin{enumerate}
        \item $A = \{ -1, 1 \}$;
        \item $A = \{ (x_1, x_2, x_3) : |x_1| \le 2, |x_2| \le  4, |x_3| \le 1 \}$;
        \item $A = \{ a \}$;
        \item $A = \KK_r(a) = \{ x\in \RR^n : \| x - a \| \le r \}$.
    \end{enumerate}
\end{problem}

\begin{solution}
    \begin{enumerate}
        \item За визначенням, $c(A, \psi) = \Max_{x \in \{-1, 1\}} \langle x, \psi\rangle = \max(-\psi,\psi) = |\psi|$.
        \item За визначенням, $c(A, \psi) = \Max_{\substack{ x_1 : |x_1| \le 2 \\ x_2 : |x_2| \le  4 \\ x_3 : |x_3| \le 1 }} x_1 \psi_1 + x_2 \psi_2 + x_3 \psi_3  = 2 |\psi_1| + 4 |\psi_2| + |\psi_3|$.
        \item За визначенням, $c(A, \psi) = \Max_{x \in \{ a \}} \langle x, \psi\rangle = \langle a, \psi\rangle $.
        \item За визначенням, 
        \begin{align*}
            c(A, \psi) &= \Max_{x \in \RR^n : \| x - a \| \le r} \langle x, \psi\rangle = \Max_{y \in \RR^n : \| y \| \le r} \langle a + y, \psi\rangle = \\
            &= \langle a, \psi\rangle + \Max_{y \in \RR^n : \| y \| \le r} \langle y, \psi\rangle = \langle a, \psi\rangle + c(\KK_r(0), \psi) = \langle a, \psi \rangle + r\|\psi\|.
        \end{align*}
    \end{enumerate}
\end{solution}

\begin{problem}
    Знайти інтеграл Аумана $\JJ = \int_0^{\pi/2} F(x) dx$ таких багатозначних відображень:
    \begin{enumerate}
        \item $F(x) = [0, \sin x]$, $x \in [0, \pi / 2]$.
        \item $F(x) = [-\sin x, \sin x]$, $x \in [0, \pi / 2]$.
        \item $F(x) = \KK_{\sin x}(0) = \{ y \in \RR^n : \| y \| \le \sin x \}$, $x \in [0, \pi / 2]$.
    \end{enumerate}
\end{problem}

\begin{solution}
    Скористаємося теоремою про зміну порядку інтегрування і взяття опорної функції:
    \begin{enumerate}
        \item $c(\JJ, \psi) = \int_0^{\pi/2} c([0, \sin x], \psi) dx = \int_0^{\pi/2} \max(0, \psi) \sin x dx = \max(0, \psi)$, звідки $\JJ = [0, 1]$.
        \item $c(\JJ, \psi) = \int_0^{\pi/2} c([-\sin x, \sin x], \psi) dx = \int_0^{\pi/2} |\psi| \sin x dx = |\psi|$, звідки $\JJ = [-1, 1]$.
        \item $c(\JJ, \psi) = \int_0^{\pi/2} c(\KK_{\sin x}(0), \psi) dx = \int_0^{\pi/2} \sin x \|\psi\| dx = \|\psi\|$, звідки $\JJ = \KK_1(0)$. 
    \end{enumerate}
\end{solution}

\begin{problem}
    Знайти множину досяжності такої системи керування:
    \[\frac{\diff x}{\diff t} = x + bu,\] 
    де $x(0) = x_0 \in \MM_0$, $u(t)\in \UU$, $t\ge0$, $b$ -- деяке ненульове число, 
    \[ \MM_0 = \{ x : | x | \le 2 \}, \]
    \[ \UU = \{ u : |u| \le 3 \}. \]
\end{problem}

\begin{solution}
    Множину досяжності знайдемо через її опорну функцію: 
    \[ c(\XX(t, \MM_0), \psi) = c(\MM_0, \Theta^*(t, t_0) \psi) + \int_{t_0}^t c(\UU(s), C^\star(s) \Theta^*(t, s)\psi) \diff s. \]
    Для цього послідовно знаходимо: \\
    
    $\Theta(t, s) = e^{t-s}$, знайдено із рівності $\dfrac{d\Theta(t,s)}{dt} = A(t)\Theta(t,s) = \Theta(t,s)$ у нашому випадку. \\
    
    $c(\MM_0, \psi) = c([-2, 2], \psi) = 2 |\psi|$, вже достатньо відома нам опорна функція. \\
    
    $c(\UU(s), \psi) = c([-3, 3], \psi) = 3 |\psi|$, ще одна вже достатньо відома нам опорна функція. \\
    
    % Послідовно пісдтавляючи знайдені вирази в формулу вище знаходимо:
    % \begin{equation*}
    % \begin{split}
    %     c(X(t, \MM_0), \psi) &= c(\MM_0, \Theta^\star(t, t_0), \psi) + \int_{t_0}^t c(\UU(s), C^\star(s) \Theta^\star(t, s)\psi) ds = \\
    %     &= c([-2,2], \Theta^\star(t, 0), \psi) + \int_0^t c([-3, 3], b \Theta^\star(t, s)\psi) ds = \\
    %     &= 2\left|\Theta^\star(t, 0)\psi\right| + \int_0^t 3\left|b \Theta^\star(t, s)\psi\right| ds = \\
    %     &= 2\left|e^{-t}\psi\right| + \int_0^t 3\left|b e^{s-t}\psi\right| ds = 2e^{-t}|\psi| + 3|b \psi| \int_0^t e^{s-t} ds = \\
    %     &= 2e^{-t}|\psi| + 3|b \psi| \left(1 - e^{-t}\right) = \left(2e^{-t} + 3|b|\left(1 - e^{-t}\right)\right) |\psi|,
    % \end{split}
    % \end{equation*}
    % звідки $X(t, \MM_0) = \left[-2e^{-t} - 3|b|\left(1 - e^{-t}\right), 2e^{-t} + 3|b|\left(1 - e^{-t}\right)\right]$.
    
    
    Послідовно пісдтавляючи знайдені вирази в формулу вище знаходимо:
    \begin{equation*}
    \begin{split}
        c(X(t, \MM_0), \psi) &= c(\MM_0, \Theta^\star(t, t_0), \psi) + \int_{t_0}^t c(\UU(s), C^\star(s) \Theta^\star(t, s)\psi) ds = \\
        &= c([-2,2], \Theta^\star(t, 0), \psi) + \int_0^t c([-3, 3], b \Theta^\star(t, s)\psi) ds = \\
        &= 2\left|\Theta^\star(t, 0)\psi\right| + \int_0^t 3\left|b \Theta^\star(t, s)\psi\right| ds = \\
        &= 2\left|e^t\psi\right| + \int_0^t 3\left|b e^{t-s}\psi\right| ds = 2e^t|\psi| + 3|b \psi| \int_0^t e^{t-s} ds = \\
        &= 2e^t|\psi| + 3|b \psi| \left(e^t - 1\right) = \left(2e^t + 3|b|\left(e^t - 1\right)\right) |\psi|,
    \end{split}
    \end{equation*}
    звідки $\XX(t, \MM_0) = \left[-2e^t - 3|b|\left(e^t - 1\right), 2e^t + 3|b|\left(e^t - 1\right)\right]$.
\end{solution}

\begin{problem}
Знайти опорну функцію множини досяжності для системи керування:
\begin{equation*}
    \left\{
    \begin{aligned}
    \dfrac{dx_1}{dt} &= x_1 - x_2 + 2u_1, \\
    \dfrac{dx_2}{dt} &= -4x_1 + x_2 + u_2,
    \end{aligned}
    \right.
\end{equation*}
де $x(0) = (x_{01}, x_{02}) \in \mathcal{M}_0$, $u(t) = (u_1(t), u_2(t)) \in\mathcal{U}$, $t\ge0$,
\begin{align*}
    \mathcal{M}_0 &= \{(x_{01},x_{02}): x_{01}^2 + x_{02}^2 \le 4\}, \\
    \mathcal{U} &= \{(u_1, u_2): u_1^2 + u_2^2 \le 1\}.
\end{align*}
\end{problem}

\begin{solution}
    Одразу помітимо, що $C=\begin{pmatrix}2&0\\0&1\end{pmatrix}$.\\

    $\Theta(t,s)$ знайдемо розв'язавши однорідну систему:
    \begin{equation*}
        \left\{
        \begin{aligned}
        \dfrac{dx_1}{dt} &= x_1 - x_2, \\
        \dfrac{dx_2}{dt} &= -4x_1 + x_2,
        \end{aligned}
        \right.
    \end{equation*}
    
    Її визначник $\begin{vmatrix} 1 - \lambda & - 1 \\ - 4 & 1 - \lambda \end{vmatrix} = (1 - \lambda)^2 - 4 = (\lambda + 1) (\lambda - 3) = 0$, звідки $\lambda_1 = -1$, $\lambda_2 = 3$. \\
    
    Підставляючи знайдені числа у систему, знаходимо власні вектори: $\begin{pmatrix} 1 \\ 2 \end{pmatrix}$ та $\begin{pmatrix} 1 \\ -2 \end{pmatrix}$ відповідно. \\

    Отже загальний розв'язок має вигляд \[\begin{pmatrix} x_1 \\ x_2 \end{pmatrix}(t) = c_1 \begin{pmatrix} e^{-t} \\ 2e^{-t} \end{pmatrix} + c_2 \begin{pmatrix} e^{3t} \\ -2e^{3t} \end{pmatrix}\]
    
    Розв'язуючи рівняння
    \[ c_1 \begin{pmatrix} e^{-s} \\ 2e^{-s} \end{pmatrix} + c_2 \begin{pmatrix} e^{3s} \\ -2e^{3s} \end{pmatrix} = \begin{pmatrix} 1 \\ 0 \end{pmatrix} \]
    і
    \[ c_1 \begin{pmatrix} e^{-s} \\ 2e^{-s} \end{pmatrix} + c_2 \begin{pmatrix} e^{3s} \\ -2e^{3s} \end{pmatrix} = \begin{pmatrix} 0 \\ 1 \end{pmatrix}, \]
    знаходимо фундаментальну матрицю системи, нормовану за моментом $s$, а саме 
    \[ \Theta(t,s) = \begin{pmatrix} \dfrac{e^{s-t} + e^{3(t-s)}}{2} & \dfrac{e^{s-t} - e^{3(t-s)}}{4} \\ e^{s-t} - e^{3(t-s)} & \dfrac{e^{s-t} + e^{3(t-s)}}{2} \end{pmatrix} \]
    
    
    Далі знаходимо $c(\mathcal{M}_0, \psi) = c(\mathcal{K}_2(0), \psi) = 2\|\psi\|$, та $c(\mathcal{U}, \psi) = c(\mathcal{K}_1(0), \psi) = \|\psi\|$, вже достатньо відомі нам опорні функції. \\
    
    Нарешті, можемо зібрати це все докупи: 
    \begin{align*}
        c(\mathcal{X}(t, \mathcal{M}_0), \psi) &= c(\mathcal{M}_0, \Theta^\star(t, 0) \psi) + \int_{0}^t c(\mathcal{U}(s), C^\star(s) \Theta^\star(t, s)\psi) ds = \\
        \\
        &= 2 \|\Theta^\star(t, 0) \psi\| + \int_{0}^t \left\|C^\star(s) \Theta^\star(t, s)\psi\right\| ds = \\
        \\
        &= 2 \left\|\begin{pmatrix} \dfrac{e^{-t} + e^{3t}}{2} & e^{3t} - e^{-t} \\ \dfrac{e^{3t} - e^{-t}}{4} & \dfrac{e^{-t} + e^{3t}}{2} \end{pmatrix} \begin{pmatrix} \psi_1 \\ \psi_2 \end{pmatrix}\right\| + \\
        \\
        &+ \int_{0}^t \left\|\begin{pmatrix} e^{s-t} + e^{3(t-s)} & 2(e^{3(t-s)} - e^{s-t}) \\ \dfrac{e^{3(t-s)} - e^{s-t}}{4} & \dfrac{e^{s-t} + e^{3(t-s)}}{2} \end{pmatrix} \begin{pmatrix} \psi_1 \\ \psi_2 \end{pmatrix}\right\| ds = \\
        \\
        &= 2 \left\| \begin{pmatrix} \dfrac{e^{-t} + e^{3t}}{2} \cdot \psi_1 + (e^{3t} - e^{-t}) \cdot \psi_2 \\ \dfrac{e^{3t} - e^{-t}}{4}\cdot\psi_1 + \dfrac{e^{-t} + e^{3t}}{2}\cdot\psi_2 \end{pmatrix} \right\| + ...
    \end{align*}
\end{solution}  \newpage

\subsection{Аудиторне заняття}

\begin{problem}
	Перевести систему \[ \frac{\diff x}{\diff t} = u, \quad t \in [0, T], \] з точки $x (0) = x_0$ в точку $x (T) = y_0$ за допомогою керування з класу:
	\begin{enumerate}
		\item постійних функцій $u (t) = c$, $c$ -- константа;

		\item кусково-постійних функцій \[ u (t) = \begin{cases} c_1, & t \in [0, t_1], \\ c_2, & t \in [t_1, T]. \end{cases} \]

		Тут $c_1$, $c_2$ -- константи, $c_1 \ne c_2$, $0 < t_1 < T$;

		\item програмних керувань $u(t) = c t$, $c$ -- константа;

		\item керувань з оберненим зв'язком $u(x) = c x$, $c$ -- константа.
	\end{enumerate}
\end{problem}

\begin{solution}
	Скористаємося формулою $x (T) = x (0) + \int_0^T \frac{\diff x}{\diff t} \diff t$:

	\begin{enumerate}
		\item \[x (T) = x (0) + \int_0^T c \diff t = x (0) + c T,\] звідки \[c = \frac{x (T) - x (0)}{T} = \frac{y_0 - x_0}{T};\]

		\item \[ x (T) = x (0) + \int_0^{t_1} c_1 \diff t + \int_{t_1}^T c_2 \diff t = x (0) + c_1 t_1 + c_2 (T - t_1). \] Розв'язок не єдиний, \[ c_2 = \frac{x (T) - x (0) - c_1 t_1}{T - t_1}, \] де $c_1$ -- довільна стала, наприклад $c_1 = 0$, тоді \[ c_2 = \frac{x (T) - x (0)}{T - t_1} = \frac{y_0 - x_0}{T - t_1}. \]

		\item \[ x (T) = x (0) + \int_0^T c t \diff t = x (0) + \frac{cT^2}{2}, \] звідки \[ c = \frac{2 (x (T) - x (0))}{T^2} = \frac{2 (y_0 - x_0)}{T^2}. \]

		\item У цьому випадку проінтегрувати не можна, бо $u$ залежить від $x$, тому просто запишемо за формулою Коші \[ x (T) = x (0) \cdot e^{c T}, \] звідки \[c = \frac{\ln(x (T) / x (0))}{T} = \frac{\ln(y_0) - \ln(x_0)}{T}. \]

		Варто зауважити, не для всіх пар $x_0$ і $y_0$ коректно визначається значення $c$. А саме, необхідно щоб $y_0$ було того ж знаку, що і $x_0$.
	\end{enumerate}
\end{solution}

\begin{problem}
	\begin{enumerate}
		\item Використовуючи означення, знайти грамміан керованості для системи керування \[ \frac{\diff x(t)}{\diff t} = t x (t) + \cos (t) \cdot u(t), \quad t \ge 0. \]

		\item Записати диференціальне рівняння для грамміана керованості і за його допомогою знайти грамміан керованості.

		\item Використовуючи критерій керованості, вказати інтервал повної керованості вказаної системи керування. Для цього інтервала записати керування, яке розв'язує задачу про переведення системи з точки $x_0$ у стан $x_T$.
	\end{enumerate}
\end{problem}

\begin{solution}
	\begin{enumerate}
		\item Скористаємося формулою \[\Phi(T, t_0) = \int_{t_0}^T \Theta(T, s) B(s) B^*(s) \Theta^*(T, s) \diff s.\] $\Theta(T, s)$ знаходимо з системи \[ \frac{\diff \Theta(t,s)}{\diff t} = A(t) \cdot \Theta(t, s) = t \cdot \Theta(t, s),\] а саме $\Theta(t, s) = \exp\left\{\frac{t^2 - s^2}{2}\right\}$. Підставляючи всі знайдені значення, отримаємо \[\Phi(T, t_0) = \cos^2(T) \cdot e^{T^2} \int_{t_0}^T e^{-s^2} \diff s = \frac{1}{2} \sqrt{\pi} \cdot \cos^2(T) \cdot e^{T^2} \cdot \erf(T).\]

		\item Запишемо систему \[ \frac{\diff \Phi (t, t_0)}{\diff t} = A(t) \cdot \Phi(t, t_0) + \Phi(t, t_0) \cdot A^*(t) + B(t) \cdot B^*(t), \quad \Phi(t_0, t_0) = 0. \]

		І підставимо відомі значення: \[ \frac{\diff \Phi (t, 0)}{\diff t} = 2 t \Phi(t, 0)  + \cos^2(t), \quad \Phi(0, 0) = 0. \]

		Звідси \[ \Phi(t, 0) = \frac{1}{8} \sqrt{\pi} e^{t^2 - 1} ( - 2 e \erf(t) + i (\erfi(1 + i t) - i \erfi(1 - i t)), \]

		а \[ \Phi(T, 0) = \frac{1}{8} \sqrt{\pi} e^{T^2 - 1} ( - 2 e \erf(T) + i (\erfi(1 + i T) - i \erfi(1 - i T)), \]

		\item З вигляду грамміану керованості отриманого у першому пункті очевидно, що система цілком керована на півінтервалі $[0, \pi / 2)$, зокрема на інтервалі $[0, 1]$. \\

		Підставимо тепер граміан у формулу для керування що розв'язує задачу про переведення системи із стану $x_0$ у стан $x_T$:
		\begin{multline*} u(t) = B^*(t) \Theta^*(T, t) \Phi^{-1}(T, t_0) (x_T - \Theta(T, t_0) x_0) = \\ = \cos(t) \cdot \exp\left\{\frac{T^2-t^2}{2}\right\} \Phi^{-1} (T, 0) \left(x_T - \exp\left\{\frac{T^2}{2}\right\} x_0\right) \end{multline*}
	\end{enumerate}
\end{solution}

\begin{problem}
	За допомогою грамміана керованості розв'язати таку задачу оптимального керування: мінімізувати критерій якості \[ \JJ (u) = \int_0^T u^2(s) \diff s\] за умов, що \[ \frac{\diff x(t)}{\diff t} = \sin(t) \cdot x(t) + u(t), \quad x (0) = x_0, x (T) = x_T. \]

	Тут $x$ -- стан системи, $u(t)$ -- скалярне керування, $x_0$, $X_T$ -- задані точки, $t \in [t_0, T]$.
\end{problem}

\begin{solution}
	Знайдемо шукане керування за формулою \[ u(t) = B^*(t) \Theta^*(T, t) \Phi^{-1}(T, t_0) (x_T - \Theta(T, t_0) x_0). \]

	У цій задачі $\Theta(t, s) = e^{\cos (s) - \cos (t)}$, знайдене з системи $\dot \Theta = A \Theta$, $\Phi(T, t_0) = e^{-2 \cos (T)} \int_0^T e^{2 \cos (s)} \diff s$, підставляючи знаходимо \[ u(t) = \frac{e^{\cos (t) + \cos (T)} \cdot (x_T - e^{1 - \cos (T)} x_0)}{\int_0^T e^{2 \cos (s)} \diff s}. \]
\end{solution}

\begin{problem}
    За допомогою грамміана керованості розв'язати таку задачу оптимального керування: мінімізувати критерій якості
    \[ \mathcal{J}(u) = \int_0^T u^2(s) dx \]
    за умов, що
    \[ \dfrac{d^2x(t)}{dt^2} - 5\dfrac{dx(t)}{dt} + 6x(t) = u(t), \]
    \[ x(0) = x_0, x'(0) = y_0, x(T) = x'(T) = 0.\]
    Тут $x$ -- стан системми, $u(t)$ -- скалярне керування, $t \in [0, T]$.
\end{problem}

\begin{solution}
    Почнемо з того що зведемо рівняння другого порядку до системи рівнянь заміною $x_1 = x$, $x_2 = \dot x_1$, тоді маємо систему
    \[ \begin{pmatrix} \dot x_1 \\ \dot x_2 \end{pmatrix} (t) = \begin{pmatrix} 0 & 1 \\ -6 & 5 \end{pmatrix} \begin{pmatrix} x_1 \\ x_2 \end{pmatrix} (t) + \begin{pmatrix} 0 \\ 1 \end{pmatrix} u(t). \]
    
    Знайдемо власні числа матриці $A - \lambda E$: $\det(A - \lambda E) = \begin{vmatrix} -\lambda & 1 \\ -6 & 5-\lambda \end{vmatrix} = \lambda^2 - 5\lambda + 6 = (\lambda - 2) (\lambda - 3) = 0$, звідки $\lambda_1 = 2$, $\lambda_2 = 3$. Знайдемо власні вектори, вони будуть $\begin{pmatrix} 1 \\ 2 \end{pmatrix}$ і $\begin{pmatrix} 1 \\ 3 \end{pmatrix}$ відповідно. Звідси знаходимо загальний розв'язок
    \[ \begin{pmatrix} x_1 \\ x_2 \end{pmatrix} (t) = c_1 \begin{pmatrix} e^{2t} \\ 2e^{2t} \end{pmatrix} + c_2 \begin{pmatrix} e^{3t} \\ 3e^{3t} \end{pmatrix}. \]
    
    З рівняння
    \[c_1 \begin{pmatrix} e^{2s} \\ 2e^{2s} \end{pmatrix} + c_2 \begin{pmatrix} e^{3s} \\ 3e^{3s} \end{pmatrix} = \begin{pmatrix} 1 \\ 0 \end{pmatrix} \]
    знаходимо $c_1 = 3e^{-2s}$, $c_2 = -2e^{-3s}$, а з рівняння
    \[c_1 \begin{pmatrix} e^{2s} \\ 2e^{2s} \end{pmatrix} + c_2 \begin{pmatrix} e^{3s} \\ 3e^{3s} \end{pmatrix} = \begin{pmatrix} 0 \\ 1 \end{pmatrix} \]
    знаходимо $c_1 = -e^{-2s}$, $c_2 = e^{-3s}$, тобто
    \[ \Theta(t, s) = \begin{pmatrix} 3e^{2(t-s)} - 2e^{3(t-s)} & -e^{2(t-s)} + e^{3(t-s)} \\ 6e^{2(t-s)} - 6e^{3(t-s)} & -2e^{2(t-s)} + 3e^{3(t-s)} \end{pmatrix}. \]
    
    Знайдемо грамміан за формулою \[\Phi(T, 0) = \int_0^T \Theta(T, s) B(s) B^* (s) \Theta^*(T, s) ds. \]
    
    \[ \Theta(T, s) B(s) = \begin{pmatrix} -e^{2(T - s)} + e^{3(T - s)} \\ -2e^{2(T - s)} + 3e^{3(T - s)} \end{pmatrix}. \]
    \[ B^* (s) \Theta^*(T, s)  =  (\Theta(T, s) B(s))^\star = \begin{pmatrix} -e^{2(T - s)} + e^{3(T - s)} & -2e^{2(T - s)} + 3e^{3(T - s)} \end{pmatrix}. \]
    
    \begin{align*} 
        \Phi(T, 0) &= \int_0^T \begin{pmatrix} -e^{2(T - s)} + e^{3(T - s)} \\ -2e^{2(T - s)} + 3e^{3(T - s)} \end{pmatrix} \begin{pmatrix} -e^{2(T - s)} + e^{3(T - s)} & -2e^{2(T - s)} + 3e^{3(T - s)} \end{pmatrix} ds = \\
        &= \int_0^T \begin{pmatrix} e^{4(T - s)} - 2 e^{5(T - s)} + e^{6(T - s)} & 2 e^{4(T - s)} - 5 e^{5(T - s)} + 3 e^{6(T - s)} \\ 2e^{4(T - s)} - 5 e^{5(T - s)} + 3 e^{6(T - s)} & 4 e^{4(T - s)} - 12 e^{5(T - s)} + 9  e^{6(T - s)} \end{pmatrix} ds = \\
        &= \begin{pmatrix} \dfrac{e^{4T} - 1}{4} - \dfrac{2(e^{5T} - 1)}{5} + \dfrac{e^{6T} - 1}{6} & \dfrac{e^{4T} - 1}{2} - (e^{5T} - 1) + \dfrac{e^{6T} - 1}{2} \\ \\ \dfrac{e^{4T} - 1}{2} - (e^{5T} - 1) + \dfrac{e^{6T} - 1}{2} & (e^{4T} - 1) - \dfrac{12(e^{5T} - 1)}{5} + \dfrac{3(e^{6T} - 1)}{2} \end{pmatrix}
    \end{align*}
    
    Чесно кажучи вже обчислення визначника грамміану є надто складною обчислювальною задачею, не бачу сенсу її робити вручну.
\end{solution}

\begin{problem}
	Записати систему диференціальних рівнянь для знаходження першої матриці керованості (грамміана керованості) і сформулювати критерій керованості на інтервалі $[0, T]$ у випадку, якщо система керування має вигляд:
	\begin{enumerate}
		\item \[ 
		\left\{
			\begin{aligned}
				\frac{\diff x_1(t)}{\diff t} = t x_1 (t) + x_2 (t) + u_1 (t), \\
				\frac{\diff x_2(t)}{\diff t} = - x_1 (t) + 2 x_2 (t) + t^2 u_2 (t).
			\end{aligned}
		\right.
		\]

		Тут $x = (x_1, x_2)^*$ -- вектор стану, $u = (u_1, u_2)^*$ -- вектор керування, $t \in [0, T]$.

		\item \[ \frac{\diff^2 x(t)}{\diff t^2} + \sin(t) \cdot x(t) = u(t). \]

		Тут $x$ -- стан системи, $u(t)$ -- скалярне керування, $t \in [0, T]$.
	\end{enumerate}
\end{problem}

\begin{solution}
	\begin{enumerate}
		\item $A = \begin{pmatrix} t & 1 \\ -1 & 2 \end{pmatrix}$, $B = \begin{pmatrix} 1 & 0 \\ 0 & t^2 \end{pmatrix}$, \[
		\left\{
			\begin{aligned}
				\dot \phi_{11} &= 2 t \phi_{11} + 2 \phi_{12} + 1, \\
				\dot \phi_{12} &= - \phi_{11} + (t + 2) \phi_{12} + \phi_{22}, \\
				\dot \phi_{21} &= \ldots 
			\end{aligned}
		\right.
		\]

		\item Введемо нову змінну $x_2 = \dot x$, тоді $A = \begin{pmatrix} 0 & 1 \\ -\sin(t) & 0 \end{pmatrix}$, $B = \begin{pmatrix} 0 \\ 1 \end{pmatrix}$, \[
		\left\{
			\begin{aligned}
				\dot \phi_{11} &= 2 \phi_{12}, \\
				\dot \phi_{12} &= (1 - \sin(t)) \phi_{11}, \\
				\ldots 
			\end{aligned}
		\right.
		\]
	\end{enumerate}
\end{solution}

\begin{problem}
 	Знайти диференціальне рівняння грамміана керованості для системи керування \[ \left\{ \begin{aligned}
 		\frac{\diff x_1 (t)}{\diff t} &= \cos(t) \cdot x_1(t)-\sin(t) \cdot x_2(t) + u_1(t) - 2 u_2(t), \\
 		\frac{\diff x_2 (t)}{\diff t} &= \sin(t) \cdot x_1(t)+\cos(t) \cdot x_2(t) - 3 u_1(t) + 4 u_2(t).
 	\end{aligned} \right. \]
\end{problem}

\begin{solution}
 	% 3.6
\end{solution}


\begin{problem}
    Дослідити системи на керованість. використовуючи другий критерій керованості:
    \begin{enumerate}
        \item \[\ddot x + a \dot x + b x = u; \]
        \item \[ \left\{ \begin{aligned} \dot x_1 &= 2x_1 + x_2 + au \\ \dot x_2 &= x_1 + 4 x_2 + u \end{aligned} \right. \]
        \item \[ \left\{ \begin{aligned} \dot x_1 &= 2x_1 + x_2 + u_1 \\ \dot x_2 &= x_1 + 3 x_3 + u_2 \\ \dot x_3 &= x_2 + x_3 + u_2  \end{aligned} \right. \]
    \end{enumerate}
\end{problem}

\begin{solution}
    \begin{enumerate}
        \item Почнемо з того що зведемо рівняння другого порядку до системи рівнянь заміною $x_1 = x$, $x_2 = \dot x_1$, тоді маємо систему
        \[ \left\{ \begin{aligned} \dot x_1 &= x_2 \\ \dot x_2 &= - a x_2 - b x_1 + u  \end{aligned} \right. \]
        Тоді
        \[ A = \begin{pmatrix} 0 & 1 \\ -b & -a \end{pmatrix} \qquad B = \begin{pmatrix} 0 \\ 1 \end{pmatrix}. \]
        \[ D = \begin{pmatrix} B & AB \end{pmatrix} = \begin{pmatrix} 0 & 1 \\ 1 & - a \end{pmatrix}. \]
        Її ранг дорівнює 2 якщо за будь-яких $a$ і $b$, тобто система завжди цілком керована.
        \item 
        \[ A = \begin{pmatrix} 2 & 1 \\ 1 & 4 \end{pmatrix} \qquad B = \begin{pmatrix} a \\ 1 \end{pmatrix}. \]
        \[ D = \begin{pmatrix} B & AB \end{pmatrix} = \begin{pmatrix} a & 2 a + 1 \\ 1 & a + 4 \end{pmatrix}. \]
        Її визначник $a^2 + 4a - 2a - 1 = a^2 + 2a - 1 = 0$ якщо $a = -1 \pm \sqrt 2$, тоді система не є цілком керованою, а інакше є.
        \item 
        \[ A = \begin{pmatrix} 2 & 1 & 0 \\ 1 & 0 & 3 \\ 0 & 1 & 1 \end{pmatrix} \qquad B = \begin{pmatrix} 1 & 0 \\ 0 & 1 \\ 0 & 1  \end{pmatrix}. \]
        \[ D = \begin{pmatrix} B & AB & A^2B \end{pmatrix} = \begin{pmatrix} 1 & 0 & 2 & 1 & 5 & 5 \\ 0 & 1 & 1 & 3 & 2 & 7 \\ 0 & 1 & 0 & 2 & 1 & 5 \end{pmatrix} .\]
        Її ранг дорівнює 3, тобто система цілком керована.
    \end{enumerate}    
\end{solution} \newpage

% OK, incomplete, missing 3.10, 3.12

\subsection{Домашнє завдання}

\begin{problem}
    Перевести систему 
    \[ \dfrac{dx}{dt} = 2tx + u, t \in [0, T], \]
    з точки $x(0) = x_0$ в точку $x(T) = y_0$ за допомогою керування з класу:
    \begin{enumerate}
        \item постійних функцій $u(t) = c$, $c$ -- константа; 
        \item кусково-постійних функцій 
        \[
        u(t) = \begin{cases}
            c_1 & t \in [0, t_1), \\
            c_2 & t \in (t_1, T].
        \end{cases}
        \]
        Тут $c_1$, $c_2$ -- константи, $c_1 \ne c_2$, $0 < t_1 < T$;
        \item програмних керувань $u(t) = ct$, $c$ -- константа;
        \item керувань з оберненим зв'язок $u(x) = cx$, $c$ -- константа.
    \end{enumerate}
\end{problem}

\begin{solution}
Будемо просто підставляти керування у диференційне рівняння і розв'язувати його:
\begin{enumerate}
\item Зводимо до канонічного вигляду лінійного рівняння:
\[ \dfrac{dx}{dt} - 2t \cdot x(t) = c. \]
Домножаємо на множник що інтегрує:
\begin{align*}
    \exp\{-t^2\} \cdot \dfrac{dx}{dt} - 2 t \cdot \exp\{-t^2\} \cdot x(t) &= c \cdot \exp\{-t^2\} \\
    \\
    \exp\{-t^2\} \cdot \dfrac{dx}{dt} + x(t) \cdot \dfrac {d \exp\{-t^2\}} {dt} &= c \cdot \exp\{-t^2\}.
\end{align*}
Згортаємо похідну добутку:
\[ \dfrac {d (\exp\{-t^2\} \cdot x(t))} {dt} = c \cdot \exp\{-t^2\}. \]
Інтегруємо:
\begin{align*}
    \left.(\exp\{-t^2\} \cdot x(t))\right|_0^T &= \int_0^T c \cdot \exp\{-t^2\} dt \\
    \\
    \exp\{-T^2\} \cdot y_0 - x_0 &= c \cdot \dfrac {\sqrt \pi} 2 \cdot \erf (T),
\end{align*}
і виражаємо звідси $c$:
\[ c = 2 \cdot \dfrac{\exp\{-T^2\} \cdot y_0 - x_0} {\sqrt \pi \cdot \erf (T)}, \]
де $\erf$ позначає функцію помилок, тобто $\erf(x) = \dfrac 2 {\sqrt \pi} \cdot \int_0^x \exp\{-t^2\} dt$.\\

Зауважимо, що задача має розв'язок завжди.
\item Нескладно зрозуміти, що нас задовольнить довільне керування вигляду
\[ c_2 = 2 \cdot \dfrac{\exp\{-T^2\} \cdot y_0 - \exp\{-t_1^2\} \cdot x_1} {\sqrt \pi \cdot (\erf (T) - \erf (t_1))},\] де \[ x_1 = \dfrac{2x_0 + c_1\sqrt \pi \cdot \erf (T)}{2\cdot \exp\{-t_1^2\}}, \]
тобто  ми просто дозволили $c_1$ бути довільною сталою, обчислили $x(t_1)$, а потім розв'язали задачу переведення системи з точки $(t_1, x_1)$ у точку $(T, y_0)$ як у першому пункті, з мінімальними поправками на межі інтегрування. \\

Зокрема, якщо $c_1 = 0$, то $x_1 = \dfrac {x_0} {\exp\{-t_1^2\}}$, тому $c_2 = 2 \cdot \dfrac{\exp\{-T^2\} \cdot y_0 - x_0} {\sqrt \pi \cdot (\erf (T) - \erf (t_1))}$.\\

Зауважимо, що задача має розв'язок завжди.
\item 
\begin{align*}
    \dfrac{dx}{2x(t) + c} &= t dt \\
    \\
    \int_0^T \dfrac{dx}{2x(t) + c} &= \int_0^T t dt \\
    \\
    \left.\left(\dfrac 12 \ln(2x(t) + c)\right)\right|_0^T &= \dfrac {T^2} 2 \\
    \\
    \ln(2y_0 + c) - \ln(2x_0 + c) &= T^2 \\
    \\
    \ln\left(\dfrac{2y_0 + c}{2x_0 + c}\right) &= T^2 \\
    \\
    \dfrac{2y_0 + c}{2x_0 + c} &= \exp\{T^2\} \\
    \\
    2y_0 + c &= (2x_0 + c)\cdot \exp\{T^2\} \\
    \\
    2(y_0 - x_0 \cdot \exp\{T^2\}) &= c \cdot (\exp\{T^2\} - 1) 
\end{align*}
звідки
\[c = 2\cdot \dfrac{y_0 - x_0 \cdot \exp\{T^2\}}{\exp\{T^2\} - 1}. \]
Зауважимо, що задача має розв'язок завжди.
\item 
\begin{align*}
    \dfrac{dx}{dt} &= 2t\cdot x(t) + c\cdot x(t) \\
    \\
    \dfrac{dx}{x(t)} &= (2t + c) dt \\
    \\
    \int_0^T \dfrac{dx}{x(t)} &= \int_0^T (2t + c) dt \\
    \\
    (\ln(x(t))|_0^T &= T^2 + cT \\
    \\
    \ln(y_0) - \ln(x_0) &= T^2 + cT \\ 
    \\
    \ln\left(y_0/x_0\right) &= T^2 + cT 
\end{align*}
звідки
\[ c = \dfrac{\ln\left(y_0/x_0\right) - T^2}{T}. \]
Зауважимо, що задача має розв'язок тільки якщо $\signum(x_0) = \signum(y_0)$.
\end{enumerate}
\end{solution} 

\begin{problem}
\begin{enumerate}
    \item Знайти грамміан керованості для системи керування \[\dfrac{dx(t)}{dt} = tx(t) + u(t)\] і дослідити її на керованість, використовуючи перший критерій керованості.
    \item За допомогою грамміана керованості розв'язати таку задачу оптимального керування: \[\mathcal{J}(u) = \int_0^T u^2(s) ds \to \min\]
    за умов, що \[\dfrac{dx(t)}{dt} = tx(t) + u(t), x(0) = x_0, x(T) = x_T. \]
    Тут $x$ -- стан системи. $u(t)$ -- скалярне керування, $x_0$, $x_T$ -- задані точки, $t\in[0,T]$.
\end{enumerate}
\end{problem}

\begin{solution}
\begin{enumerate}
    \item Одразу помітимо, що $A(t) = (t)$, $B(t) = (1)$. Далі, з рівняння $\dfrac{d\Theta(t,s)}{dt} = A(t) \cdot \Theta(t,s)$ знаходимо $\Theta(t,s) = \exp\{t^2 / 2 - s^2 / 2\}$. Залишилося всього нічого, знайти власне грамміан:
    \begin{multline*} \Phi(T,0) = \int_0^T \Theta(T,s)B(s)B^*(s),\Theta^*(T,s) ds = \\ = \int_0^T (\exp\{T^2 - s^2\}) ds = \begin{pmatrix}\dfrac{\sqrt \pi}{2} \cdot \exp\{T^2\} \cdot \erf(T) \end{pmatrix}, \end{multline*} і $\det\Phi(T,0)\ne0$, тобто система цілком керована на $[0, T]$.
    \item Пригадаємо наступний результат: розв'язком вищезгаданої задачі про оптимальне керування є функція
    \begin{align*}
        u(t) &= B^*(t) \Theta^*(T,t)\Phi^{-1}(T,0)(x_T-\Theta(T,0) x_0) = \\
        \\
        &= \exp\{T^2 / 2 - t^2 / 2\} \begin{pmatrix}\dfrac2{\sqrt \pi} \cdot \exp\{-T^2\} \cdot \dfrac1{\erf(T)} \end{pmatrix} (x_T - \exp\{T^2 / 2\} x_0) = \\
        \\
        &= \dfrac{2}{\sqrt{\pi}\cdot \erf(T)} \cdot \left(x_T \cdot \exp\left\{-\dfrac{T^2+t^2}2\right\} - x_0 \cdot \exp\left\{- \dfrac{t^2}2\right\}\right).
    \end{align*} 
\end{enumerate}
\end{solution} 

\begin{problem}
    За допомогою грамміана керованості розв'язати таку задачу оптимального керування: мінімізувати критерій якості \[ \JJ (u) = \int_0^T u^2(s) \diff s \] за умов, що \[ \frac{\diff^2 x(t)}{\diff t^2} - 5 \frac{\diff x(t)}{\diff t} + 6 x(t) = u(t), \] \[ x(0) = x_0, x'(0) = y_0,x(T)=x'(T)=0.\] Тут $x$ -- стан системи, $u(t)$ -- скалярне керування, $t\in[0,T]$.
\end{problem}

\begin{solution}
    % 3.10
\end{solution}

\begin{problem}
    Мінімізувати критерій якості 
    \[ \mathcal{J}(u) = \int_0^T (u_1^1(s) + u_2^2(s)) ds \]
    за умов \[ \left\{ \begin{aligned} \dfrac{dx_1(t)}{dt} = 6x_1(t) - 2x_2(t) + u_1(t), \\ \dfrac{dx_2(t)}{dt} = 5x_1(t) - x_2(t) + u_2(t), \end{aligned} \right. \]
    \[ x_1(0) = x_{10}, x_2(0) = x_{20}, \] 
    \[ x_1(T) = x_2(T) = 0. \]
    Тут $x = (x_1, x_2)^\star$ -- вектор фазових координат з $\RR^2$, $u=(u_1,u_2)^\star$ -- вектор керування, $x = (x_{10}, x_{20})^\star$ -- відома точка, $t \in [0, T]$.
\end{problem}

\begin{solution}
    Запишемо систему у людському вигляді:
    \[ \begin{pmatrix} \dot x_1 \\ \dot x_2 \end{pmatrix} (t) = \begin{pmatrix} 6 & -2 \\ 5 & -1 \end{pmatrix} \begin{pmatrix} x_1 \\ x_2 \end{pmatrix} (t) + \begin{pmatrix} 1 & 0 \\ 0 & 1 \end{pmatrix} \begin{pmatrix} u_1 \\ u_2 \end{pmatrix} (t) \]
    
    Знайдемо власні числа матриці $A - \lambda E$: $\det(A - \lambda E) = \begin{vmatrix} 6 - \lambda & -2 \\ 5 & -1 - \lambda \end{vmatrix} = \lambda^2 - 5\lambda + 4 = (\lambda - 1) (\lambda - 4) = 0$, звідки $\lambda_1 = 1$, $\lambda_2 = 4$. Знайдемо власні вектори, вони будуть $\begin{pmatrix} 2 \\ 5 \end{pmatrix}$ і $\begin{pmatrix} 1 \\ 1 \end{pmatrix}$ відповідно. Звідси знаходимо загальний розв'язок
    \[ \begin{pmatrix} x_1 \\ x_2 \end{pmatrix} (t) = c_1 \begin{pmatrix} 2e^t \\ 5e^t \end{pmatrix} + c_2 \begin{pmatrix} e^{4t} \\ e^{4t} \end{pmatrix}. \]
    
    З рівняння
    \[ c_1 \begin{pmatrix} 2e^s \\ 5e^s \end{pmatrix} + c_2 \begin{pmatrix} e^{4s} \\ e^{4s} \end{pmatrix} = \begin{pmatrix} 1 \\ 0 \end{pmatrix} \]
    знаходимо $c_1 = -\dfrac13 e^{-s}$, $c_2 = \dfrac53 e^{-4s}$, а з рівняння
    \[ c_1 \begin{pmatrix} 2e^s \\ 5e^s \end{pmatrix} + c_2 \begin{pmatrix} e^{4s} \\ e^{4s} \end{pmatrix} = \begin{pmatrix} 0 \\ 1 \end{pmatrix} \]
    знаходимо $c_1 = \dfrac13 e^{-s}$, $c_2 = -\dfrac23 e^{-4s}$, тобто
    \[ \Theta(t, s) = \dfrac13\begin{pmatrix} -2 e^{t-s} + 5 e^{4(t-s)} & 2 e^{t-s} - 2 e^{4(t-s)} \\ -5 e^{t-s} + 5 e^{4(t-s)} & 5 e^{t-s} - 2 e^{4(t-s)} \end{pmatrix}. \]
    
    Знайдемо грамміан за формулою \[\Phi(T, 0) = \int_0^T \Theta(T, s) B(s) B^* (s) \Theta^*(T, s) ds. \]
    
    \[ \Theta(T, s) B(s) =  \dfrac13\begin{pmatrix} -2 e^{t-s} + 5 e^{4(t-s)} & 2 e^{t-s} - 2 e^{4(t-s)} \\ -5 e^{t-s} + 5 e^{4(t-s)} & 5 e^{t-s} - 2 e^{4(t-s)} \end{pmatrix}. \]
    \[ B^* (s) \Theta^*(T, s)  =  (\Theta(T, s) B(s))^\star =  \dfrac13\begin{pmatrix} -2 e^{t-s} + 5 e^{4(t-s)} & 5 e^{t-s} - 5 e^{4(t-s)} \\ -2 e^{t-s} + 2 e^{4(t-s)} & 5 e^{t-s} - 2 e^{4(t-s)} \end{pmatrix}. \]
    
    Чесно кажучи вже обчислення грамміану є надто складною обчислювальною задачею, не бачу сенсу її робити вручну.
\end{solution}

\begin{problem}
    Записати систему диференціальних рівнянь для знаходження першої матриці керованості (грамміана керованості) і сформулювати критерій керованості на інтервалі $[0,T]$ у випадку, якщо системи керування має вигляд \[ \left\{ \begin{aligned} \frac{\diff x_1(t)}{\diff t} = tx_1(t) + t^2x_2(t) + u_1(t)-u_2(t), \\ \frac{\diff x_2(t)}{\diff t} = -x_1(t) + x_2(t) + 2u_2(t), \end{aligned} \right. \] Тут $x = (x_1, x_2)^\star$ -- вектор фазових координат з $\RR^2$, $u=(u_1,u_2)^\star$ -- вектор керування, $t \in [0, T]$.
\end{problem}

\begin{solution}
    % 3.12
\end{solution}

\begin{problem}
    Записати систему диференціальних рівнянь для знаходження першої матриці керованості (грамміана керованості) і сформулювати критерій керованості на інтервалі $[0, T]$ у випадку, якщо система керування має вигляд:
    \[ \dfrac{d^2x(t)}{dt^2} + tx(t) = u(t). \]
    Тут $x$ -- стан системи, $u(t)$ -- скалярне керування, $t \in [0, T]$.
\end{problem}

\begin{solution}
    Зробимо заміну $x_1 = x$, $x_2 = \dot x$, тоді маємо систему \[ \begin{pmatrix} \dot x_1 \\ \dot x_2 \end{pmatrix} (t) = \begin{pmatrix} 0 & 1 \\ -t & 0 \end{pmatrix} \begin{pmatrix} x_1 \\ x_2 \end{pmatrix} (t) + \begin{pmatrix} 0 \\ 1 \end{pmatrix} \begin{pmatrix} u \end{pmatrix} (t). \]
    Звідси можемо записати систему диференціальних рівнянь для знаходження грамміана керованості:
    \[ \dfrac{\Phi(t, t_0)}{dt} = A(t) \Phi(t, t_0) + \Phi(t, t_0) A^\star(t) + B(t) B^*(t). \]
    \[ \dfrac{\Phi(t, 0)}{dt} = \begin{pmatrix} 0 & 1 \\ -t & 0 \end{pmatrix} \Phi(t, 0) + \Phi(t, 0) \begin{pmatrix} 0 & t \\ -1 & 0 \end{pmatrix} + \begin{pmatrix} 0 & 0 \\ 0 & 1 \end{pmatrix}. \]
    Окрім цього, не забуваємо про умову $\Phi(0, 0) = 0$.\\
    
    Щодо критерію керованості, то тут все просто (чи радше стандартно), для того щоб система була цілком керованою на $[0, T]$ необхідно і достатньо, щоб грамміан керованості $\Phi(T, 0)$ був невиродженим, тобто щоб $\det \Phi(T, 0) \ne 0$ або (що те саме у випадку невід'ємно-визначеної матриці) щоб $\det \Phi(T, 0) > 0$.
\end{solution}

\begin{problem}
    Дослідити на керованість, використовуючи другий критерій керованості:
    \begin{enumerate}
        \item \[ \left\{ \begin{aligned} \dfrac{dx_1}{dt} &= -x_1 + x_2 + au, \\ \dfrac{dx_2}{dt} &= x_1 + \dfrac ua; \end{aligned} \right. \]
        \item \[ \left\{ \begin{aligned} \dfrac{dx_1}{dt} &= x_1 - x_2 + au, \\ \dfrac{dx_2}{dt} &= x_1 + \dfrac ua; \end{aligned} \right. \] 
        \item \[ \left\{ \begin{aligned} \dfrac{dx_1}{dt} &= x_1 + x_2 + au, \\ \dfrac{dx_2}{dt} &= - x_1 + x_2 + a^2u; \end{aligned} \right. \]
        \item \[ \left\{ \begin{aligned} \dfrac{dx_1}{dt} &= 2x_1 + x_2 - au, \\ \dfrac{dx_2}{dt} &= - x_1 + au; \end{aligned} \right. \]
        \item \[ x^{(n)}(t) + a_1 x^{(n-1)}(t) + \ldots + a_{n-1}x'(t) + a_n x(t) = u(t). \]
        \item \[ \left\{ \begin{aligned} \dfrac{dx_1}{dt} &= x_1 + 2x_2 - x_3 + u_1 - u_2 \\ \dfrac{dx_2}{dt} &= -x_1 + x_2 + 3x_3 + u_1 \\ \dfrac{dx_3}{dt} &= x_2 + x_3 + 2u_2 \end{aligned} \right. \]
    \end{enumerate}
\end{problem}

\begin{solution}
    \begin{enumerate}
        \item \[ A = \begin{pmatrix} -1 & 1 \\ 1 & 0 \end{pmatrix} \qquad B = \begin{pmatrix} a \\ 1 / a \end{pmatrix} \]
        \[ D = \begin{pmatrix} B & AB \end{pmatrix} = \begin{pmatrix} a & 1 / a - a \\ 1 / a & a \end{pmatrix} \]
        \[ \det D = a^2 + 1 - 1/a^2 \ne 0, \]
        тобто система цілком керована якщо тільки $a \ne \pm \sqrt{\dfrac{\sqrt{5} - 1}{2}}$.
        \item \[ A = \begin{pmatrix} 1 & -1 \\ 1 & 0 \end{pmatrix} \qquad B = \begin{pmatrix} a \\ 1 / a \end{pmatrix} \]
        \[ D = \begin{pmatrix} B & AB \end{pmatrix} = \begin{pmatrix} a & a - 1 / a \\ 1 / a & a \end{pmatrix} \]
        \[ \det D = a^2 - 1 + 1/a^2 \ne 0, \]
        тобто система цілком керована для будь-яких $a$ (навіть $\det D \ge 1$ за нерівністю Коші).
        \item \[ A = \begin{pmatrix} 1 & 1 \\ -1 & 1 \end{pmatrix} \qquad B = \begin{pmatrix} a \\ a^2 \end{pmatrix} \]
        \[ D = \begin{pmatrix} B & AB \end{pmatrix} = \begin{pmatrix} a & a + a^2 \\ a^2 & a^2 - a \end{pmatrix} \]
        \[ \det D = a^3 - a^2 - a^4 - a^3 = -a^4 - a^2 \ne 0, \]
        тобто система цілком керована якщо тільки $a \ne 0$.
        \item \[ A = \begin{pmatrix} 2 & 1 \\ -1 & 0 \end{pmatrix} \qquad B = \begin{pmatrix} -a \\ a \end{pmatrix} \]
        \[ D = \begin{pmatrix} B & AB \end{pmatrix} = \begin{pmatrix} -a & -a \\ a & a \end{pmatrix} \]
        \[ \det D = 0, \]
        тобто система не є цілком керованою для будь-яких $a$.
        \item Зробимо заміну $x_0 = x$, $x_1 = x'$, $\ldots$, $x_n = x^{(n)}$, тоді отримаємо систему
        \[ \left\{ \begin{aligned} \dot x_0 &= x_1 \\ \dot x_1 &= x_2 \\ \cdots \\ \dot x_{n-1} &= x_n \\ \dot x_n &= u - a_n x_0 - a_{n-1} x_1 - \ldots - a_1 x_{n-1} \end{aligned} \right. \]
        тобто
        \[ A = \begin{pmatrix} 0 & 1 & \ddots & 0 & 0 \\ 0 & 0 & \ddots & \ddots & 0 \\ \vdots & \vdots & \ddots & \ddots & \ddots \\ 0 & 0 & \cdots & 0 & 1 \\ -a_n & -a_{n-1} & \cdots & -a_1 & 0 \end{pmatrix} \qquad B = \begin{pmatrix} 0 \\ 0 \\ \vdots \\ 0 \\ 1 \end{pmatrix} \]
        \[ D = \begin{pmatrix} B & AB & A^2B & \cdots & A^nB \end{pmatrix} = \begin{pmatrix} 0 & \cdots & 0 & 0 & 1 \\ \vdots & \reflectbox{$\ddots$} & \reflectbox{$\ddots$} & \reflectbox{$\ddots$} & \cdots \\ 0 & 0 & 1 & 0 & \cdots \\ 0 & 1 & 0 & -a_1 & \cdots \\ 1 & 0 & -a_1 & -a_2 & \cdots \end{pmatrix} \]
        \[ \det D = -1 \ne 0, \] тобто система цілком керована для довільних $a_1$, $a_2$, $\ldots$, $a_n$.
        \item \[ A = \begin{pmatrix} 1 & 2 & -1 \\ -1 & 1 & 3 \\ 0 & 1 & 1 \end{pmatrix} \qquad B = \begin{pmatrix} 1 & -1 \\ 1 & 0 \\ 0 & 2 \end{pmatrix} \]
        \[ D = \begin{pmatrix} B & AB & A^2B  \end{pmatrix} = \begin{pmatrix} 1 & -1 & 3 & -3 & 2 & 8 \\ 1 & 0 & 0 & 7 & 0 & 16 \\ 0 & 2 & 1 & 2 & 1 & 9 \end{pmatrix} \]
        Її ранг дорівнює 3, тобто система цілком керована.
    \end{enumerate}
\end{solution} \newpage

% OK, incomplete, missing 4.6

\section{Критерії спостережуваності. Критерій двоїстості}

\subsection*{Аудиторне заняття}

\begin{problem}
	Побудувати систему для знаходження грамміана спостережуваності і записати умову спостережуваності на інтервалі для системи: \[
	\left\{
		\begin{aligned}
			\dot x_1 &= \cos(t) \cdot x_1 + \sin(t) \cdot x_2, \\
			\dot x_2 &= - \sin(t) \cdot x_1 + \cos(t) \cdot x_2, \\
			y(t) &= k x_1 + x_2,
		\end{aligned}
	\right.
	\]
	$k > 0$.
\end{problem}

\begin{solution}
	Диференціальне рівняння для знаходження грамміана спостережуваності має вигляд \[ \frac{\diff \NN(t, t_0)}{\diff t} = -A(t) \cdot \NN(t, t_0) - \NN(t, t_0) \cdot A^*(t) + H^*(t) \cdot H(t), \] а після підстановки відомих значень вона набує вигляду \begin{multline*} 
		\frac{\diff \NN(t, t_0)}{\diff t} = -\begin{pmatrix} \cos (t) & \sin(t) \\ - \sin(t) & \cos(t) \end{pmatrix} \cdot \NN(t, t_0) - \\
		- \NN(t, t_0) \cdot \begin{pmatrix} \cos (t) & -\sin(t) \\ \sin(t) & \cos(t) \end{pmatrix} + \begin{pmatrix} k \\ 1 \end{pmatrix} \cdot \begin{pmatrix} k & 1 \end{pmatrix}, 
	\end{multline*} або, у розгорнутому вигляді: \[
	\left\{
		\begin{aligned}
			\dot n_{11} (t) &= - 2 \cos(t) \cdot n_{11} (t) - 2 \sin(t) \cdot n_{12} (t) + k^2, \\
			\dot n_{12} (t) &= \sin (t) \cdot n_{11} (t) - 2\cos(t) \cdot n_{12} (t) - \sin (t) \cdot n_{22} (t) + k, \\
			\dot n_{22} (t) &= 2 \sin(t) \cdot n_{12} (t) - 2 \cos(t) \cdot n_{22} (t) + 1.
		\end{aligned}
	\right.
	\]

	Умова спостережуваності цієї системи на $[t_0, T]$ має вигляд  $n_{11} (t) \cdot n_{22} (t) - n_{12}^2 (t) \ne 0$, $t \in [t_0, T]$.
\end{solution}

\begin{problem}
    Чи буде система цілком спостережуваною?

    \begin{enumerate}
    	\item \[ \ddot x = a^2 x, y(t) = x(t); \]

    	\item \[ \left \{ \begin{aligned}
    		\dot x_1 &= x_1 + \alpha x_2, \\
    		\dot x_2 &= \alpha x_1 + x_2, \\
    		y(t) &= \beta x_1 (t) + x_2 (t).
    	\end{aligned} \right. \]

    	\item \[ \left \{ \begin{aligned}
    		\dot x_1 &= a x_1, \\
    		\dot x_2 &= b x_2, \\
    		y(t) &= x_1 (t) + x_2 (t).
    	\end{aligned} \right. \]
    \end{enumerate}
\end{problem}

\begin{solution}
    Всі системи є стаціонарними, тому будемо застосовувати другий критерій спостережуваності: \[ rang \mathcal{R} = rang \left(H^* \vdots A^* H^* \vdots (A^*)^2 H^* \vdots \ldots \vdots (A^*)^{n-1} H^*\right) = n. \]

    \begin{enumerate}
    	\item Введемо нову змінну $x_2 = \dot x_1$, тоді $\dot x_2 = a^2 x_1$, $y = x_1$, тому \[ A = \begin{pmatrix} 0 & 1 \\ a^2 & 0 \end{pmatrix}, \quad H = \begin{pmatrix} 1 & 0 \end{pmatrix}. \] Підставляючи у критерій, знаходимо: \[ \mathcal{R} = \left(\begin{pmatrix} 1 \\ 0 \end{pmatrix} \vdots \begin{pmatrix} 0 & a^2 \\ 1 & 0 \end{pmatrix} \begin{pmatrix} 1 \\ 0 \end{pmatrix}\right) = \begin{pmatrix} 1 & 0 \\ 0 & 1 \end{pmatrix}, \] її ранг 2, тому система цілком спостережувана.

    	\item \[ A = \begin{pmatrix} 1 & \alpha \\ \alpha & 1 \end{pmatrix}, \quad H = \begin{pmatrix} \beta & 1 \end{pmatrix}. \] Підставляючи у критерій, знаходимо: \[ \mathcal{R} = \left(\begin{pmatrix} \beta \\ 1 \end{pmatrix} \vdots \begin{pmatrix} 1 & \alpha \\ \alpha & 1 \end{pmatrix} \begin{pmatrix} \beta \\ 1 \end{pmatrix}\right) = \begin{pmatrix} \beta & \alpha + \beta \\ 1 & \alpha \beta + 1 \end{pmatrix}, \] її ранг 2 тоді і тільки тоді, коли її визначник $det \mathcal{R} = \alpha (\beta^2 - 1) \ne 0$, тому система цілком спостережувана тоді і тільки тоді, коли $\alpha \ne 0$, $\beta \ne \pm 1$.

    	\item \[ A = \begin{pmatrix} a & 0 \\ 0 & b \end{pmatrix}, \quad H = \begin{pmatrix} 1 & 1 \end{pmatrix}. \] Підставляючи у критерій, знаходимо: \[ \mathcal{R} = \left(\begin{pmatrix} 1 \\ 1 \end{pmatrix} \vdots \begin{pmatrix} a & 0 \\ 0 & b \end{pmatrix} \begin{pmatrix} 1 \\ 1 \end{pmatrix}\right) = \begin{pmatrix} 1 & a \\ 1 & b \end{pmatrix}, \] її ранг 2 тоді і тільки тоді, коли її визначник $det \mathcal{R} = b - a \ne 0$, тому система цілком спостережувана тоді і тільки тоді, коли $a \ne b$.
    \end{enumerate}
\end{solution}

\begin{problem}
    Дослідити на спостережуваність, використовуючи критерій двоїстості і відповідний критерій керованості: \[ \left\{ \begin{aligned}
    	\dot x_1 &= x_2 - 2 x_3, \\
    	\dot x_2 &= x_1 - x_3, \\
    	\dot x_3 &= - 2 x_3, \\
    	y(t) &= -x_1 + x_2 - x_3.
    \end{aligned} \right. \]
\end{problem}

\begin{solution}
	За принципом двоїстості Калмана, ця система є цілком спостережуваною на $[t_0, T]$ тоді і тільки тоді, коли система \[ \frac{\diff z(t)}{\diff t} = - A^*(t) \cdot z(t) + H^*(t) \cdot u(t) \] є цілком керованою на $[t_0, T]$. \\

	Підставляючи відомі значення, отримуємо систему \[ \frac{\diff z(t)}{\diff t} = \begin{pmatrix} 0 & -1 & 0 \\ -1 & 0 & 0 \\ 2 & 1 & 2 \end{pmatrix} \cdot z(t) + \begin{pmatrix} -1 \\ 1 \\ -1 \end{pmatrix} \cdot u(t), \] або, у розгорнутому вигляді \[ \left\{ \begin{aligned}
		\dot z_1 &= - z_2 - u, \\
		\dot z_2 &= - z_1 + u, \\
		\dot z_3 &= 2 z_1 + z_2 + 2 z_3 - u.
	\end{aligned} \right. \]

	Система стаціонарна, тому використаємо другий критерій керованості: \[ rang \mathcal{D} = rang \left(B \vdots AB \vdots A^2 B \vdots \ldots \vdots A^{n-1} B\right) = n. \]

	\[ A = \begin{pmatrix} 0 & -1 & 0 \\ -1 & 0 & 0 \\ 2 & 1 & 2 \end{pmatrix}, \quad B = \begin{pmatrix} -1 \\ 1 \\ -1 \end{pmatrix}. \] Підставляючи у критерій, знаходимо: \begin{multline*} \mathcal{D} = \left(\begin{pmatrix} -1 \\ 1 \\ -1 \end{pmatrix} \vdots \begin{pmatrix} 0 & -1 & 0 \\ -1 & 0 & 0 \\ 2 & 1 & 2 \end{pmatrix} \begin{pmatrix} -1 \\ 1 \\ -1 \end{pmatrix} \vdots \begin{pmatrix} 0 & -1 & 0 \\ -1 & 0 & 0 \\ 2 & 1 & 2 \end{pmatrix}^2 \begin{pmatrix} -1 \\ 1 \\ -1 \end{pmatrix} \right) = \\ 
	= \left(\begin{pmatrix} -1 \\ 1 \\ -1 \end{pmatrix} \vdots \begin{pmatrix} -1 \\ 1 \\ -3 \end{pmatrix} \vdots \begin{pmatrix} 0 & -1 & 0 \\ -1 & 0 & 0 \\ 2 & 1 & 2 \end{pmatrix} \begin{pmatrix} -1 \\ 1 \\ -3 \end{pmatrix} \right) = \begin{pmatrix} -1 & -1 & -1 \\ 1 & 1 & 1 \\ -1 & -3 & -7 \end{pmatrix}, \end{multline*} її ранг 2, тому система не цілком керована, а початкова -- не цілком спостережувана.
\end{solution}

\begin{problem}
    Побудувати спостерігач такої системи у загальному вигляді:
    \begin{enumerate}
    	\item \[ \left\{ \begin{aligned}
    		\dot x_1 &= t x_1 + x_2, \\
    		\dot x_2 &= x_1 - x_2, \\
    		y(t) &= x_1(t) + b x_2(t).
    	\end{aligned} \right. \]

    	\item \[ \left\{ \begin{aligned}
    		\frac{\diff^2 x(t)}{\diff t^2} &= -kx, \\
    		y(t) &= x(t) + \beta \frac{\diff x(t)}{\diff t}.
    	\end{aligned} \right. \]
    \end{enumerate}
\end{problem}

\begin{solution}
    \begin{enumerate}
    	\item За теоремою про структуру спостерігача, він має вигляд \[ \frac{\diff \hat x (t)}{\diff t} = (A (t) - K (t) H (t)) \cdot \hat x (t) + K(t) \cdot y(t).\] Підставляючи відомі значення, знаходимо \[ \frac{\diff \hat x (t)}{\diff t} = \left(\begin{pmatrix} t & 1 \\ 1 & -1 \end{pmatrix} - \begin{pmatrix} k_1(t) \\ k_2(t) \end{pmatrix} \begin{pmatrix} 1 & b \end{pmatrix} \right) \cdot \hat x (t) + \begin{pmatrix} k_1(t) \\ k_2(t) \end{pmatrix} \cdot y(t),\] або, у розгорнутому вигляді \[ \left\{ \begin{aligned}
    		\dot{\hat{x}}_1 &= (t - k_1(t)) \cdot \hat x_1 + (1 - b k_1(t)) \cdot \hat x_2 + k_1(t) \cdot y(t), \\
    		\dot{\hat{x}}_2 &= (1 - k_2(t)) \cdot \hat x_1 - (1 + b k_2(t)) \cdot \hat x_2 + k_2 (t) \cdot y(t).
    	\end{aligned} \right. \]
    	\item Введемо нову змінну $x_2 = \dot x_1$, тоді $\dot x_1 = x_2$, $\dot x_2 = - k x_1$, $y = x_1 + \beta x_2$. \\

    	За теоремою про структуру спостерігача, він має вигляд \[ \frac{\diff \hat x (t)}{\diff t} = A (t) \cdot \hat x (t) + K(t) \cdot (y(t) - H(t) \cdot \hat x(t)).\] Підставляючи відомі значення, знаходимо \[ \frac{\diff \hat x (t)}{\diff t} = \begin{pmatrix} 0 & 1 \\ -k & 0 \end{pmatrix} \hat x(t) + \begin{pmatrix} k_1(t) \\ k_2(t) \end{pmatrix} \cdot \left(y(t) - \hat x_1 - \beta \hat x_2\right), \] або, у розгорнутому вигляді \[ \left\{ \begin{aligned}
    		\dot{\hat{x}}_1 &= \hat x_2 + k_1(t) \cdot \left(y(t) - \hat x_1 - \beta \hat x_2\right), \\
    		\dot{\hat{x}}_2 &= -k \hat x_1 + k_2(t) \cdot \left(y(t) - \hat x_1 - \beta \hat x_2\right).
    	\end{aligned} \right. \]
    \end{enumerate}
\end{solution}

\begin{problem}
    Задана динамічна система \[ \left\{ \begin{aligned} 
    	\frac{\diff x(t)}{\diff t} &= 2 x(t), \\
    	y(t) &= \sin (t) \cdot x(t),
    \end{aligned} \right. \]
    де $x(t) \in \RR^1$ -- вектор стану, $y(t) \in \RR^1$ -- відомі спостереження, $t \in [0, 3]$. Знайти розв'язок задачі спостереження з використанням грамміана спростережуваності.
\end{problem}

\begin{solution}
    Розв'язок задачі спостереження задовольняє диференціальному рівнянню \[ \frac{\diff x(t)}{\diff t} = A(t) \cdot x(t) + R(t) \cdot H^*(t) \cdot (y(t) - H(t) \cdot x(t)), \] де $R(t) = \NN^{-1](t, 0)}$. \\

    Знайдемо $\NN(t, 0)$: з рівняння \[ \frac{\diff \Theta(t, s)}{\diff t} = A(t)\cdot \Theta(t, s) = 2\Theta(t, s) \] знаходимо $\Theta(t, s) = e^{2 (t - s)}$, тому \begin{multline*} \NN(t, 0) = \int_0^t e^{4 (s - t)} \cdot \sin^2(s) \diff s = e^{-4t} \int_0^t e^{4 s} \cdot \sin^2(s) \diff s = \\
    = \frac{-2 \sin (2t) - 4 \cos(2t) + 5 - e^{-4t}}{40}. \end{multline*} Звідси \[ \frac{\diff x(t)}{\diff t} = A(t) \cdot x(t) + \frac{40 \cdot H^*(t) \cdot (y(t) - H(t) \cdot x(t))}{-2 \sin (2t) - 4 \cos(2t) + 5) - e^{-4t}} , \] або, у розгорнутому вигляді, \[ \dot x = 2 x + \frac{40 \cdot \sin(t) \cdot (y(t) - \sin(t) \cdot x(t))}{-2 \sin (2t) - 4 \cos(2t) + 5 - e^{-4t}}. \]
\end{solution}

\begin{problem}
    % 4.6
\end{problem}

\begin{solution}
    % 4.6
\end{solution}
 \newpage

% OK, incomplete, missing 4.7, 4.12, 4.13

\subsection*{Домашнє завдання}

\begin{problem}
    % 4.7
\end{problem}

\begin{solution}
    % 4.7
\end{solution}

\begin{problem}
    Записати диференціальне рівняння для знаходження грамміана спостережуваності системи
    \begin{equation*}
        \left\{
            \begin{aligned}
                \dfrac{dx_1(t)}{dt} &= x_1(t) + x_2(t), \\
                \dfrac{dx_2(t)}{dt} &= - t^2x_2(t), \\
                y(t) &= \sin (t) \cdot x_1(t) + \cos (t) \cdot x_2(t).
            \end{aligned}   
        \right.
    \end{equation*}
    Тут $x = (x_1, x_2)^\star$ -- вектор фазових координат, $y$ -- скалярне спостереження.
\end{problem}

\begin{solution}
    Почнемо з того, що $A = \begin{pmatrix} 1 & 1 \\ 0 & -t^2 \end{pmatrix}$, $H = \begin{pmatrix} \sin(t) & \cos(t) \end{pmatrix}$, $A^\star = \begin{pmatrix} 1 & 0 \\ 1 & -t^2 \end{pmatrix}$, $H^\star = \begin{pmatrix} \sin(t) \\ \cos(t) \end{pmatrix}$.\\
    
    Тоді диференціальне рівняння для знаходження грамміана спостережуваності набуває вигляду
    \[ \dfrac{\diff\NN(t, t_0)}{dt} = -\begin{pmatrix} 1 & 0 \\ 1 & -t^2 \end{pmatrix} \NN(t, t_0) - \NN(t, t_0) \begin{pmatrix} 1 & 1 \\ 0 & -t^2 \end{pmatrix} + \begin{pmatrix} \sin(t) \\ \cos(t) \end{pmatrix} \begin{pmatrix} \sin(t) & \cos(t) \end{pmatrix}. \]
    Або, що те саме,
    \begin{multline*}
        \begin{pmatrix} \dot n_{11} & \dot n_{12} \\ \dot n_{12} & \dot n_{22} \end{pmatrix} (t, t_0) = -\begin{pmatrix} 1 & 0 \\ 1 & -t^2 \end{pmatrix} \begin{pmatrix} n_{11} & n_{12} \\ n_{12} & n_{22} \end{pmatrix} (t, t_0) - \\ - \begin{pmatrix} n_{11} & n_{12} \\ n_{12} & n_{22} \end{pmatrix}(t, t_0) \begin{pmatrix} 1 & 1 \\ 0 & -t^2 \end{pmatrix} + \begin{pmatrix} \sin^2(t) & \sin(t)\cdot \cos(t) \\ \sin(t) \cdot \cos(t) & \cos^2(t) \end{pmatrix}.
    \end{multline*} 
    
    \begin{multline*}
        \begin{pmatrix} \dot n_{11} & \dot n_{12} \\ \dot n_{12} & \dot n_{22} \end{pmatrix} (t, t_0) = - \begin{pmatrix} n_{11} & n_{12} \\ n_{11} - t^2n_{12} & n_{12} - t^2n_{22} \end{pmatrix} (t, t_0) - \\ 
        - \begin{pmatrix} n_{11} & n_{11} - t^2n_{12} \\ n_{12} & n_{12} - t^2n_{22} \end{pmatrix}(t, t_0) + \begin{pmatrix} \sin^2(t) & \sin(t)\cdot \cos(t) \\ \sin(t) \cdot \cos(t) & \cos^2(t) \end{pmatrix}.
    \end{multline*} 
    
    \begin{equation*}
        \left\{
            \begin{aligned}
                \dot n_{11} (t, t_0) &= - 2n_{11} (t, t_0) + \sin^2(t) \\
                \dot n_{12} (t, t_0) &= - n_{11} (t, t_0) + (t^2 - 1) n_{12} (t, t_0) + \sin(t)\cdot \cos(t) \\
                \dot n_{22} (t, t_0) &= - 2n_{12} (t, t_0) + 2t^2 n_{22} (t, t_0) + \cos^2(t)
            \end{aligned}
        \right.
    \end{equation*}
\end{solution}

\begin{problem}
    Чи буде система цілком спостережуваною?
    \begin{enumerate}
        \item \[\ddot x = a^2 x, \quad y(t) = p\dot x(t); \]
        \item \begin{equation*}
            \left\{
                \begin{aligned}
                    \dot x_1 &= 2x_1 + \alpha x_2, \\
                    \dot x_2 &= - \alpha x_1 - \alpha x_2, \\
                    y(t) &= x_1 + \beta x_2.
                \end{aligned}
            \right.
        \end{equation*}
        \item \begin{equation*}
            \left\{
                \begin{aligned}
                    \dot x_1 &= x_2 - 2 x_3, \\
                    \dot x_2 &= x_1 - x_3, \\
                    \dot x_3 &= - 2 x_3, \\
                    y(t) &= - x_1 + x_2 - x_3.
                \end{aligned}
            \right.
        \end{equation*}
    \end{enumerate}
\end{problem}

\begin{solution}
    \begin{enumerate}
        \item Почнемо з того, що $A = \begin{pmatrix} 0 & 1 \\ a^2 & 0 \end{pmatrix}$, $H = \begin{pmatrix} 0 & p \end{pmatrix}$. Матриці стаціонарні, тому застосуємо другий критерій спостережуваності:
        \[ R = \begin{pmatrix} H^\star & A^\star H^\star \end{pmatrix} = \begin{pmatrix} 0 & a^2 \\ p & 0 \end{pmatrix}. \]
        Як бачимо, ранг 2, тобто система є цілком спостережуваною, якщо тільки $a \ne 0$ і $p \ne 0$.
        \item Почнемо з того, що $A = \begin{pmatrix} 2 & \alpha \\ -\alpha & -\alpha \end{pmatrix}$, $H = \begin{pmatrix} 1 & \beta \end{pmatrix}$. Матриці стаціонарні, тому застосуємо другий критерій спостережуваності:
        \[ R = \begin{pmatrix} H^\star & A^\star H^\star \end{pmatrix} = \begin{pmatrix} 1 & 2 - \alpha\beta \\ \beta & \alpha - \alpha\beta \end{pmatrix}. \]
        $\det R = \alpha - 2\beta - \alpha\beta + \alpha\beta^2 \ne 0$ (тобто система є спостережуваною), якщо тільки $\alpha \ne \dfrac{2\beta}{1-\beta+\beta^2}$.
        \item Почнемо з того, що $A = \begin{pmatrix} 0 & 1 & -2 \\ 1 & 0 & - 1 \\ 0 & 0 & -2 \end{pmatrix}$, $H = \begin{pmatrix} -1 & 1 & -1 \end{pmatrix}$. Матриці стаціонарні, тому застосуємо другий критерій спостережуваності:
        \[ R = \begin{pmatrix} H^\star & A^\star H^\star & (A^\star)^2 H^\star \end{pmatrix} = \begin{pmatrix} -1 & 1 & -1 \\ 1 & -1 & 1 \\ -1 & 3 & -7 \end{pmatrix}.\]
        Як бачимо, ранг 2 а не 3, тому система не є цілком спостережуваною.
    \end{enumerate}
\end{solution}

\begin{problem}
    Для яких параметрів $a$, $b$ система
    \begin{equation*}
        \left\{
            \begin{aligned}
                \dfrac{dx_1(t)}{dt} &= ax_1(t), \\
                \dfrac{dx_2(t)}{dt} &= bx_2(t), \\
                y(t) &= x_1(t) + x_2(t)
            \end{aligned}
        \right.
    \end{equation*}
    є цілком спостережуваною? Тут $x = (x_1, x_2)^\star$ -- вектор фазових координат, $y$ -- скалярне спостереження.
\end{problem}

\begin{solution}
    Почнемо з того, що $A = \begin{pmatrix} a & 0 \\ 0 & b \end{pmatrix}$, $H = \begin{pmatrix} 1 & 1 \end{pmatrix}$. Матриці стаціонарні, тому застосуємо другий критерій спостережуваності:
        \[ R = \begin{pmatrix} H^\star & A^\star H^\star \end{pmatrix} = \begin{pmatrix} 1 & a \\ 1 & b \end{pmatrix}. \]
        $\det R = b - a$, тобто система є цілком керованою якщо тільки $a \ne b$.
\end{solution}

\begin{problem}
    Побудувати спостерігач у загальному вигляді для такої системи:
    \begin{enumerate}
        \item \begin{equation*}
            \left\{
                \begin{aligned}
                    \dot x_1 &= x_1 + t^2 x_2, \\
                    \dot x_2 &= 2x_1 - 3x_2, \\
                    y(t) &= bx_1(t) + x_2(t).
                \end{aligned}
            \right.
        \end{equation*}
        \item \begin{equation*}
            \left\{
                \begin{aligned}
                    & \dfrac{d^2x}{dt^2} + k_1 \dfrac{dx}{dt} + k_2 x = 0, \\
                    & y(t) = x(t) + \beta \dfrac{dx(t)}{dt}.
                \end{aligned}
            \right.
        \end{equation*}
    \end{enumerate}
\end{problem}

\begin{solution}
    \begin{enumerate}
        \item Почнемо з того, що $A = \begin{pmatrix} 1 & t^2 \\ 2 & -3 \end{pmatrix}$, $H = \begin{pmatrix} b & 1 \end{pmatrix}$. Далі пишемо
        \[ \begin{pmatrix} \hat x_1 \\ \hat x_2 \end{pmatrix}^\prime (t) = \begin{pmatrix} 1 & t^2 \\ 2 & -3 \end{pmatrix} \begin{pmatrix} \hat x_1 \\ \hat x_2 \end{pmatrix} (t) + \begin{pmatrix} k_1 \\ k_2 \end{pmatrix} (t) \left( y(t) - \begin{pmatrix} b & 1 \end{pmatrix} \begin{pmatrix} \hat x_1 \\ \hat x_2 \end{pmatrix} (t) \right) .\]
        
        \begin{equation*}
            \left\{
                \begin{aligned}
                    \hat x_1^\prime (t) &= \hat x_1 (t) + t^2 \hat x_2 (t) + k_1(t) (y(t) - b \hat x_1(t) - \hat x_2(t)) \\
                    \hat x_2^\prime (t) &= 2 \hat x_1 (t) - 3 \hat x_2 (t) + k_2(t) (y(t) - b \hat x_1(t) - \hat x_2(t))
                \end{aligned}
            \right.
        \end{equation*}
        
        \item Введемо заміну $x_1 = x$, $x_2 = \dot x$, тоді маємо $A = \begin{pmatrix} 0 & 1 \\ - k_2 & - k_1 \end{pmatrix}$, $H = \begin{pmatrix} 1 & b \end{pmatrix}$. Далі пишемо
        \[ \begin{pmatrix} \hat x_1 \\ \hat x_2 \end{pmatrix}^\prime (t) = \begin{pmatrix} 0 & 1 \\ - k_2 & - k_1 \end{pmatrix} \begin{pmatrix} \hat x_1 \\ \hat x_2 \end{pmatrix} (t) + \begin{pmatrix} K_1 \\ K_2 \end{pmatrix} (t) \left( y(t) - \begin{pmatrix} 1 & b \end{pmatrix} \begin{pmatrix} \hat x_1 \\ \hat x_2 \end{pmatrix} (t) \right) .\]
        
        \begin{equation*}
            \left\{
                \begin{aligned}
                    \hat x_1^\prime (t) &= \hat x_2 (t) + K_1 (t) (y(t) - \hat x_1(t) - b \hat x_2 (t)) \\
                    \hat x_2^\prime (t) &= - k_2 \hat x_1(t) - k_1 \hat x_2 (t) + K_2 (t) (y(t) - \hat x_1(t) - b \hat x_2 (t))
                \end{aligned}
            \right.
        \end{equation*}
    \end{enumerate}
\end{solution}

\begin{problem}
    % 4.12
\end{problem}

\begin{solution}
    % 4.12
\end{solution}

\begin{problem}
    % 4.13
\end{problem}

\begin{solution}
    % 4.13
\end{solution}
 \newpage

\section{Задача фільтрації. Множинний підхід}

\subsection{Аудиторне заняття}

\begin{problem}
	Задана динамічна система \[ \left\{ \begin{aligned}
		\frac{\diff x(t)}{\diff t} &= t x(t) + v(t), \\
		y(t) &= p x(t) + w(t),
	\end{aligned} \right. \]
	де $x(t) \in \RR^1$ -- вектор стану, $v(t) \in \RR^1$, $w(t) \in \RR^1$ -- невідомі шуми, $x_0 \in \RR^1$ -- невідома початкова умова, $y(t) \in \RR^1$ -- відомі спостереження. Побудувати інформаційну множину такої системи в момент $\tau \in [0, T]$ за умови, що \[ \int_0^\tau (v^2(s) + w^2(s)) \diff s + x^2(0) \le 1, \quad \tau \in [0, T]. \]
\end{problem}

\begin{solution}
	Загальна постановка задачі фільтрації має вигляд \[ \dot x (t)= A (t) \cdot x (t)+ v (t), \quad y (t)= G (t) \cdot x(t) + w(t),\] \[\int_{t_0}^t ( \langle M(t) \cdot v(t), v(t)\rangle + \langle N(t) \cdot w(t), w(t)\rangle )  + \langle p_0 x(t_0), x(t_0) \rangle \le \mu^2. \] У нашій задачі $A (t)= \begin{pmatrix} t \end{pmatrix}$, $G (t)= \begin{pmatrix} p \end{pmatrix}$, $M (t)= \begin{pmatrix} 1 \end{pmatrix}$, $N (t)= \begin{pmatrix} 1 \end{pmatrix}$, $p_0 = 1$, $\mu = 1$. \\

	Знайдемо фільтр (спостерігач) цієї задачі у вигляді \[ \dot{\hat{x}} (t) = A (t) \cdot \hat x (t) + K (t) \cdot (y (t) - G (t) \cdot \hat x (t)), \] де $K (t)= R (t) \cdot G^* (t) \cdot N(t)$, де у свою чергу \[\dot R (t)= A (t) \cdot R (t)+ R (t) \cdot A^* (t)- R (t) \cdot G^* (t) \cdot N (t) \cdot G (t) \cdot R(t), \quad R(t_0) = p_0^{-1}. \]

 	Підставляючи відомі функції знаходимо \[\dot R (t)= 2 t \cdot R (t) - p^2 \cdot R^2(t), \quad R(t_0) = 1. \]

 	Це рівняння Бернуллі, його розв'язок \[ R(t) = \frac{2 e^{t^2}}{2 e^{t_0^2} + p^2 \sqrt{\pi} (\erfi(t) - \erfi(t_0))}. \]

 	Далі, \[ K(t) = \frac{2 p e^{t^2}}{2 e^{t_0^2} + p^2 \sqrt{\pi} (\erfi(t) - \erfi(t_0))}, \] і \[ \dot{\hat{x}} (t) = t \cdot \hat x (t) + \frac{2 p e^{t^2} \cdot (y (t) - p \cdot \hat x (t))}{2 e^{t_0^2} + p^2 \sqrt{\pi} (\erfi(t) - \erfi(t_0))}. \]

 	Нарешті, \[ \XX(\tau) = \EE (\hat x(\tau), (\mu^2 - k(\tau)) \cdot R(\tau)), \] де \[ \dot k (s) = \langle N(s) (y(s) - G(s) \cdot \hat x(s)), y(s) - G(s) \cdot \hat x(s)\rangle, \quad k(t_0) = 0, \] тобто \[ \dot k (s) = \langle (y(s) - p \hat x(s)), y(s) - p \hat x(s)\rangle = |y(s) - p \hat x(s)|^2, \quad k(t_0) = 0. \]
\end{solution}

\begin{problem}
	Задана динамічна система \[ \left\{ \begin{aligned}
		\frac{\diff x(t)}{\diff t} &= x(t) + v(t), \\
		y(t) &= 2 x(t) + w(t),
	\end{aligned} \right. \]
	де $x(t) \in \RR^1$ -- вектор стану, $v(t) \in \RR^1$, $w(t) \in \RR^1$ -- невідомі шуми, $x_0 \in \RR^1$ -- невідома початкова умова. Побудувати оцінку стану заданої системи (фільтр) за заданими спостереженнями $y(t) \in \RR^1$ за умови, що \[ \int_0^\tau (v^2(s) + w^2(s)) \diff s + x^2(0) \le 2, \] $\tau \in [0, T]$. Знайти похибку оцінювання.
\end{problem}

\begin{solution}
	Загальна постановка задачі фільтрації має вигляд \[ \dot x (t)= A (t) \cdot x (t)+ v (t), \quad y (t)= G (t) \cdot x(t) + w(t),\] \[\int_{t_0}^t ( \langle M(t) \cdot v(t), v(t)\rangle + \langle N(t) \cdot w(t), w(t)\rangle )  + \langle p_0 x(t_0), x(t_0) \rangle \le \mu^2. \] У нашій задачі $A (t)= \begin{pmatrix} 1 \end{pmatrix}$, $G (t)= \begin{pmatrix} 2 \end{pmatrix}$, $M (t)= \begin{pmatrix} 1 \end{pmatrix}$, $N (t)= \begin{pmatrix} 1 \end{pmatrix}$, $p_0 = 1$, $\mu = \sqrt{2}$. \\

	Знайдемо фільтр (спостерігач) цієї задачі у вигляді \[ \dot{\hat{x}} (t) = A (t) \cdot \hat x (t) + K (t) \cdot (y (t) - G (t) \cdot \hat x (t)), \] де $K (t)= R (t) \cdot G^* (t) \cdot N(t)$, де у свою чергу \[\dot R (t)= A (t) \cdot R (t)+ R (t) \cdot A^* (t)- R (t) \cdot G^* (t) \cdot N (t) \cdot G (t) \cdot R(t), \quad R(t_0) = p_0^{-1}. \]

 	Підставляючи відомі функції знаходимо \[\dot R (t)= 2 \cdot R (t) - 4 \cdot R^2(t), \quad R(t_0) = 1. \]

 	Це рівняння зі змінними що роздяліються, його розв'язок \[ R(t) = \frac{e^{2t}}{2e^{2t}-e^{2t_0}}. \]

 	Далі, \[ K(t) = \frac{2 e^{2t}}{2e^{2t}-e^{2t_0}}, \] і \[ \dot{\hat{x}} (t) = \hat x (t) + \frac{2 e^{2t} \cdot (y (t) - 2 \cdot \hat x (t))}{2e^{2t}-e^{2t_0}}. \]

 	Похибка $e(\tau)$ оцінювання задовольняє оцінці \[ |e(\tau)| \le \sqrt{\mu^2-k(\tau)} \cdot \sqrt{\lambda_* (R(\tau))} = \frac{ \sqrt{2 - k(\tau)} \cdot e^{2\tau}}{2e^{2\tau}-e^{2t_0}},\] де \[ \dot k (s) = \langle N(s) (y(s) - G(s) \cdot \hat x(s)), y(s) - G(s) \cdot \hat x(s)\rangle, \quad k(t_0) = 0, \] тобто \[ \dot k (s) = \langle (y(s) - 2 \hat x(s)), y(s) - 2 \hat x(s)\rangle = |y(s) - 2 \hat x(s)|^2, \quad k(t_0) = 0. \]
\end{solution}

\begin{problem}
	Задана динамічна система \[ \left\{ \begin{aligned}
		\frac{\diff x_1(t)}{\diff t} &= 2 x_1(t) + x_2(t) + v_1(t), \\
		\frac{\diff x_2(t)}{\diff t} &= - x_1(t) + x_2(t) + v_2(t), \\
		y(t) &= x_1(t) + 2 x_2(t) + w(t),
	\end{aligned} \right. \] і відомі спостереження за цією системою $y(t) \in \RR^1$.  Побудувати оцінку стану (фільтр) і знайти похибку оцінювання. Тут $x = (x_1, x_2)^*$ -- вектор фазових координат з $\RR^2$, $v_1 (t) \in \RR^1$, $v_2 (t) \in \RR^1$, $w(t) \in \RR^1$ -- невідомі шуми, \[ \int_0^\tau (v_1^2(s) + v_2^2(s) + w^2(s)) \diff s + x_1^2(0) + 2 x_2^2(0) \le 1,\] $\tau \in [0, T]$, момент часу $T$ є заданим.
\end{problem}

\begin{solution}
	Загальна постановка задачі фільтрації має вигляд \[ \dot x (t)= A (t) \cdot x (t)+ v (t), \quad y (t)= G (t) \cdot x(t) + w(t),\] \[\int_{t_0}^t ( \langle M(t) \cdot v(t), v(t)\rangle + \langle N(t) \cdot w(t), w(t)\rangle )  + \langle p_0 x(t_0), x(t_0) \rangle \le \mu^2. \] У нашій задачі $A (t) = \begin{pmatrix} 2 & 1 \\ -1 & 1 \end{pmatrix}$, $G (t)= \begin{pmatrix} 1 & 2 \end{pmatrix}$, $M (t)= \begin{pmatrix} 1 & 0 \\ 0 & 1 \end{pmatrix}$, $N (t)= \begin{pmatrix} 1 \end{pmatrix}$, $p_0 = \begin{pmatrix} 1 & 0 \\ 0 & 2 \end{pmatrix}$, $\mu = 1$. \\

	Знайдемо фільтр (спостерігач) цієї задачі у вигляді \[ \dot{\hat{x}} (t) = A (t) \cdot \hat x (t) + K (t) \cdot (y (t) - G (t) \cdot \hat x (t)), \] де $K (t)= R (t) \cdot G^* (t) \cdot N(t)$, де у свою чергу \[\dot R (t)= A (t) \cdot R (t)+ R (t) \cdot A^* (t)- R (t) \cdot G^* (t) \cdot N (t) \cdot G (t) \cdot R(t), \quad R(t_0) = p_0^{-1}. \]

 	Підставляючи відомі функції знаходимо \[\dot R (t) = \begin{pmatrix} 2 & 1 \\ -1 & 1 \end{pmatrix} \cdot R(t) + R(t) \cdot \begin{pmatrix} 2 & -1 \\ 1 & 1 \end{pmatrix} - R(t) \cdot \begin{pmatrix} 1 & 2 \\ 2 & 4 \end{pmatrix} \cdot R(t), \] або, у розгорнутому вигляді \[ \left\{ \begin{aligned}
 		\dot r_{11} &= 4 r_{11} + 2 r_{12} + r_{11}^2 + 4 r_{11}r_{12} + 4 r_{12}^2, \\
 		\dot r_{12} &=  - r_{11} + 3 r_{12} + r_{22} + r_{11}r_{12}+2r_{12}^2+2r_{11}r_{22}+4r_{12}r_{22}, \\
 		\dot r_{22} &= - 2 r_{12} + 2 r_{22} + r_{12}^2 + 4 r_{12} r_{22} + r_{22}^2, \\
 		r_{11} (0) &= 1, \\
 		r_{12} (0) &= 0, \\
 		r_{22} (0) &= 1 / 2.
	\end{aligned} \right. \] 	

	Похибка $e(\tau)$ оцінювання задовольняє оцінці \[ |e(\tau)| \le \sqrt{\mu^2-k(\tau)} \cdot \sqrt{\lambda_* (R(\tau))},\] де \[ \dot k (s) = \langle N(s) (y(s) - G(s) \cdot \hat x(s)), y(s) - G(s) \cdot \hat x(s)\rangle, \quad k(t_0) = 0, \] тобто \[ \dot k (s) = \langle (y(s) - \hat x_1(s) - 2 \hat x_2(s)), y(s) - \hat x_1(s) - 2 \hat x_2(s)\rangle = |y(s) - \hat x_1(s) - 2 \hat x_2(s)|^2. \]

\end{solution}
 \newpage

\setcounter{section}{4}

\setcounter{problem}{6}

\begin{problem}
    Записати диференціальне рівняння для знаходження грамміана спостережуваності системи
    \begin{equation*}
        \left\{
            \begin{aligned}
                \dfrac{dx_1(t)}{dt} &= x_1(t) + x_2(t), \\
                \dfrac{dx_2(t)}{dt} &= - t^2x_2(t), \\
                y(t) &= \sin (t) \cdot x_1(t) + \cos (t) \cdot x_2(t).
            \end{aligned}   
        \right.
    \end{equation*}
    Тут $x = (x_1, x_2)^\star$ -- вектор фазових координат, $y$ -- скалярне спостереження.
\end{problem}

\begin{solution}
    Почнемо з того, що $A = \begin{pmatrix} 1 & 1 \\ 0 & -t^2 \end{pmatrix}$, $H = \begin{pmatrix} \sin(t) & \cos(t) \end{pmatrix}$, $A^\star = \begin{pmatrix} 1 & 0 \\ 1 & -t^2 \end{pmatrix}$, $H^\star = \begin{pmatrix} \sin(t) \\ \cos(t) \end{pmatrix}$.\\
    
    Тоді диференціальне рівняння для знаходження грамміана спостережуваності набуває вигляду
    \[ \dfrac{dN(t, t_0)}{dt} = -\begin{pmatrix} 1 & 0 \\ 1 & -t^2 \end{pmatrix} N(t, t_0) - N(t, t_0) \begin{pmatrix} 1 & 1 \\ 0 & -t^2 \end{pmatrix} + \begin{pmatrix} \sin(t) \\ \cos(t) \end{pmatrix} \begin{pmatrix} \sin(t) & \cos(t) \end{pmatrix}. \]
    Або, що те саме,
    \begin{align*}
        \begin{pmatrix} \dot n_{11} & \dot n_{12} \\ \dot n_{12} & \dot n_{22} \end{pmatrix} (t, t_0) = &-\begin{pmatrix} 1 & 0 \\ 1 & -t^2 \end{pmatrix} \begin{pmatrix} n_{11} & n_{12} \\ n_{12} & n_{22} \end{pmatrix} (t, t_0) - \begin{pmatrix} n_{11} & n_{12} \\ n_{12} & n_{22} \end{pmatrix}(t, t_0) \begin{pmatrix} 1 & 1 \\ 0 & -t^2 \end{pmatrix} + \\
        &+ \begin{pmatrix} \sin^2(t) & \sin(t)\cdot \cos(t) \\ \sin(t) \cdot \cos(t) & \cos^2(t) \end{pmatrix}.
    \end{align*} 
    
    \begin{align*}
        \begin{pmatrix} \dot n_{11} & \dot n_{12} \\ \dot n_{12} & \dot n_{22} \end{pmatrix} (t, t_0) = &- \begin{pmatrix} n_{11} & n_{12} \\ n_{11} - t^2n_{12} & n_{12} - t^2n_{22} \end{pmatrix} (t, t_0) - \begin{pmatrix} n_{11} & n_{11} - t^2n_{12} \\ n_{12} & n_{12} - t^2n_{22} \end{pmatrix}(t, t_0) + \\
        &+ \begin{pmatrix} \sin^2(t) & \sin(t)\cdot \cos(t) \\ \sin(t) \cdot \cos(t) & \cos^2(t) \end{pmatrix}.
    \end{align*} 
    
    \begin{equation*}
        \left\{
            \begin{aligned}
                \dot n_{11} (t, t_0) &= - 2n_{11} (t, t_0) + \sin^2(t) \\
                \dot n_{12} (t, t_0) &= - n_{11} (t, t_0) + (t^2 - 1) n_{12} (t, t_0) + \sin(t)\cdot \cos(t) \\
                \dot n_{22} (t, t_0) &= - 2n_{12} (t, t_0) + 2t^2 n_{22} (t, t_0) + \cos^2(t)
            \end{aligned}
        \right.
    \end{equation*}
\end{solution}

\begin{problem}
    Чи буде система цілком спостережуваною?
    \begin{enumerate}
        \item \[\ddot x = a^2 x, \quad y(t) = p\dot x(t); \]
        \item \begin{equation*}
            \left\{
                \begin{aligned}
                    \dot x_1 &= 2x_1 + \alpha x_2, \\
                    \dot x_2 &= - \alpha x_1 - \alpha x_2, \\
                    y(t) &= x_1 + \beta x_2.
                \end{aligned}
            \right.
        \end{equation*}
        \item \begin{equation*}
            \left\{
                \begin{aligned}
                    \dot x_1 &= x_2 - 2 x_3, \\
                    \dot x_2 &= x_1 - x_3, \\
                    \dot x_3 &= - 2 x_3, \\
                    y(t) &= - x_1 + x_2 - x_3.
                \end{aligned}
            \right.
        \end{equation*}
    \end{enumerate}
\end{problem}

\begin{solution}
    \begin{enumerate}
        \item Почнемо з того, що $A = \begin{pmatrix} 0 & 1 \\ a^2 & 0 \end{pmatrix}$, $H = \begin{pmatrix} 0 & p \end{pmatrix}$. Матриці стаціонарні, тому застосуємо другий критерій спостережуваності:
        \[ R = \begin{pmatrix} H^\star & A^\star H^\star \end{pmatrix} = \begin{pmatrix} 0 & a^2 \\ p & 0 \end{pmatrix}. \]
        Як бачимо, ранг 2, тобто система є цілком спостережуваною, якщо тільки $a \ne 0$ і $p \ne 0$.
        \item Почнемо з того, що $A = \begin{pmatrix} 2 & \alpha \\ -\alpha & -\alpha \end{pmatrix}$, $H = \begin{pmatrix} 1 & \beta \end{pmatrix}$. Матриці стаціонарні, тому застосуємо другий критерій спостережуваності:
        \[ R = \begin{pmatrix} H^\star & A^\star H^\star \end{pmatrix} = \begin{pmatrix} 1 & 2 - \alpha\beta \\ \beta & \alpha - \alpha\beta \end{pmatrix}. \]
        $\det R = \alpha - 2\beta - \alpha\beta + \alpha\beta^2 \ne 0$ (тобто система є спостережуваною), якщо тільки $\alpha \ne \dfrac{2\beta}{1-\beta+\beta^2}$.
        \item Почнемо з того, що $A = \begin{pmatrix} 0 & 1 & -2 \\ 1 & 0 & - 1 \\ 0 & 0 & -2 \end{pmatrix}$, $H = \begin{pmatrix} -1 & 1 & -1 \end{pmatrix}$. Матриці стаціонарні, тому застосуємо другий критерій спостережуваності:
        \[ R = \begin{pmatrix} H^\star & A^\star H^\star & (A^\star)^2 H^\star \end{pmatrix} = \begin{pmatrix} -1 & 1 & -1 \\ 1 & -1 & 1 \\ -1 & 3 & -7 \end{pmatrix}.\]
        Як бачимо, ранг 2 а не 3, тому система не є цілком спостережуваною.
    \end{enumerate}
\end{solution}

\begin{problem}
    Для яких параметрів $a$, $b$ система
    \begin{equation*}
        \left\{
            \begin{aligned}
                \dfrac{dx_1(t)}{dt} &= ax_1(t), \\
                \dfrac{dx_2(t)}{dt} &= bx_2(t), \\
                y(t) &= x_1(t) + x_2(t)
            \end{aligned}
        \right.
    \end{equation*}
    є цілком спостережуваною? Тут $x = (x_1, x_2)^\star$ -- вектор фазових координат, $y$ -- скалярне спостереження.
\end{problem}

\begin{solution}
    Почнемо з того, що $A = \begin{pmatrix} a & 0 \\ 0 & b \end{pmatrix}$, $H = \begin{pmatrix} 1 & 1 \end{pmatrix}$. Матриці стаціонарні, тому застосуємо другий критерій спостережуваності:
        \[ R = \begin{pmatrix} H^\star & A^\star H^\star \end{pmatrix} = \begin{pmatrix} 1 & a \\ 1 & b \end{pmatrix}. \]
        $\det R = b - a$, тобто система є цілком керованою якщо тільки $a \ne b$.
\end{solution}

\begin{problem}
    Побудувати спостерігач у загальному вигляді для такої системи:
    \begin{enumerate}
        \item \begin{equation*}
            \left\{
                \begin{aligned}
                    \dot x_1 &= x_1 + t^2 x_2, \\
                    \dot x_2 &= 2x_1 - 3x_2, \\
                    y(t) &= bx_1(t) + x_2(t).
                \end{aligned}
            \right.
        \end{equation*}
        \item \begin{equation*}
            \left\{
                \begin{aligned}
                    & \dfrac{d^2x}{dt^2} + k_1 \dfrac{dx}{dt} + k_2 x = 0, \\
                    & y(t) = x(t) + \beta \dfrac{dx(t)}{dt}.
                \end{aligned}
            \right.
        \end{equation*}
    \end{enumerate}
\end{problem}

\begin{solution}
    \begin{enumerate}
        \item Почнемо з того, що $A = \begin{pmatrix} 1 & t^2 \\ 2 & -3 \end{pmatrix}$, $H = \begin{pmatrix} b & 1 \end{pmatrix}$. Далі пишемо
        \[ \begin{pmatrix} \hat x_1 \\ \hat x_2 \end{pmatrix}^\prime (t) = \begin{pmatrix} 1 & t^2 \\ 2 & -3 \end{pmatrix} \begin{pmatrix} \hat x_1 \\ \hat x_2 \end{pmatrix} (t) + \begin{pmatrix} k_1 \\ k_2 \end{pmatrix} (t) \left( y(t) - \begin{pmatrix} b & 1 \end{pmatrix} \begin{pmatrix} \hat x_1 \\ \hat x_2 \end{pmatrix} (t) \right) .\]
        
        \begin{equation*}
            \left\{
                \begin{aligned}
                    \hat x_1^\prime (t) &= \hat x_1 (t) + t^2 \hat x_2 (t) + k_1(t) (y(t) - b \hat x_1(t) - \hat x_2(t)) \\
                    \hat x_2^\prime (t) &= 2 \hat x_1 (t) - 3 \hat x_2 (t) + k_2(t) (y(t) - b \hat x_1(t) - \hat x_2(t))
                \end{aligned}
            \right.
        \end{equation*}
        
        \item Введемо заміну $x_1 = x$, $x_2 = \dot x$, тоді маємо $A = \begin{pmatrix} 0 & 1 \\ - k_2 & - k_1 \end{pmatrix}$, $H = \begin{pmatrix} 1 & b \end{pmatrix}$. Далі пишемо
        \[ \begin{pmatrix} \hat x_1 \\ \hat x_2 \end{pmatrix}^\prime (t) = \begin{pmatrix} 0 & 1 \\ - k_2 & - k_1 \end{pmatrix} \begin{pmatrix} \hat x_1 \\ \hat x_2 \end{pmatrix} (t) + \begin{pmatrix} K_1 \\ K_2 \end{pmatrix} (t) \left( y(t) - \begin{pmatrix} 1 & b \end{pmatrix} \begin{pmatrix} \hat x_1 \\ \hat x_2 \end{pmatrix} (t) \right) .\]
        
        \begin{equation*}
            \left\{
                \begin{aligned}
                    \hat x_1^\prime (t) &= \hat x_2 (t) + K_1 (t) (y(t) - \hat x_1(t) - b \hat x_2 (t)) \\
                    \hat x_2^\prime (t) &= - k_2 \hat x_1(t) - k_1 \hat x_2 (t) + K_2 (t) (y(t) - \hat x_1(t) - b \hat x_2 (t))
                \end{aligned}
            \right.
        \end{equation*}
    \end{enumerate}
\end{solution} \newpage

\subsection{Аудиторне заняття}

\begin{problem}
	Знайти першу варіацію за Лагранжем і похідну Фреше в просторі інтегрованих з квадратом функцій для функціоналів:
	\begin{enumerate}
		\item $\JJ (u) = \int_0^T u^3(s) \diff s$;

		\item $\JJ (u) = \int_0^T (\sin^2 u_1(s) + u_2^2(s)) \diff s$, $u = (u_1, u_2)^*$.
	\end{enumerate}
\end{problem}

\begin{solution}
	Першою варіацією (за Лагранжем) функціоналу $\JJ(u)$ в точці $u$ називається \[\delta \JJ(u, \psi) = \lim\limits_{\alpha \to 0} \frac{\JJ(u + \alpha \psi) - \JJ(u)}{\alpha}\] (якщо, звичайно, вона існує для довільного напрямку $\psi$). Також можна записати \[ \delta \JJ(u, \psi) = \frac{\diff}{\diff \alpha} \left.\JJ(u + \alpha \psi)\right|_{\alpha = 0}. \]

	\begin{enumerate}
		\item Перша варіація за Лагранжем:
		\begin{multline*} 
			\delta \JJ(u, \psi) = \frac{\diff}{\diff \alpha} \left.\JJ(u + \alpha \psi)\right|_{\alpha = 0} = \frac{\diff}{\diff \alpha} \left.\int_0^T(u + \alpha \psi)^3(s) \diff s\right|_{\alpha = 0} = \\
			= \left.\int_0^T 3 \psi(s) \cdot (u + \alpha \psi)^2(s) \diff s\right|_{\alpha = 0} = \int_0^T 3 \psi(s) \cdot u^2(s) \diff s
		\end{multline*} 
		Як наслідок, похідна за Фреше $\JJ'(u) = 3 u^2(\cdot)$.

		\item Перша варіація за Лагранжем:
		\begin{multline*} 
			\delta \JJ(u, \psi) = \frac{\diff}{\diff \alpha} \left.\JJ(u + \alpha \psi)\right|_{\alpha = 0} = \\
			= \frac{\diff}{\diff \alpha} \left. \int_0^T (\sin^2 (u_1 + \alpha\psi_1) + (u_2 + \alpha \psi_2)^2) \diff s\right|_{\alpha = 0} = \\
			= \left. \int_0^T (2 \psi_1 \sin(u_1 + \alpha \psi_1)\cos(u_1 + \alpha \psi_1) + 2\psi_2 \cdot (u_2 + \alpha \psi_2)) \diff s\right|_{\alpha = 0} = \\
			= \int_0^T (2 \psi_1(s) \sin(u(s)) \cos(u(s)) + 2 \psi_2(s) \cdot u_2(s)) \diff s.
		\end{multline*}
	\end{enumerate} 
\end{solution}

\begin{problem}
	Побудувати рівняння у варіаціях для системи керування \[ \frac{\diff x(t)}{\diff t} = (x(t)+u(t))^3, \quad x(0) = x_0. \] Тут $x(t)\in\RR^1$, $u(t)\in\RR^1$, $t\in[0,T]$. Точки $x_0\in\RR^1$ і момент часу $T$ є заданими.
\end{problem}

\begin{solution}
	Рівняння у варіаціях має загальний вигляд \[ \frac{\diff z(t)}{\diff t} = \frac{\partial f(x(t),u_*(t),t)}{\partial x} \cdot z(t) + \frac{\partial f(x(t),u_*(t),t)}{\partial u} \cdot h(t), \quad z(0) = 0. \] У нашій задачі \[f(x(t), u(t), t) = (x(t) + u(t))^3, \] тому маємо \[ \frac{\diff z(t)}{\diff t} = 3 (x(t) + u(t))^2 \cdot z(t) + 3 (x(t) + u(t))^2 \cdot h(t), \quad z(0) = 0. \]
\end{solution}

\begin{problem}
	Побудувати рівняння у варіаціях для системи керування \[ \left\{ \begin{aligned}
		\frac{\diff x_1(t)}{\diff t} &= x_1^2(t) + x_2^2(t) + u_1(t), \\
		\frac{\diff x_1(t)}{\diff t} &= x_1(t) - x_2(t) + u_2(t),
	\end{aligned} \right. \]
	\[ x_1(0) = 1, x_2(0) = -3. \]
	Тут $x = (x_1, x_2)^*$ -- вектор фазових координат з $\RR^2$, $u = (u_1, u_2)^*$, $t\in[0, T]$, момент часу $T$ є заданим.
\end{problem}

\begin{solution}
	Рівняння у варіаціях має загальний вигляд \[ \frac{\diff z(t)}{\diff t} = \frac{\partial f(x(t),u_*(t),t)}{\partial x} \cdot z(t) + \frac{\partial f(x(t),u_*(t),t)}{\partial u} \cdot h(t), \quad z(0) = 0. \] У нашій задачі \[f(x(t), u(t), t) = \begin{pmatrix} x_1^2 + x_2^2 + u_1 \\ x_1 - x_2 + u_2 \end{pmatrix}, \] тому маємо \[ \frac{\diff z(t)}{\diff t} = \begin{pmatrix} 2 x_1 & 2 x_2 \\ 1 & -1 \end{pmatrix} \cdot z(t) + \begin{pmatrix} 1 & 0 \\ 0 & 1 \end{pmatrix} \cdot h(t), \quad z(0) = 0, \] або, у розгорнутому вигляді, \[ \left\{ \begin{aligned}
		\dot z_1 &= 2 x_1 z_1 + 2 x_2 z_2 + h_1, \\
		\dot z_2 &= z_1 - z_2 + h_2, \\
		z_1(0) &= z_2(0) = 0.
	\end{aligned} \right. \]
\end{solution}

\begin{problem}
	Знайти першу варіацію за Лагранжем і похідну Фреше в просторі інтегрованих з квадратом функцій для задачи оптимального керування варіаційним методом \[ \JJ(u) = \int_0^T u^2(s) \diff s + x^2(T) \to \inf \] за умови, що \[ \frac{\diff x(t)}{\diff t} = \sin (x(t)) + u(t), \quad x(0) = x_0. \] Тут $x(t) \in \RR^1$, $u(t)\in\RR^1$, $t\in[0,T]$. Точки $x_0\in\RR^1$ і момент часу $T$ є заданими.
\end{problem}

\begin{solution}
	Перш за все запишемо
	\[ \phi(\alpha) = \JJ(u + \alpha h) = \int_0^T (u(s) + \alpha h(s))^2 \diff s + x^2(T, \alpha). \]

	Далі, \[ \phi'(\alpha) = \int_0^T 2 h(s) \cdot (u(s) + \alpha h(s)) \diff s + 2 x (T, \alpha) \frac{\partial x(T, \alpha)}{\partial \alpha}. \]

	Підставимо $\alpha = 0$: \[ \delta \JJ(u, h) = \int_0^T 2 h(s) u(s) \diff s + \underset{=-\psi(T)}{\underbrace{2 x(T)}} \cdot z(T).\] 

	Тоді рівняння у варіаціях \[ \left\{ \begin{aligned} 
		z' &= \cos(x) \cdot z + 1 \cdot h, \\
		z(0) &= 0.
	\end{aligned} \right. \]

	\begin{multline*} 
		\psi(T) \cdot z(T) = \psi(T) \cdot z(T) - \psi(0) \cdot z(0) = \int_0^T (\psi(s) \cdot z(s))' \diff s = \\
		= \int_0^T (\psi'(s) \cdot z(s) + \psi(s) \cdot z'(s)) \diff s = \\
		= \int_0^T \psi'(s) \cdot z(s) + \psi(s) (\cos(x(s)) \cdot z(s) + h(s)) \diff s = \\
		= \int_0^T \psi'(s) \cdot z(s) + \psi(s) \cos(x(s)) \diff s + \int_0^T h(s) \cdot \psi(s) \diff s.
	\end{multline*}

	\begin{multline*} 
		\delta \JJ(u, h) = \int_0^T 2 h(s) \cdot u(s) \diff s - \int_0^T z(s) \cdot (\psi'(s) + \psi(s) \cos (x(s))) \diff s + \\ + \int_0^T h(s) \cdot (2 u(s) - \psi(s)) \diff s.
	\end{multline*}

	Тоді спряжена система \[ \left\{ \begin{aligned} 
		0 &= \psi'(s) + \psi(s) \cdot \cos(x(s)), \\
		\psi(T) &= - 2 x(T).
	\end{aligned} \right. \]

	Остаточно, $\JJ'(u) = 2 u(\cdot) - \psi(\cdot)$.
\end{solution}

\begin{problem}
	Розв'язати задачу оптимального керування варіаційним методом: \[ \JJ(u) = \int_0^T u^2(s) \diff s + (x(T) - 1)^2 \to \inf\] за умови, що \[ \frac{\diff x(t)}{\diff t} = u(t), \quad x(0) = x_0.\] Тут $x(t) \in \RR^1$, $u(t)\in\RR^1$, $t\in[0,T]$. Точки $x_0\in\RR^1$ і момент часу $T$ є заданими.
\end{problem}

\begin{solution}
	$\delta \JJ(u, h) = \phi'(0)$. \\

	$\phi(\alpha) = \JJ(u + \alpha h) = \int_0^T (u(s) + \alpha h(s))^2 \diff s + (x(T, \alpha) - 1)^2$. \\

	$\phi'(\alpha) = \int_0^T 2 h(s) (u(s) + \alpha h(s)) \diff s + 2 (x(T, \alpha) - 1) x_\alpha'(T, \alpha)$. \\

	$\phi'(0) = \int_0^T 2 h(s) u(s) \diff s + \underset{=-\psi(T)}{\underbrace{2 (x(T) - 1)}} \cdot z(T)$. \\

	$z' = h$, $z(0) = 0$.

	\begin{multline*} 
		\psi(T) \cdot z(T) = \psi(T) \cdot z(T) - \psi(0) \cdot z(0) = \int_0^T (\psi(s) \cdot z(s))' \diff s = \\
		= \int_0^T (\psi'(s) \cdot z(s) + \psi(s) \cdot z'(s)) \diff s.
	\end{multline*}

	$ \phi'(0) = \int_0^T 2 h(s) u(s) \diff s - \int_0^T (\psi'(s) z(1) + \psi(s) h) \diff s$. \\

	$x_\alpha'(T, \alpha) = \int_0^T h(s) (2 u(s) - \psi(s)) \diff s - \int_0^T \psi' (s) z(s) \diff s$. \\

	$\JJ'(u) = 2 u (\cdot) - \psi(\cdot)$. \\

	$\JJ'(u) = 0$:

	\[ \left\{ \begin{aligned}
		\frac{\diff x}{\diff t} &= \frac{\psi(\cdot)}{2}, \\
		\psi' &= 0, \\
		x(0) &= x_0, \\
		\phi(T) &= 2 (x(T) - 1).
	\end{aligned} \right. \]

	\[ \left\{ \begin{aligned}
		\psi &= const, \\
		\frac{\diff x}{\diff t} &= \frac{const}{2},
		x(t) &= (const / 2) t + u.
	\end{aligned} \right. \]

	$c_1 = - 2 \left( \frac{c_1}{2} T + x_0 - 1\right) = - c_1 T - 2 x_0 + 2$, $c_1 = \frac{-2x_0+2}{1+T}$. \\

	$u(t) = \frac{\psi(t)}{2} = \frac{-x_0+1}{1+T}$. 

\end{solution}

\begin{problem}
	Розв'язати задачу оптимального керування варіаційним методом: \[ \JJ(u) = \int_0^T u^2(s) \diff s + (x(T) + 2)^2 \to \inf\] за умови, що \[ \frac{\diff x(t)}{\diff t} = x(t) + u(t), \quad x(0) = x_0.\] Тут $x(t) \in \RR^1$, $u(t)\in\RR^1$, $t\in[0,T]$. Точки $x_0\in\RR^1$ і момент часу $T$ є заданими.
\end{problem}

\begin{solution}
	Застосовуємо вже добре відомий алгоритм:
	\begin{enumerate}
		\item $\phi(\alpha) = \JJ(u + \alpha h) = \int_0^T (u + \alpha h)^2(s) \diff s + (x(T, \alpha) + 2)^2$.
		\item $\phi'(\alpha) = \int_0^T 2 h (s) (u + \alpha h)(s) \diff s + 2 (x(T, \alpha) + 2) \frac{\partial x(T, \alpha)}{\partial \alpha}$.
		\item $\phi'(0) = \int_0^T 2 h (s) u(s) \diff s + \underset{=-\psi(T)}{\underbrace{2 (x(T) + 2)}} z(T)$.
		\item (рівняння у варіаціях): $z' = z + h$, $z(0) = 0$;
		\item \begin{multline*}
			\psi(T) z (T) = ... = \int_0^T \psi' (s) z(s) \diff s + \int_0^T \psi(s) (z(s) + h(s)) \diff s = \\
			= \int_0^T z(s) (\psi'(s) + \psi(s)) \diff s + \int_0^T h(s) \psi(s) \diff s.
		\end{multline*}
		\item \begin{multline*}
			\delta \JJ(u, \alpha) = \int_0^T 2 h(s) u(s) \diff s - \int_0^T z(s) (\psi'(s) + \psi(s)) \diff s - \\
			- \int_0^T h(s) \psi(s) \diff s = ... + \int_0^T (2u(s)-\psi(s)) h(s) \diff s.
		\end{multline*}
		\item $\phi'(0) = \int_0^T (2u(s)-\psi(s)) h(s) \diff s$.
		\item $\JJ'(u) = 2 u (\cdot) - \psi(\cdot)$;
		\item ...
	\end{enumerate}
\end{solution}

\begin{problem}
	% 6.7
\end{problem}

\begin{solution}
	% 6.7
\end{solution} \newpage

\subsection{Домашнє завдання}

\begin{problem}
	Знайти першу варіацію за Лагранжем і похідну Фреше в просторі інтегрованмх з квадратом функцій для функціоналів:
	\begin{enumerate}
		\item $\JJ(u) = \int_0^T \cos(u(s)) \diff s$;
		\item $\JJ(u) = \int_0^T (s^2 u_1^4(s) + u_2^2(s)) \diff s$, $u=(u_1,u_2)^*$.
	\end{enumerate}
\end{problem}

\begin{solution}
	\begin{enumerate}
		\item Перша варіація за Лагранжем:
		\begin{multline*} 
			\delta \JJ(u, \psi) = \frac{\diff}{\diff \alpha} \left.\JJ(u + \alpha \psi)\right|_{\alpha = 0} = \frac{\diff}{\diff \alpha} \left.\int_0^T \cos(u(s) + \alpha \psi(s)) \diff s\right|_{\alpha = 0} = \\
			= \left.\int_0^T - \psi(s) \sin(u(s) + \alpha \psi(s)) \diff s\right|_{\alpha = 0} = - \int_0^T \psi(s) \sin(u(s)) \diff s.
		\end{multline*}
		Як наслідок, похідна за Фреше $\JJ'(u) = - \sin u(\cdot)$.

		\item \begin{multline*} 
			\delta \JJ(u, \psi) = \frac{\diff}{\diff \alpha} \left.\JJ(u + \alpha \psi)\right|_{\alpha = 0} = \\
			= \frac{\diff}{\diff \alpha} \left.\int_0^T (s^2 (u_1 + \alpha \psi_1)^4(s) + (u_2 + \alpha \psi_2)^2(s)) \diff s\right|_{\alpha = 0} = \\
			= \left.\int_0^T (4 s^2 \psi_1(s) (u_1 + \alpha \psi_1)^3(s) + 2 \psi_2(s) (u_2 + \alpha \psi_2)(s)) \diff s\right|_{\alpha = 0} = \\
			= \int_0^T (4 s^2 \psi_1(s) u_1^3(s) + 2 \psi_2(s) u_2 (s)) \diff s.
		\end{multline*}
	\end{enumerate}
\end{solution}

\begin{problem}
	Побудувати рівняння у варіаціях для системи керування \[ \frac{\diff x(t)}{\diff t} = \cos(x(t) + u(t)), \quad x(0) = x_0. \] Тут $x(t)\in\RR^1$, $u(t)\in\RR^1$, $t\in[0,T]$. Точки $x_0\in\RR^1$ і момент часу $T$ є заданими.
\end{problem}

\begin{solution}
	Рівняння у варіаціях має загальний вигляд \[ \frac{\diff z(t)}{\diff t} = \frac{\partial f(x(t),u_*(t),t)}{\partial x} \cdot z(t) + \frac{\partial f(x(t),u_*(t),t)}{\partial u} \cdot h(t), \quad z(0) = 0. \] У нашій задачі \[f(x(t), u(t), t) = \cos(x(t) + u(t)), \] тому маємо \[ \frac{\diff z(t)}{\diff t} = -\sin(x(t) + u(t)) \cdot z(t) -\sin(x(t) + u(t)) \cdot h(t), \quad z(0) = 0. \]
\end{solution}

\begin{problem}
	Побудувати рівняння у варіаціях для системи керування \[ \left\{ \begin{aligned}
		\frac{\diff x_1(t)}{\diff t} &= x_1(t)\cdot x_2(t) + u_1(t), \\
		\frac{\diff x_2(t)}{\diff t} &= x_1(t) - x_2(t) \cdot u_2(t),
	\end{aligned} \right. \]
	\[ x_1(0) = -1, x_2 (0) = 4. \]

	Тут $x = (x_1, x_2)^*$ -- вектор фазових координат з $\RR^2$, $u = (u_1, u_2)^*$, $t\in[0, T]$, момент часу $T$ є заданим.
\end{problem}

\begin{solution}
	Рівняння у варіаціях має загальний вигляд \[ \frac{\diff z(t)}{\diff t} = \frac{\partial f(x(t),u_*(t),t)}{\partial x} \cdot z(t) + \frac{\partial f(x(t),u_*(t),t)}{\partial u} \cdot h(t), \quad z(0) = 0. \] У нашій задачі \[f(x(t), u(t), t) = \begin{pmatrix} x_1 \cdot x_2 + u_1 \\ x_1 - x_2 \cdot u_2 \end{pmatrix}, \] тому маємо \[ \frac{\diff z(t)}{\diff t} = \begin{pmatrix} x_2 & x_1 \\ 1 & -u_2 \end{pmatrix} \cdot z(t) + \begin{pmatrix} 1 & 0 \\ 0 & - x_2 \end{pmatrix} \cdot h(t), \quad z(0) = 0, \] або, у розгорнутому вигляді, \[ \left\{ \begin{aligned}
		\dot z_1 &= x_2 z_1 + x_1 z_2 + h_1, \\
		\dot z_2 &= z_1 - u_2 z_2 - x_2 h_2, \\
		0 &= z_1(0) = z_2(0).
	\end{aligned} \right. \]
\end{solution}

\begin{problem}
	Знайти першу варіацію за Лагранжем і похідну Фреше в просторі інтегрованих з квадратом функцій для задачи оптимального керування варіаційним методом \[ \JJ(u) = \int_0^T (u^2(s) + x^4(s)) \diff s + x^4(T) \to \inf \] за умови, що \[ \frac{\diff x(t)}{\diff t} = x(t) \cdot u(t), \quad x(0) = x_0. \] Тут $x(t) \in \RR^1$, $u(t)\in\RR^1$, $t\in[0,T]$. Точки $x_0\in\RR^1$ і момент часу $T$ є заданими.
\end{problem}

\begin{solution}
	Перш за все запишемо
	\[ \phi(\alpha) = \JJ(u + \alpha h) = \int_0^T ((u + \alpha h)^2(s) + x^4(s, \alpha)) \diff s + x^4(T, \alpha). \]

	Далі, \[ \phi'(\alpha) = \int_0^T (2 h(s) \cdot (u(s) + \alpha h(s)) + 4 x^3(s, \alpha) x_\alpha'(s, \alpha)) \diff s + 4 x^3 (T, \alpha) x_\alpha'(T, \alpha). \]

	Підставимо $\alpha = 0$: \[ \delta \JJ(u, h) = \int_0^T (2 h(s) u(s) + 4 x^3(s) z(s)) \diff s + \underset{=-\psi(T)}{\underbrace{4 x^3(T)}} \cdot z(T).\] 

	Тоді рівняння у варіаціях \[ \left\{ \begin{aligned} 
		z' &= u \cdot z + x \cdot h, \\
		z(0) &= 0.
	\end{aligned} \right. \]

	\begin{multline*} 
		\psi(T) \cdot z(T) = \psi(T) \cdot z(T) - \psi(0) \cdot z(0) = \int_0^T (\psi(s) \cdot z(s))' \diff s = \\
		= \int_0^T (\psi'(s) \cdot z(s) + \psi(s) \cdot z'(s)) \diff s = \\
		= \int_0^T \phi'(s) z(s) + \phi(s) (u(s) z(s) + x(s) h(s)) = \\
		= \int_0^T \phi'(s) z(s) u(s) z(s) \diff s + \int_0^T \psi(s) x(s) h(s) \diff s.
	\end{multline*}

	\begin{multline*} 
		\delta \JJ(u, h) = \int_0^T 2 h(s) u(s) + 4 x^3(s) z(s) \diff s - \\
		- \int_0^T \psi'(s) z(s) u(s) z(s) \diff s - \int_0^T \psi(s) x(s) h(s) \diff s = \\
		= \int_0^T (\psi'(s) z(s) + \psi(s) x(s) h(s) + u(s) z(s)) \diff s.
	\end{multline*}

	\[ \int_0^T z(s) (\psi'(s) + \psi(s) u(s)) \diff s + \int_0^T h(s) \psi(s) x(s( \diff s + \int)) \]

	\[ ... ??? ... \]

	% Тоді спряжена система \[ \left\{ \begin{aligned} 
	% 	[]
	% \end{aligned} \right. \]

	% Остаточно, $\JJ'(u) = \psi(s) s$.
\end{solution}

\begin{problem}
	Розв'язати задачу оптимального керування варіаційним методом: \[ \JJ (u) = \int_0^T (u (s) - v (s))^2 \diff s + (x (T) - 3)^2 \to \inf \] за умови, що \[ \frac{\diff x (t)}{\diff t} = u(t), \quad x(0) = x_0. \] Тут $x (t) \in \RR^1$, $u (t) \in \RR^1$, $t \in [0, T]$. Точки $x_0 \in \RR^1$, момент часу $T$ і функція $v (t) \in \RR^1$ є заданими.
\end{problem}

\begin{solution}
	Нагадаємо постановку задачі варіаційного методу: \[ \JJ (u) = \int_{t_0}^T f_0 (x(t), u(t), t) \diff t + \Phi(x(T)) \to \inf, \] \[ \frac{\diff x (t)}{\diff t} = f(x(t), u(t), t), \quad x(t_0) = x_0. \]

	Спочатку випишемо всі функції з теоретичної частини які фігурують в задачі: 
	\begin{align*}
		f_0(x(t), u(t), t) &= (u (t) - v (t))^2, \\
		\Phi(x(T)) &= (x (T) - 3)^2, \\
		f(x(t), u(t), t) &= u(t).
	\end{align*}

	Позначимо \[ \phi (\alpha) = \JJ (u + \alpha h) = \int_0^T ((u + \alpha h) (s) - v (s))^2 \diff s + (x (T, \alpha) - 3)^2. \]

	Необхідна умова екстремуму через першу варіацію функціоналу має вигляд $\delta \JJ (u_*, h) = \phi' (0) = 0$, тому знайдемо \[ \phi' (\alpha) = \int_0^T \left( 2 h (s) \cdot ((u + \alpha h) (s) - v (s)) \right) \diff s + 2 (x (T, \alpha) - 3) \cdot \underset{=z(T)}{\underbrace{\frac{\partial x (T, \alpha)}{\partial \alpha}}}. \]

	Підставляючи $\alpha = 0$, знаходимо \[ \phi' (0) = \int_0^T \left( 2 h (s) \cdot (u(s) - v (s)) \right) \diff s + \underset{=-\psi(T)}{\underbrace{2 (x (T) - 3)}} \cdot z (T). \]

	Запишемо рівняння у варіаціях на функцію $z(t)$. Його загальний вигляд \[ \frac{\diff z(t)}{\diff t} = \frac{\partial f(x(t), u(t), t)}{\partial x} \cdot z(t) + \frac{\partial f(x(t), u(t), t)}{\partial u} \cdot h(t), \quad z(t_0) = 0. \]

	У контексті нашої задачі маємо \[ \frac{\diff z(t)}{\diff t} = 0 \cdot z(t) + 1 \cdot h(t), \quad z(0) = 0. \]

	Введемо додаткові, спряжені змінні $\psi$ такі, що \[ \psi (T) = - \frac{\partial \Phi (x (T))}{\partial x}. \] 

	Тоді $\left\langle \frac{\partial \Phi (x (T))}{\partial x}, z (T) \right\rangle = - \langle \psi (T), z (T) \rangle$ (у контексті нашої задачі ``скалярний'' добуток зайвий бо функції і так скалярні). Враховуючи рівняння у варіаціях, маємо
	\begin{align*}
		\psi (T) \cdot z (T) &= \psi (T) \cdot z (T) - \psi (t_0) \cdot z (t_0) = \\
		&= \int_{t_0}^T \left( \psi (s) \cdot z' (s) + \psi' (s) \cdot z (s) \right) \diff s = \\
		&= \int_0^T \left( \psi (s) \cdot h (s) + \psi' (s) \cdot z (s) \right) \diff s.
	\end{align*}

	Підставимо це у вигляд $\phi' (0)$:
	\begin{align*}
		\phi' (0) &= \int_{t_0}^T \left( \frac{\partial f_0 (x (t), u (t), t)}{\partial x} \cdot z(t) + \frac{\partial f_0 (x (t), u (t), t)}{\partial u} \cdot h(t) \right) + \\
		& \left.\right. \quad + \frac{\partial \Phi (x (T))}{ \partial x} \cdot z (T) = \\
		&= \int_{t_0}^T \left( \frac{\partial f_0 (x (t), u (t), t)}{\partial x} \cdot z(t) + \frac{\partial f_0 (x (t), u (t), t)}{\partial u} \cdot h(t) \right) - \\
		& \left.\right. \quad - \int_0^T \left( \psi (s) \cdot h (s) + \psi' (s) \cdot z (s) \right) \diff s = \\
		&= \int_0^T \left( 2 h (s) \cdot (u(s) - v (s)) \right) \diff s - \\
		& \left.\right. \quad - \int_0^T \left( \psi (s) \cdot h (s) + \psi' (s) \cdot z (s) \right) \diff s = \\
		&= \int_0^T - \psi'(s) \cdot z(s) \diff s + \int_0^T (2 (u(s) - v(s)) - \psi(s)) \cdot h(s) \diff s.
	\end{align*}

	Накладаємо на функцію $\psi(t)$ умову (спряжену систему) \[ \frac{\diff \psi (t)}{\diff t} = - \frac{\partial f (x (t), u (t), t)}{\partial x} \cdot \psi (t) + \frac{\partial f_0 (x (t), u (t), t)}{\partial x} = 0, \] \[ \psi(T) = - \frac{\partial \Phi( x (T))}{\partial x} = 2 (x (T) - 3), \] звідки $\psi (t) = 2 (x (T) - 3)$. \\

	Завдяки цьому \[ \delta \JJ (u, h) = \phi' (0) = \int_0^T (2 (u(s) - v(s)) - \psi(s)) \cdot h(s) \diff s. \]

	Як наслідок, \[ \JJ ' (u) = 2 (u (\cdot) - v (\cdot)) - \psi(\cdot)). \]

	Пригадуючи необхідну умову екстремуму функціоналу, знаходимо \[ u_* (t) = v (t) + \psi (t) / 2 = v (t) + x (T) - 3. \]

	Далі \begin{multline*} 
		x_*(t) = x_0 + \int_0^t f(x(s), u_*(s), s) \diff s = \\
		= x_0 + \int_0^t (v(s) + x(T) - 3) \diff s = t x (T) - 3 t + \int_0^t v(s) \diff s.
	\end{multline*}

	покладаючи $t = T$ знаходимо \[ x (T) = T x (T) - 3 T + \int_0^T v(s) \diff s, \] звідки \[ x(T) = \frac{\int_0^T v(s) \diff s - 3  T}{1 - T}, \] і остаточно \[ u_* (t) = v (t) + \frac{\int_0^T v(s) \diff s - 3  T}{1 - T} - 3, \] \[ x_* (t) =  \frac{t \cdot \left(\int_0^T v(s) \diff s - 3  T\right)}{1 - T} - 3 t + \int_0^t v(s) \diff s .\]
\end{solution}

\begin{problem}
	% 6.13
\end{problem}

\begin{solution}
	% 6.13
\end{solution}

\begin{problem}
	% 6.14
\end{problem}

\begin{solution}
	% 6.14
\end{solution}

\begin{problem}
	% 6.15
\end{problem}

\begin{solution}
	% 6.15
\end{solution}
 \newpage

\section{Принцип максимуму Понтрягіна для задачі з вільним правим кінцем}

\subsection{Алгоритми}

\begin{problem*}
    Записати крайову задачу принципу максимуму для задачі оптимального керування: \[ \JJ = \int f \diff s + \Phi(T) \to \inf \] за умови, що \[ \dot x = f_0. \] Розв'язати задачу оптимального керування.
\end{problem*}

\begin{algorithm} \tt
    \begin{enumerate}
        \item Записуємо функцію Гамільтона-Понтрягіна: \[ \mathcal{H} (x, u, \psi, t) = - f_0(x, u, t) + \langle \psi, f(x, u, t) \rangle. \]
    
        \item Записуємо спряжену систему: \[ \dot \psi = - \nabla_x \mathcal{H}, \quad \psi(T) = - \nabla \Phi(x(T)). \]
    
    
        \item Знаходимо $u(\psi)$ з умови оптимальності: \[ \dfrac{\partial \mathcal{H}(x, u, \psi, t)}{\partial u} = 0. \]

        \item Підставляємо знайдене керування у початкову систему, от\-ри\-ма\-ли \allowbreak край\-о\-ву задачу, систему диференціальних рівнянь на $x$ і $\psi$ з гра\-нич\-ни\-ми  \allowbreak у\-мо\-ва\-ми.

        \item Розв'язуємо крайову задачу і знаходимо $x$.

        \item Відновлюємо $u = u (\psi)$ за знайденим $\psi$.
    \end{enumerate}
\end{algorithm}

\newpage

\subsection{Аудиторне заняття}

\begin{problem}
    Записати крайову задачу принципу максимуму для задачі оптимального керування:
    \begin{equation*}
        \JJ(u) = \int_0^T (u^2(s) + x_1^4(s)) \diff s + x_2^4(T) \to \inf
    \end{equation*}
    за умови, що
    \[ \left\{ \begin{aligned}
        \dot x_1 &= \sin(x_1 - x_2) + u, \\
        \dot x_2 &= \cos(-4x_1 + x_2),
    \end{aligned} \right. \]
    \begin{equation*}
        x_1(0) = 1, x_2(0) = 2.
    \end{equation*}
    Тут $x = (x_1, x_2)^*$ -- вектор фазових координат з $\RR^2$, $u(t)$ -- функція керування, $t \in [0, T]$, момент часу $T$ є заданим.
\end{problem}

\begin{solution}
    Для початку випишемо всі функції з теоретичної частини:
    \begin{multline}
        f_0(x, u, t) = u^2(t) + x_1^4(t), \\ f(x(t), u(t), t) = \begin{pmatrix} \sin(x_1(t) - x_2(t)) + u(t) \\ \cos(-4x_1(t) + x_2(t)) \end{pmatrix}, \\ \Phi(x(T)) = x_2^4(T).
    \end{multline}

    Функція Гамільтона-Понтрягіна має вигляд
    \begin{equation}
        \begin{aligned}
        \mathcal{H} (x, u, \psi, t) &= - f_0(x, u, t) + \langle \psi, f(x, u, t) \rangle = \\
        &= - u^2 - x_1^4 + \langle \psi, f(x, u, t) \rangle = \\
        &= - u^2 - x_1^4 + \left\langle \psi, \begin{pmatrix} \sin(x_1 - x_2) + u \\ \cos(-4x_1 + x_2) \end{pmatrix} \right\rangle = \\
        &= - u^2 - x_1^4 + \left\langle \begin{pmatrix} \psi_1 \\ \psi_2 \end{pmatrix}, \begin{pmatrix} \sin(x_1 - x_2) + u \\ \cos(-4x_1 + x_2) \end{pmatrix} \right\rangle = \\
        &= - u^2 - x_1^4 + \psi_1 \cdot \sin(x_1 - x_2) + \psi_1 \cdot u + \psi_2 \cdot \cos(-4x_1 + x_2).
        \end{aligned}
    \end{equation}
    
    Спряжена система записується так:
    \begin{equation} 
        \dot \psi = - \nabla_x \mathcal{H} = \begin{pmatrix} 4 x_1^3 - \psi_1 \cdot \cos(x_1 - x_2) - 4 \psi_2 \cdot \sin(-4x_1 + x_2) \\ \psi_1 \cdot \cos(x_1 - x_2) + \psi_2 \cdot \sin(-4x_1 + x_2) \end{pmatrix},
    \end{equation}
    \begin{equation} 
        \psi(T) = - \nabla \Phi(x(T)) = \begin{pmatrix} 0 \\ - 4 x_2^3(T) \end{pmatrix}.
    \end{equation}
    
    Згідно принципу максимуму, функція Гамільтона-Понтрягіна на оптимальному керуванні досягає свого максимуму, тобто, за відсутності обмежень на керування
    \begin{equation} 
        \dfrac{\partial \mathcal{H}(x, u, \psi, t)}{\partial u} = -2 u + \psi_1 = 0,
    \end{equation}
    звідки $u = \psi_1 / 2$. Підставляємо знайдене керування у початкову систему:
    \[ \left\{ \begin{aligned}
        \dot x_1 &= \sin(x_1 - x_2) + \psi_1 / 2, \\
        \dot x_2 &= \cos(-4x_1 + x_2),
    \end{aligned} \right. \]
\end{solution}

\begin{problem}
    Записати крайову задачу принципу максимуму для задачі оптимального керування:
    \begin{equation*}
        \JJ(u) = \gamma^2 \int_0^T x^2(s) \diff s \to \inf
    \end{equation*}
    за умови, що
    \begin{equation*}
        \dot x = u, \quad x(0) = x_0
    \end{equation*}
    Тут $x(t) \in \RR^1$, $u(t) \in \RR^1$,
    \begin{equation*}
        |u(t)|\le \rho,
    \end{equation*}
    $t \in [0, T]$. Точка $x_0 \in \RR^1$ і момент часу $T$ є заданими.
\end{problem}

\begin{solution}
    Для початку випишемо всі функції з теоретичної частини:
    \begin{equation}
        f_0(x, u, t) = \gamma^2 x^2(t), \quad f(x(t), u(t), t) = u(t), \quad \Phi(x(T)) = 0, \quad \mathcal{U} = \mathcal{U}(t) = [-\rho, \rho].
    \end{equation}
    
    Функція Гамільтона-Понтрягіна має вигляд
    \begin{equation}
        \mathcal{H} (x, u, \psi, t) = - f_0(x, u, t) + \langle \psi, f(x, u, t) \rangle = - \gamma^2 x^2 + \psi u.
    \end{equation}
    
    Спряжена система записується так:
    \begin{equation} 
        \dot \psi = - \nabla_x \mathcal{H} = - 2 \gamma^2 x,
    \end{equation}
    \begin{equation} 
        \psi(T) = - \nabla \Phi(x(T)) = 0,
    \end{equation}
    % її розв'язок 
    % \begin{equation}
    %     \psi(t) = - 2 \gamma^2 x(t) t.
    % \end{equation}
    
    Згідно принципу максимуму, функція Гамільтона-Понтрягіна на оптимальному керуванні досягає свого максимуму, тобто
    \begin{equation} 
        u = \rho \cdot \signum \psi.
    \end{equation} 
    Підставляємо знайдене керування у початкову систему:
    \begin{equation}
        \dot x = \rho \cdot \signum \psi.
    \end{equation}
\end{solution}

\begin{problem}
    Розв'язати задачу оптимального керування за допомогою принципу максимуму Понтрягіна:
    \begin{equation*}
        \JJ(u) = \dfrac12 \int_0^T u^2(s) \diff s + \dfrac{x^2(T)}{2} \to \inf
    \end{equation*}
    за умови, що
    \begin{equation*}
        \dot x = u, x(0) = x_0
    \end{equation*}
    Тут $x(t) \in \RR^1$, $u(t) \in \RR^1$, $t \in [0, T]$. Точка $x_0 \in \RR^1$ і момент часу $T$ є заданими.
\end{problem}

\begin{solution}
    Функція Гамільтона-Понтрягіна має вигляд
    \begin{equation}
        \mathcal{H} (x, u, \psi, t) = - f_0(x, u, t) + \langle \psi, f(x, u, t) \rangle = - \dfrac{u^2}{2} + \psi u.
    \end{equation}
    
    Спряжена система записується так:
    \begin{equation} 
        \dot \psi = - \nabla_x \mathcal{H} = 0,
    \end{equation}
    \begin{equation} 
        \psi(T) = - \nabla \Phi(x(T)) = - x(T),
    \end{equation}
    
    Згідно принципу максимуму, функція Гамільтона-Понтрягіна на оптимальному керуванні досягає свого максимуму, тобто, за відсутності обмежень на керування
    \begin{equation} 
        \dfrac{\partial \mathcal{H}(x, u, \psi, t)}{\partial u} = - u + \psi = 0,
    \end{equation}
    звідки $u = \psi$. Підставляємо знайдене керування у початкову систему:
    \begin{equation}
        \dot x = \psi.
    \end{equation}
    
    З рівнянь на $\psi$ знаходимо $\psi(t) = - x(T)$. \\
    
    Підставляючи це у рівняння на $x$ знаходимо $x(t) = x_0 - x(T) \cdot t$. \\
    
    Звідси, при $t = T$ маємо $x(T) = x_0 - T \cdot x(T)$, тобто $x(T) = \dfrac{x_0}{1 + T}$. \\
    
    Остаточно, $(u_*(t), x_*(t)) = \left(- \dfrac{x_0}{1 + T}, - \dfrac{x_0 \cdot t}{1 + T}\right)$.
    
\end{solution}

\begin{problem}
    Розв'язати задачу оптимального керування за допомогою принципу максимуму Понтрягіна:
    \begin{equation*}
        \JJ(u) = \dfrac 12 \int_0^T (u_1^2(s) + u_2^2(s) \diff s + \dfrac{x_1^2(T)}{2} \to \inf
    \end{equation*}
    \[ \left\{ \begin{aligned}
        \dot x_1 &= x_2 + u_1, \\
        \dot x_2 &= x_1 + u_2,
    \end{aligned} \right. \]
    \begin{equation*}
        x_1(0) = 1, x_2(0) = 1.
    \end{equation*}
    Тут $x = (x_1, x_2)^*$ -- вектор фазових координат з $\RR^2$, $u = (u_1, u_2)^* \in \RR^2$ -- вектор керування, $t \in [0, T]$, момент часу $T$ є заданим.
\end{problem}

\begin{solution}
    Функція Гамільтона-Понтрягіна має вигляд
    \begin{equation}
        \mathcal{H} (x, u, \psi, t) = - f_0(x, u, t) + \langle \psi, f(x, u, t) \rangle = - \dfrac{u_1^2 + u_2^2}{2} + \psi_1(x_2 + u_1) + \psi_2(x_1 + u_2).
    \end{equation}
    
    Спряжена система записується так:
    \begin{equation} 
        \dot \psi = - \nabla_x \mathcal{H} = \begin{pmatrix} \psi_2 \\ \psi_1 \end{pmatrix},
    \end{equation}
    \begin{equation} 
        \psi(T) = - \nabla \Phi(x(T)) = \begin{pmatrix} x_1(T) \\ 0 \end{pmatrix}.
    \end{equation}
    
    Згідно принципу максимуму, функція Гамільтона-Понтрягіна на оптимальному керуванні досягає свого максимуму, тобто, за відсутності обмежень на керування
    \begin{equation} 
        \dfrac{\partial \mathcal{H}(x, u, \psi, t)}{\partial u} = \begin{pmatrix} - u_1 + \psi_1 \\ -u_2 + \psi_2 \end{pmatrix} = \begin{pmatrix} 0 \\ 0 \end{pmatrix},
    \end{equation}
    звідки $u_1 = \psi_1$, $u_2 = \psi_2$. Підставляємо знайдене керування у початкову систему:
    \[ \left\{ \begin{aligned}
        \dot x_1 &= x_2 + \psi_1, \\
        \dot x_2 &= x_1 + \psi_2, \\
        \dot \psi_1 &= - \psi_2, \\
        \dot \psi_2 &= - \psi_1.
    \end{aligned} \right. \]
    
    Розв'язуємо систему на $\psi_1$ і $\psi_2$:
    \begin{equation}
        \begin{pmatrix} \psi_1 \\ \psi_2 \end{pmatrix} = c_1 \begin{pmatrix} 1 \\ -1 \end{pmatrix} e^t + c_2 \begin{pmatrix} 1 \\ 1 \end{pmatrix} e^{-t}.
    \end{equation}
    
    Підставляємо це у систему на $x_1$, $x_2$:
    \[ \left\{ \begin{aligned}
        \dot x_1 &= x_2 + c_1 e^t + c_2 e^{-t}, \\
        \dot x_2 &= x_1 - c_1 e^t + c_2 e^{-t}.
    \end{aligned} \right. \]
    
    Знаходимо загальний розв'язок однорідної системи:
    \begin{equation}
        \begin{pmatrix} x_1 \\ x_2 \end{pmatrix} = c_3 \begin{pmatrix} 1 \\ 1 \end{pmatrix} e^t + c_4 \begin{pmatrix} 1 \\ -1 \end{pmatrix} e^{-t}.
    \end{equation}
    
    Шукаємо тепер частинний розв'язок неоднорідної системи методом невизначених коефіцієнтів у вигляді:
    \begin{equation}
        \begin{pmatrix} x_1(t) \\ x_2(t) \end{pmatrix} = \begin{pmatrix} e^t (A_{11} t + B_{11}) + e^{-t} (A_{12} t + B_{12}) \\ e^t (A_{21} t + B_{21}) + e^{-t} (A_{22} t + B_{22})  \end{pmatrix}
    \end{equation}
    Підставляючи це у системи, знаходимо
    \[ \left\{ \begin{aligned}
        A_{11} &= A_{21}, \\
        B_{11} + A_{11} &= B_{21} - c_1, \\
        -A_{12} &= A_{22}, \\
        - B_{12} + A_{12} &= B_{22} + c_2, \\
        A_{21} &= A_{11}, \\
        B_{21} + A_{21} &= B_{11} + c_1, \\
        -A_{22} &= A_{12}, \\
        - B_{22} + A_{22} &= B_{12} + c_2.
    \end{aligned} \right. \]
    
    Беремо розв'язок $A_{11} = A_{12} = A_{21} = A_{22} = 0$, $B_{11} = B_{12} = 0$, $B_{21} = c_1$, $B_{22} = c_2$, тоді
    \begin{equation}
        \begin{pmatrix} x_1 \\ x_2 \end{pmatrix} = c_3 \begin{pmatrix} 1 \\ 1 \end{pmatrix} e^t + c_4 \begin{pmatrix} 1 \\ -1 \end{pmatrix} e^{-t} + \begin{pmatrix} - c_2 e^{-t} \\ c_1 e^t \end{pmatrix}.
    \end{equation}
    Пригадаємо, що $x_1(0) = x_2(0) = 1$, це дає систему
    \[ \left\{ \begin{aligned}
        c_3 + c_4 - c_2 &= 1, \\
        c_3 - c_4 + c_1 &= 1,
    \end{aligned} \right. \]

    з якої $c_3 = 1 - \dfrac{c_1}{2} + \dfrac{c_2}{2}$, $c_4 = \dfrac{c_1}{2} + \dfrac{c_2}{2}$. \\
    
    І далі це якось (як ??) розв'язується.
\end{solution}

\begin{problem}
    % 7.5
\end{problem}

\begin{solution}
    % 7.5
\end{solution}

\begin{problem}
    % 7.6
\end{problem}

\begin{solution}
    % 7.6
\end{solution}
 \newpage

\subsection{Домашнє завдання}

\begin{problem}
    Записати крайову задачу принципу максимуму для задачі оптимального керування:
    \begin{equation*}
        \JJ(u) = \int_0^T (4 u_1^2(s) + u_2^2(s) + \cos^2(x_1(s))) \diff s + \sin^2(x_2(T)) \to \inf
    \end{equation*}
    за умови, що
    \[ \left\{ \begin{aligned}
        \dot x_1 &= x_1 + x_2 + 3 x_1 x_2 + 2 u_1, \\
        \dot x_2 &= - x_1 + 6 x_2 - 3 x_1 x_2 + u_2,
    \end{aligned} \right. \]
    \begin{equation*}
        x_1(0) = 4, x_2(0) = -2.
    \end{equation*}
    Тут $x = (x_1, x_2)^*$ -- вектор фазових координат з $\RR^2$, $u_1(t), u_2(t)$ -- функції керування, $t \in [0, T]$, момент часу $T$ є заданим.
\end{problem}

\begin{solution}
    Для початку випишемо всі функції з теоретичної частини:
    \begin{equation}
        \begin{aligned}
            f_0(x, u, t) &=  4 u_1^2(s) + u_2^2(s) + \cos^2(x_1(s)), \\
            f(x(t), u(t), t) &= \begin{pmatrix} x_1 + x_2 + 3 x_1 x_2 + 2 u_1 \\ - x_1 + 6 x_2 - 3 x_1 x_2 + u_2 \end{pmatrix}, \\
            \Phi(x(T)) &= \sin^2(x_2(T)).
        \end{aligned}
    \end{equation}

    Функція Гамільтона-Понтрягіна має вигляд
    \begin{equation}
        \begin{aligned}
        \mathcal{H} (x, u, \psi, t) &= - f_0(x, u, t) + \langle \psi, f(x, u, t) \rangle = \\
        &= - 4 u_1^2(t) - u_2^2(t) - \cos^2(x_1(t)) + \langle \psi, f(x, u, t) \rangle = \\
        &= - 4 u_1^2(t) - u_2^2(t) - \cos^2(x_1(t)) + \\
        &+ \left \langle \psi, \begin{pmatrix} x_1 + x_2 + 3 x_1 x_2 + 2 u_1 \\ - x_1 + 6 x_2 - 3 x_1 x_2 + u_2 \end{pmatrix} \right \rangle = \\
        &= - 4 u_1^2(t) - u_2^2(t) - \cos^2(x_1(t)) + \\
        &+ \left \langle \begin{pmatrix} \psi_1 \\ \psi_2 \end{pmatrix}, \begin{pmatrix} x_1 + x_2 + 3 x_1 x_2 + 2 u_1 \\ - x_1 + 6 x_2 - 3 x_1 x_2 + u_2 \end{pmatrix} \right \rangle = \\
        &= - 4 u_1^2(t) - u_2^2(t) - \cos^2(x_1(t)) + \\
        &+ \psi_1 ( x_1 + x_2 + 3 x_1 x_2 + 2 u_1 ) + \psi_2 ( - x_1 + 6 x_2 - 3 x_1 x_2 + u_2 ).
        \end{aligned}
    \end{equation}
    
    Спряжена система записується так:
    \begin{equation} 
        \dot \psi = - \nabla_x \mathcal{H} = \begin{pmatrix} \sin (2 x_1) + \psi_1 + 3 \psi_1 x_2 - \psi_2 - 3 \psi_2 x_2 \\ \psi_1 + 3 \psi_1 x_1 + 6 \psi_2 - 3 \psi_2 x_1 \end{pmatrix},
    \end{equation}
    \begin{equation} 
        \psi(T) = - \nabla \Phi(x(T)) = \begin{pmatrix} 0 \\ \sin (2 x_2) \end{pmatrix}.
    \end{equation} 
    
    Згідно принципу максимуму, функція Гамільтона-Понтрягіна на оптимальному керуванні досягає свого максимуму, тобто, за відсутності обмежень на керування
    \begin{equation} 
        \dfrac{\partial \mathcal{H}(x, u, \psi, t)}{\partial u} = \begin{pmatrix} - 8 u_1 + 2 \psi_1 \\ - 2 u_2 + \psi_2 \end{pmatrix} = \begin{pmatrix} 0 \\ 0 \end{pmatrix},
    \end{equation}
    звідки $u_1 = \psi_1 / 4$, $u_2 = \psi_2 / 2$. Підставляємо знайдене керування у початкову систему:
    \[ \left\{ \begin{aligned}
        \dot x_1 &= x_1 + x_2 + 3 x_1 x_2 + \psi_1 / 2, \\
        \dot x_2 &= - x_1 + 6 x_2 - 3 x_1 x_2 + \psi_2 / 2.
    \end{aligned} \right. \]
\end{solution}

\begin{problem}
    Записати крайову задачу принципу максимуму для задачі оптимального керування:
    \begin{equation*}
        \JJ(u) = \gamma^2 \int_0^T (x(s) - z(s))^2 \diff s \to \inf
    \end{equation*}
    за умови, що
    \begin{equation*}
        \dot x = u, x(0) = x_0.
    \end{equation*}
    Тут $x(t) \in \RR^1$, $u(t) \in \RR^1$,
    \begin{equation*}
        |u(t)| \le \rho,
    \end{equation*}
    $t \in [0, T]$. Точка $x_0 \in \RR^1$, неперервна функція $z(t) \in \RR^1$ і момент часу $T$ є заданими.
\end{problem}

\begin{solution}
    Для початку випишемо всі функції з теоретичної частини:
    \begin{equation}
        f_0(x, u, t) = \gamma^2 (x - z)^2, \quad f(x(t), u(t), t) = u(t), \quad \Phi(x(T)) = 0, \quad \mathcal{U} = \mathcal{U}(t) = [-\rho, \rho].
    \end{equation}
    
    Функція Гамільтона-Понтрягіна має вигляд
    \begin{equation}
        \mathcal{H} (x, u, \psi, t) = - f_0(x, u, t) + \langle \psi, f(x, u, t) \rangle = - \gamma^2 (x - z)^2 + \psi u.
    \end{equation}
    
    Спряжена система записується так:
    \begin{equation}
        \dot \psi = - \nabla_x \mathcal{H} = - 2 \gamma^2 x + 2 \gamma^2 z,
    \end{equation}
    \begin{equation} 
        \psi(T) = - \nabla \Phi(x(T)) = 0,
    \end{equation}
    % її розв'язок 
    % \begin{equation}
    %     \psi (t) = 2 \gamma^2 (z(t) - x(t)) t,
    % \end{equation}
    % причому, з крайової умови,
    % \begin{equation}
    %     x(T) = z(T).
    % \end{equation}
    
    Згідно принципу максимуму, функція Гамільтона-Понтрягіна на оптимальному керуванні досягає свого максимуму, тобто
    \begin{equation} 
        u = \rho \cdot \signum \psi.
    \end{equation} 
    Підставляємо знайдене керування у початкову систему:
    \begin{equation}
        \dot x = \rho \cdot \signum \psi.
    \end{equation}
\end{solution}

\begin{problem}
    Розв'язати задачу оптимального керування за допомогою принципу максимуму Понтрягіна:
    \begin{equation*}
        \JJ(u) = \dfrac12 \int_0^T u^2(s) \diff s + \dfrac{(x(T) - x_1)^2}{2} \to \inf
    \end{equation*}
    за умови, що
    \begin{equation*}
        \dot x = a x + u, x(0) = x_0.
    \end{equation*}
    Тут $x(t) \in \RR^1$, $u(t) \in \RR^1$, $t \in [0, T]$. Точки $x_0 \in \RR^1$, $x_1 \in \RR^1$ і момент часу $T$ є заданими.
\end{problem}

\begin{solution}
	Для початку випишемо всі функції з теоретичної частини:
	\begin{equation}
		f_0(x, u, t) = \dfrac{u^2}{2}, \quad f(x(t), u(t), t) = a x(t) + u(t), \quad \Phi(x(T)) = \dfrac{(x(T) - x_1)^2}{2}.
	\end{equation}

	Функція Гамільтона-Понтрягіна має вигляд:
	\begin{equation}
		\mathcal{H}(x, u, \psi, t) = - f_0(x, u, t) + \langle \psi, f(x, u, t) \rangle = - \dfrac{u^2}{2} + a \psi x + \psi u.
	\end{equation}

	Спряжена система записується так:
	\begin{equation}
		\dot \psi = - \nabla_x \mathcal{H} = - a \psi,
	\end{equation}
	\begin{equation}
		\psi(T) = - \nabla \Phi(x(T)) = x(T) - x_1.
	\end{equation}

	Згідно принципу максимуму, функція Гамільтона-Понтрягіна на оптимальному керуванні досягає свого максимуму, тобто, за відсутності обмежень на керування,
	\begin{equation}
		\dfrac{\partial \mathcal{H}(x, u, \psi, t)}{\partial u} = - u + \psi = 0,
	\end{equation}
	звідки $u = \psi$. Підставляємо знайдене керування у початкову систему:
	\[ \left\{ \begin{aligned}
		\dot x &= a x + \psi, \\
		x(0) &= x_0, \\
		\dot \psi &= - a \psi, \\
		\psi(T) &= x(T) - x_1.
	\end{aligned} \right. \]

	З рівнянь на $\psi$ знаходимо, що $\psi(t) = C_1 \cdot e^{-a t}$, де $C_1$ визначається з рівності $\psi(T) = x(T) - x_1$, тобто $C_1 = (x(T) - x_1) \cdot e^{a T}$. \\

	Підставляючи це у рівняння на $x$ знаходимо, що 
	\begin{equation}
		x(t) = C_2 e^{a t} - \dfrac{C_1 e^{-a t}}{2 a} = C_2 e^{a t} - \dfrac{(x(T) - x_1) \cdot e^{a (T - t)}}{2 a},
	\end{equation}
	де $C_2$ визначається з рівності $x(0) = x_0$, тобто $C_2 = x_0 + \dfrac{(x(T) - x_1) \cdot e^{a T}}{2 a}$. \\

	% x(t) = x_0 e^{a t} + \dfrac{(x(T) - x_1) \cdot (e^{a (T + t)} - e^{a (T - t)})}{2 a},

	Залишається знайти $x(T)$ з рівності 
	\begin{equation}
		x(T) = x_0 e^{a T} + \dfrac{(x(T) - x_1) \cdot (e^{2 a T} - 1)}{2 a}.
	\end{equation}
	Зробивши це, отримаємо
	\begin{equation}
		x(T) = \dfrac{x_1 (e^{2 a T} - 1) - 2 a x_0 e^{a T}}{e^{2 a T} - 2 a - 1}.
	\end{equation}

	Остаточно,
	\begin{equation}
		x(t) = x_0 e^{a t} + \dfrac{\left(\dfrac{x_1 (e^{2 a T} - 1) - 2 a x_0 e^{a T}}{e^{2 a T} - 2 a - 1} - x_1\right) \cdot \left(e^{a (T + t)} - e^{a (T - t)}\right)}{2 a},
	\end{equation}	
	\begin{equation}
		u(t) = \left(\dfrac{x_1 (e^{2 a T} - 1) - 2 a x_0 e^{a T}}{e^{2 a T} - 2 a - 1} - x_1\right) \cdot e^{a (T - t)}.
	\end{equation}
\end{solution}

\begin{problem}
    Записати крайову задачу принципу максимуму для задачі оптимального керування:
    \begin{equation*}
        \JJ(u) = \dfrac12 \int_0^T (u^2(s) + x^2(s)) \diff s + \dfrac{(\dot x(T) - x_1)^2}{2} \to \inf
    \end{equation*}
    за умови, що
    \begin{equation*}
        \ddot x = u, x(0) = x_0, \dot x(0) = y_0.
    \end{equation*}
    Тут $x(t) \in \RR^1$, $u(t) \in \RR^1$, $t \in [0, T]$. Точки $x_0 \in \RR^1$, $y_0 \in \RR^1$ і момент часу $T$ є заданими.
\end{problem}

\begin{solution}
	Позначимо $\textbf{x} = \begin{pmatrix} \textbf{x}_1 & \textbf{x}_2 \end{pmatrix}^* = \begin{pmatrix} x & \dot x \end{pmatrix}^*$. \\

	Для початку випишемо всі функції з теоретичної частини:
	\begin{equation}
		f_0(\textbf{x}, u, t) = \dfrac{u^2 + \textbf{x}_1^2}{2}, \quad f(\textbf{x}(t), u(t), t) = \begin{pmatrix} \textbf{x}_1(t) \\ u(t) \end{pmatrix}, \quad \Phi(\textbf{x}(T)) = \dfrac{(\textbf{x}_2(T) - x_1)^2}{2}.
	\end{equation}

	Функція Гамільтона-Понтрягіна має вигляд:
	\begin{equation}
		\mathcal{H}(\textbf{x}, u, \psi, t) = - f_0(\textbf{x}, u, t) + \langle \psi, f(\textbf{x}, u, t) \rangle = - \dfrac{u^2 + \textbf{x}_1^2}{2} + \psi_1 \textbf{x}_1 + \psi_2 u.
	\end{equation}

	Спряжена система записується так:
	\begin{equation}
		\dot \psi = - \nabla_\textbf{x} \mathcal{H} = \begin{pmatrix} \textbf{x}_1 - \psi_1 \\ 0 \end{pmatrix}.
	\end{equation}
	\begin{equation}
		\psi(T) = - \nabla \Phi(\textbf{x}(T)) = \begin{pmatrix} 0 \\ \textbf{x}_2(T) - x_1 \end{pmatrix}.
	\end{equation}

	Згідно принципу максимуму, функція Гамільтона-Понтрягіна на оптимальному керуванні досягає свого максимуму, тобто, за відсутності обмежень на керування,
	\begin{equation}
    	\dfrac{\partial \mathcal{H}(\textbf{x}, u, \psi, t)}{\partial u} = - u + \psi_2
	\end{equation}
	звідки $u = \psi_2$. Підставляємо знайдене керування у початкову систему:
	\[ \left\{ \begin{aligned}
	    \dot{\textbf{x}}_1 &= \textbf{x}_2. \\
	    \dot{\textbf{x}}_2 &= \psi_2.
	\end{aligned} \right. \]
\end{solution}

\begin{problem}
    Розв'язати задачу оптимального керування за допомогою принципу максимуму Понтрягіна:
    \begin{equation*}
        \JJ(u) = \dfrac 12 \int_0^T u^2(s) \diff s + x_2^2(T) \to \inf
    \end{equation*}
    \[ \left\{ \begin{aligned}
        \dot x_1 &= x_1 - x_2 + u, \\
        \dot x_2 &= - 4 x_1 + x_2,
    \end{aligned} \right. \]
    \begin{equation*}
        x_1(0) = 2, x_2(0) = 4.
    \end{equation*}
    Тут $x = (x_1, x_2)^*$ -- вектор фазових координат з $\RR^2$, $u(t)$ -- функція керування, $t \in [0, T]$, момент часу $T$ є заданим.
\end{problem}

\begin{solution}
    Для початку випишемо всі функції з теоретичної частини:
	\begin{equation}
		f_0(x, u, t) = \dfrac{u^2}{2}, \quad f(x(t), u(t), t) = \begin{pmatrix} x_1(t) - x_2(t) + u(t) \\ -4x_1(t) + x_2(t) \end{pmatrix}, \quad \Phi(x(T)) = x_2(T))^2.
	\end{equation}
	
	Функція Гамільтона-Понтрягіна має вигляд:
	\begin{equation}
		\mathcal{H}(x, u, \psi, t) = - f_0(x, u, t) + \langle \psi, f(x, u, t) \rangle = -\dfrac{u^2}{2} + \psi_1 x_1 - \psi_1 x_2 + \psi_1 u - 4 \psi_2 x_1 + \psi_2 x_2.
	\end{equation}

	Спряжена система записується так:
	\begin{equation}
		\dot \psi = - \nabla_x \mathcal{H} = \begin{pmatrix} - \psi_1 + 4 \psi_2 \\ \psi_1 - \psi_2 \end{pmatrix},
	\end{equation}
	\begin{equation}
		\psi(T) = - \nabla \Phi(x(T)) = 2 x_2(T).
	\end{equation}

	Згідно принципу максимуму, функція Гамільтона-Понтрягіна на оптимальному керуванні досягає свого максимуму, тобто, за відсутності обмежень на керування,
	\begin{equation}
		\dfrac{\partial \mathcal{H}(x, u, \psi, t)}{\partial u} = - u + \psi_1 = 0,
	\end{equation}
	звідки $u = \psi_1$. Підставляємо знайдене керування у початкову систему:
	\[ \left\{ \begin{aligned}
		\dot x_1 &= x_1 - x_2 + \psi_1, \\
		\dot x_2 &= - 4 x_1 + x_2.
	\end{aligned} \right. \]

 	З рівнянь на $\psi$ знаходимо
 	\begin{equation}
 	    \psi = C_1 \begin{pmatrix} 2 \\ -1 \end{pmatrix} e^{-3t} + C_2 \begin{pmatrix} 2 \\ 1 \end{pmatrix} e^{t},
 	\end{equation}
 	де $C_1$ і $C_2$ визначаються з крайових умов. \\

	Підставляючи це у рівняння на $x$ знаходимо, що 
    \[ \left\{ \begin{aligned}
        \dot x_1 &= x_1 - x_2 + 2 C_1 e^{-3t} + 2 C_2 e^{t}, \\
		\dot x_2 &= - 4 x_1 + x_2.
    \end{aligned} \right. \]
    
    Загальний розв'язок однорідного рівняння має вигляд 
    \begin{equation}
        x = C_3 \begin{pmatrix} 1 \\ 2 \end{pmatrix} e^{-t} + C_4 \begin{pmatrix} 1 \\ -2 \end{pmatrix} e^{3t}.
    \end{equation}
    
    Частинний розв'язок неоднорідного знайдемо методом невизначених коефіцієнтів. Покладемо
    \begin{equation}
        x = \begin{pmatrix} A_1 \\ A_2 \end{pmatrix} e^{-3t} + \begin{pmatrix} B_{11} \\ B_{21} \end{pmatrix} e^{t} + \begin{pmatrix} B_{12} \\ B_{22} \end{pmatrix} x e^{t},
    \end{equation}
    тоді, при підстановці у систему, отримаємо:
    \[ \left\{ \begin{aligned}
        - 3 A_1 e^{-3 t} + B_1 e^t &= A_1 e^{-3 t} + B_1 e^t - A_2 e^{-3 t} - B_2 e^t + 2 C_1 e^{-3t} + 2 C_2 e^{t}, \\
		- 3 A_2 e^{-3 t} + B_2 e^t &= - 4 A_1 e^{-3 t} - 4 B_1 e^t + A_2 e^{-3 t} + B_2 e^t.
    \end{aligned} \right. \]
    Ця система рівносильна системі
    \[ \left\{ \begin{aligned}
        - 3 A_1 &= A_1 - A_2 + 2 C_1, \\
        B_1 &= B_1 - B_2 + 2 C_2, \\
		- 3 A_2 &= - 4 A_1 + A_2, \\
		B_2 &= - 4 B_1 + B_2. \\
    \end{aligned} \right. \]
\end{solution}

\begin{problem}
    % 7.12
\end{problem}

\begin{solution}
    % 7.12
\end{solution}
 \newpage

\subsection{Аудиторне заняття}

\begin{problem}
	Розв'язати задачу оптимального керування за допомогою принципу максимуму Понтрягіна: \[ \JJ (u) = \int_0^1 (u^2 (s) + x^2 (s)) \diff s \to \inf \] за умови, що \[ \frac{\diff x(t)}{\diff t} = u (t), \quad x(0) = 0, x(1) = \frac12. \] Тут $x (t) \in \RR^1$, $u (t) \in \RR^1$, $t \in [0, 1]$.
\end{problem}

\begin{solution}
	Нагадаємо загальну постановку задачі принципу максимуму Понтрягіна: \[ \JJ_0 \to \min, \dot x = f, \JJ_i \le 0 \, (i = \overline{1..k}), \JJ_i = 0 \, (i = \overline{k+1..k+r}), \JJ_i = \int f_i + \Phi_i. \]

	У нашій задачі \[ f_0 = u^2 + x^2, f = u, \] і треба щось зробити з $x(0) = 0$ і $x(1) = \frac 12$. Насправді це інтегральні обмеження вигляду \[ \JJ_1 = 0, f_1 = 0, \Phi_1 = x_0, \quad \JJ_2 = 0, f_2 = 0, \Phi_2 = x_T - \frac{1}{2}. \]

	Запишемо функцію Гамільтона-Понтрягіна і термінант: \[ F = \lambda_0 (u^2 + x^2), \quad \ell = \lambda_1 x_0 + \lambda_2 \left(x_T - \frac{1}{2}\right), \] \[ \HH = -F + \langle \psi, f \rangle = - \lambda_0 (u^2 + x^2) + \psi u. \]

	Випишемо тепер всі (необхідні) умови принципу максимуму:
	\begin{enumerate}
		\item оптимальність: $\frac{\partial \HH}{\partial u} = - 2 \lambda_0 u + \psi  = 0$;
		\item стаціонарність (спряжена система): $\dot \psi = - \nabla_x \HH = 2 \lambda_0 x$;
		\item трансверсальність: $\psi(t_0) = \psi(0) = \frac{\partial \ell}{\partial x_0} = \lambda_1$, $\psi(T) = \psi(1) = - \frac{\partial \ell}{\partial x_T} = - \lambda_2$;
		\item стаціонарність за кінцями: відсутня, бо час фіксований, $[t_0,T]=[0,1]$;
		\item доповнююча нежорсткість: відсутня, бо немає інтегральних обмежень виду нерівність на задачу, $k = 0$;
		\item невід'ємність: $\lambda_0 \ge 0$.
	\end{enumerate}

	Нескладно пересвідчитися, що якщо $\lambda_0 = 0$, то $\psi \equiv 0$, вироджений випадок, тобто виконується умова нерівності нулеві множників Лагранжа. Покладемо тоді без обмеження загальності $\lambda_0 = \frac{1}{2}$, тоді $u = \psi$. \\

	Запишемо тепер крайову задачу принципу максимуму: \[ \left\{ \begin{aligned}
		\dot \psi &= x, \\
		\dot x &= \psi, \\
		x(0) &= 0, x(1) = \frac{1}{2}.
	\end{aligned} \right. \]

	Її загальний розв'язок \[ \left\{ \begin{aligned}
		x(t) &= c_1 e^{-t} + c_2 e^t, \\
		\psi(t) &= - c_1 e^{-t} + c_2 e^t.
	\end{aligned} \right. \]

	З крайових умов \[ \left\{ \begin{aligned}
		x(0) &= c_1 + c_2 = 0, \\
		x(1) &= c_1 / e + c_2 e = \frac{1}{2},
	\end{aligned} \right. \] знаходимо \[c_1 = - \frac{1}{2(e-e^{-1})}, \quad c_2 = \frac{1}{2(e-e^{-1})}. \]

	Остаточно, \[ \left\{ \begin{aligned}
		x(t) &= \frac{e^t - e^{-t}}{2(e-e^{-1})}, \\
		u(t) &= \frac{e^t + e^{-t}}{2(e-e^{-1})}.
	\end{aligned} \right. \]
\end{solution}

\begin{problem}
	Розв'язати задачу оптимального керування за допомогою принципу максимуму Понтрягіна: \[ \JJ (u) = \frac{1}{2} \int_0^1 (u^2 (s) - 12 s x (s)) \diff s \to \inf \] за умови, що \[ \frac{\diff x(t)}{\diff t} = u (t), \quad x(0) = 0, x(1) = 0. \] Тут $x (t) \in \RR^1$, $u (t) \in \RR^1$, $t \in [0, 1]$.
\end{problem}

\begin{solution}
	Зауважимо, що на множник $\frac{1}{2}$ можна заплющити очі, адже від нього $\arg\inf\JJ$ явно не зміниться. \\

	Функція Гамільтона-Понтрягіна і термінант записують так: \[ \HH = - \lambda_0 (u^2 - 12 t x), \quad \ell = \lambda_1 x_0 + \lambda_2 x_T. \]

	Випишемо тепер всі (необхідні) умови принципу максимуму:
	\begin{enumerate}
		\item оптимальність: $\frac{\partial \HH}{\partial u} = - 2 \lambda_0 u + \psi  = 0$;
		\item стаціонарність (спряжена система): $\dot \psi = - \nabla_x \HH = - 12 \lambda_0 t$;
		\item трансверсальність: $\psi(t_0) = \psi(0) = \frac{\partial \ell}{\partial x_0} = \lambda_1$, $\psi(T) = \psi(1) = - \frac{\partial \ell}{\partial x_T} = - \lambda_2$;
		\item стаціонарність за кінцями: відсутня, бо час фіксований, $[t_0,T]=[0,1]$;
		\item доповнююча нежорсткість: відсутня, бо немає інтегральних обмежень виду нерівність на задачу, $k = 0$;
		\item невід'ємність: $\lambda_0 \ge 0$.
	\end{enumerate}

	Перевіряємо умову нерівності нулеві множників Лагранжа. Від супротивного, якщо $\lambda_0 = 0$, то $\psi \equiv 0$, а тоді і $\lambda_1 = \lambda_2 = 0$, вироджений випадок. Покладемо тоді без обмеження загальності $\lambda_0 = 1$, тоді $u = \psi / 2$. \\

	Запишемо тепер крайову задачу принципу максимуму: \[ \left\{ \begin{aligned}
		\dot \psi &= -12t, \\
		\dot x &= \psi / 2, \\
		x(0) &= x(1) = 0.
	\end{aligned} \right. \]

	З рівняння на $\dot \psi$, $\psi = - 6 t^2 + C_1$. Підставляючи в рівняння на $\dot x$ і розв'язуючи його, знаходимо $x = - t^3 + C_1 t / 2 + C_2$.

	З крайових умов \[ \left\{ \begin{aligned}
		x(0) &= C_2 = 0, \\
		x(1) &= -1 + C_1 / 2 + C_2 = 0,
	\end{aligned} \right. \] знаходимо \[C_1 = 2, \quad C_2 = 0. \]

	Отже керування \[ u_*(t) = \frac{\psi(t)}{2} = - 3 t^2 + 1, \] і відповідна йому траєкторія \[ x_*(t) = -t^3 + t \] є оптимальними.

\end{solution}

\begin{problem}
	Розв'язати задачу оптимального керування за допомогою принципу максимуму Понтрягіна: \[ \JJ (u) = \frac{1}{2} \int_0^1 u^2 (s) \diff s \to \inf \] за умови, що \[ \frac{\diff^2 x(t)}{\diff t^2} = u (t),\quad  x(0) = -1, \dot x(0) = 2, x(1) = 0, \dot x(1) = 1. \] Тут $x (t) \in \RR^1$, $u (t) \in \RR^1$, $t \in [0, 1]$.
\end{problem}

\begin{solution}
	Позначимо $x_1 = x$, $x_2 = \dot x$, тоді $f_0 = u^2 / 2$, $\Phi_0 = 0$, $f_1 = 0$, $\Phi_1 = x_{10} + 1$, $f_2 = 0$, $\Phi_2 = x_{20} - 2$, $f_3 = 0$, $\Phi_3 = x_{1T}$, $f_4 = 0$, $\Phi_4 = x_{2T} - 1$, $f = \begin{pmatrix} x_2 \\ u \end{pmatrix}$. \\

	Запишемо тепер функцію Гамільтона-Понтрягіна і термінант: \[ F = \lambda_0 u^2 / 2, \quad \ell = \lambda_1 (x_{10} + 1) + \lambda_2 (x_{20} - 2) + \lambda_3 x_{1T} + \lambda_4 (x_{2T} - 1), \] \[ \HH = -F + \langle \psi, f \rangle = - \lambda_0 u^2 / 2 + \psi_1 x_2 + \psi_2 u. \]

	Випишемо тепер всі (необхідні) умови принципу максимуму:
	\begin{enumerate}
		\item оптимальність: $\frac{\partial \HH}{\partial u} = - \lambda_0 u + \psi_2  = 0$;
		\item стаціонарність (спряжена система): $\dot \psi = - \nabla_x \HH = - \begin{pmatrix} 0 \\ \psi_1 \end{pmatrix}$;
		\item трансверсальність: \[\psi(t_0) = \psi(0) = \nabla_{x_0} \ell = \begin{pmatrix} \lambda_1 \\ \lambda_2 \end{pmatrix}, \quad \psi(T) = \psi(1) = - \nabla_{x_T} \ell = \begin{pmatrix} - \lambda_3 \\ - \lambda_4 \end{pmatrix};\]
		\item стаціонарність за кінцями: відсутня, бо час фіксований, $[t_0,T]=[0,1]$;
		\item доповнююча нежорсткість: відсутня, бо немає інтегральних обмежень виду нерівність на задачу, $k = 0$;
		\item невід'ємність: $\lambda_0 \ge 0$.
	\end{enumerate}

	Без обмеження загальності покладемо $\lambda_0 = 1$, тоді $u = \psi_2$. \\

	Запишемо тепер крайову задачу принципу максимуму: \[ \left\{ \begin{aligned}
		\dot \psi_1 &= 0, \\
		\dot \psi_2 &= - \psi_1, \\
		\dot x_1 &= x_2, \\
		\dot x_2 &= \psi_2, \\
		x_{10} &= -1, x_{20} = 2, \\
		x_{1T} &= 0, x_{2T} = 1.
	\end{aligned} \right. \]

	З першого рівняння $\psi_1 = C_1$, тоді з другого $\psi_2 = - C_1 t + C_2$, далі з четвертого $x_2 = - C_1 t^2 / 2 + C_2 t + C_3$, і нарешті з третього $x_1 = - C_1 t^3 / 6 + C_2 t^2 / 2 + C_3 t + C_4$. \\

	З крайових умов \[ \left\{ \begin{aligned}
		x_1(0) &= C_4 = -1, \\
		x_2(0) &= C_3 = 2, \\
		x_1(1) &= - C_1 / 6 + C_2 / 2 + C_3 + C_4 = 0, \\
		x_2(1) &= - C_1 / 2 + C_2 + C_3 = 1.
	\end{aligned} \right. \] знаходимо \[C_1 = -6, \quad C_2 = -4, \quad C_3 = 2, \quad C_4 = -1. \]

	Отже \[ u_*(t) = \psi_2(t) = 6 t - 4, \] \[ x_*(t) = x_1(t) = t^3 - 2 t^2 + 2 t - 1. \]
\end{solution}

\begin{problem}
	Розв'язати задачу оптимальної швидкодії за допомогою принципу максимуму Понтрягіна: \[ \JJ (u) = T = \int_0^T \diff s \to \inf \] за умови, що \[ \frac{\diff x(t)}{\diff t} = u (t), \quad x(0) = 0, x(T) = 1, \] \[ \int_0^T u^2(s) \diff s = 1. \] Тут $x (t) \in \RR^1$, $u (t) \in \RR^1$, $t \in [0, 1]$.
\end{problem}

\begin{solution}
	У нашій задачі $f_0 = 1$, $\Phi_0 = T$, $f_1 = 0$, $\Phi_1 = x_0$, $f_2 = 0$, $\Phi_2 = x_T - 1$, $f_3 = u^2$, $\Phi_3 = -1$, $f = u$. \\

	Запишемо тепер функцію Гамільтона-Понтрягіна і термінант: \[ F = \lambda_0 + \lambda_3 u^2, \quad \ell = \lambda_0 T + \lambda_1 x_0 + \lambda_2 (x_T - 1) - \lambda_3, \] \[ \HH = -F + \langle \psi, f \rangle = - \lambda_0 - \lambda_3 u^2 + \psi u. \]

	Випишемо тепер всі (необхідні) умови принципу максимуму:
	\begin{enumerate}
		\item оптимальність: $\frac{\partial \HH}{\partial u} = - 2 \lambda_3 u + \psi = 0$;
		\item стаціонарність (спряжена система): $\dot \psi = - \nabla_x \HH = - 0$;
		\item трансверсальність: \[\psi(t_0) = \psi(0) = \nabla_{x_0} \ell =  \lambda_1 , \quad \psi(T) = - \nabla_{x_T} \ell = - \lambda_2;\]
		\item стаціонарність за кінцями: $\HH(T) = \frac{\partial \ell}{\partial T} = \lambda_0$;
		\item доповнююча нежорсткість: відсутня, бо немає інтегральних обмежень виду нерівність на задачу, $k = 0$;
		\item невід'ємність: $\lambda_0 \ge 0$.
	\end{enumerate}

	$u = \frac{\psi}{2 \lambda_3}$. \\

	Запишемо тепер крайову задачу принципу максимуму: \[ \left\{ \begin{aligned}
		\dot \psi &= 0, \\
		\dot x &= \frac{\psi}{2 \lambda_3}, \\
		x(0) &= 0, x(T) = 1.
	\end{aligned} \right. \]

	З першого рівняння $\psi = C_1$, підставляючи в друге і розв'язуючи його знаходимо $x = \frac{C_1 t + C_2}{2 \lambda_3}$. \\

	З крайових умов \[ \left\{ \begin{aligned}
		x(0) &= C_2 / 2 \lambda_3 = 0, \\
		x(T) &= \frac{C_1 T + C_2}{2 \lambda_3} = 1,
	\end{aligned} \right. \] знаходимо \[C_1 = \frac{2 \lambda_3}{T}, \quad C_2 = 0. \]

	Отже \[ u(t) = \frac{\psi(t)}{2 \lambda_3} = \frac{1}{T}. \]

	Підставляючи це в умову $\int_0^T u^2(s) \diff s = 1$, знаходимо $1 / T = 1$, звідки $T = 1$.
\end{solution}
 \newpage

\subsection{Домашнє завдання}

\begin{problem}
	Розв'язати задачу оптимального керування за допомогою принципу максимуму Понтрягіна: \[ \JJ (u) = \frac{1}{2} \int_{-1}^1 (u^2 (s) + x^2 (s)) \diff s \to \inf \] за умови, що \[ \frac{\diff x(t)}{\diff t} = u (t), \quad x(-1) = x(1) = 1. \] Тут $x (t) \in \RR^1$, $u (t) \in \RR^1$, $t \in [-1, 1]$.
\end{problem}

\begin{solution}
	% 8.5
\end{solution}

\begin{problem}
	Розв'язати задачу оптимального керування за допомогою принципу максимуму Понтрягіна: \[ \JJ (u) = \frac{1}{2} \int_0^1 (u^2 (s) + x^2 (s)) \diff s \to \inf \] за умови, що \[ \frac{\diff^2 x(t)}{\diff t^2} = u (t), \quad x(0) = 1, \dot x(0) = -2, x(1) = 0, \dot x(1) = 0. \] Тут $x (t) \in \RR^1$, $u (t) \in \RR^1$, $t \in [0, 1]$.
\end{problem}

\begin{solution}
	% 8.6
\end{solution}

\begin{problem}
	Розв'язати задачу оптимального керування за допомогою принципу максимуму Понтрягіна: \[ \JJ (u) = \frac{1}{2} \int_0^1 (x(s) + u^2 (s)) \diff s \to \inf \] за умови, що \[ \frac{\diff x(t)}{\diff t} = x (t) + u (t), \quad x(1) = 0. \] Тут $x (t) \in \RR^1$, $u (t) \in \RR^1$, $t \in [0, 1]$.
\end{problem}

\begin{solution}
	% 8.7
\end{solution}

\begin{problem}
	Розв'язати задачу оптимального керування за допомогою принципу максимуму Понтрягіна: \[ \JJ (u) = \int_{-\pi}^\pi x(s) \sin(s) \diff s \to \inf \] за умови, що \[ \dot x = u, \quad x(-\pi) = x(\pi) = 0, u(t) \in [-1,1], \] де $x (t) \in \RR^1$, $t \in [-\pi, \pi]$.
\end{problem}

\begin{solution}
	% 8.8
\end{solution}


\end{document}