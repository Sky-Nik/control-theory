\section{Елементи багатозначного аналізу. Множина досяжності}

\subsection{Алгоритми}

\begin{problem*}
	Знайти
	\begin{enumerate}
		\item $A+B$;
		\item $\lambda A$;
		\item $\alpha(A,B)$;
		\item $MA$,
	\end{enumerate}
	де множини $A \subset\RR^m$, $B\subset\RR^m$, скаляр $\lambda\in\RR^1$, матриця $M\in\RR^{n\times m}$.
\end{problem*}

\begin{algorithm} \tt
	Розглянемо всі пункти задачі вище.
	\begin{enumerate}
		\item Знаходимо за визначенням, \[A+B=\{a+b|a\in A,b\in B\}.\]
		\item Знаходимо за визначенням, \[\lambda A =\{\lambda a|a\in A\}.\] 
		\item \begin{enumerate}
			\item Знаходимо $\beta(A,B)$ і $\beta(B,A)$ за визначенням, \[ \beta(A,B) = \max_{a\in A}\rho(a,B), \]
			де \[\rho(a,B) = \min_{b\in B} \rho(a,b).\]
			\item Знаходимо $\alpha(A,B)$ за визначенням, \[\alpha(A,B)=\max\{\beta(A,B),\beta(B,A).\]
		\end{enumerate} 
		\item Знаходимо за визначенням, \[MA=\{Ma|a\in A\}.\]
	\end{enumerate}
\end{algorithm}

\begin{problem*}
	Знайти опорну функцію множини $A \subset \RR^n$.
\end{problem*}

\begin{algorithm} \tt
	\begin{enumerate}
		\item \textbf{Намагаємося} знайти за визначенням, \[ c(A,\psi) = \max_{a\in A} \langle a, \psi \rangle. \]
		\item Якщо не вийшло, то намагаємося знайти за геометричною властивістю: $c(A,\psi)$ -- (орієнтована) відстань від початку координат до опорної \allowbreak пло\-щи\-ни множини $A$, для якої напрямок-вектор $\psi$ є вектором нормалі.
	\end{enumerate}
\end{algorithm}

\begin{problem*}
	Знайти інтеграл Аумана $\JJ = \int F \diff x$, де $F = F(x)\subset\RR^n$.
\end{problem*}

\begin{algorithm} \tt
	\begin{enumerate}
		\item Знаходимо опорну функцію від інтегралу: \[ c(\JJ, \psi) = \int c (F, \psi) \diff x.\]
		\item Знаходимо $\JJ$ як опуклий компакт з відомою опорною функцією $c(\JJ, \psi)$.
	\end{enumerate}
\end{algorithm}

\begin{problem*}
	Знайти множину досяжності системи $\dot x = A x + B u$, де $x(t_0) \in \mathcal{M}_0$, $u \in \mathcal{U}$.
\end{problem*}

\begin{algorithm} \tt
	\begin{enumerate}
		\item Знаходимо фундаментальну матрицю $\Theta(t,s)$ системи нор\-мо\-ва\-ну за моментом $s$.
		\item Знаходимо інтеграл Аумана \[\int_{t_0}^t \Theta(t, s) B(s) \UU(s) \diff s.\]
		\item Використовуємо теорему про вигляд множини досяжності лінійної сис\-те\-ми керування: \[ \XX(t, \MM_0) = \Theta(t, t_0) \MM_0 + \int_{t_0}^t \Theta(t, s) B(s) \UU(s) \diff s. \]
	\end{enumerate}
\end{algorithm}

\begin{problem*}
	Знайти опорну функцію множини досяжності системи $\dot x = A x + B u$, де $x(t_0) \in \mathcal{M}_0$, $u \in \mathcal{U}$.
\end{problem*}

\begin{algorithm} \tt
	\begin{enumerate}
		\item Знаходимо фундаментальну матрицю $\Theta(t,s)$ системи нор\-мо\-ва\-ну за моментом $s$.
		\item Знаходимо опорну функцію $c(\MM_0, \Theta^*(t, t_0) \psi)$.
		\item Знаходимо опорну функцію $c(\UU(s), B^*(s) \Theta^*(t, s) \psi)$.
		\item Використовуємо теорему про вигляд опорної функції множини до\-сяж\-но\-с\-ті лінійної системи керування: \[ c(\XX(t, \MM_0), \psi) = c(\MM_0, \Theta^*(t, t_0) \psi) + \int_{t_0}^t c(\UU(s), B^*(s) \Theta^*(t, s) \psi) \diff s. \]
	\end{enumerate}
\end{algorithm}

