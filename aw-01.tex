% OK, complete
\section{Системи керування. Постановка задачі оптимального керування}

\subsection{Аудиторне заняття}

\begin{problem}
	Задана скалярна система керування 
	
	\begin{equation}
		\label{eq:1.1}
		\frac{\diff x(t)}{\diff t} = u(t), \quad x(0) = 1.
	\end{equation}

	Тут $x$ -- стан системи, $t \in [0, 1]$. Керування задане у вигляді

	\begin{equation}
		\label{eq:1.2}
		u(x) = a x.
	\end{equation}

	Тут $a$ -- скалярний параметр.

	\begin{enumerate}
		\item Знайти траєкторію системи (\ref{eq:1.1}) при керуванні (\ref{eq:1.2}).

		\item Знайти програмне керування $u(t) = a x (t)$, яке відповідає знайденій траєкторії. 

		\item Оцінити, при якому значення параметра $a \in \{2, 4, -3\}$ критерій якості 

		\[ \JJ(u) = x^2 (1) \]

		буде мати менше значення.
	\end{enumerate}
\end{problem}

\begin{solution}
	\begin{enumerate}
		\item Підставляючи керування (\ref{eq:1.2}) у систему (\ref{eq:1.1}), отримаємо систему \[ \frac{\diff x(t)}{\diff t} = a x(t), \quad x(0) = 1. \]

		Її розв'язок має вигляд \[ x(t) = x(0) \cdot e^{a t} = e^{a t}. \]

		\item Просто підставляємо знайдену у попередньому пункті траєкторію у вигляд (\ref{eq:1.2}) керування: \[ u(t) = a x(t) = a \cdot e^{a t}. \]

		\item Множина $\{2, 4, -3\}$ скінченна, тому можна просто перебрати всі її елементи та обчислити значення критерію якості на кожному з них: \[ \begin{aligned} \left. \JJ (u) \right|_{a = 2} &= x^2 (1) = \left. e^{2 a t} \right|_{t = 1} = e^4, \\ \left. \JJ (u) \right|_{a = 4} &= x^2 (1) = \left. e^{2 a t} \right|_{t = 1} = e^8, \\ \left. \JJ (u) \right|_{a = -3} &= x^2 (1) = \left. e^{2 a t} \right|_{t = 1} = e^{-6}. \end{aligned} \]

		%Як бачимо, н
		Найменшим з цих значень є $e^{-6}$ яке досягається при $a = -3$. \\

		% Насправді це передбачувано, бо з загального вигляду траєкторії для $a > 0$ видно, що модуль $x(t)$ зростає з часом $t$, а для $a < 0$ -- зменшується.
	\end{enumerate}
\end{solution}

\begin{problem}
	Задана система керування

	\begin{equation}
		\label{eq:1.3}
		\left\{
			\begin{aligned}
				\frac{\diff x_1 (t)}{\diff t} &= x_1 (t) + x_2 (t) + u (t), \\
				\frac{\diff x_2 (t)}{\diff t} &= - x_1 (t) + x_2 (t) + u(t),
			\end{aligned}
		\right.
		\quad
		x_1 (0) = 2, x_2 (0) = 1.
	\end{equation}

	Тут $x = (x_1, x_2)^*$ -- вектор фазових координат з $\RR^2$, $t \in [0, T]$. Керування задане у вигляді

	\begin{equation}
		\label{eq:1.4}
		u(x_1, x_2) = 2 x_1 + x_2.
	\end{equation}

	\begin{enumerate}
		\item Знайти траєкторію системи (\ref{eq:1.3}) при керуванні (\ref{eq:1.4}).

		\item Знайти програмне керування $u(t) = 2 x_1 (t) + x_2 (t)$, яке відповідає знайденій траєкторії.

		\item Якою буде фундаментальна матриця, нормована за моментом $s$, системи. що одержана при підстановці керування (\ref{eq:1.4}) у систему (\ref{eq:1.3})?

		\item Побудувати спряжену систему до системи, одержаної при підстановці керування (\ref{eq:1.4}) у систему (\ref{eq:1.3}), та її фундаментальну матрицю.
	\end{enumerate}
\end{problem}

\begin{solution}
	\begin{enumerate}
		\item Підставляючи керування (\ref{eq:1.4}) у систему (\ref{eq:1.3}), отримаємо систему \[
		\left\{
			\begin{aligned}
				\frac{\diff x_1 (t)}{\diff t} &= 3 x_1 (t) + 2 x_2 (t), \\
				\frac{\diff x_2 (t)}{\diff t} &= x_1 (t) + 2 x_2 (t),
			\end{aligned}
		\right.
		\quad
		x_1 (0) = 2, x_2 (0) = 1.
		\]

		Її розв'язок \[
		\left\{
			\begin{aligned}
				x_1 (t) &= 2 e^{4 t}, \\
				x_2 (t) &= e^{4 t}.
			\end{aligned}
		\right.
		\]

		\item Просто підставляємо знайдену у попередньому пункті траєкторію у вигляд (\ref{eq:1.4}) керування: \[ u(t) = 2 x_1 (t) + x_2 = 2 \cdot \left( 2 e^{4 t} \right) + e^{4 t} = 5 e^{4 t}. \]

		\item Враховуючи, що загальним розв'язком системи (\ref{eq:1.3}) з підставленим керуванням (\ref{eq:1.4}) є \[
		\left\{
			\begin{aligned}
				x_1 (t) &= c_1 e^t + 2 c_2 e^{4 t}, \\
				x_2 (t) &= - c_1 e^t + c_2 e^{4 t}.
			\end{aligned}
		\right.
		\]

		Це означає, що фундаментальна матриця цієї системи матиме вигляд \[
			\Theta(t) = 
			\begin{pmatrix}
				c_{11} e^t + 2 c_{12} e^{4 t} & c_{21} e^t + 2 c_{22} e^{4 t} \\
				- c_{11} e^t + c_{12} e^{4 t} & - c_{21} e^t + c_{22} e^{4 t}
			\end{pmatrix}
		\]

		Залишається пронормувати її за моментом $s$, тобто знайти такі $c_{11} (s)$, $c_{12} (s)$, $c_{21} (s)$, $c_{22} (s)$, що $\Theta(s, s) = E$. Коли це зробити, то отримаємо \[
			\Theta(t, s) = \frac{1}{3}
			\begin{pmatrix}
				e^{t - s} + 2 e^{4 (t - s)} & -2 e^{t - s} + 2 e^{4 (t - s)} \\
				e^{t - s} - e^{4 (t - s)} & 2 e^{t - s} + e^{4 (t - s)}
			\end{pmatrix}
		\]

		\item Спряжена система буде \[
		\left\{
			\begin{aligned}
				\frac{\diff z_1 (t)}{\diff t} &= - 3 z_1 (t) - z_2 (t), \\
				\frac{\diff z_2 (t)}{\diff t} &= - 2 z_1 (t) - 2 z_2 (t),
			\end{aligned}
		\right.
		\]

		а відповідна фундаментальна матриця \[ \Psi(t, s) = \Theta^*(s, t) = \frac{1}{3}
			\begin{pmatrix}
				e^{s - t} + 2 e^{4 (s - t)} & e^{s - t} - e^{4 (s - t)} \\
				-2 e^{s - t} + 2 e^{4 (s - t)} & 2 e^{s - t} + e^{4 (s - t)}
			\end{pmatrix}
		\]
	\end{enumerate}
\end{solution}

\begin{problem}
	Розглядається задача Больца \[ \JJ (u) = \int_0^1 u^2 (s) \diff s + ( x (1) - 2 )^2 \to \inf \]

	за умови, що \[ \frac{\diff x (t)}{\diff t} = x^2 (t) + u (t), \quad x (0) = x_0. \]

	Тут $x (t) \in \RR^1$, $u (t) \in \RR^1$, $t \in [0, 1]$. Точка $x_0 \in \RR^1$ задана. Звести цю задачу до задачі Майєра.
\end{problem}

\begin{solution}
	Введемо нову змінну \[ x_2 = \int_0^t u^2 (s) \diff s, \] 

	тоді \[ \JJ (u) = x_2 (1) + ( x_1 (1) - 2)^2 \to \inf, \]

	за умов, що \[ 
	\left\{
		\begin{aligned}
			\frac{\diff x_1 (t)}{\diff t} &= x_1^2 (t) + u (t), \\
			\frac{\diff x_2 (t)}{\diff t} &= u^2 (t).
		\end{aligned}
	\right.
	\quad
	x_1 (0) = x_0, x_2 (0) = 0.
	\]
\end{solution}

\begin{problem}
	Задана система керування 
	\[
		\left\{
			\begin{aligned}
				\dfrac{dx_1(t)}{dt} &= 2x_1(t) + x_2(t) + u(t), \\
				\dfrac{dx_2(t)}{dt} &= 3x_1(t) + 4x_2(t),
			\end{aligned}
		\right.
		\quad
		x_1(0) = 1, x_2(0) = -1.
	\]

	Тут $x = (x_1, x_2)^*$ -- вектор фазових координат з $\RR^2$, $t \in [0, 2]$. Керування задане у вигляді
	\[
		u(t) 
		=
		\begin{cases}
			0, & \text{якщо } t \in [0, 1], \\
			1, & \text{якщо } t \in (1, 2].
		\end{cases}
	\]
	
	\begin{enumerate}
		\item Знайти траєкторію системи, яка відповідає цьому керуванню.
		\item Чи буде ця траєкторія неперервно диференційовною?
		\item Чи буде таке керування кращим в порівнянні з керуванням 
		\[
			u(t) = 0, t \in [0, 2]
		\]
		за умови, що критерій якості має вигляд 
		\[
			\mathcal{J}(u) = x_1^2(2) + x_2^2(2) \to \min.
		\]
	\end{enumerate}
\end{problem}

\begin{solution}
	\begin{enumerate}
		\item При $t \in [0, 1]$ маємо 
		\[
			\begin{pmatrix}
				\dot x_1 \\ 
				\dot x_2
			\end{pmatrix}
			=
			\begin{pmatrix}
				2 & 1 \\
				3 & 4
			\end{pmatrix}
			\begin{pmatrix}
				x_1 \\
				x_2
			\end{pmatrix}.
		\]
		Характеристичне рівняння 
		\[
			\begin{vmatrix}
				2 - \lambda & 1 \\
				3 & 4 - \lambda 
			\end{vmatrix}
			=
			\lambda^2 - 6\lambda + 5 
			=
			(\lambda - 1) (\lambda - 5)
			=
			0,
		\]
		звідки $\lambda_1 = 1$, $\lambda_2 = 5$.\\
		
		З курсу диференційних рівнянь відомо, що тоді загальний розв'язок має вигляд
		\[
			\begin{pmatrix}
			x_1 \\
			x_2
			\end{pmatrix}
			=
			c_1 v_1 e^t + c_2 v_2 e^{5t},
		\]
		де $v_1$, $v_2$ -- власні вектори, що відповідають $\lambda_1$ та $\lambda_2$ відповідно.\\
		
		Нескладно бачити, що 
		\[
			\begin{pmatrix}
				x_1 \\
				x_2
			\end{pmatrix}
			=
			c_1 
			\begin{pmatrix}
				1 \\
				-1
			\end{pmatrix} 
			e^t 
			+ 
			c_2 
			\begin{pmatrix}
				1 \\
				3
			\end{pmatrix} 
			e^{5t}.
		\]
		
		Підставляючи $t = 0$ отримуємо $c_1 = 1$, $c_2 = 0$.
		
		При $t \in (1, 2]$ маємо 
		\[
			\begin{pmatrix}
				x_1 \\
				x_2
			\end{pmatrix}
			=
			c_1 v_1 e^t 
			+ 
			c_2 v_2 e^{5t} 
			+ 
			\begin{pmatrix} 
				c_3 \\
				c_4
			\end{pmatrix},
		\]
		де $c_3$, $c_4$ задовольняють систему 
		\[
			\left\{
				\begin{aligned}
					2c_3 &+ c_4 + 1 &= 0 \\
					3c_3 &+ 4c_4 &= 0
				\end{aligned}
			\right.,
		\]
		звідки $c_3 = -4/5$, $c_4 = 3/5$ і 
		\[
			\begin{pmatrix}
				x_1 \\
				x_2
			\end{pmatrix}
			=
			c_1 v_1 e^t 
			+ 
			c_2 v_2 e^{5t} 
			+ 
			\begin{pmatrix} 
				-4/5 \\
				3/5
			\end{pmatrix},
		\]
		Підставляючи $t = 1$ отримуємо $c_1 = \left(1 + \dfrac{3}{4e}\right)$, $c_2 = \dfrac{1}{20e^5}$.\\
		
		Остаточно маємо 
		\[
			\begin{pmatrix}
			x_1 \\
			x_2
			\end{pmatrix}
			=
			\begin{cases}
				\begin{pmatrix}
					1 \\
					-1
				\end{pmatrix} 
				e^t, & t \in [0, 1] \\
				\left(1 + \dfrac{3}{4e}\right) 
				\begin{pmatrix}
				1 \\
				-1
				\end{pmatrix} 
				e^t 
				+ 
				\dfrac{1}{20e^5} 
				\begin{pmatrix}
				1 \\
				3
				\end{pmatrix} 
				e^{5t} 
				+ 
				\begin{pmatrix} 
					-4/5 \\
					3/5
				\end{pmatrix}			
				, & t \in (1, 2]
			\end{cases}.
		\]
		\item 
		\[
			\begin{pmatrix}
				\dot x_1 \\
				\dot x_2
			\end{pmatrix}
			(1-) 
			= 
			\begin{pmatrix}
				2 & 1 \\
				3 & 4
			\end{pmatrix}
			\begin{pmatrix}
				x_1(1-) \\
				x_2(1-)
			\end{pmatrix}
		\]
		З неперервності $x_1$, $x_2$ маємо:
		\[
			\begin{pmatrix}
				2 & 1 \\
				3 & 4
			\end{pmatrix}
			\begin{pmatrix}
				x_1(1-) \\
				x_2(1-)
			\end{pmatrix}
			=
			\begin{pmatrix}
			2 & 1 \\
			3 & 4
			\end{pmatrix}
			\begin{pmatrix}
			x_1(1) \\
			x_2(1)
			\end{pmatrix}
		\]
		З іншого боку,
		\[
			\begin{pmatrix}
				\dot x_1 \\
				\dot x_2
			\end{pmatrix}
			(1+) 
			= 
			\begin{pmatrix}
				2 & 1 \\
				3 & 4
			\end{pmatrix}
			\begin{pmatrix}
				x_1(1+) \\
				x_2(1+)
			\end{pmatrix}
			+
			\begin{pmatrix}
				1 \\
				0
			\end{pmatrix}
			= 
			\begin{pmatrix}
				2 & 1 \\
				3 & 4
			\end{pmatrix}
			\begin{pmatrix}
				x_1(1) \\
				x_2(1)
			\end{pmatrix}
			+
			\begin{pmatrix}
				1 \\
				0
			\end{pmatrix}
		\]
		Нескладно бачити, що 
		\[
			\begin{pmatrix}
				\dot x_1 \\
				\dot x_2
			\end{pmatrix}
			(1-)
			\ne
			\begin{pmatrix}
				\dot x_1 \\
				\dot x_2
			\end{pmatrix}
			(1+),
		\]
		тобто траєкторія не є неперервно диференційовною в точці $1$.
		\item 
		Просто підставимо $t=2$ в розв'язки для обох керувань (попутно зауваживши, що для нового керування розв'язок ми вже знаємо, це просто продовження вже знайденого розв'язку для $t \in [0, 1]$):
		\[
			\left(e^2 + \dfrac34e + \dfrac{e^5}{20} - \dfrac45\right)^2 + \left(-e^2 - \dfrac34e + \dfrac{3e^5}{20} + \dfrac35\right)^2
			\lor
			(e^2)^2 + (-e^2)^2
		\]
		Після марудних обчислень знаходимо, що права частина менше, тобто нове керування є кращим за початкове.
	\end{enumerate}
\end{solution}
