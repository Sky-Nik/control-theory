\setcounter{section}{1}

\section{Домашнє завдання за 9/14}

\setcounter{problem}{0}

\begin{problem}
    $A = \{-5, -3, 2\}$, $B = \{-2, 0, 3\}$, $\lambda = -3$. Знайти $A + B$ і $\lambda A$.
\end{problem}

\begin{solution}
    \begin{equation*}
        \begin{aligned}
            A + B &= \{(-5) + (-2), (-5) + 0, (-5) + 3, (-3) + (-2), (-3) + 0, (-3) + 3, 2 + (-2), 2 + 0, 2 + 3\} = \\
            &= \{-7, -5, -2, -5, -3, 0, 0, 2, 5\} = \{-7, -5, -3, -2, 0, 2, 5\}, \\
            \lambda A &= \{(-3) \cdot (-5), (-3) \cdot (-3), (-3) \cdot 2\} = \{ 15, 9, -6 \}.
        \end{aligned}
    \end{equation*}
\end{solution}


\begin{problem}
    $A = [-4, -3]$, $B = [-2, 6]$, $\lambda = 2$. Знайти $A + B$ і $\lambda A$.
\end{problem}

\begin{solution}
    $A + B = [(-4) + (-2), (-3) + 6] = [-6, 3]$, $\lambda A = [2\cdot(-4), 2\cdot(-3)] = [-8, -6]$.
\end{solution}

\begin{problem}
    $M = \begin{pmatrix} -2 & 1 \\ 3 & 7 \\ 0 & 4 \end{pmatrix}$, $A = \left\{ \begin{pmatrix} 1 \\ 0 \end{pmatrix}, \begin{pmatrix} -1 \\ 2 \end{pmatrix}, \begin{pmatrix} 3 \\ -2 \end{pmatrix} \right\}$. Знайти $MA$.
\end{problem}

\begin{solution}
    $MA = \left\{ \begin{pmatrix} -2 & 1 \\ 3 & 7 \\ 0 & 4 \end{pmatrix} \begin{pmatrix} 1 \\ 0 \end{pmatrix}, \begin{pmatrix} -2 & 1 \\ 3 & 7 \\ 0 & 4 \end{pmatrix} \begin{pmatrix} -1 \\ 2 \end{pmatrix}, \begin{pmatrix} -2 & 1 \\ 3 & 7 \\ 0 & 4 \end{pmatrix}  \begin{pmatrix} 3 \\ -2 \end{pmatrix} \right\} = \left\{ \begin{pmatrix} -2 \\ 3 \\ 0 \end{pmatrix}, \begin{pmatrix} 4 \\ 11 \\ 8 \end{pmatrix}, \begin{pmatrix} -8 \\ -5 \\ -8 \end{pmatrix}\right\}$.
\end{solution}

\setcounter{problem}{4}

\begin{problem}
    Знайти опорні функції таких множин:
    \begin{enumerate}
        \item $A = \{ -1, 1 \}$;
        \item $A = \{ (x_1, x_2, x_3) : |x_1| \le 2, |x_2| \le  4, |x_3| \le 1 \}$;
        \item $A = \{ a \}$;
        \item $A = \mathcal{K}_r(a) = \{ x\in \RR^n : \| x - a \| \le r \}$.
    \end{enumerate}
\end{problem}

\begin{solution}
    \begin{enumerate}
        \item За визначенням, $c(A, \psi) = \Max_{x \in \{-1, 1\}} \langle x, \psi\rangle = \max(-\psi,\psi) = |\psi|$.
        \item За визначенням, $c(A, \psi) = \Max_{\substack{ x_1 : |x_1| \le 2 \\ x_2 : |x_2| \le  4 \\ x_3 : |x_3| \le 1 }} x_1 \psi_1 + x_2 \psi_2 + x_3 \psi_3  = 2 |\psi_1| + 4 |\psi_2| + |\psi_3|$.
        \item За визначенням, $c(A, \psi) = \Max_{x \in \{ a \}} \langle x, \psi\rangle = \langle a, \psi\rangle $.
        \item За визначенням, 
        \begin{equation*}
        \begin{aligned}
        c(A, \psi) &= \Max_{x \in \RR^n : \| x - a \| \le r} \langle x, \psi\rangle = \Max_{y \in \RR^n : \| y \| \le r} \langle a + y, \psi\rangle = \langle a, \psi\rangle + \Max_{y \in \RR^n : \| y \| \le r} \langle y, \psi\rangle = \\
        &= \langle a, \psi\rangle + c(\mathcal{K}_r(0), \psi) = \langle a, \psi \rangle + r\|\psi\|.
        \end{aligned}
        \end{equation*}
    \end{enumerate}
\end{solution}

\begin{problem}
    Знайти інтеграл Аумана $\mathcal{J} = \Int_0^{\pi/2} F(x) dx$ таких багатозначних відображень:
    \begin{enumerate}
        \item $F(x) = [0, \sin x]$, $x \in [0, \pi / 2]$.
        \item $F(x) = [-\sin x, \sin x]$, $x \in [0, \pi / 2]$.
        \item $F(x) = \mathcal{K}_{\sin x}(0) = \{ y \in \RR^n : \| y \| \le \sin x \}$, $x \in [0, \pi / 2]$.
    \end{enumerate}
\end{problem}

\begin{solution}
    Скористаємося теоремою про зміну порядку інтегрування і взяття опорної функції:
    \begin{enumerate}
        \item $c(\mathcal{J}, \psi) = \Int_0^{\pi/2} c([0, \sin x], \psi) dx = \Int_0^{\pi/2} \max(0, \psi) \sin x dx = \max(0, \psi)$, звідки $\mathcal{J} = [0, 1]$.
        \item $c(\mathcal{J}, \psi) = \Int_0^{\pi/2} c([-\sin x, \sin x], \psi) dx = \Int_0^{\pi/2} |\psi| \sin x dx = |\psi|$, звідки $\mathcal{J} = [-1, 1]$.
        \item $c(\mathcal{J}, \psi) = \Int_0^{\pi/2} c(\mathcal{K}_{\sin x}(0), \psi) dx = \Int_0^{\pi/2} \sin x \|\psi\| dx = \|\psi\|$, звідки $\mathcal{J} = \mathcal{K}_1(0)$. 
    \end{enumerate}
\end{solution}

\begin{problem}
    Знайти множину досяжності такої системи керування:
    \[\dfrac{dx}{dt} = x + bu,\] 
    де $x(0) = x_0 \in \mathcal{M}_0$, $u(t)\in \mathcal{U}$, $t\ge0$, $b$ -- деяке ненульове число, 
    \[ \mathcal{M}_0 = \{ x : | x | \le 2 \}, \]
    \[ \mathcal{U} = \{ u : |u| \le 3 \}. \]
\end{problem}

\begin{solution}
    Множину досяжності знайдемо через її опорну функцію: 
    \[ c(X(t, \mathcal{M}_0), \psi) = c(\mathcal{M}_0, \Theta^\star(t, t_0) \psi) + \Int_{t_0}^t c(\mathcal{U}(s), C^\star(s) \Theta^\star(t, s)\psi) ds. \]
    Для цього послідовно знаходимо: \\
    
    $\Theta(t, s) = e^{t-s}$, знайдено із рівності $\dfrac{d\Theta(t,s)}{dt} = A(t)\Theta(t,s) = \Theta(t,s)$ у нашому випадку. \\
    
    $c(\mathcal{M}_0, \psi) = c([-2, 2], \psi) = 2 |\psi|$, вже достатньо відома нам опорна функція. \\
    
    $c(\mathcal{U}(s), \psi) = c([-3, 3], \psi) = 3 |\psi|$, ще одна вже достатньо відома нам опорна функція. \\
    
    % Послідовно пісдтавляючи знайдені вирази в формулу вище знаходимо:
    % \begin{equation*}
    % \begin{split}
    %     c(X(t, \mathcal{M}_0), \psi) &= c(\mathcal{M}_0, \Theta^\star(t, t_0), \psi) + \Int_{t_0}^t c(\mathcal{U}(s), C^\star(s) \Theta^\star(t, s)\psi) ds = \\
    %     &= c([-2,2], \Theta^\star(t, 0), \psi) + \Int_0^t c([-3, 3], b \Theta^\star(t, s)\psi) ds = \\
    %     &= 2\left|\Theta^\star(t, 0)\psi\right| + \Int_0^t 3\left|b \Theta^\star(t, s)\psi\right| ds = \\
    %     &= 2\left|e^{-t}\psi\right| + \Int_0^t 3\left|b e^{s-t}\psi\right| ds = 2e^{-t}|\psi| + 3|b \psi| \Int_0^t e^{s-t} ds = \\
    %     &= 2e^{-t}|\psi| + 3|b \psi| \left(1 - e^{-t}\right) = \left(2e^{-t} + 3|b|\left(1 - e^{-t}\right)\right) |\psi|,
    % \end{split}
    % \end{equation*}
    % звідки $X(t, \mathcal{M}_0) = \left[-2e^{-t} - 3|b|\left(1 - e^{-t}\right), 2e^{-t} + 3|b|\left(1 - e^{-t}\right)\right]$.
    
    
    Послідовно пісдтавляючи знайдені вирази в формулу вище знаходимо:
    \begin{equation*}
    \begin{split}
        c(X(t, \mathcal{M}_0), \psi) &= c(\mathcal{M}_0, \Theta^\star(t, t_0), \psi) + \Int_{t_0}^t c(\mathcal{U}(s), C^\star(s) \Theta^\star(t, s)\psi) ds = \\
        &= c([-2,2], \Theta^\star(t, 0), \psi) + \Int_0^t c([-3, 3], b \Theta^\star(t, s)\psi) ds = \\
        &= 2\left|\Theta^\star(t, 0)\psi\right| + \Int_0^t 3\left|b \Theta^\star(t, s)\psi\right| ds = \\
        &= 2\left|e^t\psi\right| + \Int_0^t 3\left|b e^{t-s}\psi\right| ds = 2e^t|\psi| + 3|b \psi| \Int_0^t e^{t-s} ds = \\
        &= 2e^t|\psi| + 3|b \psi| \left(e^t - 1\right) = \left(2e^t + 3|b|\left(e^t - 1\right)\right) |\psi|,
    \end{split}
    \end{equation*}
    звідки $X(t, \mathcal{M}_0) = \left[-2e^t - 3|b|\left(e^t - 1\right), 2e^t + 3|b|\left(e^t - 1\right)\right]$.
\end{solution}