% OK, complete

\section{Задача про переведення системи з точки в точку. Критерії керованості лінійної системи керування}

\subsection{Аудиторне заняття}

\begin{problem}
	Перевести систему \[ \frac{\diff x}{\diff t} = u, \quad t \in [0, T], \] з точки $x (0) = x_0$ в точку $x (T) = y_0$ за допомогою керування з класу:
	\begin{enumerate}
		\item постійних функцій $u (t) = c$, $c$ -- константа;

		\item кусково-постійних функцій \[ u (t) = \begin{cases} c_1, & t \in [0, t_1], \\ c_2, & t \in [t_1, T]. \end{cases} \]

		Тут $c_1$, $c_2$ -- константи, $c_1 \ne c_2$, $0 < t_1 < T$;

		\item програмних керувань $u(t) = c t$, $c$ -- константа;

		\item керувань з оберненим зв'язком $u(x) = c x$, $c$ -- константа.
	\end{enumerate}
\end{problem}

\begin{solution}
	Скористаємося формулою $x (T) = x (0) + \int_0^T \frac{\diff x}{\diff t} \diff t$:

	\begin{enumerate}
		\item \[x (T) = x (0) + \int_0^T c \diff t = x (0) + c T,\] звідки \[c = \frac{x (T) - x (0)}{T} = \frac{y_0 - x_0}{T};\]

		\item \[ x (T) = x (0) + \int_0^{t_1} c_1 \diff t + \int_{t_1}^T c_2 \diff t = x (0) + c_1 t_1 + c_2 (T - t_1). \] Розв'язок не єдиний, \[ c_2 = \frac{x (T) - x (0) - c_1 t_1}{T - t_1}, \] де $c_1$ -- довільна стала, наприклад $c_1 = 0$, тоді \[ c_2 = \frac{x (T) - x (0)}{T - t_1} = \frac{y_0 - x_0}{T - t_1}. \]

		\item \[ x (T) = x (0) + \int_0^T c t \diff t = x (0) + \frac{cT^2}{2}, \] звідки \[ c = \frac{2 (x (T) - x (0))}{T^2} = \frac{2 (y_0 - x_0)}{T^2}. \]

		\item У цьому випадку проінтегрувати не можна, бо $u$ залежить від $x$, тому просто запишемо за формулою Коші \[ x (T) = x (0) \cdot e^{c T}, \] звідки \[c = \frac{\ln(x (T) / x (0))}{T} = \frac{\ln(y_0) - \ln(x_0)}{T}. \]

		Варто зауважити, не для всіх пар $x_0$ і $y_0$ коректно визначається значення $c$. А саме, необхідно щоб $y_0$ було того ж знаку, що і $x_0$.
	\end{enumerate}
\end{solution}

\begin{problem}
	\begin{enumerate}
		\item Використовуючи означення, знайти грамміан керованості для системи керування \[ \frac{\diff x(t)}{\diff t} = t x (t) + \cos (t) \cdot u(t), \quad t \ge 0. \]

		\item Записати диференціальне рівняння для грамміана керованості і за його допомогою знайти грамміан керованості.

		\item Використовуючи критерій керованості, вказати інтервал повної керованості вказаної системи керування. Для цього інтервала записати керування, яке розв'язує задачу про переведення системи з точки $x_0$ у стан $x_T$.
	\end{enumerate}
\end{problem}

\begin{solution}
	\begin{enumerate}
		\item Скористаємося формулою \[\Phi(T, t_0) = \int_{t_0}^T \Theta(T, s) B(s) B^*(s) \Theta^*(T, s) \diff s.\] $\Theta(T, s)$ знаходимо з системи \[ \frac{\diff \Theta(t,s)}{\diff t} = A(t) \cdot \Theta(t, s) = t \cdot \Theta(t, s),\] а саме $\Theta(t, s) = \exp\left\{\frac{t^2 - s^2}{2}\right\}$. Підставляючи всі знайдені значення, отримаємо \[\Phi(T, t_0) = \cos^2(T) \cdot e^{T^2} \int_{t_0}^T e^{-s^2} \diff s = \frac{1}{2} \sqrt{\pi} \cdot \cos^2(T) \cdot e^{T^2} \cdot \erf(T).\]

		\item Запишемо систему \[ \frac{\diff \Phi (t, t_0)}{\diff t} = A(t) \cdot \Phi(t, t_0) + \Phi(t, t_0) \cdot A^*(t) + B(t) \cdot B^*(t), \quad \Phi(t_0, t_0) = 0. \]

		І підставимо відомі значення: \[ \frac{\diff \Phi (t, 0)}{\diff t} = 2 t \Phi(t, 0)  + \cos^2(t), \quad \Phi(0, 0) = 0. \]

		Звідси \[ \Phi(t, 0) = \frac{1}{8} \sqrt{\pi} e^{t^2 - 1} ( - 2 e \erf(t) + i (\erfi(1 + i t) - i \erfi(1 - i t)), \]

		а \[ \Phi(T, 0) = \frac{1}{8} \sqrt{\pi} e^{T^2 - 1} ( - 2 e \erf(T) + i (\erfi(1 + i T) - i \erfi(1 - i T)), \]

		\item З вигляду грамміану керованості отриманого у першому пункті очевидно, що система цілком керована на півінтервалі $[0, \pi / 2)$, зокрема на інтервалі $[0, 1]$. \\

		Підставимо тепер граміан у формулу для керування що розв'язує задачу про переведення системи із стану $x_0$ у стан $x_T$:
		\begin{multline*} u(t) = B^*(t) \Theta^*(T, t) \Phi^{-1}(T, t_0) (x_T - \Theta(T, t_0) x_0) = \\ = \cos(t) \cdot \exp\left\{\frac{T^2-t^2}{2}\right\} \Phi^{-1} (T, 0) \left(x_T - \exp\left\{\frac{T^2}{2}\right\} x_0\right) \end{multline*}
	\end{enumerate}
\end{solution}

\begin{problem}
	За допомогою грамміана керованості розв'язати таку задачу оптимального керування: мінімізувати критерій якості \[ \JJ (u) = \int_0^T u^2(s) \diff s\] за умов, що \[ \frac{\diff x(t)}{\diff t} = \sin(t) \cdot x(t) + u(t), \quad x (0) = x_0, x (T) = x_T. \]

	Тут $x$ -- стан системи, $u(t)$ -- скалярне керування, $x_0$, $X_T$ -- задані точки, $t \in [t_0, T]$.
\end{problem}

\begin{solution}
	Знайдемо шукане керування за формулою \[ u(t) = B^*(t) \Theta^*(T, t) \Phi^{-1}(T, t_0) (x_T - \Theta(T, t_0) x_0). \]

	У цій задачі $\Theta(t, s) = e^{\cos (s) - \cos (t)}$, знайдене з системи $\dot \Theta = A \Theta$, $\Phi(T, t_0) = e^{-2 \cos (T)} \int_0^T e^{2 \cos (s)} \diff s$, підставляючи знаходимо \[ u(t) = \frac{e^{\cos (t) + \cos (T)} \cdot (x_T - e^{1 - \cos (T)} x_0)}{\int_0^T e^{2 \cos (s)} \diff s}. \]
\end{solution}

\begin{problem}
    За допомогою грамміана керованості розв'язати таку задачу оптимального керування: мінімізувати критерій якості
    \[ \mathcal{J}(u) = \int_0^T u^2(s) dx \]
    за умов, що
    \[ \dfrac{d^2x(t)}{dt^2} - 5\dfrac{dx(t)}{dt} + 6x(t) = u(t), \]
    \[ x(0) = x_0, x'(0) = y_0, x(T) = x'(T) = 0.\]
    Тут $x$ -- стан системми, $u(t)$ -- скалярне керування, $t \in [0, T]$.
\end{problem}

\begin{solution}
    Почнемо з того що зведемо рівняння другого порядку до системи рівнянь заміною $x_1 = x$, $x_2 = \dot x_1$, тоді маємо систему
    \[ \begin{pmatrix} \dot x_1 \\ \dot x_2 \end{pmatrix} (t) = \begin{pmatrix} 0 & 1 \\ -6 & 5 \end{pmatrix} \begin{pmatrix} x_1 \\ x_2 \end{pmatrix} (t) + \begin{pmatrix} 0 \\ 1 \end{pmatrix} u(t). \]
    
    Знайдемо власні числа матриці $A - \lambda E$: $\det(A - \lambda E) = \begin{vmatrix} -\lambda & 1 \\ -6 & 5-\lambda \end{vmatrix} = \lambda^2 - 5\lambda + 6 = (\lambda - 2) (\lambda - 3) = 0$, звідки $\lambda_1 = 2$, $\lambda_2 = 3$. Знайдемо власні вектори, вони будуть $\begin{pmatrix} 1 \\ 2 \end{pmatrix}$ і $\begin{pmatrix} 1 \\ 3 \end{pmatrix}$ відповідно. Звідси знаходимо загальний розв'язок
    \[ \begin{pmatrix} x_1 \\ x_2 \end{pmatrix} (t) = c_1 \begin{pmatrix} e^{2t} \\ 2e^{2t} \end{pmatrix} + c_2 \begin{pmatrix} e^{3t} \\ 3e^{3t} \end{pmatrix}. \]
    
    З рівняння
    \[c_1 \begin{pmatrix} e^{2s} \\ 2e^{2s} \end{pmatrix} + c_2 \begin{pmatrix} e^{3s} \\ 3e^{3s} \end{pmatrix} = \begin{pmatrix} 1 \\ 0 \end{pmatrix} \]
    знаходимо $c_1 = 3e^{-2s}$, $c_2 = -2e^{-3s}$, а з рівняння
    \[c_1 \begin{pmatrix} e^{2s} \\ 2e^{2s} \end{pmatrix} + c_2 \begin{pmatrix} e^{3s} \\ 3e^{3s} \end{pmatrix} = \begin{pmatrix} 0 \\ 1 \end{pmatrix} \]
    знаходимо $c_1 = -e^{-2s}$, $c_2 = e^{-3s}$, тобто
    \[ \Theta(t, s) = \begin{pmatrix} 3e^{2(t-s)} - 2e^{3(t-s)} & -e^{2(t-s)} + e^{3(t-s)} \\ 6e^{2(t-s)} - 6e^{3(t-s)} & -2e^{2(t-s)} + 3e^{3(t-s)} \end{pmatrix}. \]
    
    Знайдемо грамміан за формулою \[\Phi(T, 0) = \int_0^T \Theta(T, s) B(s) B^* (s) \Theta^*(T, s) ds. \]
    
    \[ \Theta(T, s) B(s) = \begin{pmatrix} -e^{2(T - s)} + e^{3(T - s)} \\ -2e^{2(T - s)} + 3e^{3(T - s)} \end{pmatrix}. \]
    \[ B^* (s) \Theta^*(T, s)  =  (\Theta(T, s) B(s))^\star = \begin{pmatrix} -e^{2(T - s)} + e^{3(T - s)} & -2e^{2(T - s)} + 3e^{3(T - s)} \end{pmatrix}. \]
    
    \begin{align*} 
        \Phi(T, 0) &= \int_0^T \begin{pmatrix} -e^{2(T - s)} + e^{3(T - s)} \\ -2e^{2(T - s)} + 3e^{3(T - s)} \end{pmatrix} \begin{pmatrix} -e^{2(T - s)} + e^{3(T - s)} & -2e^{2(T - s)} + 3e^{3(T - s)} \end{pmatrix} ds = \\
        &= \int_0^T \begin{pmatrix} e^{4(T - s)} - 2 e^{5(T - s)} + e^{6(T - s)} & 2 e^{4(T - s)} - 5 e^{5(T - s)} + 3 e^{6(T - s)} \\ 2e^{4(T - s)} - 5 e^{5(T - s)} + 3 e^{6(T - s)} & 4 e^{4(T - s)} - 12 e^{5(T - s)} + 9  e^{6(T - s)} \end{pmatrix} ds = \\
        &= \begin{pmatrix} \dfrac{e^{4T} - 1}{4} - \dfrac{2(e^{5T} - 1)}{5} + \dfrac{e^{6T} - 1}{6} & \dfrac{e^{4T} - 1}{2} - (e^{5T} - 1) + \dfrac{e^{6T} - 1}{2} \\ \\ \dfrac{e^{4T} - 1}{2} - (e^{5T} - 1) + \dfrac{e^{6T} - 1}{2} & (e^{4T} - 1) - \dfrac{12(e^{5T} - 1)}{5} + \dfrac{3(e^{6T} - 1)}{2} \end{pmatrix}
    \end{align*}
    
    Чесно кажучи вже обчислення визначника грамміану є надто складною обчислювальною задачею, не бачу сенсу її робити вручну.
\end{solution}

\begin{problem}
	Записати систему диференціальних рівнянь для знаходження першої матриці керованості (грамміана керованості) і сформулювати критерій керованості на інтервалі $[0, T]$ у випадку, якщо система керування має вигляд:
	\begin{enumerate}
		\item \[ 
		\left\{
			\begin{aligned}
				\frac{\diff x_1(t)}{\diff t} = t x_1 (t) + x_2 (t) + u_1 (t), \\
				\frac{\diff x_2(t)}{\diff t} = - x_1 (t) + 2 x_2 (t) + t^2 u_2 (t).
			\end{aligned}
		\right.
		\]

		Тут $x = (x_1, x_2)^*$ -- вектор стану, $u = (u_1, u_2)^*$ -- вектор керування, $t \in [0, T]$.

		\item \[ \frac{\diff^2 x(t)}{\diff t^2} + \sin(t) \cdot x(t) = u(t). \]

		Тут $x$ -- стан системи, $u(t)$ -- скалярне керування, $t \in [0, T]$.
	\end{enumerate}
\end{problem}

\begin{solution}
	\begin{enumerate}
		\item $A = \begin{pmatrix} t & 1 \\ -1 & 2 \end{pmatrix}$, $B = \begin{pmatrix} 1 & 0 \\ 0 & t^2 \end{pmatrix}$, \[
		\left\{
			\begin{aligned}
				\dot \phi_{11} &= 2 t \phi_{11} + 2 \phi_{12} + 1, \\
				\dot \phi_{12} &= - \phi_{11} + (t + 2) \phi_{12} + \phi_{22}, \\
				\dot \phi_{21} &= \ldots 
			\end{aligned}
		\right.
		\]

		\item Введемо нову змінну $x_2 = \dot x$, тоді $A = \begin{pmatrix} 0 & 1 \\ -\sin(t) & 0 \end{pmatrix}$, $B = \begin{pmatrix} 0 \\ 1 \end{pmatrix}$, \[
		\left\{
			\begin{aligned}
				\dot \phi_{11} &= 2 \phi_{12}, \\
				\dot \phi_{12} &= (1 - \sin(t)) \phi_{11}, \\
				\ldots 
			\end{aligned}
		\right.
		\]
	\end{enumerate}
\end{solution}

\begin{problem}
 	Знайти диференціальне рівняння грамміана керованості для системи керування \[ \left\{ \begin{aligned}
 		\frac{\diff x_1 (t)}{\diff t} &= \cos(t) \cdot x_1(t)-\sin(t) \cdot x_2(t) + u_1(t) - 2 u_2(t), \\
 		\frac{\diff x_2 (t)}{\diff t} &= \sin(t) \cdot x_1(t)+\cos(t) \cdot x_2(t) - 3 u_1(t) + 4 u_2(t).
 	\end{aligned} \right. \]
\end{problem}

\begin{solution}
 	% 3.6
\end{solution}


\begin{problem}
    Дослідити системи на керованість. використовуючи другий критерій керованості:
    \begin{enumerate}
        \item \[\ddot x + a \dot x + b x = u; \]
        \item \[ \left\{ \begin{aligned} \dot x_1 &= 2x_1 + x_2 + au \\ \dot x_2 &= x_1 + 4 x_2 + u \end{aligned} \right. \]
        \item \[ \left\{ \begin{aligned} \dot x_1 &= 2x_1 + x_2 + u_1 \\ \dot x_2 &= x_1 + 3 x_3 + u_2 \\ \dot x_3 &= x_2 + x_3 + u_2  \end{aligned} \right. \]
    \end{enumerate}
\end{problem}

\begin{solution}
    \begin{enumerate}
        \item Почнемо з того що зведемо рівняння другого порядку до системи рівнянь заміною $x_1 = x$, $x_2 = \dot x_1$, тоді маємо систему
        \[ \left\{ \begin{aligned} \dot x_1 &= x_2 \\ \dot x_2 &= - a x_2 - b x_1 + u  \end{aligned} \right. \]
        Тоді
        \[ A = \begin{pmatrix} 0 & 1 \\ -b & -a \end{pmatrix} \qquad B = \begin{pmatrix} 0 \\ 1 \end{pmatrix}. \]
        \[ D = \begin{pmatrix} B & AB \end{pmatrix} = \begin{pmatrix} 0 & 1 \\ 1 & - a \end{pmatrix}. \]
        Її ранг дорівнює 2 якщо за будь-яких $a$ і $b$, тобто система завжди цілком керована.
        \item 
        \[ A = \begin{pmatrix} 2 & 1 \\ 1 & 4 \end{pmatrix} \qquad B = \begin{pmatrix} a \\ 1 \end{pmatrix}. \]
        \[ D = \begin{pmatrix} B & AB \end{pmatrix} = \begin{pmatrix} a & 2 a + 1 \\ 1 & a + 4 \end{pmatrix}. \]
        Її визначник $a^2 + 4a - 2a - 1 = a^2 + 2a - 1 = 0$ якщо $a = -1 \pm \sqrt 2$, тоді система не є цілком керованою, а інакше є.
        \item 
        \[ A = \begin{pmatrix} 2 & 1 & 0 \\ 1 & 0 & 3 \\ 0 & 1 & 1 \end{pmatrix} \qquad B = \begin{pmatrix} 1 & 0 \\ 0 & 1 \\ 0 & 1  \end{pmatrix}. \]
        \[ D = \begin{pmatrix} B & AB & A^2B \end{pmatrix} = \begin{pmatrix} 1 & 0 & 2 & 1 & 5 & 5 \\ 0 & 1 & 1 & 3 & 2 & 7 \\ 0 & 1 & 0 & 2 & 1 & 5 \end{pmatrix} .\]
        Її ранг дорівнює 3, тобто система цілком керована.
    \end{enumerate}    
\end{solution}