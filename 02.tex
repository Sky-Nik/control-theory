% cd ..\..\Users\NikitaSkybytskyi\Desktop\c3s1\control-theory

% cls && pdflatex 02.tex && cls && pdflatex 02.tex && del *.aux, *.log, *.out && start 02.pdf

\documentclass[a4paper, 12pt]{article}
\usepackage[T2A,T1]{fontenc}
\usepackage[utf8]{inputenc}
\usepackage[english, ukrainian]{babel}
\usepackage{amsmath, amssymb}

\usepackage{amsthm}
\theoremstyle{definition}
\newtheorem{algorithm}{Алгоритм}[section]
\newtheorem*{problem*}{\normalfont{\textit{Задача}}}
\newtheorem{problem}{\normalfont{\textit{Задача}}}[section]
\newtheorem*{solution}{Розв'язок}
% \usepackage[showframe]{geometry}

\allowdisplaybreaks
\numberwithin{equation}{section}

\usepackage{xcolor}
\usepackage{hyperref}
\hypersetup{unicode=true,colorlinks=true,linktoc=all,linkcolor=red}

\usepackage{graphicx}

\newcommand{\JJ}{\mathcal{J}}
\newcommand{\KK}{\mathcal{K}}
\newcommand{\MM}{\mathcal{M}}
\newcommand{\UU}{\mathcal{U}}
\newcommand{\XX}{\mathcal{X}}
\newcommand{\RR}{\mathbb{R}}

\newcommand{\Max}{\displaystyle\max\limits}

\renewcommand{\SS}{\mathcal{S}}

\newcommand*\diff{\mathop{}\!\mathrm{d}}

\begin{document}

\setcounter{section}{1}
\section{Елементи багатозначного аналізу. Множина досяжності}

\subsection{Алгоритми}

\begin{problem*}
	Знайти
	\begin{enumerate}
		\item $A+B$;
		\item $\lambda A$;
		\item $\alpha(A,B)$;
		\item $MA$,
	\end{enumerate}
	де множини $A \subset\RR^m$, $B\subset\RR^m$, скаляр $\lambda\in\RR^1$, матриця $M\in\RR^{n\times m}$.
\end{problem*}

\begin{algorithm}
	\label{algo-2-1}
	Розглянемо всі пункти задачі вище.
	\begin{enumerate}
		\item Знаходимо за визначенням, $A+B=\{a+b|a\in A,b\in B\}$.
		\item Знаходимо за визначенням, $\lambda A =\{\lambda a|a\in A\}$. 
		\item \begin{enumerate}
			\item Знаходимо відхилення $\beta(A,B)$ і $\beta(B,A)$ за визначенням, 
			\begin{equation}
				\label{eq:2.1}
				\beta(A,B) = \max_{a\in A}\rho(a,B),
			\end{equation}
			де 
			\begin{equation}
				\label{eq:2.2}
				\rho(a,B) = \min_{b\in B} \rho(a,b).
			\end{equation}
			\item Знаходимо $\alpha(A,B)$ за визначенням, 
			\begin{equation}
				\label{eq:2.3}
				\alpha(A,B)=\max\{\beta(A,B),\beta(B,A).
			\end{equation}
		\end{enumerate} 
		\item Знаходимо за визначенням, $MA=\{Ma|a\in A\}$.
	\end{enumerate}
\end{algorithm}

\vspace*{\baselineskip}

\begin{problem*}
	Знайти опорну функцію множини $A \subset \RR^n$.
\end{problem*}

\begin{algorithm}
	\label{algo-2-2}
	\begin{enumerate}
		\item Намагаємося знайти за визначенням,
		\begin{equation}
		 	\label{eq:2.4}
		 	c(A,\psi) = \Max_{a\in A} \langle a, \psi \rangle.
		\end{equation}
		\item Якщо не вийшло, то намагаємося знайти за геометричною властивістю: $c(A,\psi)$ -- (орієнтована) відстань від початку координат до опорної площини множини $A$, для якої напрямок-вектор $\psi$ є вектором нормалі.
	\end{enumerate}
\end{algorithm}

\vspace*{\baselineskip}

\begin{problem*}
	Знайти інтеграл Аумана $\JJ = \int F(x) \diff x$, де $F(x)\subset\RR^n$.
\end{problem*}

\begin{algorithm}
	\label{algo-2-3}
	\begin{enumerate}
		\item Знаходимо опорну функцію від інтегралу:
		\begin{equation}
		 	\label{eq:2.5}
		 	c(\JJ, \psi) = \int c (F(x), \psi) \diff x.
		\end{equation}
		\item Знаходимо $\JJ$ як опуклий компакт з відомою опорною функцією $c(\JJ, \psi)$.
	\end{enumerate}
\end{algorithm}

\vspace*{\baselineskip}

\begin{problem*}
	Знайти множину досяжності системи $\dot x = A x + B u$, де $x(t_0) \in \mathcal{M}_0$, $u \in \mathcal{U}$.
\end{problem*}

\begin{algorithm}
	\label{algo-2-4}
	\begin{enumerate}
		\item Знаходимо фундаментальну матрицю $\Theta(t,s)$ системи нормовану за моментом $s$.
		\item Знаходимо інтеграл Аумана
		\begin{equation}
			\label{eq:2.6}
			\int_{t_0}^t \Theta(t, s) B(s) \UU(s) \diff s
		\end{equation}
		за алгоритмом \ref{algo-2-3}.
		\item Використовуємо теорему про вигляд множини досяжності лінійної системи керування:
		\begin{equation}
			\label{eq:2.7}
		 	\XX(t, \MM_0) = \Theta(t, t_0) \MM_0 + \int_{t_0}^t \Theta(t, s)B(s)\UU(s) \diff s.
		\end{equation}
	\end{enumerate}
\end{algorithm}

\vspace*{\baselineskip}

\begin{problem*}
	Знайти опорну функцію множини досяжності системи $\dot x = A x + B u$, де $x(t_0) \in \mathcal{M}_0$, $u \in \mathcal{U}$.
\end{problem*}

\begin{algorithm}
	\label{algo-2-5}
	\begin{enumerate}
		\item Знаходимо фундаментальну матрицю $\Theta(t,s)$ системи нормовану за моментом $s$.
		\item Знаходимо опорну функцію $c(\MM_0, \Theta^*(t, t_0) \psi)$ за алгоритмом \ref{algo-2-2}.
		\item Знаходимо опорну функцію $c(\UU(s), B^*(s) \Theta^*(t, s) \psi)$ за алгоритмом \ref{algo-2-2}.
		\item Використовуємо теорему про вигляд опорної функції множини досяжності лінійної системи керування: 
		\begin{equation}
			\label{eq:2.8}
			c(\XX(t, \MM_0), \psi) = c(\MM_0, \Theta^*(t, t_0) \psi) + \int_{t_0}^t c(\UU(s), B^*(s) \Theta^*(t, s) \psi) \diff s.
		\end{equation}
	\end{enumerate}
\end{algorithm}



\subsection{Аудиторне заняття}

\begin{problem}
	Знайти $A + B$ і $\lambda A$, а також метрику Хаусдорфа $\alpha (A, B)$, якщо:
	
	\begin{enumerate}
	    \item $A = \{-3, 2, -1\},  B = \{-2, 5, 1\},  \lambda = 3$;
	    
	    \item $A = \{4, 2, -4\},  B = [-2, 3],  \lambda = -1$;
	    
	    \item $A = [-1, 2],  B = [3, 7],  \lambda = -2$;	    
	\end{enumerate}
	
\end{problem}

\begin{solution}
	Скористаємося пунктами 1-3 алгоритму \ref{algo-2-1}:
	\begin{enumerate}
	    \item \begin{align*} A + B &= \{-3-2, -3+5, -3+1, 2-2, 2+5, 2+1, -1-2, -1+5, -1+1\} =\\&= \{-5,2,-2,0,7,3,-3,4,0\}=\{-5,-3-2,0,2,3,4,7\},\end{align*}
	    \[ \lambda A = \{3 \cdot -3, 3 \cdot 2, 3 \cdot -1 \} = \{-9,6,-3\}=\{-9,-3,6\}.\]
	    
	    У нашому випадку \begin{align*}\beta (A, B) &= \max\left\{\min_{b\in B} \rho(-3,b), \min_{b\in B} \rho(2,b), \min_{b\in B} \rho(-1,b)\right\}=\\&=\max\{1,1,1\}=1, \\ \beta (B, A) &= \max\left\{\min_{a\in A} \rho(-2,a), \min_{a\in A} \rho(5,a), \min_{a\in A} \rho(1,a)\right\}=\\&=\max\{1,3,1\}=3,\end{align*} тому $\alpha (A, B) = \max\{1,3\}=3$.
	    
	    \item $A + B = [-6,7]$, $\lambda A = \{-4, -2, 4 \}$.\\
	    
	 	$\beta (A, B) = \max\{1, 0, 2\}$, $\beta (B, A) = \max[0, 3]$, тому $\alpha (A, B) = 3$.
	    
	    \item $A + B = [2,9]$, $\lambda A = [-4,2]$. \\
	    
	    $\beta (A, B) = \max[1,4]$, $\beta (B, A) = \max[1, 5]$, оскільки відповідні краї відхиляються на $4$ та $5$ відповідно ($-1$ від $3$ та $7$ від $2$), тому $\alpha (A, B) = 5$.
	\end{enumerate}
\end{solution}

\begin{problem}
	Знайти $MA$, якщо
	\[
	M=
  \begin{pmatrix}
    -2 & 4 \\
    3 & 5
  \end{pmatrix}
  , A= 
  \left\{
  \begin{pmatrix}
    -1 \\
    2 
  \end{pmatrix},
    \begin{pmatrix}
    3 \\
    -4 
  \end{pmatrix},
    \begin{pmatrix}
    0 \\
    -2 
  \end{pmatrix}
  \right\}
  .\]
\end{problem}

\begin{solution}
	Скористаємося пунктом 4 алгоритму \ref{algo-2-1}:
	\begin{align*} 
		\begin{pmatrix} -2 & 4 \\ 3 & 5 \end{pmatrix} \cdot \begin{pmatrix} -1 \\ 2 \end{pmatrix} &= \begin{pmatrix} 10 \\ 7 \end{pmatrix}, \\
		\begin{pmatrix} -2 & 4 \\ 3 & 5 \end{pmatrix} \cdot \begin{pmatrix} 3 \\ -4 \end{pmatrix} &= \begin{pmatrix} -22 \\ -11 \end{pmatrix}, \\
		\begin{pmatrix} -2 & 4 \\ 3 & 5 \end{pmatrix} \cdot \begin{pmatrix} 0 \\ -2 \end{pmatrix} &= \begin{pmatrix} -8 \\ -10 \end{pmatrix}.
	\end{align*}
  
  Отже, отримаємо \[MA = \left\{ \begin{pmatrix} 10 \\ 7 \end{pmatrix}, \begin{pmatrix} -22 \\ -11 \end{pmatrix}, \begin{pmatrix} -8 \\ -10 \end{pmatrix} \right\}.\]
\end{solution}

\begin{problem}
	Знайти опорні функції таких множин:

	\begin{enumerate}
		\item $A = [0, r]$;

		\item $A = [-r, r]$;

		\item $A = \{ (x_1, x_2): |x_1| \le 1, |x_2| \le 2 \}$;

		\item $A = \KK_r (0) = \{ x \in \RR^n: \|x\| \le r \}$;

		\item $A = \SS^n = \{ x \in \RR^n: \|x\| = 1 \}$.
	\end{enumerate}
\end{problem}

\begin{solution}
	У пунктах 1-3 скористаємося пунктом 1 алгоритму \ref{algo-2-2},  а у пунктах 4-5 -- пунктом 2 того ж алгоритму.
	\begin{enumerate}
		\item За означенням опорної функції, \[ c(A, \psi) = \max_{a \in A} \langle a, \psi \rangle = \begin{cases} 0, & \psi < 0 \\ r \psi, & 0 \le \psi \end{cases} = \max \{ 0, r \psi \}. \]

		\item За означенням опорної функції, \[ c(A, \psi) = \max_{a \in A} \langle a, \psi \rangle = \begin{cases} - r \psi, & \psi < 0 \\ r \psi, & 0 \le \psi \end{cases} = r |\psi |. \]

		\item За означенням опорної функції, \[ c(A, \psi) = \max_{a \in A} \langle a, \psi \rangle = \max_{a \in A} (\psi_1 x_1 + \psi_2 x_2) = |\psi_1| + 2 |\psi_2|. \]

		\item За властивістю опорної функції, $c(\KK_r (0), \psi) = r \| \psi \|$.

		\item За властивістю опорної функції, $c(\SS^n, \psi) = \| \psi \|$.
	\end{enumerate}
\end{solution}

\begin{problem}
	Знайти інтеграл Аумана $\JJ = \int_0^1 F(x) \diff x$ таких багатозначних відображень:

	\begin{enumerate}
		\item $F(x) = [0, x]$, $x \in [0, 1]$;

		\item $F(x) = \KK_x (0) = \{ y \in \RR^n: \|y\| \le x \}$, $x \in [0, 1]$.
	\end{enumerate}
\end{problem}

\begin{solution}
	Скористаємося алгоритмом \ref{algo-2-3}:
	\begin{enumerate}
		\item \[c(\JJ) = \int_0^1 c([0, x], \psi) \diff x = \begin{cases} \psi / 2, & 0 \le \psi \\ 0, & \psi < 0 \end{cases}. \]

		Далі знання опорних функцій підказують, що $\JJ = [0, 1 / 2]$.

		\item \[c(\JJ) = \int_0^1 c(\KK_x(0), \psi) \diff x = \int_0^1 x \| \psi \| \diff x = \| \psi \| / 2. \]

		Далі знання опорних функцій підказують, що $\JJ = \KK_{1 / 2} (0)$.
	\end{enumerate}
\end{solution}

\begin{problem}
	Знайти множину досяжності такої системи керування: \[ \frac{\diff x}{\diff t} = x + u, \]

	де $x (0) = x_0 \in \MM_0$, $u (t) \in \UU$, $t \ge 0$, \[ \MM_0 = \{ x: |x| \le 1 \}, \] \[ \UU = \{ u: |u| \le 1 \}. \]
\end{problem}

\begin{solution}
	Скористаємося алгоритмом \ref{algo-2-4}. \\

	Перш за все знаходимо $\Theta(t,s)$. Нескладно бачити, що $\Theta(t, s) = e^{t - s}$. \\

	Далі \[ \XX(t, \MM_0) = \Theta(t, t_0) \MM_0 + \int_{t_0}^t \Theta(t, s) B(s) \UU(s) \diff s. \]

	Підставимо вже відомі значення: 
	\begin{align*} 
		\XX(t, [-1, 1]) &= \Theta(t, 0) \cdot [-1, 1] + \int_0^t \left( \Theta(t, s) \cdot 1 \cdot [-1, 1] \right) \diff s = \\
		&= [-e^t, e^t] + \int_0^t [-e^{t - s}, e^{t - s}] \diff s = \\ 
		&= [-e^t, e^t] + [1 - e^t, e^t - 1] = [1 - 2 e^t, 2 e^t - 1]. 
	\end{align*}
\end{solution}

\begin{problem}
	Знайти опорну функцію множини досяжності для системи керування: \[
	\left\{
		\begin{aligned}
			\frac{\diff x_1}{\diff t} &= 2x_1 + x_2 + u_1, \\
			\frac{\diff x_2}{\diff t} &= 3x_1 + 4x_2 + u_2,
		\end{aligned}
	\right.
	\]
	де $x (0) = (x_{01}, x_{02}) \in \MM_0$, $u(t) = (u_1(t), u_2(t)) \in \UU$, $t \ge 0$, \[ \MM_0 = \{ (x_{01}, x_{02}): |x_{01}| \le 1, |x_{02}| \le 1 \}, \] \[ \UU = \{(u_1, u_2): |u_1| \le 1, |u_2| \le 1\}. \]
\end{problem}

\begin{solution}
	Скористаємося алгоритмом \ref{algo-2-5}. \\

	Перш за все знаходимо $\Theta(t,s)$ з системи \[ \frac{\diff \Theta(t, s)}{\diff t} = A(t) \cdot \Theta(t, s) = \begin{pmatrix} 2 & 1 \\ 3 & 4 \end{pmatrix} \Theta(t, s). \]

	Нескладно бачити, що \[ \Theta(t, s) = \frac{1}{4}
	\begin{pmatrix}
		3 e^{t - s} + e^{5 (t - s)} & - e^{t - s} + e^{5 (t - s)} \\ -3 e^{t - s} + 3 e^{5 (t - s)} & e^{t - s} + 3 e^{5 (t - s)}
	\end{pmatrix} 
	\]

	Скористаємося теоремою про вигляд опорної функції множини досяжності лінійної системи керування: \[ c(\XX(t, \MM_0), \psi) = c(\MM_0, \Theta^*(t, t_0) \psi) + \int_{t_0}^t c(\UU(s), B^*(s) \Theta^*(t, s) \psi) \diff s. \]

	Підставимо вже відомі значення: \[ c(\XX(t, [-1,1]^2), \psi) = c([-1,1]^2, \Theta^*(t, 0) \psi) + \int_0^t c([-1,1]^2, \begin{pmatrix} 1 & 1 \end{pmatrix} \Theta^*(t, s) \psi) \diff s, \] 

	і підставляємо $\Theta$:
	\begin{multline*} 
		c\left(\XX(t, [-1,1]^2), \begin{pmatrix} \psi_1 \\ \psi_2 \end{pmatrix}\right) = c\left([-1,1]^2, \frac{1}{4} \begin{pmatrix} 3 e^t + e^{5 t} & -3 e^t + 3 e^{5 t} \\ - e^t + e^{5 t} & e^t + 3 e^{5 t}	\end{pmatrix} \begin{pmatrix} \psi_1 \\ \psi_2 \end{pmatrix} \right) + \\
		+ \int_0^t c\left([-1,1]^2, \begin{pmatrix} 1 & 0 \\ 0 & 1 \end{pmatrix} \frac{1}{4} \begin{pmatrix} 3 e^{t - s} + e^{5 (t - s)} & -3 e^{t - s} + 3 e^{5 (t - s)} \\ - e^{t - s} + e^{5 (t - s)} & e^{t - s} + 3 e^{5 (t - s)} \end{pmatrix} \begin{pmatrix} \psi_1 \\ \psi_2 \end{pmatrix} \right) \diff s = \\
		= c\left([-1,1]^2, \frac{1}{4} \begin{pmatrix} (3 e^t + e^{5 t}) \psi_1 + (-3 e^t + 3 e^{5 t}) \psi_2 \\ (- e^t + e^{5 t}) \psi_1 + (e^t + 3 e^{5 t}) \psi_2 \end{pmatrix}  \right) + \\
		+ \int_0^t c\left([-1,1]^2, \frac{1}{4} \begin{pmatrix} (3 e^{t - s} + e^{5 (t - s)}) \psi_1 + (-3 e^{t - s} + 3 e^{5 (t - s)}) \psi_2 \\ (- e^{t - s} + e^{5 (t - s)}) \psi_1 + (e^{t - s} + 3 e^{5 (t - s)}) \psi_2 \end{pmatrix} \right) \diff s = \\
		= \frac{1}{4} \left( \left|(3 e^t + e^{5 t}) \psi_1 + (-3 e^t + 3 e^{5 t}) \psi_2\right| + \left|(- e^t + e^{5 t}) \psi_1 + (e^t + 3 e^{5 t}) \psi_2\right| \right) + \\
		+ \frac{1}{4} \int_0^t \left|(3 e^{t - s} + e^{5 (t - s)}) \psi_1 + (-3 e^{t - s} + 3 e^{5 (t - s)}) \psi_2\right| \diff s + \\
		+ \frac{1}{4} \int_0^t \left|(- e^{t - s} + e^{5 (t - s)}) \psi_1 + (e^{t - s} + 3 e^{5 (t - s)}) \psi_2\right| \diff s.
	\end{multline*}
\end{solution}

% Тут необхідно доінтегрувати, але, здається, все одно нічого путнього (тобто добре відомої нам опорної функції) не вийде

\subsection{Домашнє завдання}

\begin{problem}
    Знайти $A + B$ і $\lambda A$, а також метрику Хаусдорфа $\alpha(A, B)$, якщо
    \begin{enumerate}
        \item $A = \{4,-2,3\}$, $B = \{7,-1,1\}$, $\lambda=2$;
        \item $A = \{5,-5,2\}$, $B = [1,3]$, $\lambda=-1$;
        \item $A = [-4,-2]$, $B = [-1,5]$, $\lambda=3$;
    \end{enumerate}
\end{problem}

% \begin{solution}
%     \begin{enumerate}
%         \item $A = \{4,-2,3\}$, $B = \{7,-1,1\}$, $\lambda=2$;
%         \begin{multline*} 
%             A + B = \{4+7,4-1,4+1,-2+7,-2-1,-2+1,3+7,3-1,3+1\}= \\
%             = \{11,3,5,5,-3,-1,10,2,4\} = \{-3,-1,2,3,4,5,10,11\}.
%         \end{multline*} 
%         \[ \lambda A = \{2 \cdot 4, 2 \cdot -2, 2 \cdot 3\} = \{8, -4, 6\}. \]
%         \begin{multline*} 
%             \alpha(A,B) = \max\{\beta(A,B),\beta(B,A)\} = \\
%             = \max\{\max\{3,1,2\},\max\{3,1,2\}\}=\max\{3,3\}=3.
%         \end{multline*} 
%         \item $A = \{5,-5,2\}$, $B = [1,3]$, $\lambda=-1$;
%         \begin{multline*} 
%             A + B = (5 + [1,3]) \cup (-5 + [1,3]) \cup (2+[1,3])= \\
%             = [6,8] \cup [-4,-1] \cup [3,5] = [-4,-1] \cup [3,5] \cup [6,8].
%         \end{multline*} 
%         \[ \lambda A = \{-1 \cdot 5, -1 \cdot -5, -1 \cdot 2\} = \{-5, 5, -2\}. \]
%         \begin{multline*} 
%             \alpha(A,B) = \max\{\beta(A,B),\beta(B,A)\} = \\
%             = \max\{\max\{2,6,0\},\max_{b\in[1,3]}\{|b-2|\}\}=\max\{6,1\}=6.
%         \end{multline*} 
%         \item $A = [-4,-2]$, $B = [-1,5]$, $\lambda=3$;
%         \[ A + B = [-4-1,-2+5] = [-5,3]. \]
%         \[ \lambda A = [3 \cdot -4, 3 \cdot -2] = [-12, -6]. \]
%         \begin{multline*} 
%             \alpha(A,B) = \max\{\beta(A,B),\beta(B,A)\} = \\
%             = \max\{\max\{|-4+1|,|-2+1|\},\max\{|-1+2|,|5+2|\}\}=\max\{3,7\}=7.
%         \end{multline*}
%     \end{enumerate}
% \end{solution}

\begin{problem}
    Знайти $MA$, якщо \[ M = \begin{pmatrix} 2 & 1 \\ -5 & 3 \end{pmatrix}, A = \left\{ \begin{pmatrix} -1 \\ 1 \end{pmatrix}, \begin{pmatrix} 2 \\ -4 \end{pmatrix}, \begin{pmatrix} -3 \\ -2 \end{pmatrix} \right\}. \]
\end{problem}

% \begin{solution}
%     \begin{multline*} 
%         M A = \left\{ M \begin{pmatrix} -1 \\ 1 \end{pmatrix}, M \begin{pmatrix} 2 \\ -4 \end{pmatrix}, M \begin{pmatrix} -3 \\ -2 \end{pmatrix} \right\} = \\
%         = \left\{ \begin{pmatrix} -1 \\ 8 \end{pmatrix}, \begin{pmatrix} 0 \\ -22 \end{pmatrix}, \begin{pmatrix} -8 \\ 9 \end{pmatrix} \right\}.
%     \end{multline*}        
% \end{solution}


\begin{problem}
    Знайти опорні функції таких множин:
    \begin{enumerate}
        \item $A = \{ -1, 1 \}$;
        \item $A = \{ (x_1, x_2, x_3) : |x_1| \le 2, |x_2| \le  4, |x_3| \le 1 \}$;
        \item $A = \{ a \}$;
        \item $A = \KK_r(a) = \{ x\in \RR^n : \| x - a \| \le r \}$.
    \end{enumerate}
\end{problem}

% \begin{solution}
%     \begin{enumerate}
%         \item За визначенням, $c(A, \psi) = \Max_{x \in \{-1, 1\}} \langle x, \psi\rangle = \max(-\psi,\psi) = |\psi|$.
%         \item За визначенням, $c(A, \psi) = \Max_{\substack{ x_1 : |x_1| \le 2 \\ x_2 : |x_2| \le  4 \\ x_3 : |x_3| \le 1 }} x_1 \psi_1 + x_2 \psi_2 + x_3 \psi_3  = 2 |\psi_1| + 4 |\psi_2| + |\psi_3|$.
%         \item За визначенням, $c(A, \psi) = \Max_{x \in \{ a \}} \langle x, \psi\rangle = \langle a, \psi\rangle $.
%         \item За визначенням, 
%         \begin{align*}
%             c(A, \psi) &= \Max_{x \in \RR^n : \| x - a \| \le r} \langle x, \psi\rangle = \Max_{y \in \RR^n : \| y \| \le r} \langle a + y, \psi\rangle = \\
%             &= \langle a, \psi\rangle + \Max_{y \in \RR^n : \| y \| \le r} \langle y, \psi\rangle = \langle a, \psi\rangle + c(\KK_r(0), \psi) = \langle a, \psi \rangle + r\|\psi\|.
%         \end{align*}
%     \end{enumerate}
% \end{solution}

\begin{problem}
    Знайти інтеграл Аумана $\JJ = \int_0^{\pi/2} F(x) dx$ таких багатозначних відображень:
    \begin{enumerate}
        \item $F(x) = [0, \sin x]$, $x \in [0, \pi / 2]$.
        \item $F(x) = [-\sin x, \sin x]$, $x \in [0, \pi / 2]$.
        \item $F(x) = \KK_{\sin x}(0) = \{ y \in \RR^n : \| y \| \le \sin x \}$, $x \in [0, \pi / 2]$.
    \end{enumerate}
\end{problem}

% \begin{solution}
%     Скористаємося теоремою про зміну порядку інтегрування і взяття опорної функції:
%     \begin{enumerate}
%         \item $c(\JJ, \psi) = \int_0^{\pi/2} c([0, \sin x], \psi) dx = \int_0^{\pi/2} \max(0, \psi) \sin x dx = \max(0, \psi)$, звідки $\JJ = [0, 1]$.
%         \item $c(\JJ, \psi) = \int_0^{\pi/2} c([-\sin x, \sin x], \psi) dx = \int_0^{\pi/2} |\psi| \sin x dx = |\psi|$, звідки $\JJ = [-1, 1]$.
%         \item $c(\JJ, \psi) = \int_0^{\pi/2} c(\KK_{\sin x}(0), \psi) dx = \int_0^{\pi/2} \sin x \|\psi\| dx = \|\psi\|$, звідки $\JJ = \KK_1(0)$. 
%     \end{enumerate}
% \end{solution}

\begin{problem}
    Знайти множину досяжності такої системи керування:
    \[\frac{\diff x}{\diff t} = x + bu,\] 
    де $x(0) = x_0 \in \MM_0$, $u(t)\in \UU$, $t\ge0$, $b$ -- деяке ненульове число, 
    \[ \MM_0 = \{ x : | x | \le 2 \}, \]
    \[ \UU = \{ u : |u| \le 3 \}. \]
\end{problem}

% \begin{solution}
%     Множину досяжності знайдемо через її опорну функцію: 
%     \[ c(\XX(t, \MM_0), \psi) = c(\MM_0, \Theta^*(t, t_0) \psi) + \int_{t_0}^t c(\UU(s), C^\star(s) \Theta^*(t, s)\psi) \diff s. \]
%     Для цього послідовно знаходимо: \\
    
%     $\Theta(t, s) = e^{t-s}$, знайдено із рівності $\dfrac{d\Theta(t,s)}{dt} = A(t)\Theta(t,s) = \Theta(t,s)$ у нашому випадку. \\
    
%     $c(\MM_0, \psi) = c([-2, 2], \psi) = 2 |\psi|$, вже достатньо відома нам опорна функція. \\
    
%     $c(\UU(s), \psi) = c([-3, 3], \psi) = 3 |\psi|$, ще одна вже достатньо відома нам опорна функція. \\
    
%     % Послідовно пісдтавляючи знайдені вирази в формулу вище знаходимо:
%     % \begin{equation*}
%     % \begin{split}
%     %     c(X(t, \MM_0), \psi) &= c(\MM_0, \Theta^\star(t, t_0), \psi) + \int_{t_0}^t c(\UU(s), C^\star(s) \Theta^\star(t, s)\psi) ds = \\
%     %     &= c([-2,2], \Theta^\star(t, 0), \psi) + \int_0^t c([-3, 3], b \Theta^\star(t, s)\psi) ds = \\
%     %     &= 2\left|\Theta^\star(t, 0)\psi\right| + \int_0^t 3\left|b \Theta^\star(t, s)\psi\right| ds = \\
%     %     &= 2\left|e^{-t}\psi\right| + \int_0^t 3\left|b e^{s-t}\psi\right| ds = 2e^{-t}|\psi| + 3|b \psi| \int_0^t e^{s-t} ds = \\
%     %     &= 2e^{-t}|\psi| + 3|b \psi| \left(1 - e^{-t}\right) = \left(2e^{-t} + 3|b|\left(1 - e^{-t}\right)\right) |\psi|,
%     % \end{split}
%     % \end{equation*}
%     % звідки $X(t, \MM_0) = \left[-2e^{-t} - 3|b|\left(1 - e^{-t}\right), 2e^{-t} + 3|b|\left(1 - e^{-t}\right)\right]$.
    
    
%     Послідовно пісдтавляючи знайдені вирази в формулу вище знаходимо:
%     \begin{equation*}
%     \begin{split}
%         c(X(t, \MM_0), \psi) &= c(\MM_0, \Theta^\star(t, t_0), \psi) + \int_{t_0}^t c(\UU(s), C^\star(s) \Theta^\star(t, s)\psi) ds = \\
%         &= c([-2,2], \Theta^\star(t, 0), \psi) + \int_0^t c([-3, 3], b \Theta^\star(t, s)\psi) ds = \\
%         &= 2\left|\Theta^\star(t, 0)\psi\right| + \int_0^t 3\left|b \Theta^\star(t, s)\psi\right| ds = \\
%         &= 2\left|e^t\psi\right| + \int_0^t 3\left|b e^{t-s}\psi\right| ds = 2e^t|\psi| + 3|b \psi| \int_0^t e^{t-s} ds = \\
%         &= 2e^t|\psi| + 3|b \psi| \left(e^t - 1\right) = \left(2e^t + 3|b|\left(e^t - 1\right)\right) |\psi|,
%     \end{split}
%     \end{equation*}
%     звідки $\XX(t, \MM_0) = \left[-2e^t - 3|b|\left(e^t - 1\right), 2e^t + 3|b|\left(e^t - 1\right)\right]$.
% \end{solution}

\begin{problem}
Знайти опорну функцію множини досяжності для системи керування:
\begin{equation*}
    \left\{
    \begin{aligned}
    \dfrac{dx_1}{dt} &= x_1 - x_2 + 2u_1, \\
    \dfrac{dx_2}{dt} &= -4x_1 + x_2 + u_2,
    \end{aligned}
    \right.
\end{equation*}
де $x(0) = (x_{01}, x_{02}) \in \mathcal{M}_0$, $u(t) = (u_1(t), u_2(t)) \in\mathcal{U}$, $t\ge0$,
\begin{align*}
    \mathcal{M}_0 &= \{(x_{01},x_{02}): x_{01}^2 + x_{02}^2 \le 4\}, \\
    \mathcal{U} &= \{(u_1, u_2): u_1^2 + u_2^2 \le 1\}.
\end{align*}
\end{problem}

% \begin{solution}
%     Одразу помітимо, що $C=\begin{pmatrix}2&0\\0&1\end{pmatrix}$.\\

%     $\Theta(t,s)$ знайдемо розв'язавши однорідну систему:
%     \begin{equation*}
%         \left\{
%         \begin{aligned}
%         \dfrac{dx_1}{dt} &= x_1 - x_2, \\
%         \dfrac{dx_2}{dt} &= -4x_1 + x_2,
%         \end{aligned}
%         \right.
%     \end{equation*}
    
%     Її визначник $\begin{vmatrix} 1 - \lambda & - 1 \\ - 4 & 1 - \lambda \end{vmatrix} = (1 - \lambda)^2 - 4 = (\lambda + 1) (\lambda - 3) = 0$, звідки $\lambda_1 = -1$, $\lambda_2 = 3$. \\
    
%     Підставляючи знайдені числа у систему, знаходимо власні вектори: $\begin{pmatrix} 1 \\ 2 \end{pmatrix}$ та $\begin{pmatrix} 1 \\ -2 \end{pmatrix}$ відповідно. \\

%     Отже загальний розв'язок має вигляд \[\begin{pmatrix} x_1 \\ x_2 \end{pmatrix}(t) = c_1 \begin{pmatrix} e^{-t} \\ 2e^{-t} \end{pmatrix} + c_2 \begin{pmatrix} e^{3t} \\ -2e^{3t} \end{pmatrix}\]
    
%     Розв'язуючи рівняння
%     \[ c_1 \begin{pmatrix} e^{-s} \\ 2e^{-s} \end{pmatrix} + c_2 \begin{pmatrix} e^{3s} \\ -2e^{3s} \end{pmatrix} = \begin{pmatrix} 1 \\ 0 \end{pmatrix} \]
%     і
%     \[ c_1 \begin{pmatrix} e^{-s} \\ 2e^{-s} \end{pmatrix} + c_2 \begin{pmatrix} e^{3s} \\ -2e^{3s} \end{pmatrix} = \begin{pmatrix} 0 \\ 1 \end{pmatrix}, \]
%     знаходимо фундаментальну матрицю системи, нормовану за моментом $s$, а саме 
%     \[ \Theta(t,s) = \begin{pmatrix} \dfrac{e^{s-t} + e^{3(t-s)}}{2} & \dfrac{e^{s-t} - e^{3(t-s)}}{4} \\ e^{s-t} - e^{3(t-s)} & \dfrac{e^{s-t} + e^{3(t-s)}}{2} \end{pmatrix} \]
    
    
%     Далі знаходимо $c(\mathcal{M}_0, \psi) = c(\mathcal{K}_2(0), \psi) = 2\|\psi\|$, та $c(\mathcal{U}, \psi) = c(\mathcal{K}_1(0), \psi) = \|\psi\|$, вже достатньо відомі нам опорні функції. \\
    
%     Нарешті, можемо зібрати це все докупи: 
%     \begin{align*}
%         c(\mathcal{X}(t, \mathcal{M}_0), \psi) &= c(\mathcal{M}_0, \Theta^\star(t, 0) \psi) + \int_{0}^t c(\mathcal{U}(s), C^\star(s) \Theta^\star(t, s)\psi) ds = \\
%         \\
%         &= 2 \|\Theta^\star(t, 0) \psi\| + \int_{0}^t \left\|C^\star(s) \Theta^\star(t, s)\psi\right\| ds = \\
%         \\
%         &= 2 \left\|\begin{pmatrix} \dfrac{e^{-t} + e^{3t}}{2} & e^{3t} - e^{-t} \\ \dfrac{e^{3t} - e^{-t}}{4} & \dfrac{e^{-t} + e^{3t}}{2} \end{pmatrix} \begin{pmatrix} \psi_1 \\ \psi_2 \end{pmatrix}\right\| + \\
%         \\
%         &+ \int_{0}^t \left\|\begin{pmatrix} e^{s-t} + e^{3(t-s)} & 2(e^{3(t-s)} - e^{s-t}) \\ \dfrac{e^{3(t-s)} - e^{s-t}}{4} & \dfrac{e^{s-t} + e^{3(t-s)}}{2} \end{pmatrix} \begin{pmatrix} \psi_1 \\ \psi_2 \end{pmatrix}\right\| ds = \\
%         \\
%         &= 2 \left\| \begin{pmatrix} \dfrac{e^{-t} + e^{3t}}{2} \cdot \psi_1 + (e^{3t} - e^{-t}) \cdot \psi_2 \\ \dfrac{e^{3t} - e^{-t}}{4}\cdot\psi_1 + \dfrac{e^{-t} + e^{3t}}{2}\cdot\psi_2 \end{pmatrix} \right\| + ...
%     \end{align*}
% \end{solution} 

\end{document}
