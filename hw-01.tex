% OK, incomplete, missing 1.7

\subsection*{Домашнє завдання}

\begin{problem}
	Задана система керування 
	\begin{equation}
		\label{eq:1.5}
		\left\{
			\begin{aligned}
				\dfrac{dx_1(t)}{dt} &= -8x_1(t)  -x_2(t) + u(t),\\
				\dfrac{dx_2(t)}{dt} &= 6x_1(t) + 3x_2(t),
			\end{aligned}
		\right.
		\quad
		x_1(0) = -2, x_2(0) = 1.
	\end{equation}

	Тут $ x =(x_1, x_2)^*$ -- вектор фазових координат з $\RR^2$, $t \in [0, 1]$. Керування задане у вигляді
	\begin{equation}
		\label{eq:1.6}
		u(x_1, x_2) = 4x_1 - x_2.
	\end{equation}

	\begin{enumerate}
		\item До якого класу керувань належить керування (\ref{eq:1.6}): програмних керувань, чи керувань з оберненим зв'язком?
		\item Знайти траєкторію системи при керуванні (\ref{eq:1.6}).
		\item Знайти програмне керування $u(t) = 4x_1(t) - x_2(t)$, яке відповідає знайденій траєкторії.
		\item Якою буде фундаментальна матриця, нормована за моментом $s$, системи, що одержана при підстановці керування (\ref{eq:1.6}) в систему (\ref{eq:1.5})?
		\item Побудувати спряжену систему до системи, одержаної при підстановці керування (\ref{eq:1.6}) в систему (\ref{eq:1.5}), та її фундаментальну матрицю.
	\end{enumerate}
\end{problem}

\begin{solution}
	\begin{enumerate}
		\item З оберненим зв'язком.
		\item 
		\[
			\begin{pmatrix}
				\dot x_1 \\ 
				\dot x_2
			\end{pmatrix}
			=
			\begin{pmatrix}
				-4 & -2 \\
				6 & 3
			\end{pmatrix}
			\begin{pmatrix}
				x_1 \\
				x_2
			\end{pmatrix}.
		\]
		Характеристичне рівняння 
		\[
			\begin{vmatrix}
				-4 - \lambda & -2 \\
				6 & 3 - \lambda 
			\end{vmatrix}
			=
			\lambda^2 + \lambda 
			=
			(\lambda + 1) \lambda
			=
			0,
		\]
		звідки $\lambda_1 = -1$, $\lambda_2 = 0$.\\
		
		З курсу диференційних рівнянь відомо, що тоді загальний розв'язок має вигляд
		\[
			\begin{pmatrix}
				x_1 \\
				x_2
			\end{pmatrix}
			=
			c_1 v_1 e^{-t} + c_2 v_2,
		\]
		де $v_1$, $v_2$ -- власні вектори, що відповідають $\lambda_1$ та $\lambda_2$ відповідно.\\
		
		Нескладно бачити, що 
		\[
			\begin{pmatrix}
				x_1 \\
				x_2
			\end{pmatrix}
			=
			c_1 
			\begin{pmatrix}
				1 \\
				-2
			\end{pmatrix} 
			+ 
			c_2 
			\begin{pmatrix}
				2 \\
				-3
			\end{pmatrix}
			e^{-t} .
		\]
		
		Підставляючи $t=0$ отримуємо $c_1 = 4$, $c_2 = -3$.\\
		
		Остаточно маємо:
		\[
			\begin{pmatrix}
				x_1 \\
				x_2
			\end{pmatrix}
			=
			4
			\begin{pmatrix}
				1 \\
				-2
			\end{pmatrix} 
			-3
			\begin{pmatrix}
				2 \\
				-3
			\end{pmatrix}
			e^{-t}
			.
		\]
		\item Просто підставляємо знайдені $x_1(t)$, $x_2(t)$ в $u(x_1, x_2)$:
		\[
			u(t)
			=
			4\left(4\cdot (1) - 3\cdot (2)\cdot e^{-t}\right)
			-
			\left(4\cdot (-2) - 3\cdot (-3)\cdot e^{-t}\right)
			=
			24 - 33e^{-t}.
		\]
		\item З вигляду загального розв'язку бачимо, що вищезгадана фундаментальна матриця матиме вигляд
		\[
			\Theta(t,s)
			=
			\begin{pmatrix}
				c_1 + 2c_2 e^{s-t} & c_3 + 2c_4 e^{s-t} \\
				-2c_1 - 3c_2 e^{s-t} & -2c_3 - 3c_4 e^{s-t} 
			\end{pmatrix},
		\]
		причому 
		\[
			\left\{
				\begin{aligned}
					c_1   &+ 2c_2 &= 1 \\
					-2c_1 &- 3c_2 &= 0
				\end{aligned}
			\right.		
		\]
		(і аналогічна система для $c_3$, $c_4$).\\
		
		Знаходимо $c_1 = -3$, $c_2 = 2$, $c_3 = -2$, $c_4 = 1$ і підставляємо у матрицю:
		\[
			\Theta(t,s)
			=
			\begin{pmatrix}
				-3 + 4e^{s-t} & -2 + 2e^{s-t} \\
				6 - 6e^{s-t} & 4 - 3e^{s-t} 
			\end{pmatrix},
		\]
		
		\item Спряжена система буде
		\[
			\begin{pmatrix}
				\dot y_1 \\
				\dot y_2
			\end{pmatrix}
			=
			\begin{pmatrix}
				4 & -6 \\
				2 & -3
			\end{pmatrix}
			\begin{pmatrix}
				y_1 \\
				y_2
			\end{pmatrix},
		\]
		а відповідна фундаметальна матриця
		\[
			\Psi(t,s)
			=
			\Theta^*(s,t)
			=
			\begin{pmatrix}
				-3 + 4e^{t-s} & 6 - 6e^{t-s} \\
				-2 + 2e^{t-s} & 4 - 3 e^{t-s}
			\end{pmatrix},
		\]
	\end{enumerate}
\end{solution}

\begin{problem}
	Розглядається задача Лагранжа
		\[
		\JJ(u)
		=
		\int_0^T u^2(s) \diff s \to \inf
		\]
		за умови, що
		\[
			\left\{
				\begin{aligned}
					\dfrac{dx_1(t)}{dt} &= -x_1(t) + x_2(t) + u(t), \\
					\dfrac{dx_2(t)}{dt} &= x_1(t)x_2(t),
				\end{aligned}
			\right.
			\quad
			x_1(0)=0,x_2(0)=1.
		\]
		Тут $x=(x_1,x_2)^*$ -- вектор фазових координат з $\RR^2$, $t\in[0,T]$. Звести цю задачу до задачі Майєра.
\end{problem}

\begin{solution}
	Введемо нову фазову координату $x_3(t)=\int_0^t u^2(s) \diff s$, тоді до системи додається початкова умова $x_3(0)=0$, рівняння $\dot x_3 = u^2$, а функціонал якості переписується у вигляді $x_3(T) \to \inf$.
\end{solution}


\begin{problem}
	% 1.7
\end{problem}

\begin{solution}
	% 1.7
\end{solution}
