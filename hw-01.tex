\section{Домашнє завдання за 9/7}

\setcounter{problem}{3}

\begin{problem}
	Задана система керування 
	\[
		\left\{
			\begin{aligned}
				\dfrac{dx_1(t)}{dt} &= 2x_1(t) + x_2(t) + u(t), \\
				\dfrac{dx_2(t)}{dt} &= 3x_1(t) + 4x_2(t),
			\end{aligned}
		\right.
		\quad
		x_1(0) = 1, x_2(0) = -1.
	\]

	Тут $x = (x_1, x_2)^*$ -- вектор фазових координат з $\RR^2$, $t \in [0, 2]$. Керування задане у вигляді
	\[
		u(t) 
		=
		\begin{cases}
			0, & \text{якщо } t \in [0, 1], \\
			1, & \text{якщо } t \in (1, 2].
		\end{cases}
	\]
	
	\begin{enumerate}
		\item Знайти траєкторію системи, яка відповідає цьому керуванню.
		\item Чи буде ця траєкторія неперервно диференційовною?
		\item Чи буде таке керування кращим в порівнянні з керуванням 
		\[
			u(t) = 0, t \in [0, 2]
		\]
		за умови, що критерій якості має вигляд 
		\[
			\mathcal{J}(u) = x_1^2(2) + x_2^2(2) \to \min.
		\]
	\end{enumerate}
\end{problem}

\begin{solution}
	\begin{enumerate}
		\item При $t \in [0, 1]$ маємо 
		\[
			\begin{pmatrix}
				\dot x_1 \\ 
				\dot x_2
			\end{pmatrix}
			=
			\begin{pmatrix}
				2 & 1 \\
				3 & 4
			\end{pmatrix}
			\begin{pmatrix}
				x_1 \\
				x_2
			\end{pmatrix}.
		\]
		Характеристичне рівняння 
		\[
			\begin{vmatrix}
				2 - \lambda & 1 \\
				3 & 4 - \lambda 
			\end{vmatrix}
			=
			\lambda^2 - 6\lambda + 5 
			=
			(\lambda - 1) (\lambda - 5)
			=
			0,
		\]
		звідки $\lambda_1 = 1$, $\lambda_2 = 5$.\\
		
		З курсу диференційних рівнянь відомо, що тоді загальний розв'язок має вигляд
		\[
			\begin{pmatrix}
			x_1 \\
			x_2
			\end{pmatrix}
			=
			c_1 v_1 e^t + c_2 v_2 e^{5t},
		\]
		де $v_1$, $v_2$ -- власні вектори, що відповідають $\lambda_1$ та $\lambda_2$ відповідно.\\
		
		Нескладно бачити, що 
		\[
			\begin{pmatrix}
				x_1 \\
				x_2
			\end{pmatrix}
			=
			c_1 
			\begin{pmatrix}
				1 \\
				-1
			\end{pmatrix} 
			e^t 
			+ 
			c_2 
			\begin{pmatrix}
				1 \\
				3
			\end{pmatrix} 
			e^{5t}.
		\]
		
		Підставляючи $t = 0$ отримуємо $c_1 = 1$, $c_2 = 0$.
		
		При $t \in (1, 2]$ маємо 
		\[
			\begin{pmatrix}
				x_1 \\
				x_2
			\end{pmatrix}
			=
			c_1 v_1 e^t 
			+ 
			c_2 v_2 e^{5t} 
			+ 
			\begin{pmatrix} 
				c_3 \\
				c_4
			\end{pmatrix},
		\]
		де $c_3$, $c_4$ задовольняють систему 
		\[
			\left\{
				\begin{aligned}
					2c_3 &+ c_4 + 1 &= 0 \\
					3c_3 &+ 4c_4 &= 0
				\end{aligned}
			\right.,
		\]
		звідки $c_3 = -4/5$, $c_4 = 3/5$ і 
		\[
			\begin{pmatrix}
				x_1 \\
				x_2
			\end{pmatrix}
			=
			c_1 v_1 e^t 
			+ 
			c_2 v_2 e^{5t} 
			+ 
			\begin{pmatrix} 
				-4/5 \\
				3/5
			\end{pmatrix},
		\]
		Підставляючи $t = 1$ отримуємо $c_1 = \left(1 + \dfrac{3}{4e}\right)$, $c_2 = \dfrac{1}{20e^5}$.\\
		
		Остаточно маємо 
		\[
			\begin{pmatrix}
			x_1 \\
			x_2
			\end{pmatrix}
			=
			\begin{cases}
				\begin{pmatrix}
					1 \\
					-1
				\end{pmatrix} 
				e^t, & t \in [0, 1] \\
				\left(1 + \dfrac{3}{4e}\right) 
				\begin{pmatrix}
				1 \\
				-1
				\end{pmatrix} 
				e^t 
				+ 
				\dfrac{1}{20e^5} 
				\begin{pmatrix}
				1 \\
				3
				\end{pmatrix} 
				e^{5t} 
				+ 
				\begin{pmatrix} 
					-4/5 \\
					3/5
				\end{pmatrix}			
				, & t \in (1, 2]
			\end{cases}.
		\]
		\item 
		\[
			\begin{pmatrix}
				\dot x_1 \\
				\dot x_2
			\end{pmatrix}
			(1-) 
			= 
			\begin{pmatrix}
				2 & 1 \\
				3 & 4
			\end{pmatrix}
			\begin{pmatrix}
				x_1(1-) \\
				x_2(1-)
			\end{pmatrix}
		\]
		З неперервності $x_1$, $x_2$ маємо:
		\[
			\begin{pmatrix}
				2 & 1 \\
				3 & 4
			\end{pmatrix}
			\begin{pmatrix}
				x_1(1-) \\
				x_2(1-)
			\end{pmatrix}
			=
			\begin{pmatrix}
			2 & 1 \\
			3 & 4
			\end{pmatrix}
			\begin{pmatrix}
			x_1(1) \\
			x_2(1)
			\end{pmatrix}
		\]
		З іншого боку,
		\[
			\begin{pmatrix}
				\dot x_1 \\
				\dot x_2
			\end{pmatrix}
			(1+) 
			= 
			\begin{pmatrix}
				2 & 1 \\
				3 & 4
			\end{pmatrix}
			\begin{pmatrix}
				x_1(1+) \\
				x_2(1+)
			\end{pmatrix}
			+
			\begin{pmatrix}
				1 \\
				0
			\end{pmatrix}
			= 
			\begin{pmatrix}
				2 & 1 \\
				3 & 4
			\end{pmatrix}
			\begin{pmatrix}
				x_1(1) \\
				x_2(1)
			\end{pmatrix}
			+
			\begin{pmatrix}
				1 \\
				0
			\end{pmatrix}
		\]
		Нескладно бачити, що 
		\[
			\begin{pmatrix}
				\dot x_1 \\
				\dot x_2
			\end{pmatrix}
			(1-)
			\ne
			\begin{pmatrix}
				\dot x_1 \\
				\dot x_2
			\end{pmatrix}
			(1+),
		\]
		тобто траєкторія не є неперервно диференційовною в точці $1$.
		\item 
		Просто підставимо $t=2$ в розв'язки для обох керувань (попутно зауваживши, що для нового керування розв'язок ми вже знаємо, це просто продовження вже знайденого розв'язку для $t \in [0, 1]$):
		\[
			\left(e^2 + \dfrac34e + \dfrac{e^5}{20} - \dfrac45\right)^2 + \left(-e^2 - \dfrac34e + \dfrac{3e^5}{20} + \dfrac35\right)^2
			\lor
			(e^2)^2 + (-e^2)^2
		\]
		Після марудних обчислень знаходимо, що права частина менше, тобто нове керування є кращим за початкове.
	\end{enumerate}
\end{solution}

\begin{problem}
	Задана система керування 
	\begin{equation}
		\label{eq:1.5}
		\left\{
			\begin{aligned}
				\dfrac{dx_1(t)}{dt} &= -8x_1(t)  -x_2(t) + u(t),\\
				\dfrac{dx_2(t)}{dt} &= 6x_1(t) + 3x_2(t),
			\end{aligned}
		\right.
		\quad
		x_1(0) = -2, x_2(0) = 1.
	\end{equation}

	Тут $ x =(x_1, x_2)^*$ -- вектор фазових координат з $\RR^2$, $t \in [0, 1]$. Керування задане у вигляді
	\begin{equation}
		\label{eq:1.6}
		u(x_1, x_2) = 4x_1 - x_2.
	\end{equation}

	\begin{enumerate}
		\item До якого класу керувань належить керування \LaReF{eq:1.6}: програмних керувань, чи керувань з оберненим зв'язком?
		\item Знайти траєкторію системи при керуванні \LaReF{eq:1.6}.
		\item Знайти програмне керування $u(t) = 4x_1(t) - x_2(t)$, яке відповідає знайденій траєкторії.
		\item Якою буде фундаментальна матриця, нормована за моментом $s$, системи, що одержана при підстановці керування \LaReF{eq:1.6} в систему \LaReF{eq:1.5}?
		\item Побудувати спряжену систему до системи, одержаної при підстановці керування \LaReF{eq:1.6} в систему \LaReF{eq:1.5}, та її фундаментальну матрицю.
	\end{enumerate}
\end{problem}

\begin{solution}
	\begin{enumerate}
		\item З оберненим зв'язком.
		\item 
		\[
			\begin{pmatrix}
				\dot x_1 \\ 
				\dot x_2
			\end{pmatrix}
			=
			\begin{pmatrix}
				-4 & -2 \\
				6 & 3
			\end{pmatrix}
			\begin{pmatrix}
				x_1 \\
				x_2
			\end{pmatrix}.
		\]
		Характеристичне рівняння 
		\[
			\begin{vmatrix}
				-4 - \lambda & -2 \\
				6 & 3 - \lambda 
			\end{vmatrix}
			=
			\lambda^2 + \lambda 
			=
			(\lambda + 1) \lambda
			=
			0,
		\]
		звідки $\lambda_1 = -1$, $\lambda_2 = 0$.\\
		
		З курсу диференційних рівнянь відомо, що тоді загальний розв'язок має вигляд
		\[
			\begin{pmatrix}
				x_1 \\
				x_2
			\end{pmatrix}
			=
			c_1 v_1 e^{-t} + c_2 v_2,
		\]
		де $v_1$, $v_2$ -- власні вектори, що відповідають $\lambda_1$ та $\lambda_2$ відповідно.\\
		
		Нескладно бачити, що 
		\[
			\begin{pmatrix}
				x_1 \\
				x_2
			\end{pmatrix}
			=
			c_1 
			\begin{pmatrix}
				1 \\
				-2
			\end{pmatrix} 
			+ 
			c_2 
			\begin{pmatrix}
				2 \\
				-3
			\end{pmatrix}
			e^{-t} .
		\]
		
		Підставляючи $t=0$ отримуємо $c_1 = 4$, $c_2 = -3$.\\
		
		Остаточно маємо:
		\[
			\begin{pmatrix}
				x_1 \\
				x_2
			\end{pmatrix}
			=
			4
			\begin{pmatrix}
				1 \\
				-2
			\end{pmatrix} 
			-3
			\begin{pmatrix}
				2 \\
				-3
			\end{pmatrix}
			e^{-t}
			.
		\]
		\item Просто підставляємо знайдені $x_1(t)$, $x_2(t)$ в $u(x_1, x_2)$:
		\[
			u(t)
			=
			4\left(4\cdot (1) - 3\cdot (2)\cdot e^{-t}\right)
			-
			\left(4\cdot (-2) - 3\cdot (-3)\cdot e^{-t}\right)
			=
			24 - 33e^{-t}.
		\]
		\item З вигляду загального розв'язку бачимо, що вищезгадана фундаментальна матриця матиме вигляд
		\[
			\Theta(t,s)
			=
			\begin{pmatrix}
				c_1 + 2c_2 e^{s-t} & c_3 + 2c_4 e^{s-t} \\
				-2c_1 - 3c_2 e^{s-t} & -2c_3 - 3c_4 e^{s-t} 
			\end{pmatrix},
		\]
		причому 
		\[
			\left\{
				\begin{aligned}
					c_1   &+ 2c_2 &= 1 \\
					-2c_1 &- 3c_2 &= 0
				\end{aligned}
			\right.		
		\]
		(і аналогічна система для $c_3$, $c_4$).\\
		
		Знаходимо $c_1 = -3$, $c_2 = 2$, $c_3 = -2$, $c_4 = 1$ і підставляємо у матрицю:
		\[
			\Theta(t,s)
			=
			\begin{pmatrix}
				-3 + 4e^{s-t} & -2 + 2e^{s-t} \\
				6 - 6e^{s-t} & 4 - 3e^{s-t} 
			\end{pmatrix},
		\]
		
		\item Спряжена система буде
		\[
			\begin{pmatrix}
				\dot y_1 \\
				\dot y_2
			\end{pmatrix}
			=
			\begin{pmatrix}
				4 & -6 \\
				2 & -3
			\end{pmatrix}
			\begin{pmatrix}
				y_1 \\
				y_2
			\end{pmatrix},
		\]
		а відповідна фундаметальна матриця
		\[
			\Psi(t,s)
			=
			\Theta^*(s,t)
			=
			\begin{pmatrix}
				-3 + 4e^{t-s} & 6 - 6e^{t-s} \\
				-2 + 2e^{t-s} & 4 - 3 e^{t-s}
			\end{pmatrix},
		\]
	\end{enumerate}
\end{solution}

\begin{problem}
	Розглядається задача Лагранжа
		\[
		\mathcal{J}(u)
		=
		\Int_0^T u^2(s) ds \to \inf
		\]
		за умови, що
		\[
			\left\{
				\begin{aligned}
					\dfrac{dx_1(t)}{dt} &= -x_1(t) + x_2(t) + u(t), \\
					\dfrac{dx_2(t)}{dt} &= x_1(t)x_2(t),
				\end{aligned}
			\right.
			\quad
			x_1(0)=0,x_2(0)=1.
		\]
		Тут $x=(x_1,x_2)^*$ -- вектор фазових координат з $\RR^2$, $t\in[0,T]$. Звести цю задачу до задачі Майєра.
\end{problem}

\begin{solution}
	Введемо нову фазову координату $x_3(t)=\Int_0^t u^2(s) ds$, тоді до системи додається початкова умова $x_3(0)=0$, рівняння $\dot x_3 = u^2$, а функціонал якості переписується у вигляді $x_3(T) \to \inf$
\end{solution}
